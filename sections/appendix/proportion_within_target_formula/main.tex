\section{Mean blocking time} \label{sec:appendix_mean_proportion}
    
In order to consider such measure though one would need to obtain the 
distribution of time in the system for all individuals. 
The complexity of such task lies on the fact that different individuals arrive 
at different states of the Markov model. 
Consider the case when an arrival occurs when the model is at a specific state.

\paragraph{Time distribution at specific state (1 server):}

\begin{figure}[H]
    \centering
    \scalebox{0.75}{
        \documentclass{article}

\usepackage{amsmath}
\usepackage{amsfonts} 
\usepackage{geometry}
\usepackage{multicol}
\usepackage{float}
% \usepackage{mathtools}
% \usepackage{graphicx}
% \usepackage{soul}
% \usepackage{indentfirst}
\usepackage{tikz}
\usetikzlibrary{calc, automata, chains, arrows.meta, math}
\setcounter{MaxMatrixCols}{20}


\title{A game theoretic model of the behavioural gaming that takes place at the EMS - ED interface}

\author{
    Michalis Panayides, 
    Paul Harper, 
    Vince Knight
}

\begin{document}

\maketitle

\documentclass{article}

\usepackage{amsmath}
\usepackage{amsfonts} 
\usepackage{geometry}
\usepackage{multicol}
\usepackage{float}
% \usepackage{mathtools}
% \usepackage{graphicx}
% \usepackage{soul}
% \usepackage{indentfirst}
\usepackage{tikz}
\usetikzlibrary{calc, automata, chains, arrows.meta, math}
\setcounter{MaxMatrixCols}{20}


\title{A game theoretic model of the behavioural gaming that takes place at the EMS - ED interface}

\author{
    Michalis Panayides, 
    Paul Harper, 
    Vince Knight
}

\begin{document}

\maketitle

\documentclass{article}

\usepackage{amsmath}
\usepackage{amsfonts} 
\usepackage{geometry}
\usepackage{multicol}
\usepackage{float}
% \usepackage{mathtools}
% \usepackage{graphicx}
% \usepackage{soul}
% \usepackage{indentfirst}
\usepackage{tikz}
\usetikzlibrary{calc, automata, chains, arrows.meta, math}
\setcounter{MaxMatrixCols}{20}


\title{A game theoretic model of the behavioural gaming that takes place at the EMS - ED interface}

\author{
    Michalis Panayides, 
    Paul Harper, 
    Vince Knight
}

\begin{document}

\maketitle

\input{Abstract/main.tex}


\newpage
\tableofcontents

\newpage
\input{Introduction/main.tex}

\newpage
\input{Game_theory_component/main.tex}

\newpage
\input{MarkovChain/markov_chain_model/main.tex}
\input{MarkovChain/expressions_from_pi/main.tex}
\input{MarkovChain/markov_example/main.tex}

\newpage
\input{BehaviouralMethodology/main.tex}

\newpage
\input{Application_EMS_ED/main.tex}

\newpage
\input{Conclusion/main.tex}


\end{document}


\newpage
\tableofcontents

\newpage
\documentclass{article}

\usepackage{amsmath}
\usepackage{amsfonts} 
\usepackage{geometry}
\usepackage{multicol}
\usepackage{float}
% \usepackage{mathtools}
% \usepackage{graphicx}
% \usepackage{soul}
% \usepackage{indentfirst}
\usepackage{tikz}
\usetikzlibrary{calc, automata, chains, arrows.meta, math}
\setcounter{MaxMatrixCols}{20}


\title{A game theoretic model of the behavioural gaming that takes place at the EMS - ED interface}

\author{
    Michalis Panayides, 
    Paul Harper, 
    Vince Knight
}

\begin{document}

\maketitle

\input{Abstract/main.tex}


\newpage
\tableofcontents

\newpage
\input{Introduction/main.tex}

\newpage
\input{Game_theory_component/main.tex}

\newpage
\input{MarkovChain/markov_chain_model/main.tex}
\input{MarkovChain/expressions_from_pi/main.tex}
\input{MarkovChain/markov_example/main.tex}

\newpage
\input{BehaviouralMethodology/main.tex}

\newpage
\input{Application_EMS_ED/main.tex}

\newpage
\input{Conclusion/main.tex}


\end{document}

\newpage
\documentclass{article}

\usepackage{amsmath}
\usepackage{amsfonts} 
\usepackage{geometry}
\usepackage{multicol}
\usepackage{float}
% \usepackage{mathtools}
% \usepackage{graphicx}
% \usepackage{soul}
% \usepackage{indentfirst}
\usepackage{tikz}
\usetikzlibrary{calc, automata, chains, arrows.meta, math}
\setcounter{MaxMatrixCols}{20}


\title{A game theoretic model of the behavioural gaming that takes place at the EMS - ED interface}

\author{
    Michalis Panayides, 
    Paul Harper, 
    Vince Knight
}

\begin{document}

\maketitle

\input{Abstract/main.tex}


\newpage
\tableofcontents

\newpage
\input{Introduction/main.tex}

\newpage
\input{Game_theory_component/main.tex}

\newpage
\input{MarkovChain/markov_chain_model/main.tex}
\input{MarkovChain/expressions_from_pi/main.tex}
\input{MarkovChain/markov_example/main.tex}

\newpage
\input{BehaviouralMethodology/main.tex}

\newpage
\input{Application_EMS_ED/main.tex}

\newpage
\input{Conclusion/main.tex}


\end{document}

\newpage
\documentclass{article}

\usepackage{amsmath}
\usepackage{amsfonts} 
\usepackage{geometry}
\usepackage{multicol}
\usepackage{float}
% \usepackage{mathtools}
% \usepackage{graphicx}
% \usepackage{soul}
% \usepackage{indentfirst}
\usepackage{tikz}
\usetikzlibrary{calc, automata, chains, arrows.meta, math}
\setcounter{MaxMatrixCols}{20}


\title{A game theoretic model of the behavioural gaming that takes place at the EMS - ED interface}

\author{
    Michalis Panayides, 
    Paul Harper, 
    Vince Knight
}

\begin{document}

\maketitle

\input{Abstract/main.tex}


\newpage
\tableofcontents

\newpage
\input{Introduction/main.tex}

\newpage
\input{Game_theory_component/main.tex}

\newpage
\input{MarkovChain/markov_chain_model/main.tex}
\input{MarkovChain/expressions_from_pi/main.tex}
\input{MarkovChain/markov_example/main.tex}

\newpage
\input{BehaviouralMethodology/main.tex}

\newpage
\input{Application_EMS_ED/main.tex}

\newpage
\input{Conclusion/main.tex}


\end{document}
\subsection{Performance Measures}
One may easily derive the average number of individuals that are at any given state 
using \( pi \). 
The average number of individuals in state \( i \) can be calculated by multiplying 
the number of individuals that are present in state \( i \) with the probability 
of being at that particular state (i.e \(\pi_i (u_i + v_i)\)). 
Using this logic it is possible to calculate any performance measures that are related 
to the mean number of individuals in the system.


Average number of people in the system: 
\begin{equation}
    L = \sum_{i=1}^{|\pi|} \pi_i (u_i + v_i)
\end{equation} 

Average number of people in the service centre: 
\begin{equation}
    L_H = \sum_{i=1}^{|\pi|} \pi_i v_i
\end{equation}

Average number of people in the buffer centre:
\begin{equation}
    L_A = \sum_{i=1}^{|\pi|} \pi_i u_i
\end{equation}

Consequently getting the performance measures that are related to the duration of 
time is not as straightforward. 
Such performance measures are the mean waiting time in the system and the mean time 
blocked in the system. 
Under the scope of this study three approaches have been considered to calculate these 
performance measures; a direct approach, a recursive algorithm and consequently a
closed-form formula.

The research question that needs to be answered here is: ``When a class 1/2 
individuals enters the system, what is the expected time that they will have to 
wait?''. 
In order to formulate the answer to that question one needs to consider all possible 
scenarios of what state the system can be in when an individual arrives. 
Furthermore, different formulas arises for class 1 individuals 
and a different one for class 2 individuals.

\subsubsection{Mean waiting time} 
Upon closer inspection of the recursive formula a more compact formula can arise. 
The equivalent closed-form formula eliminates the need for recursion and thus makes 
the computation of waiting times much more efficient. 
Just like in the recursive part there are two formulas; one for \textit{class 1} 
and one for class 2 individuals. 
The formulas are given by:

\begin{equation} \label{eq:closed_form_waiting_others}
    W^{(1)} = \frac{\sum_{\substack{(u,v) \, \in S_A^{(1)} \\ v \geq C}} 
    \frac{1}{C \mu} \times (v-C+1) \times \pi(u,v)}{\sum_{(u,v) \, 
    \in S_A^{(1)}} \pi(u,v)}
\end{equation}
    
\begin{equation}\label{eq:closed_form_waiting_ambulance}
    W^{(2)} = \frac{\sum_{\substack{(u,v) \, \in S_A^{(2)} \\ min(v,T) \geq C}} 
    \frac{1}{C \mu} \times (\min(v+1,T)-C) \times \pi(u,v)}{\sum_{(u,v) \, 
    \in S_A^{(2)}} \pi(u,v)}
\end{equation}

Note here that the summation, in both equations \ref{eq:closed_form_waiting_others} 
and \ref{eq:closed_form_waiting_ambulance}, goes through all states in the set of 
accepting 
states of either class 1 or class 2 individuals respectively, where a wait 
incurs. 
In equation \ref{eq:closed_form_waiting_others} that includes all states \((u,v)\) 
in the set of accepting states of class 1 individuals such that \( v \geq C\); i.e. 
whenever an arrival occurs and the system is at a state where the number of individuals 
in the system is more than or equal to $C$. 
Consequently, for the states that are included in the summation the expression 
\( v-C+1 \) indicates the amount of people in service one would have to wait for 
upon arrival at the hospital.

Additionally, the minimisation function in equation 
\ref{eq:closed_form_waiting_ambulance} 
ensures that when a class 2 individual arrives at any state 
that is greater than the predetermined threshold, the wait that the individual will 
have to endure remains the same. 
In essence, the expression \(\min(v+1,T) - C\) returns the number of people in line 
in front of a particular individual upon arrival.


\subsubsection{Overall Waiting Time}

Consequently, the overall waiting time should can be estimated by a linear combination 
of the waiting times of class 1 and class 2 individuals. 
The overall waiting time can be then given by the following equation where \(c_1\) 
and \(c_2\) are the coefficients of each individual's type waiting time:

\begin{equation}\label{overall_waiting_time_coeff}
    W = c_1 W^{(1)} + c_2 W^{(2)}
\end{equation}

The two coefficients represent the proportion of individuals of each type that 
traversed through the model. 
Theoretically, getting these percentages should be as simple as looking at the arrival 
rates of each type but in practise if the service centre or the buffer centre 
is full, some individuals may be lost to the system. 
Thus, one should account for the probability that an individual is lost to the system. 
This probability can be easily calculated by using the two sets of accepting states 
\(S_A^{(2)}\) and \(S_A^{(1)}\) defined earlier in equations.
Let us define here the probability, for either class type, that an individual 
is not lost in the system by:

\begin{equation*}
    P(L'_1) = \sum_{(u,v) \, \in S_A^{(1)}} \pi(u,v) \hspace{2cm}
    P(L'_2) = \sum_{(u,v) \, \in S_A^{(2)}} \pi(u,v)
\end{equation*}

Having defined these probabilities one may combine them with the arrival rates of 
each class type in such a way to get the expected proportions of class 1 and 
class 2 individuals in the model. 
Thus, by using these values as the coefficient of equation 
\ref{overall_waiting_time_coeff} 
the resultant equation can be used to get the overall waiting time. 
Note here that the equation below gets the overall waiting time for both the recursive 
and the closed-form formula.

\begin{equation}\label{overall_waiting_time}
    W = \frac{\lambda_1 P(L'_1)}{\lambda_2 P(L'_2) + \lambda_1 P(L'_1)} W^{(1)} + 
    \frac{\lambda_2 P(L'_2)}{\lambda_2 P(L'_2) + \lambda_1 P(L'_1)} W^{(2)}
\end{equation}



\subsubsection{Mean blocking time}
Unlike the waiting time, the blocking time is only calculated for class 2 individuals.  
That is because class 1 individuals cannot be blocked. 
Thus, one only needs to consider the pathway of class 2 individuals to get the 
mean blocking time of the system. 
Blocking occurs at states \((u,v)\) where \(u > 0 \). 
Thus, the set of blocking states can be defined as:

\begin{equation*}
    S_b = \{(u,v) \in S \; | \; u > 0\}
\end{equation*}
 
In order to not consider individuals that will be lost to the system, the set of 
accepting states needs to be taken into account. The set of accepting states is given by:

\begin{equation*}
    S_A^{(2)}=
    \begin{cases}
        \{(u, v) \in S \; | \; u < M \} & \textbf{if } T \leq N\\
        \{(u, v) \in S \; | \; v < N \} & \textbf{otherwise}
    \end{cases}
\end{equation*}

For the waiting time formula,
the mean sojourn time for each state was considered,
ignoring any arrivals. Here, the same approach is used but ignoring only class 2
arrivals. That is because for the waiting time formula, once an individual enters 
the service centre (i.e. starts waiting) any individual arriving after them will 
not affect their
pathway. That is not the case for blocking time. When a class 2 individual is 
blocked, 
any class 1 individual that arrives will cause the blocked individual to remain 
blocked for more time. Therefore, class 1 arrivals are considered here:

\begin{equation}\label{eq:time_in_state_blocking_time}
    c(u,v) = 
    \begin{cases}
        \frac{1}{\min(v,C) \mu}, & \text{if } v = C\\
        \frac{1}{\min(v,C) \mu + \lambda_1}, & \text{otherwise}
    \end{cases}
\end{equation}
 
In equation \ref{eq:time_in_state_blocking_time}, both service completions and 
class 1 arrivals are considered. 
Thus, from a blocked individual's perspective whenever the system moves from one 
state \((u,v)\)
to another state it can either:

\begin{itemize}
    \item be because of a service being completed: we will denote the probability 
    of this happening by \(p_s(u,v)\). 
    \item be because of an arrival of an individual of class 1: denoting such 
    probability by \(p_o(u,v)\).
\end{itemize}
The probabilities are given by:

\begin{equation*}
    p_s(u,v) = \frac{\min(v,C)\mu}{\lambda_1 + \min(v,C)\mu}, \qquad
    p_o(u,v) = \frac{\lambda_1}{\lambda_1 + \min(v,C)\mu}
\end{equation*}


Having defined \(c(u,v)\) and \(S_b\) a formula for the blocking time that is
expected to occur at each state can be given by:

\begin{equation}\label{eq:blocking-time-at-each-state}
    b(u,v) = 
    \begin{cases} 
        0, & \textbf{if } (u,v) \notin S_b \\
        c(u,v) + b(u - 1, v), & \textbf{if } v = N = T\\
        c(u,v) + b(u, v-1), & \textbf{if } v = N \neq T \\
        c(u,v) + p_s(u,v) b(u-1, v) + p_o(u,v) b(u, v+1), & \textbf{if } u > 0 
        \textbf{ and } v = T \\
        c(u,v) + p_s(u,v) b(u, v-1) + p_o(u,v) b(u, v+1), & \textbf{otherwise} \\
    \end{cases}
\end{equation}

Equation 
(\ref{eq:blocking-time-at-each-state}) will not be solved recursively. 
A direct approach will be used to solve this equation here. 
By enumerating all equations of (\ref{eq:blocking-time-at-each-state}) for all 
states \((u,v)\) that belong in \(S_b\) 
a system of linear equations arises where the unknown variables are all the \(b(u,v)\)
terms.
For instance, let us consider a Markov model where \(C=2, T=3, N=6, M=2\). 
The Markov model is shown in Figure \ref{fig:example-algeb-blocking}
and the equivalent equations are 
(\ref{eq:first_eq_of_blocking_example})-(\ref{eq:last_eq_of_blocking_example}).
The equations considered here are only the ones that correspond to the blocking 
states.

\begin{multicols*}{2}
    \begin{figure}[H]
        \scalebox{0.50}{\input{MarkovChain/expressions_from_pi/example_model_2362/main.tex}}
        \caption{Example of Markov chain}
        \label{fig:example-algeb-blocking}
    \end{figure}
    \columnbreak
    \begin{align}
        b(1,2) &= c(1,2) + p_o b(1,3) \label{eq:first_eq_of_blocking_example} \\
        b(1,3) &= c(1,3) + p_s b(1,2) + p_o b(1,4) \\
        b(1,4) &= c(1,4) + b(1,3) \\
        b(2,2) &= c(2,2) + p_s b(1,2) + p_o b(2,3) \\
        b(2,3) &= c(2,3) + p_s b(2,2) + p_o b(1,4) \\
        b(2,4) &= c(2,4) + b(2,3)\label{eq:last_eq_of_blocking_example}
    \end{align}
\end{multicols*}

Additionally, the above equations can be transformed into a linear system of the 
form \(Zx=y\) where:

\begin{equation}\label{eq:example-algebaric-approach-blocking-time}
    Z=
    \begin{pmatrix}
        -1 & p_o & 0 & 0 & 0 & 0 \\ %(1,2)
        p_s & -1 & p_o & 0 & 0 & 0 \\ %(1,3)
        0 & 1 & -1 & 0 & 0 & 0 \\ %(1,4)
        p_s & 0 & 0 & -1 & p_o & 0\\ %(2,2)
        0 & 0 & 0 & p_s & -1 & p_o \\ %(2,3)
        0 & 0 & 0 & 0 & 1 & -1 \\ %(2,4)
    \end{pmatrix},
    x=
    \begin{pmatrix}
        b(1,2) \\
        b(1,3) \\
        b(1,4) \\
        b(2,2) \\
        b(2,3) \\
        b(2,4) \\
    \end{pmatrix}, 
    y=
    \begin{pmatrix}
        -c(1,2) \\
        -c(1,3) \\
        -c(1,4) \\
        -c(2,2) \\
        -c(2,3) \\
        -c(2,4) \\
    \end{pmatrix}
\end{equation}

A more generalised form of the equations in 
(\ref{eq:example-algebaric-approach-blocking-time})
can thus be given for any value of \(C,T,N,M\) by:

\begin{align}
    b(1,T) =& c(1, T) + p_o b(1, T + 1) \label{eq:first_eq_of_blocking_general}\\
    b(1,T + 1) =& c(1, T + 1) + p_s(1, T) + p_o b(1, T + 1) \\
    b(1,T + 2) =& c(1, T + 2) + p_s(1, T + 1) + p_o b(1, T + 3) \\
    & \vdots \nonumber \\
    b(1, N) =& c(1, N) + b(1, N - 1) \\
    b(2, T) =& c(2, T) + p_s b(1, T) + p_o b(2, T + 1) \\
    b(2, T + 1) =& c(2, T + 1) + p_s b(2, T) + p_o b(2, T + 2) \\
    & \vdots \nonumber \\
    b(M, T) =& c(M, T) + b(M, T-1) \label{eq:last_eq_of_blocking_general}
\end{align}

The equivalent matrix form of the linear system of equations 
(\ref{eq:first_eq_of_blocking_general}) - (\ref{eq:last_eq_of_blocking_general})
is given by \(Zx=y\), where:
\begin{equation}\label{eq:general-algebaric-approach-blocking-time}
    \scalebox{0.9}{
        \(
        Z = 
        \begin{pmatrix}
            -1 & p_o & 0 & \dots & 0 & 0 & 0 & 0 & 0 & \dots & 0 & 0 \\ %(1,T)
            p_s & -1 & p_o & \dots & 0 & 0 & 0 & 0 & 0 & \dots & 0 & 0 \\ %(1,T+1)
            0 & p_s & -1 & \dots & 0 & 0 & 0 & 0 & 0 & \dots & 0 & 0 \\ %(1,T+2)
            \vdots & \vdots & \vdots & \ddots & \vdots & \vdots & \vdots & \vdots & 
            \vdots & \ddots & \vdots & \vdots \\ 
            0 & 0 & 0 & \dots & 1 & -1 & 0 & 0 & 0 & \dots & 0 & 0 \\ %(1,N)
            p_s & 0 & 0 & \dots & 0 & 0 & -1 & p_o & 0 & \dots & 0 & 0 \\ %(2,T)
            0 & 0 & 0 & \dots & 0 & 0 & p_s & -1 & p_o & \dots & 0 & 0 \\ %(2,T+1)
            \vdots & \vdots & \vdots & \ddots & \vdots & \vdots & \vdots & \vdots & 
            \vdots & \ddots & \vdots & \vdots \\ 
            0 & 0 & 0 & \dots & 0 & 0 & 0 & 0 & 0 & \dots & 1 & -1 \\ %(M,T)
        \end{pmatrix},
        x = 
        \begin{pmatrix}
            b(1,T) \\
            b(1,T+1) \\
            b(1,T+2) \\
            \vdots \\
            b(1,N) \\
            b(2,T) \\
            b(2,T+1) \\
            \vdots \\
            b(M,T) \\
        \end{pmatrix}, 
        y= 
        \begin{pmatrix}
            -c(1,T) \\
            -c(1,T+1) \\
            -c(1,T+2) \\
            \vdots \\
            -c(1,N) \\
            -c(2,T) \\
            -c(2,T+1) \\
            \vdots \\
            -c(M,T) \\
        \end{pmatrix}
        \)
    }
\end{equation}

Thus, having calculated the mean blocking time for all blocking states \(b(u,v)\), 
it only remains to put them together in a formula.
The resultant blocking time formula is given by:

\begin{equation}\label{eq:algebraic-blocking-time}
    B = \frac{\sum_{(u,v) \in S_A} \pi_{(u,v)} \; b(u,v)}{\sum_{(u,v) \in S_A} 
    \pi_{(u,v)}}
\end{equation}

\documentclass{article}

\usepackage{amsmath}
\usepackage{amsfonts} 
\usepackage{geometry}
\usepackage{multicol}
\usepackage{float}
% \usepackage{mathtools}
% \usepackage{graphicx}
% \usepackage{soul}
% \usepackage{indentfirst}
\usepackage{tikz}
\usetikzlibrary{calc, automata, chains, arrows.meta, math}
\setcounter{MaxMatrixCols}{20}


\title{A game theoretic model of the behavioural gaming that takes place at the EMS - ED interface}

\author{
    Michalis Panayides, 
    Paul Harper, 
    Vince Knight
}

\begin{document}

\maketitle

\input{Abstract/main.tex}


\newpage
\tableofcontents

\newpage
\input{Introduction/main.tex}

\newpage
\input{Game_theory_component/main.tex}

\newpage
\input{MarkovChain/markov_chain_model/main.tex}
\input{MarkovChain/expressions_from_pi/main.tex}
\input{MarkovChain/markov_example/main.tex}

\newpage
\input{BehaviouralMethodology/main.tex}

\newpage
\input{Application_EMS_ED/main.tex}

\newpage
\input{Conclusion/main.tex}


\end{document}

\newpage
\documentclass{article}

\usepackage{amsmath}
\usepackage{amsfonts} 
\usepackage{geometry}
\usepackage{multicol}
\usepackage{float}
% \usepackage{mathtools}
% \usepackage{graphicx}
% \usepackage{soul}
% \usepackage{indentfirst}
\usepackage{tikz}
\usetikzlibrary{calc, automata, chains, arrows.meta, math}
\setcounter{MaxMatrixCols}{20}


\title{A game theoretic model of the behavioural gaming that takes place at the EMS - ED interface}

\author{
    Michalis Panayides, 
    Paul Harper, 
    Vince Knight
}

\begin{document}

\maketitle

\input{Abstract/main.tex}


\newpage
\tableofcontents

\newpage
\input{Introduction/main.tex}

\newpage
\input{Game_theory_component/main.tex}

\newpage
\input{MarkovChain/markov_chain_model/main.tex}
\input{MarkovChain/expressions_from_pi/main.tex}
\input{MarkovChain/markov_example/main.tex}

\newpage
\input{BehaviouralMethodology/main.tex}

\newpage
\input{Application_EMS_ED/main.tex}

\newpage
\input{Conclusion/main.tex}


\end{document}

\newpage
\section{EMS-ED application}

\subsection{Application}

\subsection{Data analysis of generated problem}

\newpage
\documentclass{article}

\usepackage{amsmath}
\usepackage{amsfonts} 
\usepackage{geometry}
\usepackage{multicol}
\usepackage{float}
% \usepackage{mathtools}
% \usepackage{graphicx}
% \usepackage{soul}
% \usepackage{indentfirst}
\usepackage{tikz}
\usetikzlibrary{calc, automata, chains, arrows.meta, math}
\setcounter{MaxMatrixCols}{20}


\title{A game theoretic model of the behavioural gaming that takes place at the EMS - ED interface}

\author{
    Michalis Panayides, 
    Paul Harper, 
    Vince Knight
}

\begin{document}

\maketitle

\input{Abstract/main.tex}


\newpage
\tableofcontents

\newpage
\input{Introduction/main.tex}

\newpage
\input{Game_theory_component/main.tex}

\newpage
\input{MarkovChain/markov_chain_model/main.tex}
\input{MarkovChain/expressions_from_pi/main.tex}
\input{MarkovChain/markov_example/main.tex}

\newpage
\input{BehaviouralMethodology/main.tex}

\newpage
\input{Application_EMS_ED/main.tex}

\newpage
\input{Conclusion/main.tex}


\end{document}


\end{document}


\newpage
\tableofcontents

\newpage
\documentclass{article}

\usepackage{amsmath}
\usepackage{amsfonts} 
\usepackage{geometry}
\usepackage{multicol}
\usepackage{float}
% \usepackage{mathtools}
% \usepackage{graphicx}
% \usepackage{soul}
% \usepackage{indentfirst}
\usepackage{tikz}
\usetikzlibrary{calc, automata, chains, arrows.meta, math}
\setcounter{MaxMatrixCols}{20}


\title{A game theoretic model of the behavioural gaming that takes place at the EMS - ED interface}

\author{
    Michalis Panayides, 
    Paul Harper, 
    Vince Knight
}

\begin{document}

\maketitle

\documentclass{article}

\usepackage{amsmath}
\usepackage{amsfonts} 
\usepackage{geometry}
\usepackage{multicol}
\usepackage{float}
% \usepackage{mathtools}
% \usepackage{graphicx}
% \usepackage{soul}
% \usepackage{indentfirst}
\usepackage{tikz}
\usetikzlibrary{calc, automata, chains, arrows.meta, math}
\setcounter{MaxMatrixCols}{20}


\title{A game theoretic model of the behavioural gaming that takes place at the EMS - ED interface}

\author{
    Michalis Panayides, 
    Paul Harper, 
    Vince Knight
}

\begin{document}

\maketitle

\input{Abstract/main.tex}


\newpage
\tableofcontents

\newpage
\input{Introduction/main.tex}

\newpage
\input{Game_theory_component/main.tex}

\newpage
\input{MarkovChain/markov_chain_model/main.tex}
\input{MarkovChain/expressions_from_pi/main.tex}
\input{MarkovChain/markov_example/main.tex}

\newpage
\input{BehaviouralMethodology/main.tex}

\newpage
\input{Application_EMS_ED/main.tex}

\newpage
\input{Conclusion/main.tex}


\end{document}


\newpage
\tableofcontents

\newpage
\documentclass{article}

\usepackage{amsmath}
\usepackage{amsfonts} 
\usepackage{geometry}
\usepackage{multicol}
\usepackage{float}
% \usepackage{mathtools}
% \usepackage{graphicx}
% \usepackage{soul}
% \usepackage{indentfirst}
\usepackage{tikz}
\usetikzlibrary{calc, automata, chains, arrows.meta, math}
\setcounter{MaxMatrixCols}{20}


\title{A game theoretic model of the behavioural gaming that takes place at the EMS - ED interface}

\author{
    Michalis Panayides, 
    Paul Harper, 
    Vince Knight
}

\begin{document}

\maketitle

\input{Abstract/main.tex}


\newpage
\tableofcontents

\newpage
\input{Introduction/main.tex}

\newpage
\input{Game_theory_component/main.tex}

\newpage
\input{MarkovChain/markov_chain_model/main.tex}
\input{MarkovChain/expressions_from_pi/main.tex}
\input{MarkovChain/markov_example/main.tex}

\newpage
\input{BehaviouralMethodology/main.tex}

\newpage
\input{Application_EMS_ED/main.tex}

\newpage
\input{Conclusion/main.tex}


\end{document}

\newpage
\documentclass{article}

\usepackage{amsmath}
\usepackage{amsfonts} 
\usepackage{geometry}
\usepackage{multicol}
\usepackage{float}
% \usepackage{mathtools}
% \usepackage{graphicx}
% \usepackage{soul}
% \usepackage{indentfirst}
\usepackage{tikz}
\usetikzlibrary{calc, automata, chains, arrows.meta, math}
\setcounter{MaxMatrixCols}{20}


\title{A game theoretic model of the behavioural gaming that takes place at the EMS - ED interface}

\author{
    Michalis Panayides, 
    Paul Harper, 
    Vince Knight
}

\begin{document}

\maketitle

\input{Abstract/main.tex}


\newpage
\tableofcontents

\newpage
\input{Introduction/main.tex}

\newpage
\input{Game_theory_component/main.tex}

\newpage
\input{MarkovChain/markov_chain_model/main.tex}
\input{MarkovChain/expressions_from_pi/main.tex}
\input{MarkovChain/markov_example/main.tex}

\newpage
\input{BehaviouralMethodology/main.tex}

\newpage
\input{Application_EMS_ED/main.tex}

\newpage
\input{Conclusion/main.tex}


\end{document}

\newpage
\documentclass{article}

\usepackage{amsmath}
\usepackage{amsfonts} 
\usepackage{geometry}
\usepackage{multicol}
\usepackage{float}
% \usepackage{mathtools}
% \usepackage{graphicx}
% \usepackage{soul}
% \usepackage{indentfirst}
\usepackage{tikz}
\usetikzlibrary{calc, automata, chains, arrows.meta, math}
\setcounter{MaxMatrixCols}{20}


\title{A game theoretic model of the behavioural gaming that takes place at the EMS - ED interface}

\author{
    Michalis Panayides, 
    Paul Harper, 
    Vince Knight
}

\begin{document}

\maketitle

\input{Abstract/main.tex}


\newpage
\tableofcontents

\newpage
\input{Introduction/main.tex}

\newpage
\input{Game_theory_component/main.tex}

\newpage
\input{MarkovChain/markov_chain_model/main.tex}
\input{MarkovChain/expressions_from_pi/main.tex}
\input{MarkovChain/markov_example/main.tex}

\newpage
\input{BehaviouralMethodology/main.tex}

\newpage
\input{Application_EMS_ED/main.tex}

\newpage
\input{Conclusion/main.tex}


\end{document}
\subsection{Performance Measures}
One may easily derive the average number of individuals that are at any given state 
using \( pi \). 
The average number of individuals in state \( i \) can be calculated by multiplying 
the number of individuals that are present in state \( i \) with the probability 
of being at that particular state (i.e \(\pi_i (u_i + v_i)\)). 
Using this logic it is possible to calculate any performance measures that are related 
to the mean number of individuals in the system.


Average number of people in the system: 
\begin{equation}
    L = \sum_{i=1}^{|\pi|} \pi_i (u_i + v_i)
\end{equation} 

Average number of people in the service centre: 
\begin{equation}
    L_H = \sum_{i=1}^{|\pi|} \pi_i v_i
\end{equation}

Average number of people in the buffer centre:
\begin{equation}
    L_A = \sum_{i=1}^{|\pi|} \pi_i u_i
\end{equation}

Consequently getting the performance measures that are related to the duration of 
time is not as straightforward. 
Such performance measures are the mean waiting time in the system and the mean time 
blocked in the system. 
Under the scope of this study three approaches have been considered to calculate these 
performance measures; a direct approach, a recursive algorithm and consequently a
closed-form formula.

The research question that needs to be answered here is: ``When a class 1/2 
individuals enters the system, what is the expected time that they will have to 
wait?''. 
In order to formulate the answer to that question one needs to consider all possible 
scenarios of what state the system can be in when an individual arrives. 
Furthermore, different formulas arises for class 1 individuals 
and a different one for class 2 individuals.

\subsubsection{Mean waiting time} 
Upon closer inspection of the recursive formula a more compact formula can arise. 
The equivalent closed-form formula eliminates the need for recursion and thus makes 
the computation of waiting times much more efficient. 
Just like in the recursive part there are two formulas; one for \textit{class 1} 
and one for class 2 individuals. 
The formulas are given by:

\begin{equation} \label{eq:closed_form_waiting_others}
    W^{(1)} = \frac{\sum_{\substack{(u,v) \, \in S_A^{(1)} \\ v \geq C}} 
    \frac{1}{C \mu} \times (v-C+1) \times \pi(u,v)}{\sum_{(u,v) \, 
    \in S_A^{(1)}} \pi(u,v)}
\end{equation}
    
\begin{equation}\label{eq:closed_form_waiting_ambulance}
    W^{(2)} = \frac{\sum_{\substack{(u,v) \, \in S_A^{(2)} \\ min(v,T) \geq C}} 
    \frac{1}{C \mu} \times (\min(v+1,T)-C) \times \pi(u,v)}{\sum_{(u,v) \, 
    \in S_A^{(2)}} \pi(u,v)}
\end{equation}

Note here that the summation, in both equations \ref{eq:closed_form_waiting_others} 
and \ref{eq:closed_form_waiting_ambulance}, goes through all states in the set of 
accepting 
states of either class 1 or class 2 individuals respectively, where a wait 
incurs. 
In equation \ref{eq:closed_form_waiting_others} that includes all states \((u,v)\) 
in the set of accepting states of class 1 individuals such that \( v \geq C\); i.e. 
whenever an arrival occurs and the system is at a state where the number of individuals 
in the system is more than or equal to $C$. 
Consequently, for the states that are included in the summation the expression 
\( v-C+1 \) indicates the amount of people in service one would have to wait for 
upon arrival at the hospital.

Additionally, the minimisation function in equation 
\ref{eq:closed_form_waiting_ambulance} 
ensures that when a class 2 individual arrives at any state 
that is greater than the predetermined threshold, the wait that the individual will 
have to endure remains the same. 
In essence, the expression \(\min(v+1,T) - C\) returns the number of people in line 
in front of a particular individual upon arrival.


\subsubsection{Overall Waiting Time}

Consequently, the overall waiting time should can be estimated by a linear combination 
of the waiting times of class 1 and class 2 individuals. 
The overall waiting time can be then given by the following equation where \(c_1\) 
and \(c_2\) are the coefficients of each individual's type waiting time:

\begin{equation}\label{overall_waiting_time_coeff}
    W = c_1 W^{(1)} + c_2 W^{(2)}
\end{equation}

The two coefficients represent the proportion of individuals of each type that 
traversed through the model. 
Theoretically, getting these percentages should be as simple as looking at the arrival 
rates of each type but in practise if the service centre or the buffer centre 
is full, some individuals may be lost to the system. 
Thus, one should account for the probability that an individual is lost to the system. 
This probability can be easily calculated by using the two sets of accepting states 
\(S_A^{(2)}\) and \(S_A^{(1)}\) defined earlier in equations.
Let us define here the probability, for either class type, that an individual 
is not lost in the system by:

\begin{equation*}
    P(L'_1) = \sum_{(u,v) \, \in S_A^{(1)}} \pi(u,v) \hspace{2cm}
    P(L'_2) = \sum_{(u,v) \, \in S_A^{(2)}} \pi(u,v)
\end{equation*}

Having defined these probabilities one may combine them with the arrival rates of 
each class type in such a way to get the expected proportions of class 1 and 
class 2 individuals in the model. 
Thus, by using these values as the coefficient of equation 
\ref{overall_waiting_time_coeff} 
the resultant equation can be used to get the overall waiting time. 
Note here that the equation below gets the overall waiting time for both the recursive 
and the closed-form formula.

\begin{equation}\label{overall_waiting_time}
    W = \frac{\lambda_1 P(L'_1)}{\lambda_2 P(L'_2) + \lambda_1 P(L'_1)} W^{(1)} + 
    \frac{\lambda_2 P(L'_2)}{\lambda_2 P(L'_2) + \lambda_1 P(L'_1)} W^{(2)}
\end{equation}



\subsubsection{Mean blocking time}
Unlike the waiting time, the blocking time is only calculated for class 2 individuals.  
That is because class 1 individuals cannot be blocked. 
Thus, one only needs to consider the pathway of class 2 individuals to get the 
mean blocking time of the system. 
Blocking occurs at states \((u,v)\) where \(u > 0 \). 
Thus, the set of blocking states can be defined as:

\begin{equation*}
    S_b = \{(u,v) \in S \; | \; u > 0\}
\end{equation*}
 
In order to not consider individuals that will be lost to the system, the set of 
accepting states needs to be taken into account. The set of accepting states is given by:

\begin{equation*}
    S_A^{(2)}=
    \begin{cases}
        \{(u, v) \in S \; | \; u < M \} & \textbf{if } T \leq N\\
        \{(u, v) \in S \; | \; v < N \} & \textbf{otherwise}
    \end{cases}
\end{equation*}

For the waiting time formula,
the mean sojourn time for each state was considered,
ignoring any arrivals. Here, the same approach is used but ignoring only class 2
arrivals. That is because for the waiting time formula, once an individual enters 
the service centre (i.e. starts waiting) any individual arriving after them will 
not affect their
pathway. That is not the case for blocking time. When a class 2 individual is 
blocked, 
any class 1 individual that arrives will cause the blocked individual to remain 
blocked for more time. Therefore, class 1 arrivals are considered here:

\begin{equation}\label{eq:time_in_state_blocking_time}
    c(u,v) = 
    \begin{cases}
        \frac{1}{\min(v,C) \mu}, & \text{if } v = C\\
        \frac{1}{\min(v,C) \mu + \lambda_1}, & \text{otherwise}
    \end{cases}
\end{equation}
 
In equation \ref{eq:time_in_state_blocking_time}, both service completions and 
class 1 arrivals are considered. 
Thus, from a blocked individual's perspective whenever the system moves from one 
state \((u,v)\)
to another state it can either:

\begin{itemize}
    \item be because of a service being completed: we will denote the probability 
    of this happening by \(p_s(u,v)\). 
    \item be because of an arrival of an individual of class 1: denoting such 
    probability by \(p_o(u,v)\).
\end{itemize}
The probabilities are given by:

\begin{equation*}
    p_s(u,v) = \frac{\min(v,C)\mu}{\lambda_1 + \min(v,C)\mu}, \qquad
    p_o(u,v) = \frac{\lambda_1}{\lambda_1 + \min(v,C)\mu}
\end{equation*}


Having defined \(c(u,v)\) and \(S_b\) a formula for the blocking time that is
expected to occur at each state can be given by:

\begin{equation}\label{eq:blocking-time-at-each-state}
    b(u,v) = 
    \begin{cases} 
        0, & \textbf{if } (u,v) \notin S_b \\
        c(u,v) + b(u - 1, v), & \textbf{if } v = N = T\\
        c(u,v) + b(u, v-1), & \textbf{if } v = N \neq T \\
        c(u,v) + p_s(u,v) b(u-1, v) + p_o(u,v) b(u, v+1), & \textbf{if } u > 0 
        \textbf{ and } v = T \\
        c(u,v) + p_s(u,v) b(u, v-1) + p_o(u,v) b(u, v+1), & \textbf{otherwise} \\
    \end{cases}
\end{equation}

Equation 
(\ref{eq:blocking-time-at-each-state}) will not be solved recursively. 
A direct approach will be used to solve this equation here. 
By enumerating all equations of (\ref{eq:blocking-time-at-each-state}) for all 
states \((u,v)\) that belong in \(S_b\) 
a system of linear equations arises where the unknown variables are all the \(b(u,v)\)
terms.
For instance, let us consider a Markov model where \(C=2, T=3, N=6, M=2\). 
The Markov model is shown in Figure \ref{fig:example-algeb-blocking}
and the equivalent equations are 
(\ref{eq:first_eq_of_blocking_example})-(\ref{eq:last_eq_of_blocking_example}).
The equations considered here are only the ones that correspond to the blocking 
states.

\begin{multicols*}{2}
    \begin{figure}[H]
        \scalebox{0.50}{\input{MarkovChain/expressions_from_pi/example_model_2362/main.tex}}
        \caption{Example of Markov chain}
        \label{fig:example-algeb-blocking}
    \end{figure}
    \columnbreak
    \begin{align}
        b(1,2) &= c(1,2) + p_o b(1,3) \label{eq:first_eq_of_blocking_example} \\
        b(1,3) &= c(1,3) + p_s b(1,2) + p_o b(1,4) \\
        b(1,4) &= c(1,4) + b(1,3) \\
        b(2,2) &= c(2,2) + p_s b(1,2) + p_o b(2,3) \\
        b(2,3) &= c(2,3) + p_s b(2,2) + p_o b(1,4) \\
        b(2,4) &= c(2,4) + b(2,3)\label{eq:last_eq_of_blocking_example}
    \end{align}
\end{multicols*}

Additionally, the above equations can be transformed into a linear system of the 
form \(Zx=y\) where:

\begin{equation}\label{eq:example-algebaric-approach-blocking-time}
    Z=
    \begin{pmatrix}
        -1 & p_o & 0 & 0 & 0 & 0 \\ %(1,2)
        p_s & -1 & p_o & 0 & 0 & 0 \\ %(1,3)
        0 & 1 & -1 & 0 & 0 & 0 \\ %(1,4)
        p_s & 0 & 0 & -1 & p_o & 0\\ %(2,2)
        0 & 0 & 0 & p_s & -1 & p_o \\ %(2,3)
        0 & 0 & 0 & 0 & 1 & -1 \\ %(2,4)
    \end{pmatrix},
    x=
    \begin{pmatrix}
        b(1,2) \\
        b(1,3) \\
        b(1,4) \\
        b(2,2) \\
        b(2,3) \\
        b(2,4) \\
    \end{pmatrix}, 
    y=
    \begin{pmatrix}
        -c(1,2) \\
        -c(1,3) \\
        -c(1,4) \\
        -c(2,2) \\
        -c(2,3) \\
        -c(2,4) \\
    \end{pmatrix}
\end{equation}

A more generalised form of the equations in 
(\ref{eq:example-algebaric-approach-blocking-time})
can thus be given for any value of \(C,T,N,M\) by:

\begin{align}
    b(1,T) =& c(1, T) + p_o b(1, T + 1) \label{eq:first_eq_of_blocking_general}\\
    b(1,T + 1) =& c(1, T + 1) + p_s(1, T) + p_o b(1, T + 1) \\
    b(1,T + 2) =& c(1, T + 2) + p_s(1, T + 1) + p_o b(1, T + 3) \\
    & \vdots \nonumber \\
    b(1, N) =& c(1, N) + b(1, N - 1) \\
    b(2, T) =& c(2, T) + p_s b(1, T) + p_o b(2, T + 1) \\
    b(2, T + 1) =& c(2, T + 1) + p_s b(2, T) + p_o b(2, T + 2) \\
    & \vdots \nonumber \\
    b(M, T) =& c(M, T) + b(M, T-1) \label{eq:last_eq_of_blocking_general}
\end{align}

The equivalent matrix form of the linear system of equations 
(\ref{eq:first_eq_of_blocking_general}) - (\ref{eq:last_eq_of_blocking_general})
is given by \(Zx=y\), where:
\begin{equation}\label{eq:general-algebaric-approach-blocking-time}
    \scalebox{0.9}{
        \(
        Z = 
        \begin{pmatrix}
            -1 & p_o & 0 & \dots & 0 & 0 & 0 & 0 & 0 & \dots & 0 & 0 \\ %(1,T)
            p_s & -1 & p_o & \dots & 0 & 0 & 0 & 0 & 0 & \dots & 0 & 0 \\ %(1,T+1)
            0 & p_s & -1 & \dots & 0 & 0 & 0 & 0 & 0 & \dots & 0 & 0 \\ %(1,T+2)
            \vdots & \vdots & \vdots & \ddots & \vdots & \vdots & \vdots & \vdots & 
            \vdots & \ddots & \vdots & \vdots \\ 
            0 & 0 & 0 & \dots & 1 & -1 & 0 & 0 & 0 & \dots & 0 & 0 \\ %(1,N)
            p_s & 0 & 0 & \dots & 0 & 0 & -1 & p_o & 0 & \dots & 0 & 0 \\ %(2,T)
            0 & 0 & 0 & \dots & 0 & 0 & p_s & -1 & p_o & \dots & 0 & 0 \\ %(2,T+1)
            \vdots & \vdots & \vdots & \ddots & \vdots & \vdots & \vdots & \vdots & 
            \vdots & \ddots & \vdots & \vdots \\ 
            0 & 0 & 0 & \dots & 0 & 0 & 0 & 0 & 0 & \dots & 1 & -1 \\ %(M,T)
        \end{pmatrix},
        x = 
        \begin{pmatrix}
            b(1,T) \\
            b(1,T+1) \\
            b(1,T+2) \\
            \vdots \\
            b(1,N) \\
            b(2,T) \\
            b(2,T+1) \\
            \vdots \\
            b(M,T) \\
        \end{pmatrix}, 
        y= 
        \begin{pmatrix}
            -c(1,T) \\
            -c(1,T+1) \\
            -c(1,T+2) \\
            \vdots \\
            -c(1,N) \\
            -c(2,T) \\
            -c(2,T+1) \\
            \vdots \\
            -c(M,T) \\
        \end{pmatrix}
        \)
    }
\end{equation}

Thus, having calculated the mean blocking time for all blocking states \(b(u,v)\), 
it only remains to put them together in a formula.
The resultant blocking time formula is given by:

\begin{equation}\label{eq:algebraic-blocking-time}
    B = \frac{\sum_{(u,v) \in S_A} \pi_{(u,v)} \; b(u,v)}{\sum_{(u,v) \in S_A} 
    \pi_{(u,v)}}
\end{equation}

\documentclass{article}

\usepackage{amsmath}
\usepackage{amsfonts} 
\usepackage{geometry}
\usepackage{multicol}
\usepackage{float}
% \usepackage{mathtools}
% \usepackage{graphicx}
% \usepackage{soul}
% \usepackage{indentfirst}
\usepackage{tikz}
\usetikzlibrary{calc, automata, chains, arrows.meta, math}
\setcounter{MaxMatrixCols}{20}


\title{A game theoretic model of the behavioural gaming that takes place at the EMS - ED interface}

\author{
    Michalis Panayides, 
    Paul Harper, 
    Vince Knight
}

\begin{document}

\maketitle

\input{Abstract/main.tex}


\newpage
\tableofcontents

\newpage
\input{Introduction/main.tex}

\newpage
\input{Game_theory_component/main.tex}

\newpage
\input{MarkovChain/markov_chain_model/main.tex}
\input{MarkovChain/expressions_from_pi/main.tex}
\input{MarkovChain/markov_example/main.tex}

\newpage
\input{BehaviouralMethodology/main.tex}

\newpage
\input{Application_EMS_ED/main.tex}

\newpage
\input{Conclusion/main.tex}


\end{document}

\newpage
\documentclass{article}

\usepackage{amsmath}
\usepackage{amsfonts} 
\usepackage{geometry}
\usepackage{multicol}
\usepackage{float}
% \usepackage{mathtools}
% \usepackage{graphicx}
% \usepackage{soul}
% \usepackage{indentfirst}
\usepackage{tikz}
\usetikzlibrary{calc, automata, chains, arrows.meta, math}
\setcounter{MaxMatrixCols}{20}


\title{A game theoretic model of the behavioural gaming that takes place at the EMS - ED interface}

\author{
    Michalis Panayides, 
    Paul Harper, 
    Vince Knight
}

\begin{document}

\maketitle

\input{Abstract/main.tex}


\newpage
\tableofcontents

\newpage
\input{Introduction/main.tex}

\newpage
\input{Game_theory_component/main.tex}

\newpage
\input{MarkovChain/markov_chain_model/main.tex}
\input{MarkovChain/expressions_from_pi/main.tex}
\input{MarkovChain/markov_example/main.tex}

\newpage
\input{BehaviouralMethodology/main.tex}

\newpage
\input{Application_EMS_ED/main.tex}

\newpage
\input{Conclusion/main.tex}


\end{document}

\newpage
\section{EMS-ED application}

\subsection{Application}

\subsection{Data analysis of generated problem}

\newpage
\documentclass{article}

\usepackage{amsmath}
\usepackage{amsfonts} 
\usepackage{geometry}
\usepackage{multicol}
\usepackage{float}
% \usepackage{mathtools}
% \usepackage{graphicx}
% \usepackage{soul}
% \usepackage{indentfirst}
\usepackage{tikz}
\usetikzlibrary{calc, automata, chains, arrows.meta, math}
\setcounter{MaxMatrixCols}{20}


\title{A game theoretic model of the behavioural gaming that takes place at the EMS - ED interface}

\author{
    Michalis Panayides, 
    Paul Harper, 
    Vince Knight
}

\begin{document}

\maketitle

\input{Abstract/main.tex}


\newpage
\tableofcontents

\newpage
\input{Introduction/main.tex}

\newpage
\input{Game_theory_component/main.tex}

\newpage
\input{MarkovChain/markov_chain_model/main.tex}
\input{MarkovChain/expressions_from_pi/main.tex}
\input{MarkovChain/markov_example/main.tex}

\newpage
\input{BehaviouralMethodology/main.tex}

\newpage
\input{Application_EMS_ED/main.tex}

\newpage
\input{Conclusion/main.tex}


\end{document}


\end{document}

\newpage
\documentclass{article}

\usepackage{amsmath}
\usepackage{amsfonts} 
\usepackage{geometry}
\usepackage{multicol}
\usepackage{float}
% \usepackage{mathtools}
% \usepackage{graphicx}
% \usepackage{soul}
% \usepackage{indentfirst}
\usepackage{tikz}
\usetikzlibrary{calc, automata, chains, arrows.meta, math}
\setcounter{MaxMatrixCols}{20}


\title{A game theoretic model of the behavioural gaming that takes place at the EMS - ED interface}

\author{
    Michalis Panayides, 
    Paul Harper, 
    Vince Knight
}

\begin{document}

\maketitle

\documentclass{article}

\usepackage{amsmath}
\usepackage{amsfonts} 
\usepackage{geometry}
\usepackage{multicol}
\usepackage{float}
% \usepackage{mathtools}
% \usepackage{graphicx}
% \usepackage{soul}
% \usepackage{indentfirst}
\usepackage{tikz}
\usetikzlibrary{calc, automata, chains, arrows.meta, math}
\setcounter{MaxMatrixCols}{20}


\title{A game theoretic model of the behavioural gaming that takes place at the EMS - ED interface}

\author{
    Michalis Panayides, 
    Paul Harper, 
    Vince Knight
}

\begin{document}

\maketitle

\input{Abstract/main.tex}


\newpage
\tableofcontents

\newpage
\input{Introduction/main.tex}

\newpage
\input{Game_theory_component/main.tex}

\newpage
\input{MarkovChain/markov_chain_model/main.tex}
\input{MarkovChain/expressions_from_pi/main.tex}
\input{MarkovChain/markov_example/main.tex}

\newpage
\input{BehaviouralMethodology/main.tex}

\newpage
\input{Application_EMS_ED/main.tex}

\newpage
\input{Conclusion/main.tex}


\end{document}


\newpage
\tableofcontents

\newpage
\documentclass{article}

\usepackage{amsmath}
\usepackage{amsfonts} 
\usepackage{geometry}
\usepackage{multicol}
\usepackage{float}
% \usepackage{mathtools}
% \usepackage{graphicx}
% \usepackage{soul}
% \usepackage{indentfirst}
\usepackage{tikz}
\usetikzlibrary{calc, automata, chains, arrows.meta, math}
\setcounter{MaxMatrixCols}{20}


\title{A game theoretic model of the behavioural gaming that takes place at the EMS - ED interface}

\author{
    Michalis Panayides, 
    Paul Harper, 
    Vince Knight
}

\begin{document}

\maketitle

\input{Abstract/main.tex}


\newpage
\tableofcontents

\newpage
\input{Introduction/main.tex}

\newpage
\input{Game_theory_component/main.tex}

\newpage
\input{MarkovChain/markov_chain_model/main.tex}
\input{MarkovChain/expressions_from_pi/main.tex}
\input{MarkovChain/markov_example/main.tex}

\newpage
\input{BehaviouralMethodology/main.tex}

\newpage
\input{Application_EMS_ED/main.tex}

\newpage
\input{Conclusion/main.tex}


\end{document}

\newpage
\documentclass{article}

\usepackage{amsmath}
\usepackage{amsfonts} 
\usepackage{geometry}
\usepackage{multicol}
\usepackage{float}
% \usepackage{mathtools}
% \usepackage{graphicx}
% \usepackage{soul}
% \usepackage{indentfirst}
\usepackage{tikz}
\usetikzlibrary{calc, automata, chains, arrows.meta, math}
\setcounter{MaxMatrixCols}{20}


\title{A game theoretic model of the behavioural gaming that takes place at the EMS - ED interface}

\author{
    Michalis Panayides, 
    Paul Harper, 
    Vince Knight
}

\begin{document}

\maketitle

\input{Abstract/main.tex}


\newpage
\tableofcontents

\newpage
\input{Introduction/main.tex}

\newpage
\input{Game_theory_component/main.tex}

\newpage
\input{MarkovChain/markov_chain_model/main.tex}
\input{MarkovChain/expressions_from_pi/main.tex}
\input{MarkovChain/markov_example/main.tex}

\newpage
\input{BehaviouralMethodology/main.tex}

\newpage
\input{Application_EMS_ED/main.tex}

\newpage
\input{Conclusion/main.tex}


\end{document}

\newpage
\documentclass{article}

\usepackage{amsmath}
\usepackage{amsfonts} 
\usepackage{geometry}
\usepackage{multicol}
\usepackage{float}
% \usepackage{mathtools}
% \usepackage{graphicx}
% \usepackage{soul}
% \usepackage{indentfirst}
\usepackage{tikz}
\usetikzlibrary{calc, automata, chains, arrows.meta, math}
\setcounter{MaxMatrixCols}{20}


\title{A game theoretic model of the behavioural gaming that takes place at the EMS - ED interface}

\author{
    Michalis Panayides, 
    Paul Harper, 
    Vince Knight
}

\begin{document}

\maketitle

\input{Abstract/main.tex}


\newpage
\tableofcontents

\newpage
\input{Introduction/main.tex}

\newpage
\input{Game_theory_component/main.tex}

\newpage
\input{MarkovChain/markov_chain_model/main.tex}
\input{MarkovChain/expressions_from_pi/main.tex}
\input{MarkovChain/markov_example/main.tex}

\newpage
\input{BehaviouralMethodology/main.tex}

\newpage
\input{Application_EMS_ED/main.tex}

\newpage
\input{Conclusion/main.tex}


\end{document}
\subsection{Performance Measures}
One may easily derive the average number of individuals that are at any given state 
using \( pi \). 
The average number of individuals in state \( i \) can be calculated by multiplying 
the number of individuals that are present in state \( i \) with the probability 
of being at that particular state (i.e \(\pi_i (u_i + v_i)\)). 
Using this logic it is possible to calculate any performance measures that are related 
to the mean number of individuals in the system.


Average number of people in the system: 
\begin{equation}
    L = \sum_{i=1}^{|\pi|} \pi_i (u_i + v_i)
\end{equation} 

Average number of people in the service centre: 
\begin{equation}
    L_H = \sum_{i=1}^{|\pi|} \pi_i v_i
\end{equation}

Average number of people in the buffer centre:
\begin{equation}
    L_A = \sum_{i=1}^{|\pi|} \pi_i u_i
\end{equation}

Consequently getting the performance measures that are related to the duration of 
time is not as straightforward. 
Such performance measures are the mean waiting time in the system and the mean time 
blocked in the system. 
Under the scope of this study three approaches have been considered to calculate these 
performance measures; a direct approach, a recursive algorithm and consequently a
closed-form formula.

The research question that needs to be answered here is: ``When a class 1/2 
individuals enters the system, what is the expected time that they will have to 
wait?''. 
In order to formulate the answer to that question one needs to consider all possible 
scenarios of what state the system can be in when an individual arrives. 
Furthermore, different formulas arises for class 1 individuals 
and a different one for class 2 individuals.

\subsubsection{Mean waiting time} 
Upon closer inspection of the recursive formula a more compact formula can arise. 
The equivalent closed-form formula eliminates the need for recursion and thus makes 
the computation of waiting times much more efficient. 
Just like in the recursive part there are two formulas; one for \textit{class 1} 
and one for class 2 individuals. 
The formulas are given by:

\begin{equation} \label{eq:closed_form_waiting_others}
    W^{(1)} = \frac{\sum_{\substack{(u,v) \, \in S_A^{(1)} \\ v \geq C}} 
    \frac{1}{C \mu} \times (v-C+1) \times \pi(u,v)}{\sum_{(u,v) \, 
    \in S_A^{(1)}} \pi(u,v)}
\end{equation}
    
\begin{equation}\label{eq:closed_form_waiting_ambulance}
    W^{(2)} = \frac{\sum_{\substack{(u,v) \, \in S_A^{(2)} \\ min(v,T) \geq C}} 
    \frac{1}{C \mu} \times (\min(v+1,T)-C) \times \pi(u,v)}{\sum_{(u,v) \, 
    \in S_A^{(2)}} \pi(u,v)}
\end{equation}

Note here that the summation, in both equations \ref{eq:closed_form_waiting_others} 
and \ref{eq:closed_form_waiting_ambulance}, goes through all states in the set of 
accepting 
states of either class 1 or class 2 individuals respectively, where a wait 
incurs. 
In equation \ref{eq:closed_form_waiting_others} that includes all states \((u,v)\) 
in the set of accepting states of class 1 individuals such that \( v \geq C\); i.e. 
whenever an arrival occurs and the system is at a state where the number of individuals 
in the system is more than or equal to $C$. 
Consequently, for the states that are included in the summation the expression 
\( v-C+1 \) indicates the amount of people in service one would have to wait for 
upon arrival at the hospital.

Additionally, the minimisation function in equation 
\ref{eq:closed_form_waiting_ambulance} 
ensures that when a class 2 individual arrives at any state 
that is greater than the predetermined threshold, the wait that the individual will 
have to endure remains the same. 
In essence, the expression \(\min(v+1,T) - C\) returns the number of people in line 
in front of a particular individual upon arrival.


\subsubsection{Overall Waiting Time}

Consequently, the overall waiting time should can be estimated by a linear combination 
of the waiting times of class 1 and class 2 individuals. 
The overall waiting time can be then given by the following equation where \(c_1\) 
and \(c_2\) are the coefficients of each individual's type waiting time:

\begin{equation}\label{overall_waiting_time_coeff}
    W = c_1 W^{(1)} + c_2 W^{(2)}
\end{equation}

The two coefficients represent the proportion of individuals of each type that 
traversed through the model. 
Theoretically, getting these percentages should be as simple as looking at the arrival 
rates of each type but in practise if the service centre or the buffer centre 
is full, some individuals may be lost to the system. 
Thus, one should account for the probability that an individual is lost to the system. 
This probability can be easily calculated by using the two sets of accepting states 
\(S_A^{(2)}\) and \(S_A^{(1)}\) defined earlier in equations.
Let us define here the probability, for either class type, that an individual 
is not lost in the system by:

\begin{equation*}
    P(L'_1) = \sum_{(u,v) \, \in S_A^{(1)}} \pi(u,v) \hspace{2cm}
    P(L'_2) = \sum_{(u,v) \, \in S_A^{(2)}} \pi(u,v)
\end{equation*}

Having defined these probabilities one may combine them with the arrival rates of 
each class type in such a way to get the expected proportions of class 1 and 
class 2 individuals in the model. 
Thus, by using these values as the coefficient of equation 
\ref{overall_waiting_time_coeff} 
the resultant equation can be used to get the overall waiting time. 
Note here that the equation below gets the overall waiting time for both the recursive 
and the closed-form formula.

\begin{equation}\label{overall_waiting_time}
    W = \frac{\lambda_1 P(L'_1)}{\lambda_2 P(L'_2) + \lambda_1 P(L'_1)} W^{(1)} + 
    \frac{\lambda_2 P(L'_2)}{\lambda_2 P(L'_2) + \lambda_1 P(L'_1)} W^{(2)}
\end{equation}



\subsubsection{Mean blocking time}
Unlike the waiting time, the blocking time is only calculated for class 2 individuals.  
That is because class 1 individuals cannot be blocked. 
Thus, one only needs to consider the pathway of class 2 individuals to get the 
mean blocking time of the system. 
Blocking occurs at states \((u,v)\) where \(u > 0 \). 
Thus, the set of blocking states can be defined as:

\begin{equation*}
    S_b = \{(u,v) \in S \; | \; u > 0\}
\end{equation*}
 
In order to not consider individuals that will be lost to the system, the set of 
accepting states needs to be taken into account. The set of accepting states is given by:

\begin{equation*}
    S_A^{(2)}=
    \begin{cases}
        \{(u, v) \in S \; | \; u < M \} & \textbf{if } T \leq N\\
        \{(u, v) \in S \; | \; v < N \} & \textbf{otherwise}
    \end{cases}
\end{equation*}

For the waiting time formula,
the mean sojourn time for each state was considered,
ignoring any arrivals. Here, the same approach is used but ignoring only class 2
arrivals. That is because for the waiting time formula, once an individual enters 
the service centre (i.e. starts waiting) any individual arriving after them will 
not affect their
pathway. That is not the case for blocking time. When a class 2 individual is 
blocked, 
any class 1 individual that arrives will cause the blocked individual to remain 
blocked for more time. Therefore, class 1 arrivals are considered here:

\begin{equation}\label{eq:time_in_state_blocking_time}
    c(u,v) = 
    \begin{cases}
        \frac{1}{\min(v,C) \mu}, & \text{if } v = C\\
        \frac{1}{\min(v,C) \mu + \lambda_1}, & \text{otherwise}
    \end{cases}
\end{equation}
 
In equation \ref{eq:time_in_state_blocking_time}, both service completions and 
class 1 arrivals are considered. 
Thus, from a blocked individual's perspective whenever the system moves from one 
state \((u,v)\)
to another state it can either:

\begin{itemize}
    \item be because of a service being completed: we will denote the probability 
    of this happening by \(p_s(u,v)\). 
    \item be because of an arrival of an individual of class 1: denoting such 
    probability by \(p_o(u,v)\).
\end{itemize}
The probabilities are given by:

\begin{equation*}
    p_s(u,v) = \frac{\min(v,C)\mu}{\lambda_1 + \min(v,C)\mu}, \qquad
    p_o(u,v) = \frac{\lambda_1}{\lambda_1 + \min(v,C)\mu}
\end{equation*}


Having defined \(c(u,v)\) and \(S_b\) a formula for the blocking time that is
expected to occur at each state can be given by:

\begin{equation}\label{eq:blocking-time-at-each-state}
    b(u,v) = 
    \begin{cases} 
        0, & \textbf{if } (u,v) \notin S_b \\
        c(u,v) + b(u - 1, v), & \textbf{if } v = N = T\\
        c(u,v) + b(u, v-1), & \textbf{if } v = N \neq T \\
        c(u,v) + p_s(u,v) b(u-1, v) + p_o(u,v) b(u, v+1), & \textbf{if } u > 0 
        \textbf{ and } v = T \\
        c(u,v) + p_s(u,v) b(u, v-1) + p_o(u,v) b(u, v+1), & \textbf{otherwise} \\
    \end{cases}
\end{equation}

Equation 
(\ref{eq:blocking-time-at-each-state}) will not be solved recursively. 
A direct approach will be used to solve this equation here. 
By enumerating all equations of (\ref{eq:blocking-time-at-each-state}) for all 
states \((u,v)\) that belong in \(S_b\) 
a system of linear equations arises where the unknown variables are all the \(b(u,v)\)
terms.
For instance, let us consider a Markov model where \(C=2, T=3, N=6, M=2\). 
The Markov model is shown in Figure \ref{fig:example-algeb-blocking}
and the equivalent equations are 
(\ref{eq:first_eq_of_blocking_example})-(\ref{eq:last_eq_of_blocking_example}).
The equations considered here are only the ones that correspond to the blocking 
states.

\begin{multicols*}{2}
    \begin{figure}[H]
        \scalebox{0.50}{\input{MarkovChain/expressions_from_pi/example_model_2362/main.tex}}
        \caption{Example of Markov chain}
        \label{fig:example-algeb-blocking}
    \end{figure}
    \columnbreak
    \begin{align}
        b(1,2) &= c(1,2) + p_o b(1,3) \label{eq:first_eq_of_blocking_example} \\
        b(1,3) &= c(1,3) + p_s b(1,2) + p_o b(1,4) \\
        b(1,4) &= c(1,4) + b(1,3) \\
        b(2,2) &= c(2,2) + p_s b(1,2) + p_o b(2,3) \\
        b(2,3) &= c(2,3) + p_s b(2,2) + p_o b(1,4) \\
        b(2,4) &= c(2,4) + b(2,3)\label{eq:last_eq_of_blocking_example}
    \end{align}
\end{multicols*}

Additionally, the above equations can be transformed into a linear system of the 
form \(Zx=y\) where:

\begin{equation}\label{eq:example-algebaric-approach-blocking-time}
    Z=
    \begin{pmatrix}
        -1 & p_o & 0 & 0 & 0 & 0 \\ %(1,2)
        p_s & -1 & p_o & 0 & 0 & 0 \\ %(1,3)
        0 & 1 & -1 & 0 & 0 & 0 \\ %(1,4)
        p_s & 0 & 0 & -1 & p_o & 0\\ %(2,2)
        0 & 0 & 0 & p_s & -1 & p_o \\ %(2,3)
        0 & 0 & 0 & 0 & 1 & -1 \\ %(2,4)
    \end{pmatrix},
    x=
    \begin{pmatrix}
        b(1,2) \\
        b(1,3) \\
        b(1,4) \\
        b(2,2) \\
        b(2,3) \\
        b(2,4) \\
    \end{pmatrix}, 
    y=
    \begin{pmatrix}
        -c(1,2) \\
        -c(1,3) \\
        -c(1,4) \\
        -c(2,2) \\
        -c(2,3) \\
        -c(2,4) \\
    \end{pmatrix}
\end{equation}

A more generalised form of the equations in 
(\ref{eq:example-algebaric-approach-blocking-time})
can thus be given for any value of \(C,T,N,M\) by:

\begin{align}
    b(1,T) =& c(1, T) + p_o b(1, T + 1) \label{eq:first_eq_of_blocking_general}\\
    b(1,T + 1) =& c(1, T + 1) + p_s(1, T) + p_o b(1, T + 1) \\
    b(1,T + 2) =& c(1, T + 2) + p_s(1, T + 1) + p_o b(1, T + 3) \\
    & \vdots \nonumber \\
    b(1, N) =& c(1, N) + b(1, N - 1) \\
    b(2, T) =& c(2, T) + p_s b(1, T) + p_o b(2, T + 1) \\
    b(2, T + 1) =& c(2, T + 1) + p_s b(2, T) + p_o b(2, T + 2) \\
    & \vdots \nonumber \\
    b(M, T) =& c(M, T) + b(M, T-1) \label{eq:last_eq_of_blocking_general}
\end{align}

The equivalent matrix form of the linear system of equations 
(\ref{eq:first_eq_of_blocking_general}) - (\ref{eq:last_eq_of_blocking_general})
is given by \(Zx=y\), where:
\begin{equation}\label{eq:general-algebaric-approach-blocking-time}
    \scalebox{0.9}{
        \(
        Z = 
        \begin{pmatrix}
            -1 & p_o & 0 & \dots & 0 & 0 & 0 & 0 & 0 & \dots & 0 & 0 \\ %(1,T)
            p_s & -1 & p_o & \dots & 0 & 0 & 0 & 0 & 0 & \dots & 0 & 0 \\ %(1,T+1)
            0 & p_s & -1 & \dots & 0 & 0 & 0 & 0 & 0 & \dots & 0 & 0 \\ %(1,T+2)
            \vdots & \vdots & \vdots & \ddots & \vdots & \vdots & \vdots & \vdots & 
            \vdots & \ddots & \vdots & \vdots \\ 
            0 & 0 & 0 & \dots & 1 & -1 & 0 & 0 & 0 & \dots & 0 & 0 \\ %(1,N)
            p_s & 0 & 0 & \dots & 0 & 0 & -1 & p_o & 0 & \dots & 0 & 0 \\ %(2,T)
            0 & 0 & 0 & \dots & 0 & 0 & p_s & -1 & p_o & \dots & 0 & 0 \\ %(2,T+1)
            \vdots & \vdots & \vdots & \ddots & \vdots & \vdots & \vdots & \vdots & 
            \vdots & \ddots & \vdots & \vdots \\ 
            0 & 0 & 0 & \dots & 0 & 0 & 0 & 0 & 0 & \dots & 1 & -1 \\ %(M,T)
        \end{pmatrix},
        x = 
        \begin{pmatrix}
            b(1,T) \\
            b(1,T+1) \\
            b(1,T+2) \\
            \vdots \\
            b(1,N) \\
            b(2,T) \\
            b(2,T+1) \\
            \vdots \\
            b(M,T) \\
        \end{pmatrix}, 
        y= 
        \begin{pmatrix}
            -c(1,T) \\
            -c(1,T+1) \\
            -c(1,T+2) \\
            \vdots \\
            -c(1,N) \\
            -c(2,T) \\
            -c(2,T+1) \\
            \vdots \\
            -c(M,T) \\
        \end{pmatrix}
        \)
    }
\end{equation}

Thus, having calculated the mean blocking time for all blocking states \(b(u,v)\), 
it only remains to put them together in a formula.
The resultant blocking time formula is given by:

\begin{equation}\label{eq:algebraic-blocking-time}
    B = \frac{\sum_{(u,v) \in S_A} \pi_{(u,v)} \; b(u,v)}{\sum_{(u,v) \in S_A} 
    \pi_{(u,v)}}
\end{equation}

\documentclass{article}

\usepackage{amsmath}
\usepackage{amsfonts} 
\usepackage{geometry}
\usepackage{multicol}
\usepackage{float}
% \usepackage{mathtools}
% \usepackage{graphicx}
% \usepackage{soul}
% \usepackage{indentfirst}
\usepackage{tikz}
\usetikzlibrary{calc, automata, chains, arrows.meta, math}
\setcounter{MaxMatrixCols}{20}


\title{A game theoretic model of the behavioural gaming that takes place at the EMS - ED interface}

\author{
    Michalis Panayides, 
    Paul Harper, 
    Vince Knight
}

\begin{document}

\maketitle

\input{Abstract/main.tex}


\newpage
\tableofcontents

\newpage
\input{Introduction/main.tex}

\newpage
\input{Game_theory_component/main.tex}

\newpage
\input{MarkovChain/markov_chain_model/main.tex}
\input{MarkovChain/expressions_from_pi/main.tex}
\input{MarkovChain/markov_example/main.tex}

\newpage
\input{BehaviouralMethodology/main.tex}

\newpage
\input{Application_EMS_ED/main.tex}

\newpage
\input{Conclusion/main.tex}


\end{document}

\newpage
\documentclass{article}

\usepackage{amsmath}
\usepackage{amsfonts} 
\usepackage{geometry}
\usepackage{multicol}
\usepackage{float}
% \usepackage{mathtools}
% \usepackage{graphicx}
% \usepackage{soul}
% \usepackage{indentfirst}
\usepackage{tikz}
\usetikzlibrary{calc, automata, chains, arrows.meta, math}
\setcounter{MaxMatrixCols}{20}


\title{A game theoretic model of the behavioural gaming that takes place at the EMS - ED interface}

\author{
    Michalis Panayides, 
    Paul Harper, 
    Vince Knight
}

\begin{document}

\maketitle

\input{Abstract/main.tex}


\newpage
\tableofcontents

\newpage
\input{Introduction/main.tex}

\newpage
\input{Game_theory_component/main.tex}

\newpage
\input{MarkovChain/markov_chain_model/main.tex}
\input{MarkovChain/expressions_from_pi/main.tex}
\input{MarkovChain/markov_example/main.tex}

\newpage
\input{BehaviouralMethodology/main.tex}

\newpage
\input{Application_EMS_ED/main.tex}

\newpage
\input{Conclusion/main.tex}


\end{document}

\newpage
\section{EMS-ED application}

\subsection{Application}

\subsection{Data analysis of generated problem}

\newpage
\documentclass{article}

\usepackage{amsmath}
\usepackage{amsfonts} 
\usepackage{geometry}
\usepackage{multicol}
\usepackage{float}
% \usepackage{mathtools}
% \usepackage{graphicx}
% \usepackage{soul}
% \usepackage{indentfirst}
\usepackage{tikz}
\usetikzlibrary{calc, automata, chains, arrows.meta, math}
\setcounter{MaxMatrixCols}{20}


\title{A game theoretic model of the behavioural gaming that takes place at the EMS - ED interface}

\author{
    Michalis Panayides, 
    Paul Harper, 
    Vince Knight
}

\begin{document}

\maketitle

\input{Abstract/main.tex}


\newpage
\tableofcontents

\newpage
\input{Introduction/main.tex}

\newpage
\input{Game_theory_component/main.tex}

\newpage
\input{MarkovChain/markov_chain_model/main.tex}
\input{MarkovChain/expressions_from_pi/main.tex}
\input{MarkovChain/markov_example/main.tex}

\newpage
\input{BehaviouralMethodology/main.tex}

\newpage
\input{Application_EMS_ED/main.tex}

\newpage
\input{Conclusion/main.tex}


\end{document}


\end{document}

\newpage
\documentclass{article}

\usepackage{amsmath}
\usepackage{amsfonts} 
\usepackage{geometry}
\usepackage{multicol}
\usepackage{float}
% \usepackage{mathtools}
% \usepackage{graphicx}
% \usepackage{soul}
% \usepackage{indentfirst}
\usepackage{tikz}
\usetikzlibrary{calc, automata, chains, arrows.meta, math}
\setcounter{MaxMatrixCols}{20}


\title{A game theoretic model of the behavioural gaming that takes place at the EMS - ED interface}

\author{
    Michalis Panayides, 
    Paul Harper, 
    Vince Knight
}

\begin{document}

\maketitle

\documentclass{article}

\usepackage{amsmath}
\usepackage{amsfonts} 
\usepackage{geometry}
\usepackage{multicol}
\usepackage{float}
% \usepackage{mathtools}
% \usepackage{graphicx}
% \usepackage{soul}
% \usepackage{indentfirst}
\usepackage{tikz}
\usetikzlibrary{calc, automata, chains, arrows.meta, math}
\setcounter{MaxMatrixCols}{20}


\title{A game theoretic model of the behavioural gaming that takes place at the EMS - ED interface}

\author{
    Michalis Panayides, 
    Paul Harper, 
    Vince Knight
}

\begin{document}

\maketitle

\input{Abstract/main.tex}


\newpage
\tableofcontents

\newpage
\input{Introduction/main.tex}

\newpage
\input{Game_theory_component/main.tex}

\newpage
\input{MarkovChain/markov_chain_model/main.tex}
\input{MarkovChain/expressions_from_pi/main.tex}
\input{MarkovChain/markov_example/main.tex}

\newpage
\input{BehaviouralMethodology/main.tex}

\newpage
\input{Application_EMS_ED/main.tex}

\newpage
\input{Conclusion/main.tex}


\end{document}


\newpage
\tableofcontents

\newpage
\documentclass{article}

\usepackage{amsmath}
\usepackage{amsfonts} 
\usepackage{geometry}
\usepackage{multicol}
\usepackage{float}
% \usepackage{mathtools}
% \usepackage{graphicx}
% \usepackage{soul}
% \usepackage{indentfirst}
\usepackage{tikz}
\usetikzlibrary{calc, automata, chains, arrows.meta, math}
\setcounter{MaxMatrixCols}{20}


\title{A game theoretic model of the behavioural gaming that takes place at the EMS - ED interface}

\author{
    Michalis Panayides, 
    Paul Harper, 
    Vince Knight
}

\begin{document}

\maketitle

\input{Abstract/main.tex}


\newpage
\tableofcontents

\newpage
\input{Introduction/main.tex}

\newpage
\input{Game_theory_component/main.tex}

\newpage
\input{MarkovChain/markov_chain_model/main.tex}
\input{MarkovChain/expressions_from_pi/main.tex}
\input{MarkovChain/markov_example/main.tex}

\newpage
\input{BehaviouralMethodology/main.tex}

\newpage
\input{Application_EMS_ED/main.tex}

\newpage
\input{Conclusion/main.tex}


\end{document}

\newpage
\documentclass{article}

\usepackage{amsmath}
\usepackage{amsfonts} 
\usepackage{geometry}
\usepackage{multicol}
\usepackage{float}
% \usepackage{mathtools}
% \usepackage{graphicx}
% \usepackage{soul}
% \usepackage{indentfirst}
\usepackage{tikz}
\usetikzlibrary{calc, automata, chains, arrows.meta, math}
\setcounter{MaxMatrixCols}{20}


\title{A game theoretic model of the behavioural gaming that takes place at the EMS - ED interface}

\author{
    Michalis Panayides, 
    Paul Harper, 
    Vince Knight
}

\begin{document}

\maketitle

\input{Abstract/main.tex}


\newpage
\tableofcontents

\newpage
\input{Introduction/main.tex}

\newpage
\input{Game_theory_component/main.tex}

\newpage
\input{MarkovChain/markov_chain_model/main.tex}
\input{MarkovChain/expressions_from_pi/main.tex}
\input{MarkovChain/markov_example/main.tex}

\newpage
\input{BehaviouralMethodology/main.tex}

\newpage
\input{Application_EMS_ED/main.tex}

\newpage
\input{Conclusion/main.tex}


\end{document}

\newpage
\documentclass{article}

\usepackage{amsmath}
\usepackage{amsfonts} 
\usepackage{geometry}
\usepackage{multicol}
\usepackage{float}
% \usepackage{mathtools}
% \usepackage{graphicx}
% \usepackage{soul}
% \usepackage{indentfirst}
\usepackage{tikz}
\usetikzlibrary{calc, automata, chains, arrows.meta, math}
\setcounter{MaxMatrixCols}{20}


\title{A game theoretic model of the behavioural gaming that takes place at the EMS - ED interface}

\author{
    Michalis Panayides, 
    Paul Harper, 
    Vince Knight
}

\begin{document}

\maketitle

\input{Abstract/main.tex}


\newpage
\tableofcontents

\newpage
\input{Introduction/main.tex}

\newpage
\input{Game_theory_component/main.tex}

\newpage
\input{MarkovChain/markov_chain_model/main.tex}
\input{MarkovChain/expressions_from_pi/main.tex}
\input{MarkovChain/markov_example/main.tex}

\newpage
\input{BehaviouralMethodology/main.tex}

\newpage
\input{Application_EMS_ED/main.tex}

\newpage
\input{Conclusion/main.tex}


\end{document}
\subsection{Performance Measures}
One may easily derive the average number of individuals that are at any given state 
using \( pi \). 
The average number of individuals in state \( i \) can be calculated by multiplying 
the number of individuals that are present in state \( i \) with the probability 
of being at that particular state (i.e \(\pi_i (u_i + v_i)\)). 
Using this logic it is possible to calculate any performance measures that are related 
to the mean number of individuals in the system.


Average number of people in the system: 
\begin{equation}
    L = \sum_{i=1}^{|\pi|} \pi_i (u_i + v_i)
\end{equation} 

Average number of people in the service centre: 
\begin{equation}
    L_H = \sum_{i=1}^{|\pi|} \pi_i v_i
\end{equation}

Average number of people in the buffer centre:
\begin{equation}
    L_A = \sum_{i=1}^{|\pi|} \pi_i u_i
\end{equation}

Consequently getting the performance measures that are related to the duration of 
time is not as straightforward. 
Such performance measures are the mean waiting time in the system and the mean time 
blocked in the system. 
Under the scope of this study three approaches have been considered to calculate these 
performance measures; a direct approach, a recursive algorithm and consequently a
closed-form formula.

The research question that needs to be answered here is: ``When a class 1/2 
individuals enters the system, what is the expected time that they will have to 
wait?''. 
In order to formulate the answer to that question one needs to consider all possible 
scenarios of what state the system can be in when an individual arrives. 
Furthermore, different formulas arises for class 1 individuals 
and a different one for class 2 individuals.

\subsubsection{Mean waiting time} 
Upon closer inspection of the recursive formula a more compact formula can arise. 
The equivalent closed-form formula eliminates the need for recursion and thus makes 
the computation of waiting times much more efficient. 
Just like in the recursive part there are two formulas; one for \textit{class 1} 
and one for class 2 individuals. 
The formulas are given by:

\begin{equation} \label{eq:closed_form_waiting_others}
    W^{(1)} = \frac{\sum_{\substack{(u,v) \, \in S_A^{(1)} \\ v \geq C}} 
    \frac{1}{C \mu} \times (v-C+1) \times \pi(u,v)}{\sum_{(u,v) \, 
    \in S_A^{(1)}} \pi(u,v)}
\end{equation}
    
\begin{equation}\label{eq:closed_form_waiting_ambulance}
    W^{(2)} = \frac{\sum_{\substack{(u,v) \, \in S_A^{(2)} \\ min(v,T) \geq C}} 
    \frac{1}{C \mu} \times (\min(v+1,T)-C) \times \pi(u,v)}{\sum_{(u,v) \, 
    \in S_A^{(2)}} \pi(u,v)}
\end{equation}

Note here that the summation, in both equations \ref{eq:closed_form_waiting_others} 
and \ref{eq:closed_form_waiting_ambulance}, goes through all states in the set of 
accepting 
states of either class 1 or class 2 individuals respectively, where a wait 
incurs. 
In equation \ref{eq:closed_form_waiting_others} that includes all states \((u,v)\) 
in the set of accepting states of class 1 individuals such that \( v \geq C\); i.e. 
whenever an arrival occurs and the system is at a state where the number of individuals 
in the system is more than or equal to $C$. 
Consequently, for the states that are included in the summation the expression 
\( v-C+1 \) indicates the amount of people in service one would have to wait for 
upon arrival at the hospital.

Additionally, the minimisation function in equation 
\ref{eq:closed_form_waiting_ambulance} 
ensures that when a class 2 individual arrives at any state 
that is greater than the predetermined threshold, the wait that the individual will 
have to endure remains the same. 
In essence, the expression \(\min(v+1,T) - C\) returns the number of people in line 
in front of a particular individual upon arrival.


\subsubsection{Overall Waiting Time}

Consequently, the overall waiting time should can be estimated by a linear combination 
of the waiting times of class 1 and class 2 individuals. 
The overall waiting time can be then given by the following equation where \(c_1\) 
and \(c_2\) are the coefficients of each individual's type waiting time:

\begin{equation}\label{overall_waiting_time_coeff}
    W = c_1 W^{(1)} + c_2 W^{(2)}
\end{equation}

The two coefficients represent the proportion of individuals of each type that 
traversed through the model. 
Theoretically, getting these percentages should be as simple as looking at the arrival 
rates of each type but in practise if the service centre or the buffer centre 
is full, some individuals may be lost to the system. 
Thus, one should account for the probability that an individual is lost to the system. 
This probability can be easily calculated by using the two sets of accepting states 
\(S_A^{(2)}\) and \(S_A^{(1)}\) defined earlier in equations.
Let us define here the probability, for either class type, that an individual 
is not lost in the system by:

\begin{equation*}
    P(L'_1) = \sum_{(u,v) \, \in S_A^{(1)}} \pi(u,v) \hspace{2cm}
    P(L'_2) = \sum_{(u,v) \, \in S_A^{(2)}} \pi(u,v)
\end{equation*}

Having defined these probabilities one may combine them with the arrival rates of 
each class type in such a way to get the expected proportions of class 1 and 
class 2 individuals in the model. 
Thus, by using these values as the coefficient of equation 
\ref{overall_waiting_time_coeff} 
the resultant equation can be used to get the overall waiting time. 
Note here that the equation below gets the overall waiting time for both the recursive 
and the closed-form formula.

\begin{equation}\label{overall_waiting_time}
    W = \frac{\lambda_1 P(L'_1)}{\lambda_2 P(L'_2) + \lambda_1 P(L'_1)} W^{(1)} + 
    \frac{\lambda_2 P(L'_2)}{\lambda_2 P(L'_2) + \lambda_1 P(L'_1)} W^{(2)}
\end{equation}



\subsubsection{Mean blocking time}
Unlike the waiting time, the blocking time is only calculated for class 2 individuals.  
That is because class 1 individuals cannot be blocked. 
Thus, one only needs to consider the pathway of class 2 individuals to get the 
mean blocking time of the system. 
Blocking occurs at states \((u,v)\) where \(u > 0 \). 
Thus, the set of blocking states can be defined as:

\begin{equation*}
    S_b = \{(u,v) \in S \; | \; u > 0\}
\end{equation*}
 
In order to not consider individuals that will be lost to the system, the set of 
accepting states needs to be taken into account. The set of accepting states is given by:

\begin{equation*}
    S_A^{(2)}=
    \begin{cases}
        \{(u, v) \in S \; | \; u < M \} & \textbf{if } T \leq N\\
        \{(u, v) \in S \; | \; v < N \} & \textbf{otherwise}
    \end{cases}
\end{equation*}

For the waiting time formula,
the mean sojourn time for each state was considered,
ignoring any arrivals. Here, the same approach is used but ignoring only class 2
arrivals. That is because for the waiting time formula, once an individual enters 
the service centre (i.e. starts waiting) any individual arriving after them will 
not affect their
pathway. That is not the case for blocking time. When a class 2 individual is 
blocked, 
any class 1 individual that arrives will cause the blocked individual to remain 
blocked for more time. Therefore, class 1 arrivals are considered here:

\begin{equation}\label{eq:time_in_state_blocking_time}
    c(u,v) = 
    \begin{cases}
        \frac{1}{\min(v,C) \mu}, & \text{if } v = C\\
        \frac{1}{\min(v,C) \mu + \lambda_1}, & \text{otherwise}
    \end{cases}
\end{equation}
 
In equation \ref{eq:time_in_state_blocking_time}, both service completions and 
class 1 arrivals are considered. 
Thus, from a blocked individual's perspective whenever the system moves from one 
state \((u,v)\)
to another state it can either:

\begin{itemize}
    \item be because of a service being completed: we will denote the probability 
    of this happening by \(p_s(u,v)\). 
    \item be because of an arrival of an individual of class 1: denoting such 
    probability by \(p_o(u,v)\).
\end{itemize}
The probabilities are given by:

\begin{equation*}
    p_s(u,v) = \frac{\min(v,C)\mu}{\lambda_1 + \min(v,C)\mu}, \qquad
    p_o(u,v) = \frac{\lambda_1}{\lambda_1 + \min(v,C)\mu}
\end{equation*}


Having defined \(c(u,v)\) and \(S_b\) a formula for the blocking time that is
expected to occur at each state can be given by:

\begin{equation}\label{eq:blocking-time-at-each-state}
    b(u,v) = 
    \begin{cases} 
        0, & \textbf{if } (u,v) \notin S_b \\
        c(u,v) + b(u - 1, v), & \textbf{if } v = N = T\\
        c(u,v) + b(u, v-1), & \textbf{if } v = N \neq T \\
        c(u,v) + p_s(u,v) b(u-1, v) + p_o(u,v) b(u, v+1), & \textbf{if } u > 0 
        \textbf{ and } v = T \\
        c(u,v) + p_s(u,v) b(u, v-1) + p_o(u,v) b(u, v+1), & \textbf{otherwise} \\
    \end{cases}
\end{equation}

Equation 
(\ref{eq:blocking-time-at-each-state}) will not be solved recursively. 
A direct approach will be used to solve this equation here. 
By enumerating all equations of (\ref{eq:blocking-time-at-each-state}) for all 
states \((u,v)\) that belong in \(S_b\) 
a system of linear equations arises where the unknown variables are all the \(b(u,v)\)
terms.
For instance, let us consider a Markov model where \(C=2, T=3, N=6, M=2\). 
The Markov model is shown in Figure \ref{fig:example-algeb-blocking}
and the equivalent equations are 
(\ref{eq:first_eq_of_blocking_example})-(\ref{eq:last_eq_of_blocking_example}).
The equations considered here are only the ones that correspond to the blocking 
states.

\begin{multicols*}{2}
    \begin{figure}[H]
        \scalebox{0.50}{\input{MarkovChain/expressions_from_pi/example_model_2362/main.tex}}
        \caption{Example of Markov chain}
        \label{fig:example-algeb-blocking}
    \end{figure}
    \columnbreak
    \begin{align}
        b(1,2) &= c(1,2) + p_o b(1,3) \label{eq:first_eq_of_blocking_example} \\
        b(1,3) &= c(1,3) + p_s b(1,2) + p_o b(1,4) \\
        b(1,4) &= c(1,4) + b(1,3) \\
        b(2,2) &= c(2,2) + p_s b(1,2) + p_o b(2,3) \\
        b(2,3) &= c(2,3) + p_s b(2,2) + p_o b(1,4) \\
        b(2,4) &= c(2,4) + b(2,3)\label{eq:last_eq_of_blocking_example}
    \end{align}
\end{multicols*}

Additionally, the above equations can be transformed into a linear system of the 
form \(Zx=y\) where:

\begin{equation}\label{eq:example-algebaric-approach-blocking-time}
    Z=
    \begin{pmatrix}
        -1 & p_o & 0 & 0 & 0 & 0 \\ %(1,2)
        p_s & -1 & p_o & 0 & 0 & 0 \\ %(1,3)
        0 & 1 & -1 & 0 & 0 & 0 \\ %(1,4)
        p_s & 0 & 0 & -1 & p_o & 0\\ %(2,2)
        0 & 0 & 0 & p_s & -1 & p_o \\ %(2,3)
        0 & 0 & 0 & 0 & 1 & -1 \\ %(2,4)
    \end{pmatrix},
    x=
    \begin{pmatrix}
        b(1,2) \\
        b(1,3) \\
        b(1,4) \\
        b(2,2) \\
        b(2,3) \\
        b(2,4) \\
    \end{pmatrix}, 
    y=
    \begin{pmatrix}
        -c(1,2) \\
        -c(1,3) \\
        -c(1,4) \\
        -c(2,2) \\
        -c(2,3) \\
        -c(2,4) \\
    \end{pmatrix}
\end{equation}

A more generalised form of the equations in 
(\ref{eq:example-algebaric-approach-blocking-time})
can thus be given for any value of \(C,T,N,M\) by:

\begin{align}
    b(1,T) =& c(1, T) + p_o b(1, T + 1) \label{eq:first_eq_of_blocking_general}\\
    b(1,T + 1) =& c(1, T + 1) + p_s(1, T) + p_o b(1, T + 1) \\
    b(1,T + 2) =& c(1, T + 2) + p_s(1, T + 1) + p_o b(1, T + 3) \\
    & \vdots \nonumber \\
    b(1, N) =& c(1, N) + b(1, N - 1) \\
    b(2, T) =& c(2, T) + p_s b(1, T) + p_o b(2, T + 1) \\
    b(2, T + 1) =& c(2, T + 1) + p_s b(2, T) + p_o b(2, T + 2) \\
    & \vdots \nonumber \\
    b(M, T) =& c(M, T) + b(M, T-1) \label{eq:last_eq_of_blocking_general}
\end{align}

The equivalent matrix form of the linear system of equations 
(\ref{eq:first_eq_of_blocking_general}) - (\ref{eq:last_eq_of_blocking_general})
is given by \(Zx=y\), where:
\begin{equation}\label{eq:general-algebaric-approach-blocking-time}
    \scalebox{0.9}{
        \(
        Z = 
        \begin{pmatrix}
            -1 & p_o & 0 & \dots & 0 & 0 & 0 & 0 & 0 & \dots & 0 & 0 \\ %(1,T)
            p_s & -1 & p_o & \dots & 0 & 0 & 0 & 0 & 0 & \dots & 0 & 0 \\ %(1,T+1)
            0 & p_s & -1 & \dots & 0 & 0 & 0 & 0 & 0 & \dots & 0 & 0 \\ %(1,T+2)
            \vdots & \vdots & \vdots & \ddots & \vdots & \vdots & \vdots & \vdots & 
            \vdots & \ddots & \vdots & \vdots \\ 
            0 & 0 & 0 & \dots & 1 & -1 & 0 & 0 & 0 & \dots & 0 & 0 \\ %(1,N)
            p_s & 0 & 0 & \dots & 0 & 0 & -1 & p_o & 0 & \dots & 0 & 0 \\ %(2,T)
            0 & 0 & 0 & \dots & 0 & 0 & p_s & -1 & p_o & \dots & 0 & 0 \\ %(2,T+1)
            \vdots & \vdots & \vdots & \ddots & \vdots & \vdots & \vdots & \vdots & 
            \vdots & \ddots & \vdots & \vdots \\ 
            0 & 0 & 0 & \dots & 0 & 0 & 0 & 0 & 0 & \dots & 1 & -1 \\ %(M,T)
        \end{pmatrix},
        x = 
        \begin{pmatrix}
            b(1,T) \\
            b(1,T+1) \\
            b(1,T+2) \\
            \vdots \\
            b(1,N) \\
            b(2,T) \\
            b(2,T+1) \\
            \vdots \\
            b(M,T) \\
        \end{pmatrix}, 
        y= 
        \begin{pmatrix}
            -c(1,T) \\
            -c(1,T+1) \\
            -c(1,T+2) \\
            \vdots \\
            -c(1,N) \\
            -c(2,T) \\
            -c(2,T+1) \\
            \vdots \\
            -c(M,T) \\
        \end{pmatrix}
        \)
    }
\end{equation}

Thus, having calculated the mean blocking time for all blocking states \(b(u,v)\), 
it only remains to put them together in a formula.
The resultant blocking time formula is given by:

\begin{equation}\label{eq:algebraic-blocking-time}
    B = \frac{\sum_{(u,v) \in S_A} \pi_{(u,v)} \; b(u,v)}{\sum_{(u,v) \in S_A} 
    \pi_{(u,v)}}
\end{equation}

\documentclass{article}

\usepackage{amsmath}
\usepackage{amsfonts} 
\usepackage{geometry}
\usepackage{multicol}
\usepackage{float}
% \usepackage{mathtools}
% \usepackage{graphicx}
% \usepackage{soul}
% \usepackage{indentfirst}
\usepackage{tikz}
\usetikzlibrary{calc, automata, chains, arrows.meta, math}
\setcounter{MaxMatrixCols}{20}


\title{A game theoretic model of the behavioural gaming that takes place at the EMS - ED interface}

\author{
    Michalis Panayides, 
    Paul Harper, 
    Vince Knight
}

\begin{document}

\maketitle

\input{Abstract/main.tex}


\newpage
\tableofcontents

\newpage
\input{Introduction/main.tex}

\newpage
\input{Game_theory_component/main.tex}

\newpage
\input{MarkovChain/markov_chain_model/main.tex}
\input{MarkovChain/expressions_from_pi/main.tex}
\input{MarkovChain/markov_example/main.tex}

\newpage
\input{BehaviouralMethodology/main.tex}

\newpage
\input{Application_EMS_ED/main.tex}

\newpage
\input{Conclusion/main.tex}


\end{document}

\newpage
\documentclass{article}

\usepackage{amsmath}
\usepackage{amsfonts} 
\usepackage{geometry}
\usepackage{multicol}
\usepackage{float}
% \usepackage{mathtools}
% \usepackage{graphicx}
% \usepackage{soul}
% \usepackage{indentfirst}
\usepackage{tikz}
\usetikzlibrary{calc, automata, chains, arrows.meta, math}
\setcounter{MaxMatrixCols}{20}


\title{A game theoretic model of the behavioural gaming that takes place at the EMS - ED interface}

\author{
    Michalis Panayides, 
    Paul Harper, 
    Vince Knight
}

\begin{document}

\maketitle

\input{Abstract/main.tex}


\newpage
\tableofcontents

\newpage
\input{Introduction/main.tex}

\newpage
\input{Game_theory_component/main.tex}

\newpage
\input{MarkovChain/markov_chain_model/main.tex}
\input{MarkovChain/expressions_from_pi/main.tex}
\input{MarkovChain/markov_example/main.tex}

\newpage
\input{BehaviouralMethodology/main.tex}

\newpage
\input{Application_EMS_ED/main.tex}

\newpage
\input{Conclusion/main.tex}


\end{document}

\newpage
\section{EMS-ED application}

\subsection{Application}

\subsection{Data analysis of generated problem}

\newpage
\documentclass{article}

\usepackage{amsmath}
\usepackage{amsfonts} 
\usepackage{geometry}
\usepackage{multicol}
\usepackage{float}
% \usepackage{mathtools}
% \usepackage{graphicx}
% \usepackage{soul}
% \usepackage{indentfirst}
\usepackage{tikz}
\usetikzlibrary{calc, automata, chains, arrows.meta, math}
\setcounter{MaxMatrixCols}{20}


\title{A game theoretic model of the behavioural gaming that takes place at the EMS - ED interface}

\author{
    Michalis Panayides, 
    Paul Harper, 
    Vince Knight
}

\begin{document}

\maketitle

\input{Abstract/main.tex}


\newpage
\tableofcontents

\newpage
\input{Introduction/main.tex}

\newpage
\input{Game_theory_component/main.tex}

\newpage
\input{MarkovChain/markov_chain_model/main.tex}
\input{MarkovChain/expressions_from_pi/main.tex}
\input{MarkovChain/markov_example/main.tex}

\newpage
\input{BehaviouralMethodology/main.tex}

\newpage
\input{Application_EMS_ED/main.tex}

\newpage
\input{Conclusion/main.tex}


\end{document}


\end{document}
\subsection{Performance Measures}
One may easily derive the average number of individuals that are at any given state 
using \( pi \). 
The average number of individuals in state \( i \) can be calculated by multiplying 
the number of individuals that are present in state \( i \) with the probability 
of being at that particular state (i.e \(\pi_i (u_i + v_i)\)). 
Using this logic it is possible to calculate any performance measures that are related 
to the mean number of individuals in the system.


Average number of people in the system: 
\begin{equation}
    L = \sum_{i=1}^{|\pi|} \pi_i (u_i + v_i)
\end{equation} 

Average number of people in the service centre: 
\begin{equation}
    L_H = \sum_{i=1}^{|\pi|} \pi_i v_i
\end{equation}

Average number of people in the buffer centre:
\begin{equation}
    L_A = \sum_{i=1}^{|\pi|} \pi_i u_i
\end{equation}

Consequently getting the performance measures that are related to the duration of 
time is not as straightforward. 
Such performance measures are the mean waiting time in the system and the mean time 
blocked in the system. 
Under the scope of this study three approaches have been considered to calculate these 
performance measures; a direct approach, a recursive algorithm and consequently a
closed-form formula.

The research question that needs to be answered here is: ``When a class 1/2 
individuals enters the system, what is the expected time that they will have to 
wait?''. 
In order to formulate the answer to that question one needs to consider all possible 
scenarios of what state the system can be in when an individual arrives. 
Furthermore, different formulas arises for class 1 individuals 
and a different one for class 2 individuals.

\subsubsection{Mean waiting time} 
Upon closer inspection of the recursive formula a more compact formula can arise. 
The equivalent closed-form formula eliminates the need for recursion and thus makes 
the computation of waiting times much more efficient. 
Just like in the recursive part there are two formulas; one for \textit{class 1} 
and one for class 2 individuals. 
The formulas are given by:

\begin{equation} \label{eq:closed_form_waiting_others}
    W^{(1)} = \frac{\sum_{\substack{(u,v) \, \in S_A^{(1)} \\ v \geq C}} 
    \frac{1}{C \mu} \times (v-C+1) \times \pi(u,v)}{\sum_{(u,v) \, 
    \in S_A^{(1)}} \pi(u,v)}
\end{equation}
    
\begin{equation}\label{eq:closed_form_waiting_ambulance}
    W^{(2)} = \frac{\sum_{\substack{(u,v) \, \in S_A^{(2)} \\ min(v,T) \geq C}} 
    \frac{1}{C \mu} \times (\min(v+1,T)-C) \times \pi(u,v)}{\sum_{(u,v) \, 
    \in S_A^{(2)}} \pi(u,v)}
\end{equation}

Note here that the summation, in both equations \ref{eq:closed_form_waiting_others} 
and \ref{eq:closed_form_waiting_ambulance}, goes through all states in the set of 
accepting 
states of either class 1 or class 2 individuals respectively, where a wait 
incurs. 
In equation \ref{eq:closed_form_waiting_others} that includes all states \((u,v)\) 
in the set of accepting states of class 1 individuals such that \( v \geq C\); i.e. 
whenever an arrival occurs and the system is at a state where the number of individuals 
in the system is more than or equal to $C$. 
Consequently, for the states that are included in the summation the expression 
\( v-C+1 \) indicates the amount of people in service one would have to wait for 
upon arrival at the hospital.

Additionally, the minimisation function in equation 
\ref{eq:closed_form_waiting_ambulance} 
ensures that when a class 2 individual arrives at any state 
that is greater than the predetermined threshold, the wait that the individual will 
have to endure remains the same. 
In essence, the expression \(\min(v+1,T) - C\) returns the number of people in line 
in front of a particular individual upon arrival.


\subsubsection{Overall Waiting Time}

Consequently, the overall waiting time should can be estimated by a linear combination 
of the waiting times of class 1 and class 2 individuals. 
The overall waiting time can be then given by the following equation where \(c_1\) 
and \(c_2\) are the coefficients of each individual's type waiting time:

\begin{equation}\label{overall_waiting_time_coeff}
    W = c_1 W^{(1)} + c_2 W^{(2)}
\end{equation}

The two coefficients represent the proportion of individuals of each type that 
traversed through the model. 
Theoretically, getting these percentages should be as simple as looking at the arrival 
rates of each type but in practise if the service centre or the buffer centre 
is full, some individuals may be lost to the system. 
Thus, one should account for the probability that an individual is lost to the system. 
This probability can be easily calculated by using the two sets of accepting states 
\(S_A^{(2)}\) and \(S_A^{(1)}\) defined earlier in equations.
Let us define here the probability, for either class type, that an individual 
is not lost in the system by:

\begin{equation*}
    P(L'_1) = \sum_{(u,v) \, \in S_A^{(1)}} \pi(u,v) \hspace{2cm}
    P(L'_2) = \sum_{(u,v) \, \in S_A^{(2)}} \pi(u,v)
\end{equation*}

Having defined these probabilities one may combine them with the arrival rates of 
each class type in such a way to get the expected proportions of class 1 and 
class 2 individuals in the model. 
Thus, by using these values as the coefficient of equation 
\ref{overall_waiting_time_coeff} 
the resultant equation can be used to get the overall waiting time. 
Note here that the equation below gets the overall waiting time for both the recursive 
and the closed-form formula.

\begin{equation}\label{overall_waiting_time}
    W = \frac{\lambda_1 P(L'_1)}{\lambda_2 P(L'_2) + \lambda_1 P(L'_1)} W^{(1)} + 
    \frac{\lambda_2 P(L'_2)}{\lambda_2 P(L'_2) + \lambda_1 P(L'_1)} W^{(2)}
\end{equation}



\subsubsection{Mean blocking time}
Unlike the waiting time, the blocking time is only calculated for class 2 individuals.  
That is because class 1 individuals cannot be blocked. 
Thus, one only needs to consider the pathway of class 2 individuals to get the 
mean blocking time of the system. 
Blocking occurs at states \((u,v)\) where \(u > 0 \). 
Thus, the set of blocking states can be defined as:

\begin{equation*}
    S_b = \{(u,v) \in S \; | \; u > 0\}
\end{equation*}
 
In order to not consider individuals that will be lost to the system, the set of 
accepting states needs to be taken into account. The set of accepting states is given by:

\begin{equation*}
    S_A^{(2)}=
    \begin{cases}
        \{(u, v) \in S \; | \; u < M \} & \textbf{if } T \leq N\\
        \{(u, v) \in S \; | \; v < N \} & \textbf{otherwise}
    \end{cases}
\end{equation*}

For the waiting time formula,
the mean sojourn time for each state was considered,
ignoring any arrivals. Here, the same approach is used but ignoring only class 2
arrivals. That is because for the waiting time formula, once an individual enters 
the service centre (i.e. starts waiting) any individual arriving after them will 
not affect their
pathway. That is not the case for blocking time. When a class 2 individual is 
blocked, 
any class 1 individual that arrives will cause the blocked individual to remain 
blocked for more time. Therefore, class 1 arrivals are considered here:

\begin{equation}\label{eq:time_in_state_blocking_time}
    c(u,v) = 
    \begin{cases}
        \frac{1}{\min(v,C) \mu}, & \text{if } v = C\\
        \frac{1}{\min(v,C) \mu + \lambda_1}, & \text{otherwise}
    \end{cases}
\end{equation}
 
In equation \ref{eq:time_in_state_blocking_time}, both service completions and 
class 1 arrivals are considered. 
Thus, from a blocked individual's perspective whenever the system moves from one 
state \((u,v)\)
to another state it can either:

\begin{itemize}
    \item be because of a service being completed: we will denote the probability 
    of this happening by \(p_s(u,v)\). 
    \item be because of an arrival of an individual of class 1: denoting such 
    probability by \(p_o(u,v)\).
\end{itemize}
The probabilities are given by:

\begin{equation*}
    p_s(u,v) = \frac{\min(v,C)\mu}{\lambda_1 + \min(v,C)\mu}, \qquad
    p_o(u,v) = \frac{\lambda_1}{\lambda_1 + \min(v,C)\mu}
\end{equation*}


Having defined \(c(u,v)\) and \(S_b\) a formula for the blocking time that is
expected to occur at each state can be given by:

\begin{equation}\label{eq:blocking-time-at-each-state}
    b(u,v) = 
    \begin{cases} 
        0, & \textbf{if } (u,v) \notin S_b \\
        c(u,v) + b(u - 1, v), & \textbf{if } v = N = T\\
        c(u,v) + b(u, v-1), & \textbf{if } v = N \neq T \\
        c(u,v) + p_s(u,v) b(u-1, v) + p_o(u,v) b(u, v+1), & \textbf{if } u > 0 
        \textbf{ and } v = T \\
        c(u,v) + p_s(u,v) b(u, v-1) + p_o(u,v) b(u, v+1), & \textbf{otherwise} \\
    \end{cases}
\end{equation}

Equation 
(\ref{eq:blocking-time-at-each-state}) will not be solved recursively. 
A direct approach will be used to solve this equation here. 
By enumerating all equations of (\ref{eq:blocking-time-at-each-state}) for all 
states \((u,v)\) that belong in \(S_b\) 
a system of linear equations arises where the unknown variables are all the \(b(u,v)\)
terms.
For instance, let us consider a Markov model where \(C=2, T=3, N=6, M=2\). 
The Markov model is shown in Figure \ref{fig:example-algeb-blocking}
and the equivalent equations are 
(\ref{eq:first_eq_of_blocking_example})-(\ref{eq:last_eq_of_blocking_example}).
The equations considered here are only the ones that correspond to the blocking 
states.

\begin{multicols*}{2}
    \begin{figure}[H]
        \scalebox{0.50}{\documentclass{article}

\usepackage{amsmath}
\usepackage{amsfonts} 
\usepackage{geometry}
\usepackage{multicol}
\usepackage{float}
% \usepackage{mathtools}
% \usepackage{graphicx}
% \usepackage{soul}
% \usepackage{indentfirst}
\usepackage{tikz}
\usetikzlibrary{calc, automata, chains, arrows.meta, math}
\setcounter{MaxMatrixCols}{20}


\title{A game theoretic model of the behavioural gaming that takes place at the EMS - ED interface}

\author{
    Michalis Panayides, 
    Paul Harper, 
    Vince Knight
}

\begin{document}

\maketitle

\input{Abstract/main.tex}


\newpage
\tableofcontents

\newpage
\input{Introduction/main.tex}

\newpage
\input{Game_theory_component/main.tex}

\newpage
\input{MarkovChain/markov_chain_model/main.tex}
\input{MarkovChain/expressions_from_pi/main.tex}
\input{MarkovChain/markov_example/main.tex}

\newpage
\input{BehaviouralMethodology/main.tex}

\newpage
\input{Application_EMS_ED/main.tex}

\newpage
\input{Conclusion/main.tex}


\end{document}}
        \caption{Example of Markov chain}
        \label{fig:example-algeb-blocking}
    \end{figure}
    \columnbreak
    \begin{align}
        b(1,2) &= c(1,2) + p_o b(1,3) \label{eq:first_eq_of_blocking_example} \\
        b(1,3) &= c(1,3) + p_s b(1,2) + p_o b(1,4) \\
        b(1,4) &= c(1,4) + b(1,3) \\
        b(2,2) &= c(2,2) + p_s b(1,2) + p_o b(2,3) \\
        b(2,3) &= c(2,3) + p_s b(2,2) + p_o b(1,4) \\
        b(2,4) &= c(2,4) + b(2,3)\label{eq:last_eq_of_blocking_example}
    \end{align}
\end{multicols*}

Additionally, the above equations can be transformed into a linear system of the 
form \(Zx=y\) where:

\begin{equation}\label{eq:example-algebaric-approach-blocking-time}
    Z=
    \begin{pmatrix}
        -1 & p_o & 0 & 0 & 0 & 0 \\ %(1,2)
        p_s & -1 & p_o & 0 & 0 & 0 \\ %(1,3)
        0 & 1 & -1 & 0 & 0 & 0 \\ %(1,4)
        p_s & 0 & 0 & -1 & p_o & 0\\ %(2,2)
        0 & 0 & 0 & p_s & -1 & p_o \\ %(2,3)
        0 & 0 & 0 & 0 & 1 & -1 \\ %(2,4)
    \end{pmatrix},
    x=
    \begin{pmatrix}
        b(1,2) \\
        b(1,3) \\
        b(1,4) \\
        b(2,2) \\
        b(2,3) \\
        b(2,4) \\
    \end{pmatrix}, 
    y=
    \begin{pmatrix}
        -c(1,2) \\
        -c(1,3) \\
        -c(1,4) \\
        -c(2,2) \\
        -c(2,3) \\
        -c(2,4) \\
    \end{pmatrix}
\end{equation}

A more generalised form of the equations in 
(\ref{eq:example-algebaric-approach-blocking-time})
can thus be given for any value of \(C,T,N,M\) by:

\begin{align}
    b(1,T) =& c(1, T) + p_o b(1, T + 1) \label{eq:first_eq_of_blocking_general}\\
    b(1,T + 1) =& c(1, T + 1) + p_s(1, T) + p_o b(1, T + 1) \\
    b(1,T + 2) =& c(1, T + 2) + p_s(1, T + 1) + p_o b(1, T + 3) \\
    & \vdots \nonumber \\
    b(1, N) =& c(1, N) + b(1, N - 1) \\
    b(2, T) =& c(2, T) + p_s b(1, T) + p_o b(2, T + 1) \\
    b(2, T + 1) =& c(2, T + 1) + p_s b(2, T) + p_o b(2, T + 2) \\
    & \vdots \nonumber \\
    b(M, T) =& c(M, T) + b(M, T-1) \label{eq:last_eq_of_blocking_general}
\end{align}

The equivalent matrix form of the linear system of equations 
(\ref{eq:first_eq_of_blocking_general}) - (\ref{eq:last_eq_of_blocking_general})
is given by \(Zx=y\), where:
\begin{equation}\label{eq:general-algebaric-approach-blocking-time}
    \scalebox{0.9}{
        \(
        Z = 
        \begin{pmatrix}
            -1 & p_o & 0 & \dots & 0 & 0 & 0 & 0 & 0 & \dots & 0 & 0 \\ %(1,T)
            p_s & -1 & p_o & \dots & 0 & 0 & 0 & 0 & 0 & \dots & 0 & 0 \\ %(1,T+1)
            0 & p_s & -1 & \dots & 0 & 0 & 0 & 0 & 0 & \dots & 0 & 0 \\ %(1,T+2)
            \vdots & \vdots & \vdots & \ddots & \vdots & \vdots & \vdots & \vdots & 
            \vdots & \ddots & \vdots & \vdots \\ 
            0 & 0 & 0 & \dots & 1 & -1 & 0 & 0 & 0 & \dots & 0 & 0 \\ %(1,N)
            p_s & 0 & 0 & \dots & 0 & 0 & -1 & p_o & 0 & \dots & 0 & 0 \\ %(2,T)
            0 & 0 & 0 & \dots & 0 & 0 & p_s & -1 & p_o & \dots & 0 & 0 \\ %(2,T+1)
            \vdots & \vdots & \vdots & \ddots & \vdots & \vdots & \vdots & \vdots & 
            \vdots & \ddots & \vdots & \vdots \\ 
            0 & 0 & 0 & \dots & 0 & 0 & 0 & 0 & 0 & \dots & 1 & -1 \\ %(M,T)
        \end{pmatrix},
        x = 
        \begin{pmatrix}
            b(1,T) \\
            b(1,T+1) \\
            b(1,T+2) \\
            \vdots \\
            b(1,N) \\
            b(2,T) \\
            b(2,T+1) \\
            \vdots \\
            b(M,T) \\
        \end{pmatrix}, 
        y= 
        \begin{pmatrix}
            -c(1,T) \\
            -c(1,T+1) \\
            -c(1,T+2) \\
            \vdots \\
            -c(1,N) \\
            -c(2,T) \\
            -c(2,T+1) \\
            \vdots \\
            -c(M,T) \\
        \end{pmatrix}
        \)
    }
\end{equation}

Thus, having calculated the mean blocking time for all blocking states \(b(u,v)\), 
it only remains to put them together in a formula.
The resultant blocking time formula is given by:

\begin{equation}\label{eq:algebraic-blocking-time}
    B = \frac{\sum_{(u,v) \in S_A} \pi_{(u,v)} \; b(u,v)}{\sum_{(u,v) \in S_A} 
    \pi_{(u,v)}}
\end{equation}

\documentclass{article}

\usepackage{amsmath}
\usepackage{amsfonts} 
\usepackage{geometry}
\usepackage{multicol}
\usepackage{float}
% \usepackage{mathtools}
% \usepackage{graphicx}
% \usepackage{soul}
% \usepackage{indentfirst}
\usepackage{tikz}
\usetikzlibrary{calc, automata, chains, arrows.meta, math}
\setcounter{MaxMatrixCols}{20}


\title{A game theoretic model of the behavioural gaming that takes place at the EMS - ED interface}

\author{
    Michalis Panayides, 
    Paul Harper, 
    Vince Knight
}

\begin{document}

\maketitle

\documentclass{article}

\usepackage{amsmath}
\usepackage{amsfonts} 
\usepackage{geometry}
\usepackage{multicol}
\usepackage{float}
% \usepackage{mathtools}
% \usepackage{graphicx}
% \usepackage{soul}
% \usepackage{indentfirst}
\usepackage{tikz}
\usetikzlibrary{calc, automata, chains, arrows.meta, math}
\setcounter{MaxMatrixCols}{20}


\title{A game theoretic model of the behavioural gaming that takes place at the EMS - ED interface}

\author{
    Michalis Panayides, 
    Paul Harper, 
    Vince Knight
}

\begin{document}

\maketitle

\input{Abstract/main.tex}


\newpage
\tableofcontents

\newpage
\input{Introduction/main.tex}

\newpage
\input{Game_theory_component/main.tex}

\newpage
\input{MarkovChain/markov_chain_model/main.tex}
\input{MarkovChain/expressions_from_pi/main.tex}
\input{MarkovChain/markov_example/main.tex}

\newpage
\input{BehaviouralMethodology/main.tex}

\newpage
\input{Application_EMS_ED/main.tex}

\newpage
\input{Conclusion/main.tex}


\end{document}


\newpage
\tableofcontents

\newpage
\documentclass{article}

\usepackage{amsmath}
\usepackage{amsfonts} 
\usepackage{geometry}
\usepackage{multicol}
\usepackage{float}
% \usepackage{mathtools}
% \usepackage{graphicx}
% \usepackage{soul}
% \usepackage{indentfirst}
\usepackage{tikz}
\usetikzlibrary{calc, automata, chains, arrows.meta, math}
\setcounter{MaxMatrixCols}{20}


\title{A game theoretic model of the behavioural gaming that takes place at the EMS - ED interface}

\author{
    Michalis Panayides, 
    Paul Harper, 
    Vince Knight
}

\begin{document}

\maketitle

\input{Abstract/main.tex}


\newpage
\tableofcontents

\newpage
\input{Introduction/main.tex}

\newpage
\input{Game_theory_component/main.tex}

\newpage
\input{MarkovChain/markov_chain_model/main.tex}
\input{MarkovChain/expressions_from_pi/main.tex}
\input{MarkovChain/markov_example/main.tex}

\newpage
\input{BehaviouralMethodology/main.tex}

\newpage
\input{Application_EMS_ED/main.tex}

\newpage
\input{Conclusion/main.tex}


\end{document}

\newpage
\documentclass{article}

\usepackage{amsmath}
\usepackage{amsfonts} 
\usepackage{geometry}
\usepackage{multicol}
\usepackage{float}
% \usepackage{mathtools}
% \usepackage{graphicx}
% \usepackage{soul}
% \usepackage{indentfirst}
\usepackage{tikz}
\usetikzlibrary{calc, automata, chains, arrows.meta, math}
\setcounter{MaxMatrixCols}{20}


\title{A game theoretic model of the behavioural gaming that takes place at the EMS - ED interface}

\author{
    Michalis Panayides, 
    Paul Harper, 
    Vince Knight
}

\begin{document}

\maketitle

\input{Abstract/main.tex}


\newpage
\tableofcontents

\newpage
\input{Introduction/main.tex}

\newpage
\input{Game_theory_component/main.tex}

\newpage
\input{MarkovChain/markov_chain_model/main.tex}
\input{MarkovChain/expressions_from_pi/main.tex}
\input{MarkovChain/markov_example/main.tex}

\newpage
\input{BehaviouralMethodology/main.tex}

\newpage
\input{Application_EMS_ED/main.tex}

\newpage
\input{Conclusion/main.tex}


\end{document}

\newpage
\documentclass{article}

\usepackage{amsmath}
\usepackage{amsfonts} 
\usepackage{geometry}
\usepackage{multicol}
\usepackage{float}
% \usepackage{mathtools}
% \usepackage{graphicx}
% \usepackage{soul}
% \usepackage{indentfirst}
\usepackage{tikz}
\usetikzlibrary{calc, automata, chains, arrows.meta, math}
\setcounter{MaxMatrixCols}{20}


\title{A game theoretic model of the behavioural gaming that takes place at the EMS - ED interface}

\author{
    Michalis Panayides, 
    Paul Harper, 
    Vince Knight
}

\begin{document}

\maketitle

\input{Abstract/main.tex}


\newpage
\tableofcontents

\newpage
\input{Introduction/main.tex}

\newpage
\input{Game_theory_component/main.tex}

\newpage
\input{MarkovChain/markov_chain_model/main.tex}
\input{MarkovChain/expressions_from_pi/main.tex}
\input{MarkovChain/markov_example/main.tex}

\newpage
\input{BehaviouralMethodology/main.tex}

\newpage
\input{Application_EMS_ED/main.tex}

\newpage
\input{Conclusion/main.tex}


\end{document}
\subsection{Performance Measures}
One may easily derive the average number of individuals that are at any given state 
using \( pi \). 
The average number of individuals in state \( i \) can be calculated by multiplying 
the number of individuals that are present in state \( i \) with the probability 
of being at that particular state (i.e \(\pi_i (u_i + v_i)\)). 
Using this logic it is possible to calculate any performance measures that are related 
to the mean number of individuals in the system.


Average number of people in the system: 
\begin{equation}
    L = \sum_{i=1}^{|\pi|} \pi_i (u_i + v_i)
\end{equation} 

Average number of people in the service centre: 
\begin{equation}
    L_H = \sum_{i=1}^{|\pi|} \pi_i v_i
\end{equation}

Average number of people in the buffer centre:
\begin{equation}
    L_A = \sum_{i=1}^{|\pi|} \pi_i u_i
\end{equation}

Consequently getting the performance measures that are related to the duration of 
time is not as straightforward. 
Such performance measures are the mean waiting time in the system and the mean time 
blocked in the system. 
Under the scope of this study three approaches have been considered to calculate these 
performance measures; a direct approach, a recursive algorithm and consequently a
closed-form formula.

The research question that needs to be answered here is: ``When a class 1/2 
individuals enters the system, what is the expected time that they will have to 
wait?''. 
In order to formulate the answer to that question one needs to consider all possible 
scenarios of what state the system can be in when an individual arrives. 
Furthermore, different formulas arises for class 1 individuals 
and a different one for class 2 individuals.

\subsubsection{Mean waiting time} 
Upon closer inspection of the recursive formula a more compact formula can arise. 
The equivalent closed-form formula eliminates the need for recursion and thus makes 
the computation of waiting times much more efficient. 
Just like in the recursive part there are two formulas; one for \textit{class 1} 
and one for class 2 individuals. 
The formulas are given by:

\begin{equation} \label{eq:closed_form_waiting_others}
    W^{(1)} = \frac{\sum_{\substack{(u,v) \, \in S_A^{(1)} \\ v \geq C}} 
    \frac{1}{C \mu} \times (v-C+1) \times \pi(u,v)}{\sum_{(u,v) \, 
    \in S_A^{(1)}} \pi(u,v)}
\end{equation}
    
\begin{equation}\label{eq:closed_form_waiting_ambulance}
    W^{(2)} = \frac{\sum_{\substack{(u,v) \, \in S_A^{(2)} \\ min(v,T) \geq C}} 
    \frac{1}{C \mu} \times (\min(v+1,T)-C) \times \pi(u,v)}{\sum_{(u,v) \, 
    \in S_A^{(2)}} \pi(u,v)}
\end{equation}

Note here that the summation, in both equations \ref{eq:closed_form_waiting_others} 
and \ref{eq:closed_form_waiting_ambulance}, goes through all states in the set of 
accepting 
states of either class 1 or class 2 individuals respectively, where a wait 
incurs. 
In equation \ref{eq:closed_form_waiting_others} that includes all states \((u,v)\) 
in the set of accepting states of class 1 individuals such that \( v \geq C\); i.e. 
whenever an arrival occurs and the system is at a state where the number of individuals 
in the system is more than or equal to $C$. 
Consequently, for the states that are included in the summation the expression 
\( v-C+1 \) indicates the amount of people in service one would have to wait for 
upon arrival at the hospital.

Additionally, the minimisation function in equation 
\ref{eq:closed_form_waiting_ambulance} 
ensures that when a class 2 individual arrives at any state 
that is greater than the predetermined threshold, the wait that the individual will 
have to endure remains the same. 
In essence, the expression \(\min(v+1,T) - C\) returns the number of people in line 
in front of a particular individual upon arrival.


\subsubsection{Overall Waiting Time}

Consequently, the overall waiting time should can be estimated by a linear combination 
of the waiting times of class 1 and class 2 individuals. 
The overall waiting time can be then given by the following equation where \(c_1\) 
and \(c_2\) are the coefficients of each individual's type waiting time:

\begin{equation}\label{overall_waiting_time_coeff}
    W = c_1 W^{(1)} + c_2 W^{(2)}
\end{equation}

The two coefficients represent the proportion of individuals of each type that 
traversed through the model. 
Theoretically, getting these percentages should be as simple as looking at the arrival 
rates of each type but in practise if the service centre or the buffer centre 
is full, some individuals may be lost to the system. 
Thus, one should account for the probability that an individual is lost to the system. 
This probability can be easily calculated by using the two sets of accepting states 
\(S_A^{(2)}\) and \(S_A^{(1)}\) defined earlier in equations.
Let us define here the probability, for either class type, that an individual 
is not lost in the system by:

\begin{equation*}
    P(L'_1) = \sum_{(u,v) \, \in S_A^{(1)}} \pi(u,v) \hspace{2cm}
    P(L'_2) = \sum_{(u,v) \, \in S_A^{(2)}} \pi(u,v)
\end{equation*}

Having defined these probabilities one may combine them with the arrival rates of 
each class type in such a way to get the expected proportions of class 1 and 
class 2 individuals in the model. 
Thus, by using these values as the coefficient of equation 
\ref{overall_waiting_time_coeff} 
the resultant equation can be used to get the overall waiting time. 
Note here that the equation below gets the overall waiting time for both the recursive 
and the closed-form formula.

\begin{equation}\label{overall_waiting_time}
    W = \frac{\lambda_1 P(L'_1)}{\lambda_2 P(L'_2) + \lambda_1 P(L'_1)} W^{(1)} + 
    \frac{\lambda_2 P(L'_2)}{\lambda_2 P(L'_2) + \lambda_1 P(L'_1)} W^{(2)}
\end{equation}



\subsubsection{Mean blocking time}
Unlike the waiting time, the blocking time is only calculated for class 2 individuals.  
That is because class 1 individuals cannot be blocked. 
Thus, one only needs to consider the pathway of class 2 individuals to get the 
mean blocking time of the system. 
Blocking occurs at states \((u,v)\) where \(u > 0 \). 
Thus, the set of blocking states can be defined as:

\begin{equation*}
    S_b = \{(u,v) \in S \; | \; u > 0\}
\end{equation*}
 
In order to not consider individuals that will be lost to the system, the set of 
accepting states needs to be taken into account. The set of accepting states is given by:

\begin{equation*}
    S_A^{(2)}=
    \begin{cases}
        \{(u, v) \in S \; | \; u < M \} & \textbf{if } T \leq N\\
        \{(u, v) \in S \; | \; v < N \} & \textbf{otherwise}
    \end{cases}
\end{equation*}

For the waiting time formula,
the mean sojourn time for each state was considered,
ignoring any arrivals. Here, the same approach is used but ignoring only class 2
arrivals. That is because for the waiting time formula, once an individual enters 
the service centre (i.e. starts waiting) any individual arriving after them will 
not affect their
pathway. That is not the case for blocking time. When a class 2 individual is 
blocked, 
any class 1 individual that arrives will cause the blocked individual to remain 
blocked for more time. Therefore, class 1 arrivals are considered here:

\begin{equation}\label{eq:time_in_state_blocking_time}
    c(u,v) = 
    \begin{cases}
        \frac{1}{\min(v,C) \mu}, & \text{if } v = C\\
        \frac{1}{\min(v,C) \mu + \lambda_1}, & \text{otherwise}
    \end{cases}
\end{equation}
 
In equation \ref{eq:time_in_state_blocking_time}, both service completions and 
class 1 arrivals are considered. 
Thus, from a blocked individual's perspective whenever the system moves from one 
state \((u,v)\)
to another state it can either:

\begin{itemize}
    \item be because of a service being completed: we will denote the probability 
    of this happening by \(p_s(u,v)\). 
    \item be because of an arrival of an individual of class 1: denoting such 
    probability by \(p_o(u,v)\).
\end{itemize}
The probabilities are given by:

\begin{equation*}
    p_s(u,v) = \frac{\min(v,C)\mu}{\lambda_1 + \min(v,C)\mu}, \qquad
    p_o(u,v) = \frac{\lambda_1}{\lambda_1 + \min(v,C)\mu}
\end{equation*}


Having defined \(c(u,v)\) and \(S_b\) a formula for the blocking time that is
expected to occur at each state can be given by:

\begin{equation}\label{eq:blocking-time-at-each-state}
    b(u,v) = 
    \begin{cases} 
        0, & \textbf{if } (u,v) \notin S_b \\
        c(u,v) + b(u - 1, v), & \textbf{if } v = N = T\\
        c(u,v) + b(u, v-1), & \textbf{if } v = N \neq T \\
        c(u,v) + p_s(u,v) b(u-1, v) + p_o(u,v) b(u, v+1), & \textbf{if } u > 0 
        \textbf{ and } v = T \\
        c(u,v) + p_s(u,v) b(u, v-1) + p_o(u,v) b(u, v+1), & \textbf{otherwise} \\
    \end{cases}
\end{equation}

Equation 
(\ref{eq:blocking-time-at-each-state}) will not be solved recursively. 
A direct approach will be used to solve this equation here. 
By enumerating all equations of (\ref{eq:blocking-time-at-each-state}) for all 
states \((u,v)\) that belong in \(S_b\) 
a system of linear equations arises where the unknown variables are all the \(b(u,v)\)
terms.
For instance, let us consider a Markov model where \(C=2, T=3, N=6, M=2\). 
The Markov model is shown in Figure \ref{fig:example-algeb-blocking}
and the equivalent equations are 
(\ref{eq:first_eq_of_blocking_example})-(\ref{eq:last_eq_of_blocking_example}).
The equations considered here are only the ones that correspond to the blocking 
states.

\begin{multicols*}{2}
    \begin{figure}[H]
        \scalebox{0.50}{\input{MarkovChain/expressions_from_pi/example_model_2362/main.tex}}
        \caption{Example of Markov chain}
        \label{fig:example-algeb-blocking}
    \end{figure}
    \columnbreak
    \begin{align}
        b(1,2) &= c(1,2) + p_o b(1,3) \label{eq:first_eq_of_blocking_example} \\
        b(1,3) &= c(1,3) + p_s b(1,2) + p_o b(1,4) \\
        b(1,4) &= c(1,4) + b(1,3) \\
        b(2,2) &= c(2,2) + p_s b(1,2) + p_o b(2,3) \\
        b(2,3) &= c(2,3) + p_s b(2,2) + p_o b(1,4) \\
        b(2,4) &= c(2,4) + b(2,3)\label{eq:last_eq_of_blocking_example}
    \end{align}
\end{multicols*}

Additionally, the above equations can be transformed into a linear system of the 
form \(Zx=y\) where:

\begin{equation}\label{eq:example-algebaric-approach-blocking-time}
    Z=
    \begin{pmatrix}
        -1 & p_o & 0 & 0 & 0 & 0 \\ %(1,2)
        p_s & -1 & p_o & 0 & 0 & 0 \\ %(1,3)
        0 & 1 & -1 & 0 & 0 & 0 \\ %(1,4)
        p_s & 0 & 0 & -1 & p_o & 0\\ %(2,2)
        0 & 0 & 0 & p_s & -1 & p_o \\ %(2,3)
        0 & 0 & 0 & 0 & 1 & -1 \\ %(2,4)
    \end{pmatrix},
    x=
    \begin{pmatrix}
        b(1,2) \\
        b(1,3) \\
        b(1,4) \\
        b(2,2) \\
        b(2,3) \\
        b(2,4) \\
    \end{pmatrix}, 
    y=
    \begin{pmatrix}
        -c(1,2) \\
        -c(1,3) \\
        -c(1,4) \\
        -c(2,2) \\
        -c(2,3) \\
        -c(2,4) \\
    \end{pmatrix}
\end{equation}

A more generalised form of the equations in 
(\ref{eq:example-algebaric-approach-blocking-time})
can thus be given for any value of \(C,T,N,M\) by:

\begin{align}
    b(1,T) =& c(1, T) + p_o b(1, T + 1) \label{eq:first_eq_of_blocking_general}\\
    b(1,T + 1) =& c(1, T + 1) + p_s(1, T) + p_o b(1, T + 1) \\
    b(1,T + 2) =& c(1, T + 2) + p_s(1, T + 1) + p_o b(1, T + 3) \\
    & \vdots \nonumber \\
    b(1, N) =& c(1, N) + b(1, N - 1) \\
    b(2, T) =& c(2, T) + p_s b(1, T) + p_o b(2, T + 1) \\
    b(2, T + 1) =& c(2, T + 1) + p_s b(2, T) + p_o b(2, T + 2) \\
    & \vdots \nonumber \\
    b(M, T) =& c(M, T) + b(M, T-1) \label{eq:last_eq_of_blocking_general}
\end{align}

The equivalent matrix form of the linear system of equations 
(\ref{eq:first_eq_of_blocking_general}) - (\ref{eq:last_eq_of_blocking_general})
is given by \(Zx=y\), where:
\begin{equation}\label{eq:general-algebaric-approach-blocking-time}
    \scalebox{0.9}{
        \(
        Z = 
        \begin{pmatrix}
            -1 & p_o & 0 & \dots & 0 & 0 & 0 & 0 & 0 & \dots & 0 & 0 \\ %(1,T)
            p_s & -1 & p_o & \dots & 0 & 0 & 0 & 0 & 0 & \dots & 0 & 0 \\ %(1,T+1)
            0 & p_s & -1 & \dots & 0 & 0 & 0 & 0 & 0 & \dots & 0 & 0 \\ %(1,T+2)
            \vdots & \vdots & \vdots & \ddots & \vdots & \vdots & \vdots & \vdots & 
            \vdots & \ddots & \vdots & \vdots \\ 
            0 & 0 & 0 & \dots & 1 & -1 & 0 & 0 & 0 & \dots & 0 & 0 \\ %(1,N)
            p_s & 0 & 0 & \dots & 0 & 0 & -1 & p_o & 0 & \dots & 0 & 0 \\ %(2,T)
            0 & 0 & 0 & \dots & 0 & 0 & p_s & -1 & p_o & \dots & 0 & 0 \\ %(2,T+1)
            \vdots & \vdots & \vdots & \ddots & \vdots & \vdots & \vdots & \vdots & 
            \vdots & \ddots & \vdots & \vdots \\ 
            0 & 0 & 0 & \dots & 0 & 0 & 0 & 0 & 0 & \dots & 1 & -1 \\ %(M,T)
        \end{pmatrix},
        x = 
        \begin{pmatrix}
            b(1,T) \\
            b(1,T+1) \\
            b(1,T+2) \\
            \vdots \\
            b(1,N) \\
            b(2,T) \\
            b(2,T+1) \\
            \vdots \\
            b(M,T) \\
        \end{pmatrix}, 
        y= 
        \begin{pmatrix}
            -c(1,T) \\
            -c(1,T+1) \\
            -c(1,T+2) \\
            \vdots \\
            -c(1,N) \\
            -c(2,T) \\
            -c(2,T+1) \\
            \vdots \\
            -c(M,T) \\
        \end{pmatrix}
        \)
    }
\end{equation}

Thus, having calculated the mean blocking time for all blocking states \(b(u,v)\), 
it only remains to put them together in a formula.
The resultant blocking time formula is given by:

\begin{equation}\label{eq:algebraic-blocking-time}
    B = \frac{\sum_{(u,v) \in S_A} \pi_{(u,v)} \; b(u,v)}{\sum_{(u,v) \in S_A} 
    \pi_{(u,v)}}
\end{equation}

\documentclass{article}

\usepackage{amsmath}
\usepackage{amsfonts} 
\usepackage{geometry}
\usepackage{multicol}
\usepackage{float}
% \usepackage{mathtools}
% \usepackage{graphicx}
% \usepackage{soul}
% \usepackage{indentfirst}
\usepackage{tikz}
\usetikzlibrary{calc, automata, chains, arrows.meta, math}
\setcounter{MaxMatrixCols}{20}


\title{A game theoretic model of the behavioural gaming that takes place at the EMS - ED interface}

\author{
    Michalis Panayides, 
    Paul Harper, 
    Vince Knight
}

\begin{document}

\maketitle

\input{Abstract/main.tex}


\newpage
\tableofcontents

\newpage
\input{Introduction/main.tex}

\newpage
\input{Game_theory_component/main.tex}

\newpage
\input{MarkovChain/markov_chain_model/main.tex}
\input{MarkovChain/expressions_from_pi/main.tex}
\input{MarkovChain/markov_example/main.tex}

\newpage
\input{BehaviouralMethodology/main.tex}

\newpage
\input{Application_EMS_ED/main.tex}

\newpage
\input{Conclusion/main.tex}


\end{document}

\newpage
\documentclass{article}

\usepackage{amsmath}
\usepackage{amsfonts} 
\usepackage{geometry}
\usepackage{multicol}
\usepackage{float}
% \usepackage{mathtools}
% \usepackage{graphicx}
% \usepackage{soul}
% \usepackage{indentfirst}
\usepackage{tikz}
\usetikzlibrary{calc, automata, chains, arrows.meta, math}
\setcounter{MaxMatrixCols}{20}


\title{A game theoretic model of the behavioural gaming that takes place at the EMS - ED interface}

\author{
    Michalis Panayides, 
    Paul Harper, 
    Vince Knight
}

\begin{document}

\maketitle

\input{Abstract/main.tex}


\newpage
\tableofcontents

\newpage
\input{Introduction/main.tex}

\newpage
\input{Game_theory_component/main.tex}

\newpage
\input{MarkovChain/markov_chain_model/main.tex}
\input{MarkovChain/expressions_from_pi/main.tex}
\input{MarkovChain/markov_example/main.tex}

\newpage
\input{BehaviouralMethodology/main.tex}

\newpage
\input{Application_EMS_ED/main.tex}

\newpage
\input{Conclusion/main.tex}


\end{document}

\newpage
\section{EMS-ED application}

\subsection{Application}

\subsection{Data analysis of generated problem}

\newpage
\documentclass{article}

\usepackage{amsmath}
\usepackage{amsfonts} 
\usepackage{geometry}
\usepackage{multicol}
\usepackage{float}
% \usepackage{mathtools}
% \usepackage{graphicx}
% \usepackage{soul}
% \usepackage{indentfirst}
\usepackage{tikz}
\usetikzlibrary{calc, automata, chains, arrows.meta, math}
\setcounter{MaxMatrixCols}{20}


\title{A game theoretic model of the behavioural gaming that takes place at the EMS - ED interface}

\author{
    Michalis Panayides, 
    Paul Harper, 
    Vince Knight
}

\begin{document}

\maketitle

\input{Abstract/main.tex}


\newpage
\tableofcontents

\newpage
\input{Introduction/main.tex}

\newpage
\input{Game_theory_component/main.tex}

\newpage
\input{MarkovChain/markov_chain_model/main.tex}
\input{MarkovChain/expressions_from_pi/main.tex}
\input{MarkovChain/markov_example/main.tex}

\newpage
\input{BehaviouralMethodology/main.tex}

\newpage
\input{Application_EMS_ED/main.tex}

\newpage
\input{Conclusion/main.tex}


\end{document}


\end{document}

\newpage
\documentclass{article}

\usepackage{amsmath}
\usepackage{amsfonts} 
\usepackage{geometry}
\usepackage{multicol}
\usepackage{float}
% \usepackage{mathtools}
% \usepackage{graphicx}
% \usepackage{soul}
% \usepackage{indentfirst}
\usepackage{tikz}
\usetikzlibrary{calc, automata, chains, arrows.meta, math}
\setcounter{MaxMatrixCols}{20}


\title{A game theoretic model of the behavioural gaming that takes place at the EMS - ED interface}

\author{
    Michalis Panayides, 
    Paul Harper, 
    Vince Knight
}

\begin{document}

\maketitle

\documentclass{article}

\usepackage{amsmath}
\usepackage{amsfonts} 
\usepackage{geometry}
\usepackage{multicol}
\usepackage{float}
% \usepackage{mathtools}
% \usepackage{graphicx}
% \usepackage{soul}
% \usepackage{indentfirst}
\usepackage{tikz}
\usetikzlibrary{calc, automata, chains, arrows.meta, math}
\setcounter{MaxMatrixCols}{20}


\title{A game theoretic model of the behavioural gaming that takes place at the EMS - ED interface}

\author{
    Michalis Panayides, 
    Paul Harper, 
    Vince Knight
}

\begin{document}

\maketitle

\input{Abstract/main.tex}


\newpage
\tableofcontents

\newpage
\input{Introduction/main.tex}

\newpage
\input{Game_theory_component/main.tex}

\newpage
\input{MarkovChain/markov_chain_model/main.tex}
\input{MarkovChain/expressions_from_pi/main.tex}
\input{MarkovChain/markov_example/main.tex}

\newpage
\input{BehaviouralMethodology/main.tex}

\newpage
\input{Application_EMS_ED/main.tex}

\newpage
\input{Conclusion/main.tex}


\end{document}


\newpage
\tableofcontents

\newpage
\documentclass{article}

\usepackage{amsmath}
\usepackage{amsfonts} 
\usepackage{geometry}
\usepackage{multicol}
\usepackage{float}
% \usepackage{mathtools}
% \usepackage{graphicx}
% \usepackage{soul}
% \usepackage{indentfirst}
\usepackage{tikz}
\usetikzlibrary{calc, automata, chains, arrows.meta, math}
\setcounter{MaxMatrixCols}{20}


\title{A game theoretic model of the behavioural gaming that takes place at the EMS - ED interface}

\author{
    Michalis Panayides, 
    Paul Harper, 
    Vince Knight
}

\begin{document}

\maketitle

\input{Abstract/main.tex}


\newpage
\tableofcontents

\newpage
\input{Introduction/main.tex}

\newpage
\input{Game_theory_component/main.tex}

\newpage
\input{MarkovChain/markov_chain_model/main.tex}
\input{MarkovChain/expressions_from_pi/main.tex}
\input{MarkovChain/markov_example/main.tex}

\newpage
\input{BehaviouralMethodology/main.tex}

\newpage
\input{Application_EMS_ED/main.tex}

\newpage
\input{Conclusion/main.tex}


\end{document}

\newpage
\documentclass{article}

\usepackage{amsmath}
\usepackage{amsfonts} 
\usepackage{geometry}
\usepackage{multicol}
\usepackage{float}
% \usepackage{mathtools}
% \usepackage{graphicx}
% \usepackage{soul}
% \usepackage{indentfirst}
\usepackage{tikz}
\usetikzlibrary{calc, automata, chains, arrows.meta, math}
\setcounter{MaxMatrixCols}{20}


\title{A game theoretic model of the behavioural gaming that takes place at the EMS - ED interface}

\author{
    Michalis Panayides, 
    Paul Harper, 
    Vince Knight
}

\begin{document}

\maketitle

\input{Abstract/main.tex}


\newpage
\tableofcontents

\newpage
\input{Introduction/main.tex}

\newpage
\input{Game_theory_component/main.tex}

\newpage
\input{MarkovChain/markov_chain_model/main.tex}
\input{MarkovChain/expressions_from_pi/main.tex}
\input{MarkovChain/markov_example/main.tex}

\newpage
\input{BehaviouralMethodology/main.tex}

\newpage
\input{Application_EMS_ED/main.tex}

\newpage
\input{Conclusion/main.tex}


\end{document}

\newpage
\documentclass{article}

\usepackage{amsmath}
\usepackage{amsfonts} 
\usepackage{geometry}
\usepackage{multicol}
\usepackage{float}
% \usepackage{mathtools}
% \usepackage{graphicx}
% \usepackage{soul}
% \usepackage{indentfirst}
\usepackage{tikz}
\usetikzlibrary{calc, automata, chains, arrows.meta, math}
\setcounter{MaxMatrixCols}{20}


\title{A game theoretic model of the behavioural gaming that takes place at the EMS - ED interface}

\author{
    Michalis Panayides, 
    Paul Harper, 
    Vince Knight
}

\begin{document}

\maketitle

\input{Abstract/main.tex}


\newpage
\tableofcontents

\newpage
\input{Introduction/main.tex}

\newpage
\input{Game_theory_component/main.tex}

\newpage
\input{MarkovChain/markov_chain_model/main.tex}
\input{MarkovChain/expressions_from_pi/main.tex}
\input{MarkovChain/markov_example/main.tex}

\newpage
\input{BehaviouralMethodology/main.tex}

\newpage
\input{Application_EMS_ED/main.tex}

\newpage
\input{Conclusion/main.tex}


\end{document}
\subsection{Performance Measures}
One may easily derive the average number of individuals that are at any given state 
using \( pi \). 
The average number of individuals in state \( i \) can be calculated by multiplying 
the number of individuals that are present in state \( i \) with the probability 
of being at that particular state (i.e \(\pi_i (u_i + v_i)\)). 
Using this logic it is possible to calculate any performance measures that are related 
to the mean number of individuals in the system.


Average number of people in the system: 
\begin{equation}
    L = \sum_{i=1}^{|\pi|} \pi_i (u_i + v_i)
\end{equation} 

Average number of people in the service centre: 
\begin{equation}
    L_H = \sum_{i=1}^{|\pi|} \pi_i v_i
\end{equation}

Average number of people in the buffer centre:
\begin{equation}
    L_A = \sum_{i=1}^{|\pi|} \pi_i u_i
\end{equation}

Consequently getting the performance measures that are related to the duration of 
time is not as straightforward. 
Such performance measures are the mean waiting time in the system and the mean time 
blocked in the system. 
Under the scope of this study three approaches have been considered to calculate these 
performance measures; a direct approach, a recursive algorithm and consequently a
closed-form formula.

The research question that needs to be answered here is: ``When a class 1/2 
individuals enters the system, what is the expected time that they will have to 
wait?''. 
In order to formulate the answer to that question one needs to consider all possible 
scenarios of what state the system can be in when an individual arrives. 
Furthermore, different formulas arises for class 1 individuals 
and a different one for class 2 individuals.

\subsubsection{Mean waiting time} 
Upon closer inspection of the recursive formula a more compact formula can arise. 
The equivalent closed-form formula eliminates the need for recursion and thus makes 
the computation of waiting times much more efficient. 
Just like in the recursive part there are two formulas; one for \textit{class 1} 
and one for class 2 individuals. 
The formulas are given by:

\begin{equation} \label{eq:closed_form_waiting_others}
    W^{(1)} = \frac{\sum_{\substack{(u,v) \, \in S_A^{(1)} \\ v \geq C}} 
    \frac{1}{C \mu} \times (v-C+1) \times \pi(u,v)}{\sum_{(u,v) \, 
    \in S_A^{(1)}} \pi(u,v)}
\end{equation}
    
\begin{equation}\label{eq:closed_form_waiting_ambulance}
    W^{(2)} = \frac{\sum_{\substack{(u,v) \, \in S_A^{(2)} \\ min(v,T) \geq C}} 
    \frac{1}{C \mu} \times (\min(v+1,T)-C) \times \pi(u,v)}{\sum_{(u,v) \, 
    \in S_A^{(2)}} \pi(u,v)}
\end{equation}

Note here that the summation, in both equations \ref{eq:closed_form_waiting_others} 
and \ref{eq:closed_form_waiting_ambulance}, goes through all states in the set of 
accepting 
states of either class 1 or class 2 individuals respectively, where a wait 
incurs. 
In equation \ref{eq:closed_form_waiting_others} that includes all states \((u,v)\) 
in the set of accepting states of class 1 individuals such that \( v \geq C\); i.e. 
whenever an arrival occurs and the system is at a state where the number of individuals 
in the system is more than or equal to $C$. 
Consequently, for the states that are included in the summation the expression 
\( v-C+1 \) indicates the amount of people in service one would have to wait for 
upon arrival at the hospital.

Additionally, the minimisation function in equation 
\ref{eq:closed_form_waiting_ambulance} 
ensures that when a class 2 individual arrives at any state 
that is greater than the predetermined threshold, the wait that the individual will 
have to endure remains the same. 
In essence, the expression \(\min(v+1,T) - C\) returns the number of people in line 
in front of a particular individual upon arrival.


\subsubsection{Overall Waiting Time}

Consequently, the overall waiting time should can be estimated by a linear combination 
of the waiting times of class 1 and class 2 individuals. 
The overall waiting time can be then given by the following equation where \(c_1\) 
and \(c_2\) are the coefficients of each individual's type waiting time:

\begin{equation}\label{overall_waiting_time_coeff}
    W = c_1 W^{(1)} + c_2 W^{(2)}
\end{equation}

The two coefficients represent the proportion of individuals of each type that 
traversed through the model. 
Theoretically, getting these percentages should be as simple as looking at the arrival 
rates of each type but in practise if the service centre or the buffer centre 
is full, some individuals may be lost to the system. 
Thus, one should account for the probability that an individual is lost to the system. 
This probability can be easily calculated by using the two sets of accepting states 
\(S_A^{(2)}\) and \(S_A^{(1)}\) defined earlier in equations.
Let us define here the probability, for either class type, that an individual 
is not lost in the system by:

\begin{equation*}
    P(L'_1) = \sum_{(u,v) \, \in S_A^{(1)}} \pi(u,v) \hspace{2cm}
    P(L'_2) = \sum_{(u,v) \, \in S_A^{(2)}} \pi(u,v)
\end{equation*}

Having defined these probabilities one may combine them with the arrival rates of 
each class type in such a way to get the expected proportions of class 1 and 
class 2 individuals in the model. 
Thus, by using these values as the coefficient of equation 
\ref{overall_waiting_time_coeff} 
the resultant equation can be used to get the overall waiting time. 
Note here that the equation below gets the overall waiting time for both the recursive 
and the closed-form formula.

\begin{equation}\label{overall_waiting_time}
    W = \frac{\lambda_1 P(L'_1)}{\lambda_2 P(L'_2) + \lambda_1 P(L'_1)} W^{(1)} + 
    \frac{\lambda_2 P(L'_2)}{\lambda_2 P(L'_2) + \lambda_1 P(L'_1)} W^{(2)}
\end{equation}



\subsubsection{Mean blocking time}
Unlike the waiting time, the blocking time is only calculated for class 2 individuals.  
That is because class 1 individuals cannot be blocked. 
Thus, one only needs to consider the pathway of class 2 individuals to get the 
mean blocking time of the system. 
Blocking occurs at states \((u,v)\) where \(u > 0 \). 
Thus, the set of blocking states can be defined as:

\begin{equation*}
    S_b = \{(u,v) \in S \; | \; u > 0\}
\end{equation*}
 
In order to not consider individuals that will be lost to the system, the set of 
accepting states needs to be taken into account. The set of accepting states is given by:

\begin{equation*}
    S_A^{(2)}=
    \begin{cases}
        \{(u, v) \in S \; | \; u < M \} & \textbf{if } T \leq N\\
        \{(u, v) \in S \; | \; v < N \} & \textbf{otherwise}
    \end{cases}
\end{equation*}

For the waiting time formula,
the mean sojourn time for each state was considered,
ignoring any arrivals. Here, the same approach is used but ignoring only class 2
arrivals. That is because for the waiting time formula, once an individual enters 
the service centre (i.e. starts waiting) any individual arriving after them will 
not affect their
pathway. That is not the case for blocking time. When a class 2 individual is 
blocked, 
any class 1 individual that arrives will cause the blocked individual to remain 
blocked for more time. Therefore, class 1 arrivals are considered here:

\begin{equation}\label{eq:time_in_state_blocking_time}
    c(u,v) = 
    \begin{cases}
        \frac{1}{\min(v,C) \mu}, & \text{if } v = C\\
        \frac{1}{\min(v,C) \mu + \lambda_1}, & \text{otherwise}
    \end{cases}
\end{equation}
 
In equation \ref{eq:time_in_state_blocking_time}, both service completions and 
class 1 arrivals are considered. 
Thus, from a blocked individual's perspective whenever the system moves from one 
state \((u,v)\)
to another state it can either:

\begin{itemize}
    \item be because of a service being completed: we will denote the probability 
    of this happening by \(p_s(u,v)\). 
    \item be because of an arrival of an individual of class 1: denoting such 
    probability by \(p_o(u,v)\).
\end{itemize}
The probabilities are given by:

\begin{equation*}
    p_s(u,v) = \frac{\min(v,C)\mu}{\lambda_1 + \min(v,C)\mu}, \qquad
    p_o(u,v) = \frac{\lambda_1}{\lambda_1 + \min(v,C)\mu}
\end{equation*}


Having defined \(c(u,v)\) and \(S_b\) a formula for the blocking time that is
expected to occur at each state can be given by:

\begin{equation}\label{eq:blocking-time-at-each-state}
    b(u,v) = 
    \begin{cases} 
        0, & \textbf{if } (u,v) \notin S_b \\
        c(u,v) + b(u - 1, v), & \textbf{if } v = N = T\\
        c(u,v) + b(u, v-1), & \textbf{if } v = N \neq T \\
        c(u,v) + p_s(u,v) b(u-1, v) + p_o(u,v) b(u, v+1), & \textbf{if } u > 0 
        \textbf{ and } v = T \\
        c(u,v) + p_s(u,v) b(u, v-1) + p_o(u,v) b(u, v+1), & \textbf{otherwise} \\
    \end{cases}
\end{equation}

Equation 
(\ref{eq:blocking-time-at-each-state}) will not be solved recursively. 
A direct approach will be used to solve this equation here. 
By enumerating all equations of (\ref{eq:blocking-time-at-each-state}) for all 
states \((u,v)\) that belong in \(S_b\) 
a system of linear equations arises where the unknown variables are all the \(b(u,v)\)
terms.
For instance, let us consider a Markov model where \(C=2, T=3, N=6, M=2\). 
The Markov model is shown in Figure \ref{fig:example-algeb-blocking}
and the equivalent equations are 
(\ref{eq:first_eq_of_blocking_example})-(\ref{eq:last_eq_of_blocking_example}).
The equations considered here are only the ones that correspond to the blocking 
states.

\begin{multicols*}{2}
    \begin{figure}[H]
        \scalebox{0.50}{\input{MarkovChain/expressions_from_pi/example_model_2362/main.tex}}
        \caption{Example of Markov chain}
        \label{fig:example-algeb-blocking}
    \end{figure}
    \columnbreak
    \begin{align}
        b(1,2) &= c(1,2) + p_o b(1,3) \label{eq:first_eq_of_blocking_example} \\
        b(1,3) &= c(1,3) + p_s b(1,2) + p_o b(1,4) \\
        b(1,4) &= c(1,4) + b(1,3) \\
        b(2,2) &= c(2,2) + p_s b(1,2) + p_o b(2,3) \\
        b(2,3) &= c(2,3) + p_s b(2,2) + p_o b(1,4) \\
        b(2,4) &= c(2,4) + b(2,3)\label{eq:last_eq_of_blocking_example}
    \end{align}
\end{multicols*}

Additionally, the above equations can be transformed into a linear system of the 
form \(Zx=y\) where:

\begin{equation}\label{eq:example-algebaric-approach-blocking-time}
    Z=
    \begin{pmatrix}
        -1 & p_o & 0 & 0 & 0 & 0 \\ %(1,2)
        p_s & -1 & p_o & 0 & 0 & 0 \\ %(1,3)
        0 & 1 & -1 & 0 & 0 & 0 \\ %(1,4)
        p_s & 0 & 0 & -1 & p_o & 0\\ %(2,2)
        0 & 0 & 0 & p_s & -1 & p_o \\ %(2,3)
        0 & 0 & 0 & 0 & 1 & -1 \\ %(2,4)
    \end{pmatrix},
    x=
    \begin{pmatrix}
        b(1,2) \\
        b(1,3) \\
        b(1,4) \\
        b(2,2) \\
        b(2,3) \\
        b(2,4) \\
    \end{pmatrix}, 
    y=
    \begin{pmatrix}
        -c(1,2) \\
        -c(1,3) \\
        -c(1,4) \\
        -c(2,2) \\
        -c(2,3) \\
        -c(2,4) \\
    \end{pmatrix}
\end{equation}

A more generalised form of the equations in 
(\ref{eq:example-algebaric-approach-blocking-time})
can thus be given for any value of \(C,T,N,M\) by:

\begin{align}
    b(1,T) =& c(1, T) + p_o b(1, T + 1) \label{eq:first_eq_of_blocking_general}\\
    b(1,T + 1) =& c(1, T + 1) + p_s(1, T) + p_o b(1, T + 1) \\
    b(1,T + 2) =& c(1, T + 2) + p_s(1, T + 1) + p_o b(1, T + 3) \\
    & \vdots \nonumber \\
    b(1, N) =& c(1, N) + b(1, N - 1) \\
    b(2, T) =& c(2, T) + p_s b(1, T) + p_o b(2, T + 1) \\
    b(2, T + 1) =& c(2, T + 1) + p_s b(2, T) + p_o b(2, T + 2) \\
    & \vdots \nonumber \\
    b(M, T) =& c(M, T) + b(M, T-1) \label{eq:last_eq_of_blocking_general}
\end{align}

The equivalent matrix form of the linear system of equations 
(\ref{eq:first_eq_of_blocking_general}) - (\ref{eq:last_eq_of_blocking_general})
is given by \(Zx=y\), where:
\begin{equation}\label{eq:general-algebaric-approach-blocking-time}
    \scalebox{0.9}{
        \(
        Z = 
        \begin{pmatrix}
            -1 & p_o & 0 & \dots & 0 & 0 & 0 & 0 & 0 & \dots & 0 & 0 \\ %(1,T)
            p_s & -1 & p_o & \dots & 0 & 0 & 0 & 0 & 0 & \dots & 0 & 0 \\ %(1,T+1)
            0 & p_s & -1 & \dots & 0 & 0 & 0 & 0 & 0 & \dots & 0 & 0 \\ %(1,T+2)
            \vdots & \vdots & \vdots & \ddots & \vdots & \vdots & \vdots & \vdots & 
            \vdots & \ddots & \vdots & \vdots \\ 
            0 & 0 & 0 & \dots & 1 & -1 & 0 & 0 & 0 & \dots & 0 & 0 \\ %(1,N)
            p_s & 0 & 0 & \dots & 0 & 0 & -1 & p_o & 0 & \dots & 0 & 0 \\ %(2,T)
            0 & 0 & 0 & \dots & 0 & 0 & p_s & -1 & p_o & \dots & 0 & 0 \\ %(2,T+1)
            \vdots & \vdots & \vdots & \ddots & \vdots & \vdots & \vdots & \vdots & 
            \vdots & \ddots & \vdots & \vdots \\ 
            0 & 0 & 0 & \dots & 0 & 0 & 0 & 0 & 0 & \dots & 1 & -1 \\ %(M,T)
        \end{pmatrix},
        x = 
        \begin{pmatrix}
            b(1,T) \\
            b(1,T+1) \\
            b(1,T+2) \\
            \vdots \\
            b(1,N) \\
            b(2,T) \\
            b(2,T+1) \\
            \vdots \\
            b(M,T) \\
        \end{pmatrix}, 
        y= 
        \begin{pmatrix}
            -c(1,T) \\
            -c(1,T+1) \\
            -c(1,T+2) \\
            \vdots \\
            -c(1,N) \\
            -c(2,T) \\
            -c(2,T+1) \\
            \vdots \\
            -c(M,T) \\
        \end{pmatrix}
        \)
    }
\end{equation}

Thus, having calculated the mean blocking time for all blocking states \(b(u,v)\), 
it only remains to put them together in a formula.
The resultant blocking time formula is given by:

\begin{equation}\label{eq:algebraic-blocking-time}
    B = \frac{\sum_{(u,v) \in S_A} \pi_{(u,v)} \; b(u,v)}{\sum_{(u,v) \in S_A} 
    \pi_{(u,v)}}
\end{equation}

\documentclass{article}

\usepackage{amsmath}
\usepackage{amsfonts} 
\usepackage{geometry}
\usepackage{multicol}
\usepackage{float}
% \usepackage{mathtools}
% \usepackage{graphicx}
% \usepackage{soul}
% \usepackage{indentfirst}
\usepackage{tikz}
\usetikzlibrary{calc, automata, chains, arrows.meta, math}
\setcounter{MaxMatrixCols}{20}


\title{A game theoretic model of the behavioural gaming that takes place at the EMS - ED interface}

\author{
    Michalis Panayides, 
    Paul Harper, 
    Vince Knight
}

\begin{document}

\maketitle

\input{Abstract/main.tex}


\newpage
\tableofcontents

\newpage
\input{Introduction/main.tex}

\newpage
\input{Game_theory_component/main.tex}

\newpage
\input{MarkovChain/markov_chain_model/main.tex}
\input{MarkovChain/expressions_from_pi/main.tex}
\input{MarkovChain/markov_example/main.tex}

\newpage
\input{BehaviouralMethodology/main.tex}

\newpage
\input{Application_EMS_ED/main.tex}

\newpage
\input{Conclusion/main.tex}


\end{document}

\newpage
\documentclass{article}

\usepackage{amsmath}
\usepackage{amsfonts} 
\usepackage{geometry}
\usepackage{multicol}
\usepackage{float}
% \usepackage{mathtools}
% \usepackage{graphicx}
% \usepackage{soul}
% \usepackage{indentfirst}
\usepackage{tikz}
\usetikzlibrary{calc, automata, chains, arrows.meta, math}
\setcounter{MaxMatrixCols}{20}


\title{A game theoretic model of the behavioural gaming that takes place at the EMS - ED interface}

\author{
    Michalis Panayides, 
    Paul Harper, 
    Vince Knight
}

\begin{document}

\maketitle

\input{Abstract/main.tex}


\newpage
\tableofcontents

\newpage
\input{Introduction/main.tex}

\newpage
\input{Game_theory_component/main.tex}

\newpage
\input{MarkovChain/markov_chain_model/main.tex}
\input{MarkovChain/expressions_from_pi/main.tex}
\input{MarkovChain/markov_example/main.tex}

\newpage
\input{BehaviouralMethodology/main.tex}

\newpage
\input{Application_EMS_ED/main.tex}

\newpage
\input{Conclusion/main.tex}


\end{document}

\newpage
\section{EMS-ED application}

\subsection{Application}

\subsection{Data analysis of generated problem}

\newpage
\documentclass{article}

\usepackage{amsmath}
\usepackage{amsfonts} 
\usepackage{geometry}
\usepackage{multicol}
\usepackage{float}
% \usepackage{mathtools}
% \usepackage{graphicx}
% \usepackage{soul}
% \usepackage{indentfirst}
\usepackage{tikz}
\usetikzlibrary{calc, automata, chains, arrows.meta, math}
\setcounter{MaxMatrixCols}{20}


\title{A game theoretic model of the behavioural gaming that takes place at the EMS - ED interface}

\author{
    Michalis Panayides, 
    Paul Harper, 
    Vince Knight
}

\begin{document}

\maketitle

\input{Abstract/main.tex}


\newpage
\tableofcontents

\newpage
\input{Introduction/main.tex}

\newpage
\input{Game_theory_component/main.tex}

\newpage
\input{MarkovChain/markov_chain_model/main.tex}
\input{MarkovChain/expressions_from_pi/main.tex}
\input{MarkovChain/markov_example/main.tex}

\newpage
\input{BehaviouralMethodology/main.tex}

\newpage
\input{Application_EMS_ED/main.tex}

\newpage
\input{Conclusion/main.tex}


\end{document}


\end{document}

\newpage
\section{EMS-ED application}

\subsection{Application}

\subsection{Data analysis of generated problem}

\newpage
\documentclass{article}

\usepackage{amsmath}
\usepackage{amsfonts} 
\usepackage{geometry}
\usepackage{multicol}
\usepackage{float}
% \usepackage{mathtools}
% \usepackage{graphicx}
% \usepackage{soul}
% \usepackage{indentfirst}
\usepackage{tikz}
\usetikzlibrary{calc, automata, chains, arrows.meta, math}
\setcounter{MaxMatrixCols}{20}


\title{A game theoretic model of the behavioural gaming that takes place at the EMS - ED interface}

\author{
    Michalis Panayides, 
    Paul Harper, 
    Vince Knight
}

\begin{document}

\maketitle

\documentclass{article}

\usepackage{amsmath}
\usepackage{amsfonts} 
\usepackage{geometry}
\usepackage{multicol}
\usepackage{float}
% \usepackage{mathtools}
% \usepackage{graphicx}
% \usepackage{soul}
% \usepackage{indentfirst}
\usepackage{tikz}
\usetikzlibrary{calc, automata, chains, arrows.meta, math}
\setcounter{MaxMatrixCols}{20}


\title{A game theoretic model of the behavioural gaming that takes place at the EMS - ED interface}

\author{
    Michalis Panayides, 
    Paul Harper, 
    Vince Knight
}

\begin{document}

\maketitle

\input{Abstract/main.tex}


\newpage
\tableofcontents

\newpage
\input{Introduction/main.tex}

\newpage
\input{Game_theory_component/main.tex}

\newpage
\input{MarkovChain/markov_chain_model/main.tex}
\input{MarkovChain/expressions_from_pi/main.tex}
\input{MarkovChain/markov_example/main.tex}

\newpage
\input{BehaviouralMethodology/main.tex}

\newpage
\input{Application_EMS_ED/main.tex}

\newpage
\input{Conclusion/main.tex}


\end{document}


\newpage
\tableofcontents

\newpage
\documentclass{article}

\usepackage{amsmath}
\usepackage{amsfonts} 
\usepackage{geometry}
\usepackage{multicol}
\usepackage{float}
% \usepackage{mathtools}
% \usepackage{graphicx}
% \usepackage{soul}
% \usepackage{indentfirst}
\usepackage{tikz}
\usetikzlibrary{calc, automata, chains, arrows.meta, math}
\setcounter{MaxMatrixCols}{20}


\title{A game theoretic model of the behavioural gaming that takes place at the EMS - ED interface}

\author{
    Michalis Panayides, 
    Paul Harper, 
    Vince Knight
}

\begin{document}

\maketitle

\input{Abstract/main.tex}


\newpage
\tableofcontents

\newpage
\input{Introduction/main.tex}

\newpage
\input{Game_theory_component/main.tex}

\newpage
\input{MarkovChain/markov_chain_model/main.tex}
\input{MarkovChain/expressions_from_pi/main.tex}
\input{MarkovChain/markov_example/main.tex}

\newpage
\input{BehaviouralMethodology/main.tex}

\newpage
\input{Application_EMS_ED/main.tex}

\newpage
\input{Conclusion/main.tex}


\end{document}

\newpage
\documentclass{article}

\usepackage{amsmath}
\usepackage{amsfonts} 
\usepackage{geometry}
\usepackage{multicol}
\usepackage{float}
% \usepackage{mathtools}
% \usepackage{graphicx}
% \usepackage{soul}
% \usepackage{indentfirst}
\usepackage{tikz}
\usetikzlibrary{calc, automata, chains, arrows.meta, math}
\setcounter{MaxMatrixCols}{20}


\title{A game theoretic model of the behavioural gaming that takes place at the EMS - ED interface}

\author{
    Michalis Panayides, 
    Paul Harper, 
    Vince Knight
}

\begin{document}

\maketitle

\input{Abstract/main.tex}


\newpage
\tableofcontents

\newpage
\input{Introduction/main.tex}

\newpage
\input{Game_theory_component/main.tex}

\newpage
\input{MarkovChain/markov_chain_model/main.tex}
\input{MarkovChain/expressions_from_pi/main.tex}
\input{MarkovChain/markov_example/main.tex}

\newpage
\input{BehaviouralMethodology/main.tex}

\newpage
\input{Application_EMS_ED/main.tex}

\newpage
\input{Conclusion/main.tex}


\end{document}

\newpage
\documentclass{article}

\usepackage{amsmath}
\usepackage{amsfonts} 
\usepackage{geometry}
\usepackage{multicol}
\usepackage{float}
% \usepackage{mathtools}
% \usepackage{graphicx}
% \usepackage{soul}
% \usepackage{indentfirst}
\usepackage{tikz}
\usetikzlibrary{calc, automata, chains, arrows.meta, math}
\setcounter{MaxMatrixCols}{20}


\title{A game theoretic model of the behavioural gaming that takes place at the EMS - ED interface}

\author{
    Michalis Panayides, 
    Paul Harper, 
    Vince Knight
}

\begin{document}

\maketitle

\input{Abstract/main.tex}


\newpage
\tableofcontents

\newpage
\input{Introduction/main.tex}

\newpage
\input{Game_theory_component/main.tex}

\newpage
\input{MarkovChain/markov_chain_model/main.tex}
\input{MarkovChain/expressions_from_pi/main.tex}
\input{MarkovChain/markov_example/main.tex}

\newpage
\input{BehaviouralMethodology/main.tex}

\newpage
\input{Application_EMS_ED/main.tex}

\newpage
\input{Conclusion/main.tex}


\end{document}
\subsection{Performance Measures}
One may easily derive the average number of individuals that are at any given state 
using \( pi \). 
The average number of individuals in state \( i \) can be calculated by multiplying 
the number of individuals that are present in state \( i \) with the probability 
of being at that particular state (i.e \(\pi_i (u_i + v_i)\)). 
Using this logic it is possible to calculate any performance measures that are related 
to the mean number of individuals in the system.


Average number of people in the system: 
\begin{equation}
    L = \sum_{i=1}^{|\pi|} \pi_i (u_i + v_i)
\end{equation} 

Average number of people in the service centre: 
\begin{equation}
    L_H = \sum_{i=1}^{|\pi|} \pi_i v_i
\end{equation}

Average number of people in the buffer centre:
\begin{equation}
    L_A = \sum_{i=1}^{|\pi|} \pi_i u_i
\end{equation}

Consequently getting the performance measures that are related to the duration of 
time is not as straightforward. 
Such performance measures are the mean waiting time in the system and the mean time 
blocked in the system. 
Under the scope of this study three approaches have been considered to calculate these 
performance measures; a direct approach, a recursive algorithm and consequently a
closed-form formula.

The research question that needs to be answered here is: ``When a class 1/2 
individuals enters the system, what is the expected time that they will have to 
wait?''. 
In order to formulate the answer to that question one needs to consider all possible 
scenarios of what state the system can be in when an individual arrives. 
Furthermore, different formulas arises for class 1 individuals 
and a different one for class 2 individuals.

\subsubsection{Mean waiting time} 
Upon closer inspection of the recursive formula a more compact formula can arise. 
The equivalent closed-form formula eliminates the need for recursion and thus makes 
the computation of waiting times much more efficient. 
Just like in the recursive part there are two formulas; one for \textit{class 1} 
and one for class 2 individuals. 
The formulas are given by:

\begin{equation} \label{eq:closed_form_waiting_others}
    W^{(1)} = \frac{\sum_{\substack{(u,v) \, \in S_A^{(1)} \\ v \geq C}} 
    \frac{1}{C \mu} \times (v-C+1) \times \pi(u,v)}{\sum_{(u,v) \, 
    \in S_A^{(1)}} \pi(u,v)}
\end{equation}
    
\begin{equation}\label{eq:closed_form_waiting_ambulance}
    W^{(2)} = \frac{\sum_{\substack{(u,v) \, \in S_A^{(2)} \\ min(v,T) \geq C}} 
    \frac{1}{C \mu} \times (\min(v+1,T)-C) \times \pi(u,v)}{\sum_{(u,v) \, 
    \in S_A^{(2)}} \pi(u,v)}
\end{equation}

Note here that the summation, in both equations \ref{eq:closed_form_waiting_others} 
and \ref{eq:closed_form_waiting_ambulance}, goes through all states in the set of 
accepting 
states of either class 1 or class 2 individuals respectively, where a wait 
incurs. 
In equation \ref{eq:closed_form_waiting_others} that includes all states \((u,v)\) 
in the set of accepting states of class 1 individuals such that \( v \geq C\); i.e. 
whenever an arrival occurs and the system is at a state where the number of individuals 
in the system is more than or equal to $C$. 
Consequently, for the states that are included in the summation the expression 
\( v-C+1 \) indicates the amount of people in service one would have to wait for 
upon arrival at the hospital.

Additionally, the minimisation function in equation 
\ref{eq:closed_form_waiting_ambulance} 
ensures that when a class 2 individual arrives at any state 
that is greater than the predetermined threshold, the wait that the individual will 
have to endure remains the same. 
In essence, the expression \(\min(v+1,T) - C\) returns the number of people in line 
in front of a particular individual upon arrival.


\subsubsection{Overall Waiting Time}

Consequently, the overall waiting time should can be estimated by a linear combination 
of the waiting times of class 1 and class 2 individuals. 
The overall waiting time can be then given by the following equation where \(c_1\) 
and \(c_2\) are the coefficients of each individual's type waiting time:

\begin{equation}\label{overall_waiting_time_coeff}
    W = c_1 W^{(1)} + c_2 W^{(2)}
\end{equation}

The two coefficients represent the proportion of individuals of each type that 
traversed through the model. 
Theoretically, getting these percentages should be as simple as looking at the arrival 
rates of each type but in practise if the service centre or the buffer centre 
is full, some individuals may be lost to the system. 
Thus, one should account for the probability that an individual is lost to the system. 
This probability can be easily calculated by using the two sets of accepting states 
\(S_A^{(2)}\) and \(S_A^{(1)}\) defined earlier in equations.
Let us define here the probability, for either class type, that an individual 
is not lost in the system by:

\begin{equation*}
    P(L'_1) = \sum_{(u,v) \, \in S_A^{(1)}} \pi(u,v) \hspace{2cm}
    P(L'_2) = \sum_{(u,v) \, \in S_A^{(2)}} \pi(u,v)
\end{equation*}

Having defined these probabilities one may combine them with the arrival rates of 
each class type in such a way to get the expected proportions of class 1 and 
class 2 individuals in the model. 
Thus, by using these values as the coefficient of equation 
\ref{overall_waiting_time_coeff} 
the resultant equation can be used to get the overall waiting time. 
Note here that the equation below gets the overall waiting time for both the recursive 
and the closed-form formula.

\begin{equation}\label{overall_waiting_time}
    W = \frac{\lambda_1 P(L'_1)}{\lambda_2 P(L'_2) + \lambda_1 P(L'_1)} W^{(1)} + 
    \frac{\lambda_2 P(L'_2)}{\lambda_2 P(L'_2) + \lambda_1 P(L'_1)} W^{(2)}
\end{equation}



\subsubsection{Mean blocking time}
Unlike the waiting time, the blocking time is only calculated for class 2 individuals.  
That is because class 1 individuals cannot be blocked. 
Thus, one only needs to consider the pathway of class 2 individuals to get the 
mean blocking time of the system. 
Blocking occurs at states \((u,v)\) where \(u > 0 \). 
Thus, the set of blocking states can be defined as:

\begin{equation*}
    S_b = \{(u,v) \in S \; | \; u > 0\}
\end{equation*}
 
In order to not consider individuals that will be lost to the system, the set of 
accepting states needs to be taken into account. The set of accepting states is given by:

\begin{equation*}
    S_A^{(2)}=
    \begin{cases}
        \{(u, v) \in S \; | \; u < M \} & \textbf{if } T \leq N\\
        \{(u, v) \in S \; | \; v < N \} & \textbf{otherwise}
    \end{cases}
\end{equation*}

For the waiting time formula,
the mean sojourn time for each state was considered,
ignoring any arrivals. Here, the same approach is used but ignoring only class 2
arrivals. That is because for the waiting time formula, once an individual enters 
the service centre (i.e. starts waiting) any individual arriving after them will 
not affect their
pathway. That is not the case for blocking time. When a class 2 individual is 
blocked, 
any class 1 individual that arrives will cause the blocked individual to remain 
blocked for more time. Therefore, class 1 arrivals are considered here:

\begin{equation}\label{eq:time_in_state_blocking_time}
    c(u,v) = 
    \begin{cases}
        \frac{1}{\min(v,C) \mu}, & \text{if } v = C\\
        \frac{1}{\min(v,C) \mu + \lambda_1}, & \text{otherwise}
    \end{cases}
\end{equation}
 
In equation \ref{eq:time_in_state_blocking_time}, both service completions and 
class 1 arrivals are considered. 
Thus, from a blocked individual's perspective whenever the system moves from one 
state \((u,v)\)
to another state it can either:

\begin{itemize}
    \item be because of a service being completed: we will denote the probability 
    of this happening by \(p_s(u,v)\). 
    \item be because of an arrival of an individual of class 1: denoting such 
    probability by \(p_o(u,v)\).
\end{itemize}
The probabilities are given by:

\begin{equation*}
    p_s(u,v) = \frac{\min(v,C)\mu}{\lambda_1 + \min(v,C)\mu}, \qquad
    p_o(u,v) = \frac{\lambda_1}{\lambda_1 + \min(v,C)\mu}
\end{equation*}


Having defined \(c(u,v)\) and \(S_b\) a formula for the blocking time that is
expected to occur at each state can be given by:

\begin{equation}\label{eq:blocking-time-at-each-state}
    b(u,v) = 
    \begin{cases} 
        0, & \textbf{if } (u,v) \notin S_b \\
        c(u,v) + b(u - 1, v), & \textbf{if } v = N = T\\
        c(u,v) + b(u, v-1), & \textbf{if } v = N \neq T \\
        c(u,v) + p_s(u,v) b(u-1, v) + p_o(u,v) b(u, v+1), & \textbf{if } u > 0 
        \textbf{ and } v = T \\
        c(u,v) + p_s(u,v) b(u, v-1) + p_o(u,v) b(u, v+1), & \textbf{otherwise} \\
    \end{cases}
\end{equation}

Equation 
(\ref{eq:blocking-time-at-each-state}) will not be solved recursively. 
A direct approach will be used to solve this equation here. 
By enumerating all equations of (\ref{eq:blocking-time-at-each-state}) for all 
states \((u,v)\) that belong in \(S_b\) 
a system of linear equations arises where the unknown variables are all the \(b(u,v)\)
terms.
For instance, let us consider a Markov model where \(C=2, T=3, N=6, M=2\). 
The Markov model is shown in Figure \ref{fig:example-algeb-blocking}
and the equivalent equations are 
(\ref{eq:first_eq_of_blocking_example})-(\ref{eq:last_eq_of_blocking_example}).
The equations considered here are only the ones that correspond to the blocking 
states.

\begin{multicols*}{2}
    \begin{figure}[H]
        \scalebox{0.50}{\input{MarkovChain/expressions_from_pi/example_model_2362/main.tex}}
        \caption{Example of Markov chain}
        \label{fig:example-algeb-blocking}
    \end{figure}
    \columnbreak
    \begin{align}
        b(1,2) &= c(1,2) + p_o b(1,3) \label{eq:first_eq_of_blocking_example} \\
        b(1,3) &= c(1,3) + p_s b(1,2) + p_o b(1,4) \\
        b(1,4) &= c(1,4) + b(1,3) \\
        b(2,2) &= c(2,2) + p_s b(1,2) + p_o b(2,3) \\
        b(2,3) &= c(2,3) + p_s b(2,2) + p_o b(1,4) \\
        b(2,4) &= c(2,4) + b(2,3)\label{eq:last_eq_of_blocking_example}
    \end{align}
\end{multicols*}

Additionally, the above equations can be transformed into a linear system of the 
form \(Zx=y\) where:

\begin{equation}\label{eq:example-algebaric-approach-blocking-time}
    Z=
    \begin{pmatrix}
        -1 & p_o & 0 & 0 & 0 & 0 \\ %(1,2)
        p_s & -1 & p_o & 0 & 0 & 0 \\ %(1,3)
        0 & 1 & -1 & 0 & 0 & 0 \\ %(1,4)
        p_s & 0 & 0 & -1 & p_o & 0\\ %(2,2)
        0 & 0 & 0 & p_s & -1 & p_o \\ %(2,3)
        0 & 0 & 0 & 0 & 1 & -1 \\ %(2,4)
    \end{pmatrix},
    x=
    \begin{pmatrix}
        b(1,2) \\
        b(1,3) \\
        b(1,4) \\
        b(2,2) \\
        b(2,3) \\
        b(2,4) \\
    \end{pmatrix}, 
    y=
    \begin{pmatrix}
        -c(1,2) \\
        -c(1,3) \\
        -c(1,4) \\
        -c(2,2) \\
        -c(2,3) \\
        -c(2,4) \\
    \end{pmatrix}
\end{equation}

A more generalised form of the equations in 
(\ref{eq:example-algebaric-approach-blocking-time})
can thus be given for any value of \(C,T,N,M\) by:

\begin{align}
    b(1,T) =& c(1, T) + p_o b(1, T + 1) \label{eq:first_eq_of_blocking_general}\\
    b(1,T + 1) =& c(1, T + 1) + p_s(1, T) + p_o b(1, T + 1) \\
    b(1,T + 2) =& c(1, T + 2) + p_s(1, T + 1) + p_o b(1, T + 3) \\
    & \vdots \nonumber \\
    b(1, N) =& c(1, N) + b(1, N - 1) \\
    b(2, T) =& c(2, T) + p_s b(1, T) + p_o b(2, T + 1) \\
    b(2, T + 1) =& c(2, T + 1) + p_s b(2, T) + p_o b(2, T + 2) \\
    & \vdots \nonumber \\
    b(M, T) =& c(M, T) + b(M, T-1) \label{eq:last_eq_of_blocking_general}
\end{align}

The equivalent matrix form of the linear system of equations 
(\ref{eq:first_eq_of_blocking_general}) - (\ref{eq:last_eq_of_blocking_general})
is given by \(Zx=y\), where:
\begin{equation}\label{eq:general-algebaric-approach-blocking-time}
    \scalebox{0.9}{
        \(
        Z = 
        \begin{pmatrix}
            -1 & p_o & 0 & \dots & 0 & 0 & 0 & 0 & 0 & \dots & 0 & 0 \\ %(1,T)
            p_s & -1 & p_o & \dots & 0 & 0 & 0 & 0 & 0 & \dots & 0 & 0 \\ %(1,T+1)
            0 & p_s & -1 & \dots & 0 & 0 & 0 & 0 & 0 & \dots & 0 & 0 \\ %(1,T+2)
            \vdots & \vdots & \vdots & \ddots & \vdots & \vdots & \vdots & \vdots & 
            \vdots & \ddots & \vdots & \vdots \\ 
            0 & 0 & 0 & \dots & 1 & -1 & 0 & 0 & 0 & \dots & 0 & 0 \\ %(1,N)
            p_s & 0 & 0 & \dots & 0 & 0 & -1 & p_o & 0 & \dots & 0 & 0 \\ %(2,T)
            0 & 0 & 0 & \dots & 0 & 0 & p_s & -1 & p_o & \dots & 0 & 0 \\ %(2,T+1)
            \vdots & \vdots & \vdots & \ddots & \vdots & \vdots & \vdots & \vdots & 
            \vdots & \ddots & \vdots & \vdots \\ 
            0 & 0 & 0 & \dots & 0 & 0 & 0 & 0 & 0 & \dots & 1 & -1 \\ %(M,T)
        \end{pmatrix},
        x = 
        \begin{pmatrix}
            b(1,T) \\
            b(1,T+1) \\
            b(1,T+2) \\
            \vdots \\
            b(1,N) \\
            b(2,T) \\
            b(2,T+1) \\
            \vdots \\
            b(M,T) \\
        \end{pmatrix}, 
        y= 
        \begin{pmatrix}
            -c(1,T) \\
            -c(1,T+1) \\
            -c(1,T+2) \\
            \vdots \\
            -c(1,N) \\
            -c(2,T) \\
            -c(2,T+1) \\
            \vdots \\
            -c(M,T) \\
        \end{pmatrix}
        \)
    }
\end{equation}

Thus, having calculated the mean blocking time for all blocking states \(b(u,v)\), 
it only remains to put them together in a formula.
The resultant blocking time formula is given by:

\begin{equation}\label{eq:algebraic-blocking-time}
    B = \frac{\sum_{(u,v) \in S_A} \pi_{(u,v)} \; b(u,v)}{\sum_{(u,v) \in S_A} 
    \pi_{(u,v)}}
\end{equation}

\documentclass{article}

\usepackage{amsmath}
\usepackage{amsfonts} 
\usepackage{geometry}
\usepackage{multicol}
\usepackage{float}
% \usepackage{mathtools}
% \usepackage{graphicx}
% \usepackage{soul}
% \usepackage{indentfirst}
\usepackage{tikz}
\usetikzlibrary{calc, automata, chains, arrows.meta, math}
\setcounter{MaxMatrixCols}{20}


\title{A game theoretic model of the behavioural gaming that takes place at the EMS - ED interface}

\author{
    Michalis Panayides, 
    Paul Harper, 
    Vince Knight
}

\begin{document}

\maketitle

\input{Abstract/main.tex}


\newpage
\tableofcontents

\newpage
\input{Introduction/main.tex}

\newpage
\input{Game_theory_component/main.tex}

\newpage
\input{MarkovChain/markov_chain_model/main.tex}
\input{MarkovChain/expressions_from_pi/main.tex}
\input{MarkovChain/markov_example/main.tex}

\newpage
\input{BehaviouralMethodology/main.tex}

\newpage
\input{Application_EMS_ED/main.tex}

\newpage
\input{Conclusion/main.tex}


\end{document}

\newpage
\documentclass{article}

\usepackage{amsmath}
\usepackage{amsfonts} 
\usepackage{geometry}
\usepackage{multicol}
\usepackage{float}
% \usepackage{mathtools}
% \usepackage{graphicx}
% \usepackage{soul}
% \usepackage{indentfirst}
\usepackage{tikz}
\usetikzlibrary{calc, automata, chains, arrows.meta, math}
\setcounter{MaxMatrixCols}{20}


\title{A game theoretic model of the behavioural gaming that takes place at the EMS - ED interface}

\author{
    Michalis Panayides, 
    Paul Harper, 
    Vince Knight
}

\begin{document}

\maketitle

\input{Abstract/main.tex}


\newpage
\tableofcontents

\newpage
\input{Introduction/main.tex}

\newpage
\input{Game_theory_component/main.tex}

\newpage
\input{MarkovChain/markov_chain_model/main.tex}
\input{MarkovChain/expressions_from_pi/main.tex}
\input{MarkovChain/markov_example/main.tex}

\newpage
\input{BehaviouralMethodology/main.tex}

\newpage
\input{Application_EMS_ED/main.tex}

\newpage
\input{Conclusion/main.tex}


\end{document}

\newpage
\section{EMS-ED application}

\subsection{Application}

\subsection{Data analysis of generated problem}

\newpage
\documentclass{article}

\usepackage{amsmath}
\usepackage{amsfonts} 
\usepackage{geometry}
\usepackage{multicol}
\usepackage{float}
% \usepackage{mathtools}
% \usepackage{graphicx}
% \usepackage{soul}
% \usepackage{indentfirst}
\usepackage{tikz}
\usetikzlibrary{calc, automata, chains, arrows.meta, math}
\setcounter{MaxMatrixCols}{20}


\title{A game theoretic model of the behavioural gaming that takes place at the EMS - ED interface}

\author{
    Michalis Panayides, 
    Paul Harper, 
    Vince Knight
}

\begin{document}

\maketitle

\input{Abstract/main.tex}


\newpage
\tableofcontents

\newpage
\input{Introduction/main.tex}

\newpage
\input{Game_theory_component/main.tex}

\newpage
\input{MarkovChain/markov_chain_model/main.tex}
\input{MarkovChain/expressions_from_pi/main.tex}
\input{MarkovChain/markov_example/main.tex}

\newpage
\input{BehaviouralMethodology/main.tex}

\newpage
\input{Application_EMS_ED/main.tex}

\newpage
\input{Conclusion/main.tex}


\end{document}


\end{document}


\end{document}
    }
    \caption{Example Markov model \(C=1, T=2, N=4, M=2\)}
    \label{fig:distribution_of_time_at_specific_state_1_server}
\end{figure}

Consider the Markov model of figure 
\ref{fig:distribution_of_time_at_specific_state_1_server} with one server and a 
threshold of two individuals. 
Assume that an individual of the first type arrives when the model is at state 
\((0,3)\), thus forcing the model to move to state \((0,4)\). 
The distribution of the time needed for the specified individual to exit the 
system from state \((0,4)\) is given by the sum of exponentially distributed 
random variables with the same parameter \(\mu\). 
The sum of such random variables forms an Erlang distribution which is defined 
by the number of random variables that are added and their exponential 
parameter.
Note here that these random variables represent the individual's pathway from 
the perspective of the individual. 
Thus, \(X_i\) represents the time that it takes to move from the 
\(i^{\text{th}}\) position of the queue to the \((i-1)^{\text{th}}\) position 
(i.e. for someone in front of them to finish their service) and \(X_0\) is the 
time it takes to move from having a service to exiting the system.

\begin{align}
    (0,4) \Rightarrow \quad & X_3 \sim Exp(\mu) \nonumber \\
    (0,3) \Rightarrow \quad & X_2 \sim Exp(\mu) \nonumber \\
    (0,2) \Rightarrow \quad & X_1 \sim Exp(\mu) \nonumber \\
    (0,1) \Rightarrow \quad & X_0 \sim Exp(\mu) \nonumber \\
    S = X_3 + X_2 + & X_1 + X_0 = Erlang(4, \mu)
\end{align}

Thus, the waiting and service time of an individual in the model of figure 
\ref{fig:distribution_of_time_at_specific_state_1_server} can be captured by an 
erlang distributed random variable. 
The general CDF of the erlang distribution \(Erlang(k, \mu)\) is given by:

\begin{equation} \label{eq:cdf_erlang}
    P(S < t) = 1 - \sum_{i=0}^{k-1} \frac{1}{i!} e^{-\mu t} (\mu t)^i
\end{equation}

Unfortunately, the erlang distribution can only be used for the sum of 
identically distributed random variables from the exponential distribution. 
Therefore, this approach cannot be used when one of the random variables has a 
different parameter than the others. 
In fact the only case where it can be used is only when the number of servers 
are \(C=1\), or when an individual arrives and goes straight to service 
(i.e. when there is no other individual waiting and there is an empty server).


\paragraph{Time distribution at a state (multiple servers):}

\begin{figure}[H]
    \centering
    \scalebox{0.75}{\documentclass{article}

\usepackage{amsmath}
\usepackage{amsfonts} 
\usepackage{geometry}
\usepackage{multicol}
\usepackage{float}
% \usepackage{mathtools}
% \usepackage{graphicx}
% \usepackage{soul}
% \usepackage{indentfirst}
\usepackage{tikz}
\usetikzlibrary{calc, automata, chains, arrows.meta, math}
\setcounter{MaxMatrixCols}{20}


\title{A game theoretic model of the behavioural gaming that takes place at the EMS - ED interface}

\author{
    Michalis Panayides, 
    Paul Harper, 
    Vince Knight
}

\begin{document}

\maketitle

\documentclass{article}

\usepackage{amsmath}
\usepackage{amsfonts} 
\usepackage{geometry}
\usepackage{multicol}
\usepackage{float}
% \usepackage{mathtools}
% \usepackage{graphicx}
% \usepackage{soul}
% \usepackage{indentfirst}
\usepackage{tikz}
\usetikzlibrary{calc, automata, chains, arrows.meta, math}
\setcounter{MaxMatrixCols}{20}


\title{A game theoretic model of the behavioural gaming that takes place at the EMS - ED interface}

\author{
    Michalis Panayides, 
    Paul Harper, 
    Vince Knight
}

\begin{document}

\maketitle

\documentclass{article}

\usepackage{amsmath}
\usepackage{amsfonts} 
\usepackage{geometry}
\usepackage{multicol}
\usepackage{float}
% \usepackage{mathtools}
% \usepackage{graphicx}
% \usepackage{soul}
% \usepackage{indentfirst}
\usepackage{tikz}
\usetikzlibrary{calc, automata, chains, arrows.meta, math}
\setcounter{MaxMatrixCols}{20}


\title{A game theoretic model of the behavioural gaming that takes place at the EMS - ED interface}

\author{
    Michalis Panayides, 
    Paul Harper, 
    Vince Knight
}

\begin{document}

\maketitle

\input{Abstract/main.tex}


\newpage
\tableofcontents

\newpage
\input{Introduction/main.tex}

\newpage
\input{Game_theory_component/main.tex}

\newpage
\input{MarkovChain/markov_chain_model/main.tex}
\input{MarkovChain/expressions_from_pi/main.tex}
\input{MarkovChain/markov_example/main.tex}

\newpage
\input{BehaviouralMethodology/main.tex}

\newpage
\input{Application_EMS_ED/main.tex}

\newpage
\input{Conclusion/main.tex}


\end{document}


\newpage
\tableofcontents

\newpage
\documentclass{article}

\usepackage{amsmath}
\usepackage{amsfonts} 
\usepackage{geometry}
\usepackage{multicol}
\usepackage{float}
% \usepackage{mathtools}
% \usepackage{graphicx}
% \usepackage{soul}
% \usepackage{indentfirst}
\usepackage{tikz}
\usetikzlibrary{calc, automata, chains, arrows.meta, math}
\setcounter{MaxMatrixCols}{20}


\title{A game theoretic model of the behavioural gaming that takes place at the EMS - ED interface}

\author{
    Michalis Panayides, 
    Paul Harper, 
    Vince Knight
}

\begin{document}

\maketitle

\input{Abstract/main.tex}


\newpage
\tableofcontents

\newpage
\input{Introduction/main.tex}

\newpage
\input{Game_theory_component/main.tex}

\newpage
\input{MarkovChain/markov_chain_model/main.tex}
\input{MarkovChain/expressions_from_pi/main.tex}
\input{MarkovChain/markov_example/main.tex}

\newpage
\input{BehaviouralMethodology/main.tex}

\newpage
\input{Application_EMS_ED/main.tex}

\newpage
\input{Conclusion/main.tex}


\end{document}

\newpage
\documentclass{article}

\usepackage{amsmath}
\usepackage{amsfonts} 
\usepackage{geometry}
\usepackage{multicol}
\usepackage{float}
% \usepackage{mathtools}
% \usepackage{graphicx}
% \usepackage{soul}
% \usepackage{indentfirst}
\usepackage{tikz}
\usetikzlibrary{calc, automata, chains, arrows.meta, math}
\setcounter{MaxMatrixCols}{20}


\title{A game theoretic model of the behavioural gaming that takes place at the EMS - ED interface}

\author{
    Michalis Panayides, 
    Paul Harper, 
    Vince Knight
}

\begin{document}

\maketitle

\input{Abstract/main.tex}


\newpage
\tableofcontents

\newpage
\input{Introduction/main.tex}

\newpage
\input{Game_theory_component/main.tex}

\newpage
\input{MarkovChain/markov_chain_model/main.tex}
\input{MarkovChain/expressions_from_pi/main.tex}
\input{MarkovChain/markov_example/main.tex}

\newpage
\input{BehaviouralMethodology/main.tex}

\newpage
\input{Application_EMS_ED/main.tex}

\newpage
\input{Conclusion/main.tex}


\end{document}

\newpage
\documentclass{article}

\usepackage{amsmath}
\usepackage{amsfonts} 
\usepackage{geometry}
\usepackage{multicol}
\usepackage{float}
% \usepackage{mathtools}
% \usepackage{graphicx}
% \usepackage{soul}
% \usepackage{indentfirst}
\usepackage{tikz}
\usetikzlibrary{calc, automata, chains, arrows.meta, math}
\setcounter{MaxMatrixCols}{20}


\title{A game theoretic model of the behavioural gaming that takes place at the EMS - ED interface}

\author{
    Michalis Panayides, 
    Paul Harper, 
    Vince Knight
}

\begin{document}

\maketitle

\input{Abstract/main.tex}


\newpage
\tableofcontents

\newpage
\input{Introduction/main.tex}

\newpage
\input{Game_theory_component/main.tex}

\newpage
\input{MarkovChain/markov_chain_model/main.tex}
\input{MarkovChain/expressions_from_pi/main.tex}
\input{MarkovChain/markov_example/main.tex}

\newpage
\input{BehaviouralMethodology/main.tex}

\newpage
\input{Application_EMS_ED/main.tex}

\newpage
\input{Conclusion/main.tex}


\end{document}
\subsection{Performance Measures}
One may easily derive the average number of individuals that are at any given state 
using \( pi \). 
The average number of individuals in state \( i \) can be calculated by multiplying 
the number of individuals that are present in state \( i \) with the probability 
of being at that particular state (i.e \(\pi_i (u_i + v_i)\)). 
Using this logic it is possible to calculate any performance measures that are related 
to the mean number of individuals in the system.


Average number of people in the system: 
\begin{equation}
    L = \sum_{i=1}^{|\pi|} \pi_i (u_i + v_i)
\end{equation} 

Average number of people in the service centre: 
\begin{equation}
    L_H = \sum_{i=1}^{|\pi|} \pi_i v_i
\end{equation}

Average number of people in the buffer centre:
\begin{equation}
    L_A = \sum_{i=1}^{|\pi|} \pi_i u_i
\end{equation}

Consequently getting the performance measures that are related to the duration of 
time is not as straightforward. 
Such performance measures are the mean waiting time in the system and the mean time 
blocked in the system. 
Under the scope of this study three approaches have been considered to calculate these 
performance measures; a direct approach, a recursive algorithm and consequently a
closed-form formula.

The research question that needs to be answered here is: ``When a class 1/2 
individuals enters the system, what is the expected time that they will have to 
wait?''. 
In order to formulate the answer to that question one needs to consider all possible 
scenarios of what state the system can be in when an individual arrives. 
Furthermore, different formulas arises for class 1 individuals 
and a different one for class 2 individuals.

\subsubsection{Mean waiting time} 
Upon closer inspection of the recursive formula a more compact formula can arise. 
The equivalent closed-form formula eliminates the need for recursion and thus makes 
the computation of waiting times much more efficient. 
Just like in the recursive part there are two formulas; one for \textit{class 1} 
and one for class 2 individuals. 
The formulas are given by:

\begin{equation} \label{eq:closed_form_waiting_others}
    W^{(1)} = \frac{\sum_{\substack{(u,v) \, \in S_A^{(1)} \\ v \geq C}} 
    \frac{1}{C \mu} \times (v-C+1) \times \pi(u,v)}{\sum_{(u,v) \, 
    \in S_A^{(1)}} \pi(u,v)}
\end{equation}
    
\begin{equation}\label{eq:closed_form_waiting_ambulance}
    W^{(2)} = \frac{\sum_{\substack{(u,v) \, \in S_A^{(2)} \\ min(v,T) \geq C}} 
    \frac{1}{C \mu} \times (\min(v+1,T)-C) \times \pi(u,v)}{\sum_{(u,v) \, 
    \in S_A^{(2)}} \pi(u,v)}
\end{equation}

Note here that the summation, in both equations \ref{eq:closed_form_waiting_others} 
and \ref{eq:closed_form_waiting_ambulance}, goes through all states in the set of 
accepting 
states of either class 1 or class 2 individuals respectively, where a wait 
incurs. 
In equation \ref{eq:closed_form_waiting_others} that includes all states \((u,v)\) 
in the set of accepting states of class 1 individuals such that \( v \geq C\); i.e. 
whenever an arrival occurs and the system is at a state where the number of individuals 
in the system is more than or equal to $C$. 
Consequently, for the states that are included in the summation the expression 
\( v-C+1 \) indicates the amount of people in service one would have to wait for 
upon arrival at the hospital.

Additionally, the minimisation function in equation 
\ref{eq:closed_form_waiting_ambulance} 
ensures that when a class 2 individual arrives at any state 
that is greater than the predetermined threshold, the wait that the individual will 
have to endure remains the same. 
In essence, the expression \(\min(v+1,T) - C\) returns the number of people in line 
in front of a particular individual upon arrival.


\subsubsection{Overall Waiting Time}

Consequently, the overall waiting time should can be estimated by a linear combination 
of the waiting times of class 1 and class 2 individuals. 
The overall waiting time can be then given by the following equation where \(c_1\) 
and \(c_2\) are the coefficients of each individual's type waiting time:

\begin{equation}\label{overall_waiting_time_coeff}
    W = c_1 W^{(1)} + c_2 W^{(2)}
\end{equation}

The two coefficients represent the proportion of individuals of each type that 
traversed through the model. 
Theoretically, getting these percentages should be as simple as looking at the arrival 
rates of each type but in practise if the service centre or the buffer centre 
is full, some individuals may be lost to the system. 
Thus, one should account for the probability that an individual is lost to the system. 
This probability can be easily calculated by using the two sets of accepting states 
\(S_A^{(2)}\) and \(S_A^{(1)}\) defined earlier in equations.
Let us define here the probability, for either class type, that an individual 
is not lost in the system by:

\begin{equation*}
    P(L'_1) = \sum_{(u,v) \, \in S_A^{(1)}} \pi(u,v) \hspace{2cm}
    P(L'_2) = \sum_{(u,v) \, \in S_A^{(2)}} \pi(u,v)
\end{equation*}

Having defined these probabilities one may combine them with the arrival rates of 
each class type in such a way to get the expected proportions of class 1 and 
class 2 individuals in the model. 
Thus, by using these values as the coefficient of equation 
\ref{overall_waiting_time_coeff} 
the resultant equation can be used to get the overall waiting time. 
Note here that the equation below gets the overall waiting time for both the recursive 
and the closed-form formula.

\begin{equation}\label{overall_waiting_time}
    W = \frac{\lambda_1 P(L'_1)}{\lambda_2 P(L'_2) + \lambda_1 P(L'_1)} W^{(1)} + 
    \frac{\lambda_2 P(L'_2)}{\lambda_2 P(L'_2) + \lambda_1 P(L'_1)} W^{(2)}
\end{equation}



\subsubsection{Mean blocking time}
Unlike the waiting time, the blocking time is only calculated for class 2 individuals.  
That is because class 1 individuals cannot be blocked. 
Thus, one only needs to consider the pathway of class 2 individuals to get the 
mean blocking time of the system. 
Blocking occurs at states \((u,v)\) where \(u > 0 \). 
Thus, the set of blocking states can be defined as:

\begin{equation*}
    S_b = \{(u,v) \in S \; | \; u > 0\}
\end{equation*}
 
In order to not consider individuals that will be lost to the system, the set of 
accepting states needs to be taken into account. The set of accepting states is given by:

\begin{equation*}
    S_A^{(2)}=
    \begin{cases}
        \{(u, v) \in S \; | \; u < M \} & \textbf{if } T \leq N\\
        \{(u, v) \in S \; | \; v < N \} & \textbf{otherwise}
    \end{cases}
\end{equation*}

For the waiting time formula,
the mean sojourn time for each state was considered,
ignoring any arrivals. Here, the same approach is used but ignoring only class 2
arrivals. That is because for the waiting time formula, once an individual enters 
the service centre (i.e. starts waiting) any individual arriving after them will 
not affect their
pathway. That is not the case for blocking time. When a class 2 individual is 
blocked, 
any class 1 individual that arrives will cause the blocked individual to remain 
blocked for more time. Therefore, class 1 arrivals are considered here:

\begin{equation}\label{eq:time_in_state_blocking_time}
    c(u,v) = 
    \begin{cases}
        \frac{1}{\min(v,C) \mu}, & \text{if } v = C\\
        \frac{1}{\min(v,C) \mu + \lambda_1}, & \text{otherwise}
    \end{cases}
\end{equation}
 
In equation \ref{eq:time_in_state_blocking_time}, both service completions and 
class 1 arrivals are considered. 
Thus, from a blocked individual's perspective whenever the system moves from one 
state \((u,v)\)
to another state it can either:

\begin{itemize}
    \item be because of a service being completed: we will denote the probability 
    of this happening by \(p_s(u,v)\). 
    \item be because of an arrival of an individual of class 1: denoting such 
    probability by \(p_o(u,v)\).
\end{itemize}
The probabilities are given by:

\begin{equation*}
    p_s(u,v) = \frac{\min(v,C)\mu}{\lambda_1 + \min(v,C)\mu}, \qquad
    p_o(u,v) = \frac{\lambda_1}{\lambda_1 + \min(v,C)\mu}
\end{equation*}


Having defined \(c(u,v)\) and \(S_b\) a formula for the blocking time that is
expected to occur at each state can be given by:

\begin{equation}\label{eq:blocking-time-at-each-state}
    b(u,v) = 
    \begin{cases} 
        0, & \textbf{if } (u,v) \notin S_b \\
        c(u,v) + b(u - 1, v), & \textbf{if } v = N = T\\
        c(u,v) + b(u, v-1), & \textbf{if } v = N \neq T \\
        c(u,v) + p_s(u,v) b(u-1, v) + p_o(u,v) b(u, v+1), & \textbf{if } u > 0 
        \textbf{ and } v = T \\
        c(u,v) + p_s(u,v) b(u, v-1) + p_o(u,v) b(u, v+1), & \textbf{otherwise} \\
    \end{cases}
\end{equation}

Equation 
(\ref{eq:blocking-time-at-each-state}) will not be solved recursively. 
A direct approach will be used to solve this equation here. 
By enumerating all equations of (\ref{eq:blocking-time-at-each-state}) for all 
states \((u,v)\) that belong in \(S_b\) 
a system of linear equations arises where the unknown variables are all the \(b(u,v)\)
terms.
For instance, let us consider a Markov model where \(C=2, T=3, N=6, M=2\). 
The Markov model is shown in Figure \ref{fig:example-algeb-blocking}
and the equivalent equations are 
(\ref{eq:first_eq_of_blocking_example})-(\ref{eq:last_eq_of_blocking_example}).
The equations considered here are only the ones that correspond to the blocking 
states.

\begin{multicols*}{2}
    \begin{figure}[H]
        \scalebox{0.50}{\input{MarkovChain/expressions_from_pi/example_model_2362/main.tex}}
        \caption{Example of Markov chain}
        \label{fig:example-algeb-blocking}
    \end{figure}
    \columnbreak
    \begin{align}
        b(1,2) &= c(1,2) + p_o b(1,3) \label{eq:first_eq_of_blocking_example} \\
        b(1,3) &= c(1,3) + p_s b(1,2) + p_o b(1,4) \\
        b(1,4) &= c(1,4) + b(1,3) \\
        b(2,2) &= c(2,2) + p_s b(1,2) + p_o b(2,3) \\
        b(2,3) &= c(2,3) + p_s b(2,2) + p_o b(1,4) \\
        b(2,4) &= c(2,4) + b(2,3)\label{eq:last_eq_of_blocking_example}
    \end{align}
\end{multicols*}

Additionally, the above equations can be transformed into a linear system of the 
form \(Zx=y\) where:

\begin{equation}\label{eq:example-algebaric-approach-blocking-time}
    Z=
    \begin{pmatrix}
        -1 & p_o & 0 & 0 & 0 & 0 \\ %(1,2)
        p_s & -1 & p_o & 0 & 0 & 0 \\ %(1,3)
        0 & 1 & -1 & 0 & 0 & 0 \\ %(1,4)
        p_s & 0 & 0 & -1 & p_o & 0\\ %(2,2)
        0 & 0 & 0 & p_s & -1 & p_o \\ %(2,3)
        0 & 0 & 0 & 0 & 1 & -1 \\ %(2,4)
    \end{pmatrix},
    x=
    \begin{pmatrix}
        b(1,2) \\
        b(1,3) \\
        b(1,4) \\
        b(2,2) \\
        b(2,3) \\
        b(2,4) \\
    \end{pmatrix}, 
    y=
    \begin{pmatrix}
        -c(1,2) \\
        -c(1,3) \\
        -c(1,4) \\
        -c(2,2) \\
        -c(2,3) \\
        -c(2,4) \\
    \end{pmatrix}
\end{equation}

A more generalised form of the equations in 
(\ref{eq:example-algebaric-approach-blocking-time})
can thus be given for any value of \(C,T,N,M\) by:

\begin{align}
    b(1,T) =& c(1, T) + p_o b(1, T + 1) \label{eq:first_eq_of_blocking_general}\\
    b(1,T + 1) =& c(1, T + 1) + p_s(1, T) + p_o b(1, T + 1) \\
    b(1,T + 2) =& c(1, T + 2) + p_s(1, T + 1) + p_o b(1, T + 3) \\
    & \vdots \nonumber \\
    b(1, N) =& c(1, N) + b(1, N - 1) \\
    b(2, T) =& c(2, T) + p_s b(1, T) + p_o b(2, T + 1) \\
    b(2, T + 1) =& c(2, T + 1) + p_s b(2, T) + p_o b(2, T + 2) \\
    & \vdots \nonumber \\
    b(M, T) =& c(M, T) + b(M, T-1) \label{eq:last_eq_of_blocking_general}
\end{align}

The equivalent matrix form of the linear system of equations 
(\ref{eq:first_eq_of_blocking_general}) - (\ref{eq:last_eq_of_blocking_general})
is given by \(Zx=y\), where:
\begin{equation}\label{eq:general-algebaric-approach-blocking-time}
    \scalebox{0.9}{
        \(
        Z = 
        \begin{pmatrix}
            -1 & p_o & 0 & \dots & 0 & 0 & 0 & 0 & 0 & \dots & 0 & 0 \\ %(1,T)
            p_s & -1 & p_o & \dots & 0 & 0 & 0 & 0 & 0 & \dots & 0 & 0 \\ %(1,T+1)
            0 & p_s & -1 & \dots & 0 & 0 & 0 & 0 & 0 & \dots & 0 & 0 \\ %(1,T+2)
            \vdots & \vdots & \vdots & \ddots & \vdots & \vdots & \vdots & \vdots & 
            \vdots & \ddots & \vdots & \vdots \\ 
            0 & 0 & 0 & \dots & 1 & -1 & 0 & 0 & 0 & \dots & 0 & 0 \\ %(1,N)
            p_s & 0 & 0 & \dots & 0 & 0 & -1 & p_o & 0 & \dots & 0 & 0 \\ %(2,T)
            0 & 0 & 0 & \dots & 0 & 0 & p_s & -1 & p_o & \dots & 0 & 0 \\ %(2,T+1)
            \vdots & \vdots & \vdots & \ddots & \vdots & \vdots & \vdots & \vdots & 
            \vdots & \ddots & \vdots & \vdots \\ 
            0 & 0 & 0 & \dots & 0 & 0 & 0 & 0 & 0 & \dots & 1 & -1 \\ %(M,T)
        \end{pmatrix},
        x = 
        \begin{pmatrix}
            b(1,T) \\
            b(1,T+1) \\
            b(1,T+2) \\
            \vdots \\
            b(1,N) \\
            b(2,T) \\
            b(2,T+1) \\
            \vdots \\
            b(M,T) \\
        \end{pmatrix}, 
        y= 
        \begin{pmatrix}
            -c(1,T) \\
            -c(1,T+1) \\
            -c(1,T+2) \\
            \vdots \\
            -c(1,N) \\
            -c(2,T) \\
            -c(2,T+1) \\
            \vdots \\
            -c(M,T) \\
        \end{pmatrix}
        \)
    }
\end{equation}

Thus, having calculated the mean blocking time for all blocking states \(b(u,v)\), 
it only remains to put them together in a formula.
The resultant blocking time formula is given by:

\begin{equation}\label{eq:algebraic-blocking-time}
    B = \frac{\sum_{(u,v) \in S_A} \pi_{(u,v)} \; b(u,v)}{\sum_{(u,v) \in S_A} 
    \pi_{(u,v)}}
\end{equation}

\documentclass{article}

\usepackage{amsmath}
\usepackage{amsfonts} 
\usepackage{geometry}
\usepackage{multicol}
\usepackage{float}
% \usepackage{mathtools}
% \usepackage{graphicx}
% \usepackage{soul}
% \usepackage{indentfirst}
\usepackage{tikz}
\usetikzlibrary{calc, automata, chains, arrows.meta, math}
\setcounter{MaxMatrixCols}{20}


\title{A game theoretic model of the behavioural gaming that takes place at the EMS - ED interface}

\author{
    Michalis Panayides, 
    Paul Harper, 
    Vince Knight
}

\begin{document}

\maketitle

\input{Abstract/main.tex}


\newpage
\tableofcontents

\newpage
\input{Introduction/main.tex}

\newpage
\input{Game_theory_component/main.tex}

\newpage
\input{MarkovChain/markov_chain_model/main.tex}
\input{MarkovChain/expressions_from_pi/main.tex}
\input{MarkovChain/markov_example/main.tex}

\newpage
\input{BehaviouralMethodology/main.tex}

\newpage
\input{Application_EMS_ED/main.tex}

\newpage
\input{Conclusion/main.tex}


\end{document}

\newpage
\documentclass{article}

\usepackage{amsmath}
\usepackage{amsfonts} 
\usepackage{geometry}
\usepackage{multicol}
\usepackage{float}
% \usepackage{mathtools}
% \usepackage{graphicx}
% \usepackage{soul}
% \usepackage{indentfirst}
\usepackage{tikz}
\usetikzlibrary{calc, automata, chains, arrows.meta, math}
\setcounter{MaxMatrixCols}{20}


\title{A game theoretic model of the behavioural gaming that takes place at the EMS - ED interface}

\author{
    Michalis Panayides, 
    Paul Harper, 
    Vince Knight
}

\begin{document}

\maketitle

\input{Abstract/main.tex}


\newpage
\tableofcontents

\newpage
\input{Introduction/main.tex}

\newpage
\input{Game_theory_component/main.tex}

\newpage
\input{MarkovChain/markov_chain_model/main.tex}
\input{MarkovChain/expressions_from_pi/main.tex}
\input{MarkovChain/markov_example/main.tex}

\newpage
\input{BehaviouralMethodology/main.tex}

\newpage
\input{Application_EMS_ED/main.tex}

\newpage
\input{Conclusion/main.tex}


\end{document}

\newpage
\section{EMS-ED application}

\subsection{Application}

\subsection{Data analysis of generated problem}

\newpage
\documentclass{article}

\usepackage{amsmath}
\usepackage{amsfonts} 
\usepackage{geometry}
\usepackage{multicol}
\usepackage{float}
% \usepackage{mathtools}
% \usepackage{graphicx}
% \usepackage{soul}
% \usepackage{indentfirst}
\usepackage{tikz}
\usetikzlibrary{calc, automata, chains, arrows.meta, math}
\setcounter{MaxMatrixCols}{20}


\title{A game theoretic model of the behavioural gaming that takes place at the EMS - ED interface}

\author{
    Michalis Panayides, 
    Paul Harper, 
    Vince Knight
}

\begin{document}

\maketitle

\input{Abstract/main.tex}


\newpage
\tableofcontents

\newpage
\input{Introduction/main.tex}

\newpage
\input{Game_theory_component/main.tex}

\newpage
\input{MarkovChain/markov_chain_model/main.tex}
\input{MarkovChain/expressions_from_pi/main.tex}
\input{MarkovChain/markov_example/main.tex}

\newpage
\input{BehaviouralMethodology/main.tex}

\newpage
\input{Application_EMS_ED/main.tex}

\newpage
\input{Conclusion/main.tex}


\end{document}


\end{document}


\newpage
\tableofcontents

\newpage
\documentclass{article}

\usepackage{amsmath}
\usepackage{amsfonts} 
\usepackage{geometry}
\usepackage{multicol}
\usepackage{float}
% \usepackage{mathtools}
% \usepackage{graphicx}
% \usepackage{soul}
% \usepackage{indentfirst}
\usepackage{tikz}
\usetikzlibrary{calc, automata, chains, arrows.meta, math}
\setcounter{MaxMatrixCols}{20}


\title{A game theoretic model of the behavioural gaming that takes place at the EMS - ED interface}

\author{
    Michalis Panayides, 
    Paul Harper, 
    Vince Knight
}

\begin{document}

\maketitle

\documentclass{article}

\usepackage{amsmath}
\usepackage{amsfonts} 
\usepackage{geometry}
\usepackage{multicol}
\usepackage{float}
% \usepackage{mathtools}
% \usepackage{graphicx}
% \usepackage{soul}
% \usepackage{indentfirst}
\usepackage{tikz}
\usetikzlibrary{calc, automata, chains, arrows.meta, math}
\setcounter{MaxMatrixCols}{20}


\title{A game theoretic model of the behavioural gaming that takes place at the EMS - ED interface}

\author{
    Michalis Panayides, 
    Paul Harper, 
    Vince Knight
}

\begin{document}

\maketitle

\input{Abstract/main.tex}


\newpage
\tableofcontents

\newpage
\input{Introduction/main.tex}

\newpage
\input{Game_theory_component/main.tex}

\newpage
\input{MarkovChain/markov_chain_model/main.tex}
\input{MarkovChain/expressions_from_pi/main.tex}
\input{MarkovChain/markov_example/main.tex}

\newpage
\input{BehaviouralMethodology/main.tex}

\newpage
\input{Application_EMS_ED/main.tex}

\newpage
\input{Conclusion/main.tex}


\end{document}


\newpage
\tableofcontents

\newpage
\documentclass{article}

\usepackage{amsmath}
\usepackage{amsfonts} 
\usepackage{geometry}
\usepackage{multicol}
\usepackage{float}
% \usepackage{mathtools}
% \usepackage{graphicx}
% \usepackage{soul}
% \usepackage{indentfirst}
\usepackage{tikz}
\usetikzlibrary{calc, automata, chains, arrows.meta, math}
\setcounter{MaxMatrixCols}{20}


\title{A game theoretic model of the behavioural gaming that takes place at the EMS - ED interface}

\author{
    Michalis Panayides, 
    Paul Harper, 
    Vince Knight
}

\begin{document}

\maketitle

\input{Abstract/main.tex}


\newpage
\tableofcontents

\newpage
\input{Introduction/main.tex}

\newpage
\input{Game_theory_component/main.tex}

\newpage
\input{MarkovChain/markov_chain_model/main.tex}
\input{MarkovChain/expressions_from_pi/main.tex}
\input{MarkovChain/markov_example/main.tex}

\newpage
\input{BehaviouralMethodology/main.tex}

\newpage
\input{Application_EMS_ED/main.tex}

\newpage
\input{Conclusion/main.tex}


\end{document}

\newpage
\documentclass{article}

\usepackage{amsmath}
\usepackage{amsfonts} 
\usepackage{geometry}
\usepackage{multicol}
\usepackage{float}
% \usepackage{mathtools}
% \usepackage{graphicx}
% \usepackage{soul}
% \usepackage{indentfirst}
\usepackage{tikz}
\usetikzlibrary{calc, automata, chains, arrows.meta, math}
\setcounter{MaxMatrixCols}{20}


\title{A game theoretic model of the behavioural gaming that takes place at the EMS - ED interface}

\author{
    Michalis Panayides, 
    Paul Harper, 
    Vince Knight
}

\begin{document}

\maketitle

\input{Abstract/main.tex}


\newpage
\tableofcontents

\newpage
\input{Introduction/main.tex}

\newpage
\input{Game_theory_component/main.tex}

\newpage
\input{MarkovChain/markov_chain_model/main.tex}
\input{MarkovChain/expressions_from_pi/main.tex}
\input{MarkovChain/markov_example/main.tex}

\newpage
\input{BehaviouralMethodology/main.tex}

\newpage
\input{Application_EMS_ED/main.tex}

\newpage
\input{Conclusion/main.tex}


\end{document}

\newpage
\documentclass{article}

\usepackage{amsmath}
\usepackage{amsfonts} 
\usepackage{geometry}
\usepackage{multicol}
\usepackage{float}
% \usepackage{mathtools}
% \usepackage{graphicx}
% \usepackage{soul}
% \usepackage{indentfirst}
\usepackage{tikz}
\usetikzlibrary{calc, automata, chains, arrows.meta, math}
\setcounter{MaxMatrixCols}{20}


\title{A game theoretic model of the behavioural gaming that takes place at the EMS - ED interface}

\author{
    Michalis Panayides, 
    Paul Harper, 
    Vince Knight
}

\begin{document}

\maketitle

\input{Abstract/main.tex}


\newpage
\tableofcontents

\newpage
\input{Introduction/main.tex}

\newpage
\input{Game_theory_component/main.tex}

\newpage
\input{MarkovChain/markov_chain_model/main.tex}
\input{MarkovChain/expressions_from_pi/main.tex}
\input{MarkovChain/markov_example/main.tex}

\newpage
\input{BehaviouralMethodology/main.tex}

\newpage
\input{Application_EMS_ED/main.tex}

\newpage
\input{Conclusion/main.tex}


\end{document}
\subsection{Performance Measures}
One may easily derive the average number of individuals that are at any given state 
using \( pi \). 
The average number of individuals in state \( i \) can be calculated by multiplying 
the number of individuals that are present in state \( i \) with the probability 
of being at that particular state (i.e \(\pi_i (u_i + v_i)\)). 
Using this logic it is possible to calculate any performance measures that are related 
to the mean number of individuals in the system.


Average number of people in the system: 
\begin{equation}
    L = \sum_{i=1}^{|\pi|} \pi_i (u_i + v_i)
\end{equation} 

Average number of people in the service centre: 
\begin{equation}
    L_H = \sum_{i=1}^{|\pi|} \pi_i v_i
\end{equation}

Average number of people in the buffer centre:
\begin{equation}
    L_A = \sum_{i=1}^{|\pi|} \pi_i u_i
\end{equation}

Consequently getting the performance measures that are related to the duration of 
time is not as straightforward. 
Such performance measures are the mean waiting time in the system and the mean time 
blocked in the system. 
Under the scope of this study three approaches have been considered to calculate these 
performance measures; a direct approach, a recursive algorithm and consequently a
closed-form formula.

The research question that needs to be answered here is: ``When a class 1/2 
individuals enters the system, what is the expected time that they will have to 
wait?''. 
In order to formulate the answer to that question one needs to consider all possible 
scenarios of what state the system can be in when an individual arrives. 
Furthermore, different formulas arises for class 1 individuals 
and a different one for class 2 individuals.

\subsubsection{Mean waiting time} 
Upon closer inspection of the recursive formula a more compact formula can arise. 
The equivalent closed-form formula eliminates the need for recursion and thus makes 
the computation of waiting times much more efficient. 
Just like in the recursive part there are two formulas; one for \textit{class 1} 
and one for class 2 individuals. 
The formulas are given by:

\begin{equation} \label{eq:closed_form_waiting_others}
    W^{(1)} = \frac{\sum_{\substack{(u,v) \, \in S_A^{(1)} \\ v \geq C}} 
    \frac{1}{C \mu} \times (v-C+1) \times \pi(u,v)}{\sum_{(u,v) \, 
    \in S_A^{(1)}} \pi(u,v)}
\end{equation}
    
\begin{equation}\label{eq:closed_form_waiting_ambulance}
    W^{(2)} = \frac{\sum_{\substack{(u,v) \, \in S_A^{(2)} \\ min(v,T) \geq C}} 
    \frac{1}{C \mu} \times (\min(v+1,T)-C) \times \pi(u,v)}{\sum_{(u,v) \, 
    \in S_A^{(2)}} \pi(u,v)}
\end{equation}

Note here that the summation, in both equations \ref{eq:closed_form_waiting_others} 
and \ref{eq:closed_form_waiting_ambulance}, goes through all states in the set of 
accepting 
states of either class 1 or class 2 individuals respectively, where a wait 
incurs. 
In equation \ref{eq:closed_form_waiting_others} that includes all states \((u,v)\) 
in the set of accepting states of class 1 individuals such that \( v \geq C\); i.e. 
whenever an arrival occurs and the system is at a state where the number of individuals 
in the system is more than or equal to $C$. 
Consequently, for the states that are included in the summation the expression 
\( v-C+1 \) indicates the amount of people in service one would have to wait for 
upon arrival at the hospital.

Additionally, the minimisation function in equation 
\ref{eq:closed_form_waiting_ambulance} 
ensures that when a class 2 individual arrives at any state 
that is greater than the predetermined threshold, the wait that the individual will 
have to endure remains the same. 
In essence, the expression \(\min(v+1,T) - C\) returns the number of people in line 
in front of a particular individual upon arrival.


\subsubsection{Overall Waiting Time}

Consequently, the overall waiting time should can be estimated by a linear combination 
of the waiting times of class 1 and class 2 individuals. 
The overall waiting time can be then given by the following equation where \(c_1\) 
and \(c_2\) are the coefficients of each individual's type waiting time:

\begin{equation}\label{overall_waiting_time_coeff}
    W = c_1 W^{(1)} + c_2 W^{(2)}
\end{equation}

The two coefficients represent the proportion of individuals of each type that 
traversed through the model. 
Theoretically, getting these percentages should be as simple as looking at the arrival 
rates of each type but in practise if the service centre or the buffer centre 
is full, some individuals may be lost to the system. 
Thus, one should account for the probability that an individual is lost to the system. 
This probability can be easily calculated by using the two sets of accepting states 
\(S_A^{(2)}\) and \(S_A^{(1)}\) defined earlier in equations.
Let us define here the probability, for either class type, that an individual 
is not lost in the system by:

\begin{equation*}
    P(L'_1) = \sum_{(u,v) \, \in S_A^{(1)}} \pi(u,v) \hspace{2cm}
    P(L'_2) = \sum_{(u,v) \, \in S_A^{(2)}} \pi(u,v)
\end{equation*}

Having defined these probabilities one may combine them with the arrival rates of 
each class type in such a way to get the expected proportions of class 1 and 
class 2 individuals in the model. 
Thus, by using these values as the coefficient of equation 
\ref{overall_waiting_time_coeff} 
the resultant equation can be used to get the overall waiting time. 
Note here that the equation below gets the overall waiting time for both the recursive 
and the closed-form formula.

\begin{equation}\label{overall_waiting_time}
    W = \frac{\lambda_1 P(L'_1)}{\lambda_2 P(L'_2) + \lambda_1 P(L'_1)} W^{(1)} + 
    \frac{\lambda_2 P(L'_2)}{\lambda_2 P(L'_2) + \lambda_1 P(L'_1)} W^{(2)}
\end{equation}



\subsubsection{Mean blocking time}
Unlike the waiting time, the blocking time is only calculated for class 2 individuals.  
That is because class 1 individuals cannot be blocked. 
Thus, one only needs to consider the pathway of class 2 individuals to get the 
mean blocking time of the system. 
Blocking occurs at states \((u,v)\) where \(u > 0 \). 
Thus, the set of blocking states can be defined as:

\begin{equation*}
    S_b = \{(u,v) \in S \; | \; u > 0\}
\end{equation*}
 
In order to not consider individuals that will be lost to the system, the set of 
accepting states needs to be taken into account. The set of accepting states is given by:

\begin{equation*}
    S_A^{(2)}=
    \begin{cases}
        \{(u, v) \in S \; | \; u < M \} & \textbf{if } T \leq N\\
        \{(u, v) \in S \; | \; v < N \} & \textbf{otherwise}
    \end{cases}
\end{equation*}

For the waiting time formula,
the mean sojourn time for each state was considered,
ignoring any arrivals. Here, the same approach is used but ignoring only class 2
arrivals. That is because for the waiting time formula, once an individual enters 
the service centre (i.e. starts waiting) any individual arriving after them will 
not affect their
pathway. That is not the case for blocking time. When a class 2 individual is 
blocked, 
any class 1 individual that arrives will cause the blocked individual to remain 
blocked for more time. Therefore, class 1 arrivals are considered here:

\begin{equation}\label{eq:time_in_state_blocking_time}
    c(u,v) = 
    \begin{cases}
        \frac{1}{\min(v,C) \mu}, & \text{if } v = C\\
        \frac{1}{\min(v,C) \mu + \lambda_1}, & \text{otherwise}
    \end{cases}
\end{equation}
 
In equation \ref{eq:time_in_state_blocking_time}, both service completions and 
class 1 arrivals are considered. 
Thus, from a blocked individual's perspective whenever the system moves from one 
state \((u,v)\)
to another state it can either:

\begin{itemize}
    \item be because of a service being completed: we will denote the probability 
    of this happening by \(p_s(u,v)\). 
    \item be because of an arrival of an individual of class 1: denoting such 
    probability by \(p_o(u,v)\).
\end{itemize}
The probabilities are given by:

\begin{equation*}
    p_s(u,v) = \frac{\min(v,C)\mu}{\lambda_1 + \min(v,C)\mu}, \qquad
    p_o(u,v) = \frac{\lambda_1}{\lambda_1 + \min(v,C)\mu}
\end{equation*}


Having defined \(c(u,v)\) and \(S_b\) a formula for the blocking time that is
expected to occur at each state can be given by:

\begin{equation}\label{eq:blocking-time-at-each-state}
    b(u,v) = 
    \begin{cases} 
        0, & \textbf{if } (u,v) \notin S_b \\
        c(u,v) + b(u - 1, v), & \textbf{if } v = N = T\\
        c(u,v) + b(u, v-1), & \textbf{if } v = N \neq T \\
        c(u,v) + p_s(u,v) b(u-1, v) + p_o(u,v) b(u, v+1), & \textbf{if } u > 0 
        \textbf{ and } v = T \\
        c(u,v) + p_s(u,v) b(u, v-1) + p_o(u,v) b(u, v+1), & \textbf{otherwise} \\
    \end{cases}
\end{equation}

Equation 
(\ref{eq:blocking-time-at-each-state}) will not be solved recursively. 
A direct approach will be used to solve this equation here. 
By enumerating all equations of (\ref{eq:blocking-time-at-each-state}) for all 
states \((u,v)\) that belong in \(S_b\) 
a system of linear equations arises where the unknown variables are all the \(b(u,v)\)
terms.
For instance, let us consider a Markov model where \(C=2, T=3, N=6, M=2\). 
The Markov model is shown in Figure \ref{fig:example-algeb-blocking}
and the equivalent equations are 
(\ref{eq:first_eq_of_blocking_example})-(\ref{eq:last_eq_of_blocking_example}).
The equations considered here are only the ones that correspond to the blocking 
states.

\begin{multicols*}{2}
    \begin{figure}[H]
        \scalebox{0.50}{\input{MarkovChain/expressions_from_pi/example_model_2362/main.tex}}
        \caption{Example of Markov chain}
        \label{fig:example-algeb-blocking}
    \end{figure}
    \columnbreak
    \begin{align}
        b(1,2) &= c(1,2) + p_o b(1,3) \label{eq:first_eq_of_blocking_example} \\
        b(1,3) &= c(1,3) + p_s b(1,2) + p_o b(1,4) \\
        b(1,4) &= c(1,4) + b(1,3) \\
        b(2,2) &= c(2,2) + p_s b(1,2) + p_o b(2,3) \\
        b(2,3) &= c(2,3) + p_s b(2,2) + p_o b(1,4) \\
        b(2,4) &= c(2,4) + b(2,3)\label{eq:last_eq_of_blocking_example}
    \end{align}
\end{multicols*}

Additionally, the above equations can be transformed into a linear system of the 
form \(Zx=y\) where:

\begin{equation}\label{eq:example-algebaric-approach-blocking-time}
    Z=
    \begin{pmatrix}
        -1 & p_o & 0 & 0 & 0 & 0 \\ %(1,2)
        p_s & -1 & p_o & 0 & 0 & 0 \\ %(1,3)
        0 & 1 & -1 & 0 & 0 & 0 \\ %(1,4)
        p_s & 0 & 0 & -1 & p_o & 0\\ %(2,2)
        0 & 0 & 0 & p_s & -1 & p_o \\ %(2,3)
        0 & 0 & 0 & 0 & 1 & -1 \\ %(2,4)
    \end{pmatrix},
    x=
    \begin{pmatrix}
        b(1,2) \\
        b(1,3) \\
        b(1,4) \\
        b(2,2) \\
        b(2,3) \\
        b(2,4) \\
    \end{pmatrix}, 
    y=
    \begin{pmatrix}
        -c(1,2) \\
        -c(1,3) \\
        -c(1,4) \\
        -c(2,2) \\
        -c(2,3) \\
        -c(2,4) \\
    \end{pmatrix}
\end{equation}

A more generalised form of the equations in 
(\ref{eq:example-algebaric-approach-blocking-time})
can thus be given for any value of \(C,T,N,M\) by:

\begin{align}
    b(1,T) =& c(1, T) + p_o b(1, T + 1) \label{eq:first_eq_of_blocking_general}\\
    b(1,T + 1) =& c(1, T + 1) + p_s(1, T) + p_o b(1, T + 1) \\
    b(1,T + 2) =& c(1, T + 2) + p_s(1, T + 1) + p_o b(1, T + 3) \\
    & \vdots \nonumber \\
    b(1, N) =& c(1, N) + b(1, N - 1) \\
    b(2, T) =& c(2, T) + p_s b(1, T) + p_o b(2, T + 1) \\
    b(2, T + 1) =& c(2, T + 1) + p_s b(2, T) + p_o b(2, T + 2) \\
    & \vdots \nonumber \\
    b(M, T) =& c(M, T) + b(M, T-1) \label{eq:last_eq_of_blocking_general}
\end{align}

The equivalent matrix form of the linear system of equations 
(\ref{eq:first_eq_of_blocking_general}) - (\ref{eq:last_eq_of_blocking_general})
is given by \(Zx=y\), where:
\begin{equation}\label{eq:general-algebaric-approach-blocking-time}
    \scalebox{0.9}{
        \(
        Z = 
        \begin{pmatrix}
            -1 & p_o & 0 & \dots & 0 & 0 & 0 & 0 & 0 & \dots & 0 & 0 \\ %(1,T)
            p_s & -1 & p_o & \dots & 0 & 0 & 0 & 0 & 0 & \dots & 0 & 0 \\ %(1,T+1)
            0 & p_s & -1 & \dots & 0 & 0 & 0 & 0 & 0 & \dots & 0 & 0 \\ %(1,T+2)
            \vdots & \vdots & \vdots & \ddots & \vdots & \vdots & \vdots & \vdots & 
            \vdots & \ddots & \vdots & \vdots \\ 
            0 & 0 & 0 & \dots & 1 & -1 & 0 & 0 & 0 & \dots & 0 & 0 \\ %(1,N)
            p_s & 0 & 0 & \dots & 0 & 0 & -1 & p_o & 0 & \dots & 0 & 0 \\ %(2,T)
            0 & 0 & 0 & \dots & 0 & 0 & p_s & -1 & p_o & \dots & 0 & 0 \\ %(2,T+1)
            \vdots & \vdots & \vdots & \ddots & \vdots & \vdots & \vdots & \vdots & 
            \vdots & \ddots & \vdots & \vdots \\ 
            0 & 0 & 0 & \dots & 0 & 0 & 0 & 0 & 0 & \dots & 1 & -1 \\ %(M,T)
        \end{pmatrix},
        x = 
        \begin{pmatrix}
            b(1,T) \\
            b(1,T+1) \\
            b(1,T+2) \\
            \vdots \\
            b(1,N) \\
            b(2,T) \\
            b(2,T+1) \\
            \vdots \\
            b(M,T) \\
        \end{pmatrix}, 
        y= 
        \begin{pmatrix}
            -c(1,T) \\
            -c(1,T+1) \\
            -c(1,T+2) \\
            \vdots \\
            -c(1,N) \\
            -c(2,T) \\
            -c(2,T+1) \\
            \vdots \\
            -c(M,T) \\
        \end{pmatrix}
        \)
    }
\end{equation}

Thus, having calculated the mean blocking time for all blocking states \(b(u,v)\), 
it only remains to put them together in a formula.
The resultant blocking time formula is given by:

\begin{equation}\label{eq:algebraic-blocking-time}
    B = \frac{\sum_{(u,v) \in S_A} \pi_{(u,v)} \; b(u,v)}{\sum_{(u,v) \in S_A} 
    \pi_{(u,v)}}
\end{equation}

\documentclass{article}

\usepackage{amsmath}
\usepackage{amsfonts} 
\usepackage{geometry}
\usepackage{multicol}
\usepackage{float}
% \usepackage{mathtools}
% \usepackage{graphicx}
% \usepackage{soul}
% \usepackage{indentfirst}
\usepackage{tikz}
\usetikzlibrary{calc, automata, chains, arrows.meta, math}
\setcounter{MaxMatrixCols}{20}


\title{A game theoretic model of the behavioural gaming that takes place at the EMS - ED interface}

\author{
    Michalis Panayides, 
    Paul Harper, 
    Vince Knight
}

\begin{document}

\maketitle

\input{Abstract/main.tex}


\newpage
\tableofcontents

\newpage
\input{Introduction/main.tex}

\newpage
\input{Game_theory_component/main.tex}

\newpage
\input{MarkovChain/markov_chain_model/main.tex}
\input{MarkovChain/expressions_from_pi/main.tex}
\input{MarkovChain/markov_example/main.tex}

\newpage
\input{BehaviouralMethodology/main.tex}

\newpage
\input{Application_EMS_ED/main.tex}

\newpage
\input{Conclusion/main.tex}


\end{document}

\newpage
\documentclass{article}

\usepackage{amsmath}
\usepackage{amsfonts} 
\usepackage{geometry}
\usepackage{multicol}
\usepackage{float}
% \usepackage{mathtools}
% \usepackage{graphicx}
% \usepackage{soul}
% \usepackage{indentfirst}
\usepackage{tikz}
\usetikzlibrary{calc, automata, chains, arrows.meta, math}
\setcounter{MaxMatrixCols}{20}


\title{A game theoretic model of the behavioural gaming that takes place at the EMS - ED interface}

\author{
    Michalis Panayides, 
    Paul Harper, 
    Vince Knight
}

\begin{document}

\maketitle

\input{Abstract/main.tex}


\newpage
\tableofcontents

\newpage
\input{Introduction/main.tex}

\newpage
\input{Game_theory_component/main.tex}

\newpage
\input{MarkovChain/markov_chain_model/main.tex}
\input{MarkovChain/expressions_from_pi/main.tex}
\input{MarkovChain/markov_example/main.tex}

\newpage
\input{BehaviouralMethodology/main.tex}

\newpage
\input{Application_EMS_ED/main.tex}

\newpage
\input{Conclusion/main.tex}


\end{document}

\newpage
\section{EMS-ED application}

\subsection{Application}

\subsection{Data analysis of generated problem}

\newpage
\documentclass{article}

\usepackage{amsmath}
\usepackage{amsfonts} 
\usepackage{geometry}
\usepackage{multicol}
\usepackage{float}
% \usepackage{mathtools}
% \usepackage{graphicx}
% \usepackage{soul}
% \usepackage{indentfirst}
\usepackage{tikz}
\usetikzlibrary{calc, automata, chains, arrows.meta, math}
\setcounter{MaxMatrixCols}{20}


\title{A game theoretic model of the behavioural gaming that takes place at the EMS - ED interface}

\author{
    Michalis Panayides, 
    Paul Harper, 
    Vince Knight
}

\begin{document}

\maketitle

\input{Abstract/main.tex}


\newpage
\tableofcontents

\newpage
\input{Introduction/main.tex}

\newpage
\input{Game_theory_component/main.tex}

\newpage
\input{MarkovChain/markov_chain_model/main.tex}
\input{MarkovChain/expressions_from_pi/main.tex}
\input{MarkovChain/markov_example/main.tex}

\newpage
\input{BehaviouralMethodology/main.tex}

\newpage
\input{Application_EMS_ED/main.tex}

\newpage
\input{Conclusion/main.tex}


\end{document}


\end{document}

\newpage
\documentclass{article}

\usepackage{amsmath}
\usepackage{amsfonts} 
\usepackage{geometry}
\usepackage{multicol}
\usepackage{float}
% \usepackage{mathtools}
% \usepackage{graphicx}
% \usepackage{soul}
% \usepackage{indentfirst}
\usepackage{tikz}
\usetikzlibrary{calc, automata, chains, arrows.meta, math}
\setcounter{MaxMatrixCols}{20}


\title{A game theoretic model of the behavioural gaming that takes place at the EMS - ED interface}

\author{
    Michalis Panayides, 
    Paul Harper, 
    Vince Knight
}

\begin{document}

\maketitle

\documentclass{article}

\usepackage{amsmath}
\usepackage{amsfonts} 
\usepackage{geometry}
\usepackage{multicol}
\usepackage{float}
% \usepackage{mathtools}
% \usepackage{graphicx}
% \usepackage{soul}
% \usepackage{indentfirst}
\usepackage{tikz}
\usetikzlibrary{calc, automata, chains, arrows.meta, math}
\setcounter{MaxMatrixCols}{20}


\title{A game theoretic model of the behavioural gaming that takes place at the EMS - ED interface}

\author{
    Michalis Panayides, 
    Paul Harper, 
    Vince Knight
}

\begin{document}

\maketitle

\input{Abstract/main.tex}


\newpage
\tableofcontents

\newpage
\input{Introduction/main.tex}

\newpage
\input{Game_theory_component/main.tex}

\newpage
\input{MarkovChain/markov_chain_model/main.tex}
\input{MarkovChain/expressions_from_pi/main.tex}
\input{MarkovChain/markov_example/main.tex}

\newpage
\input{BehaviouralMethodology/main.tex}

\newpage
\input{Application_EMS_ED/main.tex}

\newpage
\input{Conclusion/main.tex}


\end{document}


\newpage
\tableofcontents

\newpage
\documentclass{article}

\usepackage{amsmath}
\usepackage{amsfonts} 
\usepackage{geometry}
\usepackage{multicol}
\usepackage{float}
% \usepackage{mathtools}
% \usepackage{graphicx}
% \usepackage{soul}
% \usepackage{indentfirst}
\usepackage{tikz}
\usetikzlibrary{calc, automata, chains, arrows.meta, math}
\setcounter{MaxMatrixCols}{20}


\title{A game theoretic model of the behavioural gaming that takes place at the EMS - ED interface}

\author{
    Michalis Panayides, 
    Paul Harper, 
    Vince Knight
}

\begin{document}

\maketitle

\input{Abstract/main.tex}


\newpage
\tableofcontents

\newpage
\input{Introduction/main.tex}

\newpage
\input{Game_theory_component/main.tex}

\newpage
\input{MarkovChain/markov_chain_model/main.tex}
\input{MarkovChain/expressions_from_pi/main.tex}
\input{MarkovChain/markov_example/main.tex}

\newpage
\input{BehaviouralMethodology/main.tex}

\newpage
\input{Application_EMS_ED/main.tex}

\newpage
\input{Conclusion/main.tex}


\end{document}

\newpage
\documentclass{article}

\usepackage{amsmath}
\usepackage{amsfonts} 
\usepackage{geometry}
\usepackage{multicol}
\usepackage{float}
% \usepackage{mathtools}
% \usepackage{graphicx}
% \usepackage{soul}
% \usepackage{indentfirst}
\usepackage{tikz}
\usetikzlibrary{calc, automata, chains, arrows.meta, math}
\setcounter{MaxMatrixCols}{20}


\title{A game theoretic model of the behavioural gaming that takes place at the EMS - ED interface}

\author{
    Michalis Panayides, 
    Paul Harper, 
    Vince Knight
}

\begin{document}

\maketitle

\input{Abstract/main.tex}


\newpage
\tableofcontents

\newpage
\input{Introduction/main.tex}

\newpage
\input{Game_theory_component/main.tex}

\newpage
\input{MarkovChain/markov_chain_model/main.tex}
\input{MarkovChain/expressions_from_pi/main.tex}
\input{MarkovChain/markov_example/main.tex}

\newpage
\input{BehaviouralMethodology/main.tex}

\newpage
\input{Application_EMS_ED/main.tex}

\newpage
\input{Conclusion/main.tex}


\end{document}

\newpage
\documentclass{article}

\usepackage{amsmath}
\usepackage{amsfonts} 
\usepackage{geometry}
\usepackage{multicol}
\usepackage{float}
% \usepackage{mathtools}
% \usepackage{graphicx}
% \usepackage{soul}
% \usepackage{indentfirst}
\usepackage{tikz}
\usetikzlibrary{calc, automata, chains, arrows.meta, math}
\setcounter{MaxMatrixCols}{20}


\title{A game theoretic model of the behavioural gaming that takes place at the EMS - ED interface}

\author{
    Michalis Panayides, 
    Paul Harper, 
    Vince Knight
}

\begin{document}

\maketitle

\input{Abstract/main.tex}


\newpage
\tableofcontents

\newpage
\input{Introduction/main.tex}

\newpage
\input{Game_theory_component/main.tex}

\newpage
\input{MarkovChain/markov_chain_model/main.tex}
\input{MarkovChain/expressions_from_pi/main.tex}
\input{MarkovChain/markov_example/main.tex}

\newpage
\input{BehaviouralMethodology/main.tex}

\newpage
\input{Application_EMS_ED/main.tex}

\newpage
\input{Conclusion/main.tex}


\end{document}
\subsection{Performance Measures}
One may easily derive the average number of individuals that are at any given state 
using \( pi \). 
The average number of individuals in state \( i \) can be calculated by multiplying 
the number of individuals that are present in state \( i \) with the probability 
of being at that particular state (i.e \(\pi_i (u_i + v_i)\)). 
Using this logic it is possible to calculate any performance measures that are related 
to the mean number of individuals in the system.


Average number of people in the system: 
\begin{equation}
    L = \sum_{i=1}^{|\pi|} \pi_i (u_i + v_i)
\end{equation} 

Average number of people in the service centre: 
\begin{equation}
    L_H = \sum_{i=1}^{|\pi|} \pi_i v_i
\end{equation}

Average number of people in the buffer centre:
\begin{equation}
    L_A = \sum_{i=1}^{|\pi|} \pi_i u_i
\end{equation}

Consequently getting the performance measures that are related to the duration of 
time is not as straightforward. 
Such performance measures are the mean waiting time in the system and the mean time 
blocked in the system. 
Under the scope of this study three approaches have been considered to calculate these 
performance measures; a direct approach, a recursive algorithm and consequently a
closed-form formula.

The research question that needs to be answered here is: ``When a class 1/2 
individuals enters the system, what is the expected time that they will have to 
wait?''. 
In order to formulate the answer to that question one needs to consider all possible 
scenarios of what state the system can be in when an individual arrives. 
Furthermore, different formulas arises for class 1 individuals 
and a different one for class 2 individuals.

\subsubsection{Mean waiting time} 
Upon closer inspection of the recursive formula a more compact formula can arise. 
The equivalent closed-form formula eliminates the need for recursion and thus makes 
the computation of waiting times much more efficient. 
Just like in the recursive part there are two formulas; one for \textit{class 1} 
and one for class 2 individuals. 
The formulas are given by:

\begin{equation} \label{eq:closed_form_waiting_others}
    W^{(1)} = \frac{\sum_{\substack{(u,v) \, \in S_A^{(1)} \\ v \geq C}} 
    \frac{1}{C \mu} \times (v-C+1) \times \pi(u,v)}{\sum_{(u,v) \, 
    \in S_A^{(1)}} \pi(u,v)}
\end{equation}
    
\begin{equation}\label{eq:closed_form_waiting_ambulance}
    W^{(2)} = \frac{\sum_{\substack{(u,v) \, \in S_A^{(2)} \\ min(v,T) \geq C}} 
    \frac{1}{C \mu} \times (\min(v+1,T)-C) \times \pi(u,v)}{\sum_{(u,v) \, 
    \in S_A^{(2)}} \pi(u,v)}
\end{equation}

Note here that the summation, in both equations \ref{eq:closed_form_waiting_others} 
and \ref{eq:closed_form_waiting_ambulance}, goes through all states in the set of 
accepting 
states of either class 1 or class 2 individuals respectively, where a wait 
incurs. 
In equation \ref{eq:closed_form_waiting_others} that includes all states \((u,v)\) 
in the set of accepting states of class 1 individuals such that \( v \geq C\); i.e. 
whenever an arrival occurs and the system is at a state where the number of individuals 
in the system is more than or equal to $C$. 
Consequently, for the states that are included in the summation the expression 
\( v-C+1 \) indicates the amount of people in service one would have to wait for 
upon arrival at the hospital.

Additionally, the minimisation function in equation 
\ref{eq:closed_form_waiting_ambulance} 
ensures that when a class 2 individual arrives at any state 
that is greater than the predetermined threshold, the wait that the individual will 
have to endure remains the same. 
In essence, the expression \(\min(v+1,T) - C\) returns the number of people in line 
in front of a particular individual upon arrival.


\subsubsection{Overall Waiting Time}

Consequently, the overall waiting time should can be estimated by a linear combination 
of the waiting times of class 1 and class 2 individuals. 
The overall waiting time can be then given by the following equation where \(c_1\) 
and \(c_2\) are the coefficients of each individual's type waiting time:

\begin{equation}\label{overall_waiting_time_coeff}
    W = c_1 W^{(1)} + c_2 W^{(2)}
\end{equation}

The two coefficients represent the proportion of individuals of each type that 
traversed through the model. 
Theoretically, getting these percentages should be as simple as looking at the arrival 
rates of each type but in practise if the service centre or the buffer centre 
is full, some individuals may be lost to the system. 
Thus, one should account for the probability that an individual is lost to the system. 
This probability can be easily calculated by using the two sets of accepting states 
\(S_A^{(2)}\) and \(S_A^{(1)}\) defined earlier in equations.
Let us define here the probability, for either class type, that an individual 
is not lost in the system by:

\begin{equation*}
    P(L'_1) = \sum_{(u,v) \, \in S_A^{(1)}} \pi(u,v) \hspace{2cm}
    P(L'_2) = \sum_{(u,v) \, \in S_A^{(2)}} \pi(u,v)
\end{equation*}

Having defined these probabilities one may combine them with the arrival rates of 
each class type in such a way to get the expected proportions of class 1 and 
class 2 individuals in the model. 
Thus, by using these values as the coefficient of equation 
\ref{overall_waiting_time_coeff} 
the resultant equation can be used to get the overall waiting time. 
Note here that the equation below gets the overall waiting time for both the recursive 
and the closed-form formula.

\begin{equation}\label{overall_waiting_time}
    W = \frac{\lambda_1 P(L'_1)}{\lambda_2 P(L'_2) + \lambda_1 P(L'_1)} W^{(1)} + 
    \frac{\lambda_2 P(L'_2)}{\lambda_2 P(L'_2) + \lambda_1 P(L'_1)} W^{(2)}
\end{equation}



\subsubsection{Mean blocking time}
Unlike the waiting time, the blocking time is only calculated for class 2 individuals.  
That is because class 1 individuals cannot be blocked. 
Thus, one only needs to consider the pathway of class 2 individuals to get the 
mean blocking time of the system. 
Blocking occurs at states \((u,v)\) where \(u > 0 \). 
Thus, the set of blocking states can be defined as:

\begin{equation*}
    S_b = \{(u,v) \in S \; | \; u > 0\}
\end{equation*}
 
In order to not consider individuals that will be lost to the system, the set of 
accepting states needs to be taken into account. The set of accepting states is given by:

\begin{equation*}
    S_A^{(2)}=
    \begin{cases}
        \{(u, v) \in S \; | \; u < M \} & \textbf{if } T \leq N\\
        \{(u, v) \in S \; | \; v < N \} & \textbf{otherwise}
    \end{cases}
\end{equation*}

For the waiting time formula,
the mean sojourn time for each state was considered,
ignoring any arrivals. Here, the same approach is used but ignoring only class 2
arrivals. That is because for the waiting time formula, once an individual enters 
the service centre (i.e. starts waiting) any individual arriving after them will 
not affect their
pathway. That is not the case for blocking time. When a class 2 individual is 
blocked, 
any class 1 individual that arrives will cause the blocked individual to remain 
blocked for more time. Therefore, class 1 arrivals are considered here:

\begin{equation}\label{eq:time_in_state_blocking_time}
    c(u,v) = 
    \begin{cases}
        \frac{1}{\min(v,C) \mu}, & \text{if } v = C\\
        \frac{1}{\min(v,C) \mu + \lambda_1}, & \text{otherwise}
    \end{cases}
\end{equation}
 
In equation \ref{eq:time_in_state_blocking_time}, both service completions and 
class 1 arrivals are considered. 
Thus, from a blocked individual's perspective whenever the system moves from one 
state \((u,v)\)
to another state it can either:

\begin{itemize}
    \item be because of a service being completed: we will denote the probability 
    of this happening by \(p_s(u,v)\). 
    \item be because of an arrival of an individual of class 1: denoting such 
    probability by \(p_o(u,v)\).
\end{itemize}
The probabilities are given by:

\begin{equation*}
    p_s(u,v) = \frac{\min(v,C)\mu}{\lambda_1 + \min(v,C)\mu}, \qquad
    p_o(u,v) = \frac{\lambda_1}{\lambda_1 + \min(v,C)\mu}
\end{equation*}


Having defined \(c(u,v)\) and \(S_b\) a formula for the blocking time that is
expected to occur at each state can be given by:

\begin{equation}\label{eq:blocking-time-at-each-state}
    b(u,v) = 
    \begin{cases} 
        0, & \textbf{if } (u,v) \notin S_b \\
        c(u,v) + b(u - 1, v), & \textbf{if } v = N = T\\
        c(u,v) + b(u, v-1), & \textbf{if } v = N \neq T \\
        c(u,v) + p_s(u,v) b(u-1, v) + p_o(u,v) b(u, v+1), & \textbf{if } u > 0 
        \textbf{ and } v = T \\
        c(u,v) + p_s(u,v) b(u, v-1) + p_o(u,v) b(u, v+1), & \textbf{otherwise} \\
    \end{cases}
\end{equation}

Equation 
(\ref{eq:blocking-time-at-each-state}) will not be solved recursively. 
A direct approach will be used to solve this equation here. 
By enumerating all equations of (\ref{eq:blocking-time-at-each-state}) for all 
states \((u,v)\) that belong in \(S_b\) 
a system of linear equations arises where the unknown variables are all the \(b(u,v)\)
terms.
For instance, let us consider a Markov model where \(C=2, T=3, N=6, M=2\). 
The Markov model is shown in Figure \ref{fig:example-algeb-blocking}
and the equivalent equations are 
(\ref{eq:first_eq_of_blocking_example})-(\ref{eq:last_eq_of_blocking_example}).
The equations considered here are only the ones that correspond to the blocking 
states.

\begin{multicols*}{2}
    \begin{figure}[H]
        \scalebox{0.50}{\input{MarkovChain/expressions_from_pi/example_model_2362/main.tex}}
        \caption{Example of Markov chain}
        \label{fig:example-algeb-blocking}
    \end{figure}
    \columnbreak
    \begin{align}
        b(1,2) &= c(1,2) + p_o b(1,3) \label{eq:first_eq_of_blocking_example} \\
        b(1,3) &= c(1,3) + p_s b(1,2) + p_o b(1,4) \\
        b(1,4) &= c(1,4) + b(1,3) \\
        b(2,2) &= c(2,2) + p_s b(1,2) + p_o b(2,3) \\
        b(2,3) &= c(2,3) + p_s b(2,2) + p_o b(1,4) \\
        b(2,4) &= c(2,4) + b(2,3)\label{eq:last_eq_of_blocking_example}
    \end{align}
\end{multicols*}

Additionally, the above equations can be transformed into a linear system of the 
form \(Zx=y\) where:

\begin{equation}\label{eq:example-algebaric-approach-blocking-time}
    Z=
    \begin{pmatrix}
        -1 & p_o & 0 & 0 & 0 & 0 \\ %(1,2)
        p_s & -1 & p_o & 0 & 0 & 0 \\ %(1,3)
        0 & 1 & -1 & 0 & 0 & 0 \\ %(1,4)
        p_s & 0 & 0 & -1 & p_o & 0\\ %(2,2)
        0 & 0 & 0 & p_s & -1 & p_o \\ %(2,3)
        0 & 0 & 0 & 0 & 1 & -1 \\ %(2,4)
    \end{pmatrix},
    x=
    \begin{pmatrix}
        b(1,2) \\
        b(1,3) \\
        b(1,4) \\
        b(2,2) \\
        b(2,3) \\
        b(2,4) \\
    \end{pmatrix}, 
    y=
    \begin{pmatrix}
        -c(1,2) \\
        -c(1,3) \\
        -c(1,4) \\
        -c(2,2) \\
        -c(2,3) \\
        -c(2,4) \\
    \end{pmatrix}
\end{equation}

A more generalised form of the equations in 
(\ref{eq:example-algebaric-approach-blocking-time})
can thus be given for any value of \(C,T,N,M\) by:

\begin{align}
    b(1,T) =& c(1, T) + p_o b(1, T + 1) \label{eq:first_eq_of_blocking_general}\\
    b(1,T + 1) =& c(1, T + 1) + p_s(1, T) + p_o b(1, T + 1) \\
    b(1,T + 2) =& c(1, T + 2) + p_s(1, T + 1) + p_o b(1, T + 3) \\
    & \vdots \nonumber \\
    b(1, N) =& c(1, N) + b(1, N - 1) \\
    b(2, T) =& c(2, T) + p_s b(1, T) + p_o b(2, T + 1) \\
    b(2, T + 1) =& c(2, T + 1) + p_s b(2, T) + p_o b(2, T + 2) \\
    & \vdots \nonumber \\
    b(M, T) =& c(M, T) + b(M, T-1) \label{eq:last_eq_of_blocking_general}
\end{align}

The equivalent matrix form of the linear system of equations 
(\ref{eq:first_eq_of_blocking_general}) - (\ref{eq:last_eq_of_blocking_general})
is given by \(Zx=y\), where:
\begin{equation}\label{eq:general-algebaric-approach-blocking-time}
    \scalebox{0.9}{
        \(
        Z = 
        \begin{pmatrix}
            -1 & p_o & 0 & \dots & 0 & 0 & 0 & 0 & 0 & \dots & 0 & 0 \\ %(1,T)
            p_s & -1 & p_o & \dots & 0 & 0 & 0 & 0 & 0 & \dots & 0 & 0 \\ %(1,T+1)
            0 & p_s & -1 & \dots & 0 & 0 & 0 & 0 & 0 & \dots & 0 & 0 \\ %(1,T+2)
            \vdots & \vdots & \vdots & \ddots & \vdots & \vdots & \vdots & \vdots & 
            \vdots & \ddots & \vdots & \vdots \\ 
            0 & 0 & 0 & \dots & 1 & -1 & 0 & 0 & 0 & \dots & 0 & 0 \\ %(1,N)
            p_s & 0 & 0 & \dots & 0 & 0 & -1 & p_o & 0 & \dots & 0 & 0 \\ %(2,T)
            0 & 0 & 0 & \dots & 0 & 0 & p_s & -1 & p_o & \dots & 0 & 0 \\ %(2,T+1)
            \vdots & \vdots & \vdots & \ddots & \vdots & \vdots & \vdots & \vdots & 
            \vdots & \ddots & \vdots & \vdots \\ 
            0 & 0 & 0 & \dots & 0 & 0 & 0 & 0 & 0 & \dots & 1 & -1 \\ %(M,T)
        \end{pmatrix},
        x = 
        \begin{pmatrix}
            b(1,T) \\
            b(1,T+1) \\
            b(1,T+2) \\
            \vdots \\
            b(1,N) \\
            b(2,T) \\
            b(2,T+1) \\
            \vdots \\
            b(M,T) \\
        \end{pmatrix}, 
        y= 
        \begin{pmatrix}
            -c(1,T) \\
            -c(1,T+1) \\
            -c(1,T+2) \\
            \vdots \\
            -c(1,N) \\
            -c(2,T) \\
            -c(2,T+1) \\
            \vdots \\
            -c(M,T) \\
        \end{pmatrix}
        \)
    }
\end{equation}

Thus, having calculated the mean blocking time for all blocking states \(b(u,v)\), 
it only remains to put them together in a formula.
The resultant blocking time formula is given by:

\begin{equation}\label{eq:algebraic-blocking-time}
    B = \frac{\sum_{(u,v) \in S_A} \pi_{(u,v)} \; b(u,v)}{\sum_{(u,v) \in S_A} 
    \pi_{(u,v)}}
\end{equation}

\documentclass{article}

\usepackage{amsmath}
\usepackage{amsfonts} 
\usepackage{geometry}
\usepackage{multicol}
\usepackage{float}
% \usepackage{mathtools}
% \usepackage{graphicx}
% \usepackage{soul}
% \usepackage{indentfirst}
\usepackage{tikz}
\usetikzlibrary{calc, automata, chains, arrows.meta, math}
\setcounter{MaxMatrixCols}{20}


\title{A game theoretic model of the behavioural gaming that takes place at the EMS - ED interface}

\author{
    Michalis Panayides, 
    Paul Harper, 
    Vince Knight
}

\begin{document}

\maketitle

\input{Abstract/main.tex}


\newpage
\tableofcontents

\newpage
\input{Introduction/main.tex}

\newpage
\input{Game_theory_component/main.tex}

\newpage
\input{MarkovChain/markov_chain_model/main.tex}
\input{MarkovChain/expressions_from_pi/main.tex}
\input{MarkovChain/markov_example/main.tex}

\newpage
\input{BehaviouralMethodology/main.tex}

\newpage
\input{Application_EMS_ED/main.tex}

\newpage
\input{Conclusion/main.tex}


\end{document}

\newpage
\documentclass{article}

\usepackage{amsmath}
\usepackage{amsfonts} 
\usepackage{geometry}
\usepackage{multicol}
\usepackage{float}
% \usepackage{mathtools}
% \usepackage{graphicx}
% \usepackage{soul}
% \usepackage{indentfirst}
\usepackage{tikz}
\usetikzlibrary{calc, automata, chains, arrows.meta, math}
\setcounter{MaxMatrixCols}{20}


\title{A game theoretic model of the behavioural gaming that takes place at the EMS - ED interface}

\author{
    Michalis Panayides, 
    Paul Harper, 
    Vince Knight
}

\begin{document}

\maketitle

\input{Abstract/main.tex}


\newpage
\tableofcontents

\newpage
\input{Introduction/main.tex}

\newpage
\input{Game_theory_component/main.tex}

\newpage
\input{MarkovChain/markov_chain_model/main.tex}
\input{MarkovChain/expressions_from_pi/main.tex}
\input{MarkovChain/markov_example/main.tex}

\newpage
\input{BehaviouralMethodology/main.tex}

\newpage
\input{Application_EMS_ED/main.tex}

\newpage
\input{Conclusion/main.tex}


\end{document}

\newpage
\section{EMS-ED application}

\subsection{Application}

\subsection{Data analysis of generated problem}

\newpage
\documentclass{article}

\usepackage{amsmath}
\usepackage{amsfonts} 
\usepackage{geometry}
\usepackage{multicol}
\usepackage{float}
% \usepackage{mathtools}
% \usepackage{graphicx}
% \usepackage{soul}
% \usepackage{indentfirst}
\usepackage{tikz}
\usetikzlibrary{calc, automata, chains, arrows.meta, math}
\setcounter{MaxMatrixCols}{20}


\title{A game theoretic model of the behavioural gaming that takes place at the EMS - ED interface}

\author{
    Michalis Panayides, 
    Paul Harper, 
    Vince Knight
}

\begin{document}

\maketitle

\input{Abstract/main.tex}


\newpage
\tableofcontents

\newpage
\input{Introduction/main.tex}

\newpage
\input{Game_theory_component/main.tex}

\newpage
\input{MarkovChain/markov_chain_model/main.tex}
\input{MarkovChain/expressions_from_pi/main.tex}
\input{MarkovChain/markov_example/main.tex}

\newpage
\input{BehaviouralMethodology/main.tex}

\newpage
\input{Application_EMS_ED/main.tex}

\newpage
\input{Conclusion/main.tex}


\end{document}


\end{document}

\newpage
\documentclass{article}

\usepackage{amsmath}
\usepackage{amsfonts} 
\usepackage{geometry}
\usepackage{multicol}
\usepackage{float}
% \usepackage{mathtools}
% \usepackage{graphicx}
% \usepackage{soul}
% \usepackage{indentfirst}
\usepackage{tikz}
\usetikzlibrary{calc, automata, chains, arrows.meta, math}
\setcounter{MaxMatrixCols}{20}


\title{A game theoretic model of the behavioural gaming that takes place at the EMS - ED interface}

\author{
    Michalis Panayides, 
    Paul Harper, 
    Vince Knight
}

\begin{document}

\maketitle

\documentclass{article}

\usepackage{amsmath}
\usepackage{amsfonts} 
\usepackage{geometry}
\usepackage{multicol}
\usepackage{float}
% \usepackage{mathtools}
% \usepackage{graphicx}
% \usepackage{soul}
% \usepackage{indentfirst}
\usepackage{tikz}
\usetikzlibrary{calc, automata, chains, arrows.meta, math}
\setcounter{MaxMatrixCols}{20}


\title{A game theoretic model of the behavioural gaming that takes place at the EMS - ED interface}

\author{
    Michalis Panayides, 
    Paul Harper, 
    Vince Knight
}

\begin{document}

\maketitle

\input{Abstract/main.tex}


\newpage
\tableofcontents

\newpage
\input{Introduction/main.tex}

\newpage
\input{Game_theory_component/main.tex}

\newpage
\input{MarkovChain/markov_chain_model/main.tex}
\input{MarkovChain/expressions_from_pi/main.tex}
\input{MarkovChain/markov_example/main.tex}

\newpage
\input{BehaviouralMethodology/main.tex}

\newpage
\input{Application_EMS_ED/main.tex}

\newpage
\input{Conclusion/main.tex}


\end{document}


\newpage
\tableofcontents

\newpage
\documentclass{article}

\usepackage{amsmath}
\usepackage{amsfonts} 
\usepackage{geometry}
\usepackage{multicol}
\usepackage{float}
% \usepackage{mathtools}
% \usepackage{graphicx}
% \usepackage{soul}
% \usepackage{indentfirst}
\usepackage{tikz}
\usetikzlibrary{calc, automata, chains, arrows.meta, math}
\setcounter{MaxMatrixCols}{20}


\title{A game theoretic model of the behavioural gaming that takes place at the EMS - ED interface}

\author{
    Michalis Panayides, 
    Paul Harper, 
    Vince Knight
}

\begin{document}

\maketitle

\input{Abstract/main.tex}


\newpage
\tableofcontents

\newpage
\input{Introduction/main.tex}

\newpage
\input{Game_theory_component/main.tex}

\newpage
\input{MarkovChain/markov_chain_model/main.tex}
\input{MarkovChain/expressions_from_pi/main.tex}
\input{MarkovChain/markov_example/main.tex}

\newpage
\input{BehaviouralMethodology/main.tex}

\newpage
\input{Application_EMS_ED/main.tex}

\newpage
\input{Conclusion/main.tex}


\end{document}

\newpage
\documentclass{article}

\usepackage{amsmath}
\usepackage{amsfonts} 
\usepackage{geometry}
\usepackage{multicol}
\usepackage{float}
% \usepackage{mathtools}
% \usepackage{graphicx}
% \usepackage{soul}
% \usepackage{indentfirst}
\usepackage{tikz}
\usetikzlibrary{calc, automata, chains, arrows.meta, math}
\setcounter{MaxMatrixCols}{20}


\title{A game theoretic model of the behavioural gaming that takes place at the EMS - ED interface}

\author{
    Michalis Panayides, 
    Paul Harper, 
    Vince Knight
}

\begin{document}

\maketitle

\input{Abstract/main.tex}


\newpage
\tableofcontents

\newpage
\input{Introduction/main.tex}

\newpage
\input{Game_theory_component/main.tex}

\newpage
\input{MarkovChain/markov_chain_model/main.tex}
\input{MarkovChain/expressions_from_pi/main.tex}
\input{MarkovChain/markov_example/main.tex}

\newpage
\input{BehaviouralMethodology/main.tex}

\newpage
\input{Application_EMS_ED/main.tex}

\newpage
\input{Conclusion/main.tex}


\end{document}

\newpage
\documentclass{article}

\usepackage{amsmath}
\usepackage{amsfonts} 
\usepackage{geometry}
\usepackage{multicol}
\usepackage{float}
% \usepackage{mathtools}
% \usepackage{graphicx}
% \usepackage{soul}
% \usepackage{indentfirst}
\usepackage{tikz}
\usetikzlibrary{calc, automata, chains, arrows.meta, math}
\setcounter{MaxMatrixCols}{20}


\title{A game theoretic model of the behavioural gaming that takes place at the EMS - ED interface}

\author{
    Michalis Panayides, 
    Paul Harper, 
    Vince Knight
}

\begin{document}

\maketitle

\input{Abstract/main.tex}


\newpage
\tableofcontents

\newpage
\input{Introduction/main.tex}

\newpage
\input{Game_theory_component/main.tex}

\newpage
\input{MarkovChain/markov_chain_model/main.tex}
\input{MarkovChain/expressions_from_pi/main.tex}
\input{MarkovChain/markov_example/main.tex}

\newpage
\input{BehaviouralMethodology/main.tex}

\newpage
\input{Application_EMS_ED/main.tex}

\newpage
\input{Conclusion/main.tex}


\end{document}
\subsection{Performance Measures}
One may easily derive the average number of individuals that are at any given state 
using \( pi \). 
The average number of individuals in state \( i \) can be calculated by multiplying 
the number of individuals that are present in state \( i \) with the probability 
of being at that particular state (i.e \(\pi_i (u_i + v_i)\)). 
Using this logic it is possible to calculate any performance measures that are related 
to the mean number of individuals in the system.


Average number of people in the system: 
\begin{equation}
    L = \sum_{i=1}^{|\pi|} \pi_i (u_i + v_i)
\end{equation} 

Average number of people in the service centre: 
\begin{equation}
    L_H = \sum_{i=1}^{|\pi|} \pi_i v_i
\end{equation}

Average number of people in the buffer centre:
\begin{equation}
    L_A = \sum_{i=1}^{|\pi|} \pi_i u_i
\end{equation}

Consequently getting the performance measures that are related to the duration of 
time is not as straightforward. 
Such performance measures are the mean waiting time in the system and the mean time 
blocked in the system. 
Under the scope of this study three approaches have been considered to calculate these 
performance measures; a direct approach, a recursive algorithm and consequently a
closed-form formula.

The research question that needs to be answered here is: ``When a class 1/2 
individuals enters the system, what is the expected time that they will have to 
wait?''. 
In order to formulate the answer to that question one needs to consider all possible 
scenarios of what state the system can be in when an individual arrives. 
Furthermore, different formulas arises for class 1 individuals 
and a different one for class 2 individuals.

\subsubsection{Mean waiting time} 
Upon closer inspection of the recursive formula a more compact formula can arise. 
The equivalent closed-form formula eliminates the need for recursion and thus makes 
the computation of waiting times much more efficient. 
Just like in the recursive part there are two formulas; one for \textit{class 1} 
and one for class 2 individuals. 
The formulas are given by:

\begin{equation} \label{eq:closed_form_waiting_others}
    W^{(1)} = \frac{\sum_{\substack{(u,v) \, \in S_A^{(1)} \\ v \geq C}} 
    \frac{1}{C \mu} \times (v-C+1) \times \pi(u,v)}{\sum_{(u,v) \, 
    \in S_A^{(1)}} \pi(u,v)}
\end{equation}
    
\begin{equation}\label{eq:closed_form_waiting_ambulance}
    W^{(2)} = \frac{\sum_{\substack{(u,v) \, \in S_A^{(2)} \\ min(v,T) \geq C}} 
    \frac{1}{C \mu} \times (\min(v+1,T)-C) \times \pi(u,v)}{\sum_{(u,v) \, 
    \in S_A^{(2)}} \pi(u,v)}
\end{equation}

Note here that the summation, in both equations \ref{eq:closed_form_waiting_others} 
and \ref{eq:closed_form_waiting_ambulance}, goes through all states in the set of 
accepting 
states of either class 1 or class 2 individuals respectively, where a wait 
incurs. 
In equation \ref{eq:closed_form_waiting_others} that includes all states \((u,v)\) 
in the set of accepting states of class 1 individuals such that \( v \geq C\); i.e. 
whenever an arrival occurs and the system is at a state where the number of individuals 
in the system is more than or equal to $C$. 
Consequently, for the states that are included in the summation the expression 
\( v-C+1 \) indicates the amount of people in service one would have to wait for 
upon arrival at the hospital.

Additionally, the minimisation function in equation 
\ref{eq:closed_form_waiting_ambulance} 
ensures that when a class 2 individual arrives at any state 
that is greater than the predetermined threshold, the wait that the individual will 
have to endure remains the same. 
In essence, the expression \(\min(v+1,T) - C\) returns the number of people in line 
in front of a particular individual upon arrival.


\subsubsection{Overall Waiting Time}

Consequently, the overall waiting time should can be estimated by a linear combination 
of the waiting times of class 1 and class 2 individuals. 
The overall waiting time can be then given by the following equation where \(c_1\) 
and \(c_2\) are the coefficients of each individual's type waiting time:

\begin{equation}\label{overall_waiting_time_coeff}
    W = c_1 W^{(1)} + c_2 W^{(2)}
\end{equation}

The two coefficients represent the proportion of individuals of each type that 
traversed through the model. 
Theoretically, getting these percentages should be as simple as looking at the arrival 
rates of each type but in practise if the service centre or the buffer centre 
is full, some individuals may be lost to the system. 
Thus, one should account for the probability that an individual is lost to the system. 
This probability can be easily calculated by using the two sets of accepting states 
\(S_A^{(2)}\) and \(S_A^{(1)}\) defined earlier in equations.
Let us define here the probability, for either class type, that an individual 
is not lost in the system by:

\begin{equation*}
    P(L'_1) = \sum_{(u,v) \, \in S_A^{(1)}} \pi(u,v) \hspace{2cm}
    P(L'_2) = \sum_{(u,v) \, \in S_A^{(2)}} \pi(u,v)
\end{equation*}

Having defined these probabilities one may combine them with the arrival rates of 
each class type in such a way to get the expected proportions of class 1 and 
class 2 individuals in the model. 
Thus, by using these values as the coefficient of equation 
\ref{overall_waiting_time_coeff} 
the resultant equation can be used to get the overall waiting time. 
Note here that the equation below gets the overall waiting time for both the recursive 
and the closed-form formula.

\begin{equation}\label{overall_waiting_time}
    W = \frac{\lambda_1 P(L'_1)}{\lambda_2 P(L'_2) + \lambda_1 P(L'_1)} W^{(1)} + 
    \frac{\lambda_2 P(L'_2)}{\lambda_2 P(L'_2) + \lambda_1 P(L'_1)} W^{(2)}
\end{equation}



\subsubsection{Mean blocking time}
Unlike the waiting time, the blocking time is only calculated for class 2 individuals.  
That is because class 1 individuals cannot be blocked. 
Thus, one only needs to consider the pathway of class 2 individuals to get the 
mean blocking time of the system. 
Blocking occurs at states \((u,v)\) where \(u > 0 \). 
Thus, the set of blocking states can be defined as:

\begin{equation*}
    S_b = \{(u,v) \in S \; | \; u > 0\}
\end{equation*}
 
In order to not consider individuals that will be lost to the system, the set of 
accepting states needs to be taken into account. The set of accepting states is given by:

\begin{equation*}
    S_A^{(2)}=
    \begin{cases}
        \{(u, v) \in S \; | \; u < M \} & \textbf{if } T \leq N\\
        \{(u, v) \in S \; | \; v < N \} & \textbf{otherwise}
    \end{cases}
\end{equation*}

For the waiting time formula,
the mean sojourn time for each state was considered,
ignoring any arrivals. Here, the same approach is used but ignoring only class 2
arrivals. That is because for the waiting time formula, once an individual enters 
the service centre (i.e. starts waiting) any individual arriving after them will 
not affect their
pathway. That is not the case for blocking time. When a class 2 individual is 
blocked, 
any class 1 individual that arrives will cause the blocked individual to remain 
blocked for more time. Therefore, class 1 arrivals are considered here:

\begin{equation}\label{eq:time_in_state_blocking_time}
    c(u,v) = 
    \begin{cases}
        \frac{1}{\min(v,C) \mu}, & \text{if } v = C\\
        \frac{1}{\min(v,C) \mu + \lambda_1}, & \text{otherwise}
    \end{cases}
\end{equation}
 
In equation \ref{eq:time_in_state_blocking_time}, both service completions and 
class 1 arrivals are considered. 
Thus, from a blocked individual's perspective whenever the system moves from one 
state \((u,v)\)
to another state it can either:

\begin{itemize}
    \item be because of a service being completed: we will denote the probability 
    of this happening by \(p_s(u,v)\). 
    \item be because of an arrival of an individual of class 1: denoting such 
    probability by \(p_o(u,v)\).
\end{itemize}
The probabilities are given by:

\begin{equation*}
    p_s(u,v) = \frac{\min(v,C)\mu}{\lambda_1 + \min(v,C)\mu}, \qquad
    p_o(u,v) = \frac{\lambda_1}{\lambda_1 + \min(v,C)\mu}
\end{equation*}


Having defined \(c(u,v)\) and \(S_b\) a formula for the blocking time that is
expected to occur at each state can be given by:

\begin{equation}\label{eq:blocking-time-at-each-state}
    b(u,v) = 
    \begin{cases} 
        0, & \textbf{if } (u,v) \notin S_b \\
        c(u,v) + b(u - 1, v), & \textbf{if } v = N = T\\
        c(u,v) + b(u, v-1), & \textbf{if } v = N \neq T \\
        c(u,v) + p_s(u,v) b(u-1, v) + p_o(u,v) b(u, v+1), & \textbf{if } u > 0 
        \textbf{ and } v = T \\
        c(u,v) + p_s(u,v) b(u, v-1) + p_o(u,v) b(u, v+1), & \textbf{otherwise} \\
    \end{cases}
\end{equation}

Equation 
(\ref{eq:blocking-time-at-each-state}) will not be solved recursively. 
A direct approach will be used to solve this equation here. 
By enumerating all equations of (\ref{eq:blocking-time-at-each-state}) for all 
states \((u,v)\) that belong in \(S_b\) 
a system of linear equations arises where the unknown variables are all the \(b(u,v)\)
terms.
For instance, let us consider a Markov model where \(C=2, T=3, N=6, M=2\). 
The Markov model is shown in Figure \ref{fig:example-algeb-blocking}
and the equivalent equations are 
(\ref{eq:first_eq_of_blocking_example})-(\ref{eq:last_eq_of_blocking_example}).
The equations considered here are only the ones that correspond to the blocking 
states.

\begin{multicols*}{2}
    \begin{figure}[H]
        \scalebox{0.50}{\input{MarkovChain/expressions_from_pi/example_model_2362/main.tex}}
        \caption{Example of Markov chain}
        \label{fig:example-algeb-blocking}
    \end{figure}
    \columnbreak
    \begin{align}
        b(1,2) &= c(1,2) + p_o b(1,3) \label{eq:first_eq_of_blocking_example} \\
        b(1,3) &= c(1,3) + p_s b(1,2) + p_o b(1,4) \\
        b(1,4) &= c(1,4) + b(1,3) \\
        b(2,2) &= c(2,2) + p_s b(1,2) + p_o b(2,3) \\
        b(2,3) &= c(2,3) + p_s b(2,2) + p_o b(1,4) \\
        b(2,4) &= c(2,4) + b(2,3)\label{eq:last_eq_of_blocking_example}
    \end{align}
\end{multicols*}

Additionally, the above equations can be transformed into a linear system of the 
form \(Zx=y\) where:

\begin{equation}\label{eq:example-algebaric-approach-blocking-time}
    Z=
    \begin{pmatrix}
        -1 & p_o & 0 & 0 & 0 & 0 \\ %(1,2)
        p_s & -1 & p_o & 0 & 0 & 0 \\ %(1,3)
        0 & 1 & -1 & 0 & 0 & 0 \\ %(1,4)
        p_s & 0 & 0 & -1 & p_o & 0\\ %(2,2)
        0 & 0 & 0 & p_s & -1 & p_o \\ %(2,3)
        0 & 0 & 0 & 0 & 1 & -1 \\ %(2,4)
    \end{pmatrix},
    x=
    \begin{pmatrix}
        b(1,2) \\
        b(1,3) \\
        b(1,4) \\
        b(2,2) \\
        b(2,3) \\
        b(2,4) \\
    \end{pmatrix}, 
    y=
    \begin{pmatrix}
        -c(1,2) \\
        -c(1,3) \\
        -c(1,4) \\
        -c(2,2) \\
        -c(2,3) \\
        -c(2,4) \\
    \end{pmatrix}
\end{equation}

A more generalised form of the equations in 
(\ref{eq:example-algebaric-approach-blocking-time})
can thus be given for any value of \(C,T,N,M\) by:

\begin{align}
    b(1,T) =& c(1, T) + p_o b(1, T + 1) \label{eq:first_eq_of_blocking_general}\\
    b(1,T + 1) =& c(1, T + 1) + p_s(1, T) + p_o b(1, T + 1) \\
    b(1,T + 2) =& c(1, T + 2) + p_s(1, T + 1) + p_o b(1, T + 3) \\
    & \vdots \nonumber \\
    b(1, N) =& c(1, N) + b(1, N - 1) \\
    b(2, T) =& c(2, T) + p_s b(1, T) + p_o b(2, T + 1) \\
    b(2, T + 1) =& c(2, T + 1) + p_s b(2, T) + p_o b(2, T + 2) \\
    & \vdots \nonumber \\
    b(M, T) =& c(M, T) + b(M, T-1) \label{eq:last_eq_of_blocking_general}
\end{align}

The equivalent matrix form of the linear system of equations 
(\ref{eq:first_eq_of_blocking_general}) - (\ref{eq:last_eq_of_blocking_general})
is given by \(Zx=y\), where:
\begin{equation}\label{eq:general-algebaric-approach-blocking-time}
    \scalebox{0.9}{
        \(
        Z = 
        \begin{pmatrix}
            -1 & p_o & 0 & \dots & 0 & 0 & 0 & 0 & 0 & \dots & 0 & 0 \\ %(1,T)
            p_s & -1 & p_o & \dots & 0 & 0 & 0 & 0 & 0 & \dots & 0 & 0 \\ %(1,T+1)
            0 & p_s & -1 & \dots & 0 & 0 & 0 & 0 & 0 & \dots & 0 & 0 \\ %(1,T+2)
            \vdots & \vdots & \vdots & \ddots & \vdots & \vdots & \vdots & \vdots & 
            \vdots & \ddots & \vdots & \vdots \\ 
            0 & 0 & 0 & \dots & 1 & -1 & 0 & 0 & 0 & \dots & 0 & 0 \\ %(1,N)
            p_s & 0 & 0 & \dots & 0 & 0 & -1 & p_o & 0 & \dots & 0 & 0 \\ %(2,T)
            0 & 0 & 0 & \dots & 0 & 0 & p_s & -1 & p_o & \dots & 0 & 0 \\ %(2,T+1)
            \vdots & \vdots & \vdots & \ddots & \vdots & \vdots & \vdots & \vdots & 
            \vdots & \ddots & \vdots & \vdots \\ 
            0 & 0 & 0 & \dots & 0 & 0 & 0 & 0 & 0 & \dots & 1 & -1 \\ %(M,T)
        \end{pmatrix},
        x = 
        \begin{pmatrix}
            b(1,T) \\
            b(1,T+1) \\
            b(1,T+2) \\
            \vdots \\
            b(1,N) \\
            b(2,T) \\
            b(2,T+1) \\
            \vdots \\
            b(M,T) \\
        \end{pmatrix}, 
        y= 
        \begin{pmatrix}
            -c(1,T) \\
            -c(1,T+1) \\
            -c(1,T+2) \\
            \vdots \\
            -c(1,N) \\
            -c(2,T) \\
            -c(2,T+1) \\
            \vdots \\
            -c(M,T) \\
        \end{pmatrix}
        \)
    }
\end{equation}

Thus, having calculated the mean blocking time for all blocking states \(b(u,v)\), 
it only remains to put them together in a formula.
The resultant blocking time formula is given by:

\begin{equation}\label{eq:algebraic-blocking-time}
    B = \frac{\sum_{(u,v) \in S_A} \pi_{(u,v)} \; b(u,v)}{\sum_{(u,v) \in S_A} 
    \pi_{(u,v)}}
\end{equation}

\documentclass{article}

\usepackage{amsmath}
\usepackage{amsfonts} 
\usepackage{geometry}
\usepackage{multicol}
\usepackage{float}
% \usepackage{mathtools}
% \usepackage{graphicx}
% \usepackage{soul}
% \usepackage{indentfirst}
\usepackage{tikz}
\usetikzlibrary{calc, automata, chains, arrows.meta, math}
\setcounter{MaxMatrixCols}{20}


\title{A game theoretic model of the behavioural gaming that takes place at the EMS - ED interface}

\author{
    Michalis Panayides, 
    Paul Harper, 
    Vince Knight
}

\begin{document}

\maketitle

\input{Abstract/main.tex}


\newpage
\tableofcontents

\newpage
\input{Introduction/main.tex}

\newpage
\input{Game_theory_component/main.tex}

\newpage
\input{MarkovChain/markov_chain_model/main.tex}
\input{MarkovChain/expressions_from_pi/main.tex}
\input{MarkovChain/markov_example/main.tex}

\newpage
\input{BehaviouralMethodology/main.tex}

\newpage
\input{Application_EMS_ED/main.tex}

\newpage
\input{Conclusion/main.tex}


\end{document}

\newpage
\documentclass{article}

\usepackage{amsmath}
\usepackage{amsfonts} 
\usepackage{geometry}
\usepackage{multicol}
\usepackage{float}
% \usepackage{mathtools}
% \usepackage{graphicx}
% \usepackage{soul}
% \usepackage{indentfirst}
\usepackage{tikz}
\usetikzlibrary{calc, automata, chains, arrows.meta, math}
\setcounter{MaxMatrixCols}{20}


\title{A game theoretic model of the behavioural gaming that takes place at the EMS - ED interface}

\author{
    Michalis Panayides, 
    Paul Harper, 
    Vince Knight
}

\begin{document}

\maketitle

\input{Abstract/main.tex}


\newpage
\tableofcontents

\newpage
\input{Introduction/main.tex}

\newpage
\input{Game_theory_component/main.tex}

\newpage
\input{MarkovChain/markov_chain_model/main.tex}
\input{MarkovChain/expressions_from_pi/main.tex}
\input{MarkovChain/markov_example/main.tex}

\newpage
\input{BehaviouralMethodology/main.tex}

\newpage
\input{Application_EMS_ED/main.tex}

\newpage
\input{Conclusion/main.tex}


\end{document}

\newpage
\section{EMS-ED application}

\subsection{Application}

\subsection{Data analysis of generated problem}

\newpage
\documentclass{article}

\usepackage{amsmath}
\usepackage{amsfonts} 
\usepackage{geometry}
\usepackage{multicol}
\usepackage{float}
% \usepackage{mathtools}
% \usepackage{graphicx}
% \usepackage{soul}
% \usepackage{indentfirst}
\usepackage{tikz}
\usetikzlibrary{calc, automata, chains, arrows.meta, math}
\setcounter{MaxMatrixCols}{20}


\title{A game theoretic model of the behavioural gaming that takes place at the EMS - ED interface}

\author{
    Michalis Panayides, 
    Paul Harper, 
    Vince Knight
}

\begin{document}

\maketitle

\input{Abstract/main.tex}


\newpage
\tableofcontents

\newpage
\input{Introduction/main.tex}

\newpage
\input{Game_theory_component/main.tex}

\newpage
\input{MarkovChain/markov_chain_model/main.tex}
\input{MarkovChain/expressions_from_pi/main.tex}
\input{MarkovChain/markov_example/main.tex}

\newpage
\input{BehaviouralMethodology/main.tex}

\newpage
\input{Application_EMS_ED/main.tex}

\newpage
\input{Conclusion/main.tex}


\end{document}


\end{document}
\subsection{Performance Measures}
One may easily derive the average number of individuals that are at any given state 
using \( pi \). 
The average number of individuals in state \( i \) can be calculated by multiplying 
the number of individuals that are present in state \( i \) with the probability 
of being at that particular state (i.e \(\pi_i (u_i + v_i)\)). 
Using this logic it is possible to calculate any performance measures that are related 
to the mean number of individuals in the system.


Average number of people in the system: 
\begin{equation}
    L = \sum_{i=1}^{|\pi|} \pi_i (u_i + v_i)
\end{equation} 

Average number of people in the service centre: 
\begin{equation}
    L_H = \sum_{i=1}^{|\pi|} \pi_i v_i
\end{equation}

Average number of people in the buffer centre:
\begin{equation}
    L_A = \sum_{i=1}^{|\pi|} \pi_i u_i
\end{equation}

Consequently getting the performance measures that are related to the duration of 
time is not as straightforward. 
Such performance measures are the mean waiting time in the system and the mean time 
blocked in the system. 
Under the scope of this study three approaches have been considered to calculate these 
performance measures; a direct approach, a recursive algorithm and consequently a
closed-form formula.

The research question that needs to be answered here is: ``When a class 1/2 
individuals enters the system, what is the expected time that they will have to 
wait?''. 
In order to formulate the answer to that question one needs to consider all possible 
scenarios of what state the system can be in when an individual arrives. 
Furthermore, different formulas arises for class 1 individuals 
and a different one for class 2 individuals.

\subsubsection{Mean waiting time} 
Upon closer inspection of the recursive formula a more compact formula can arise. 
The equivalent closed-form formula eliminates the need for recursion and thus makes 
the computation of waiting times much more efficient. 
Just like in the recursive part there are two formulas; one for \textit{class 1} 
and one for class 2 individuals. 
The formulas are given by:

\begin{equation} \label{eq:closed_form_waiting_others}
    W^{(1)} = \frac{\sum_{\substack{(u,v) \, \in S_A^{(1)} \\ v \geq C}} 
    \frac{1}{C \mu} \times (v-C+1) \times \pi(u,v)}{\sum_{(u,v) \, 
    \in S_A^{(1)}} \pi(u,v)}
\end{equation}
    
\begin{equation}\label{eq:closed_form_waiting_ambulance}
    W^{(2)} = \frac{\sum_{\substack{(u,v) \, \in S_A^{(2)} \\ min(v,T) \geq C}} 
    \frac{1}{C \mu} \times (\min(v+1,T)-C) \times \pi(u,v)}{\sum_{(u,v) \, 
    \in S_A^{(2)}} \pi(u,v)}
\end{equation}

Note here that the summation, in both equations \ref{eq:closed_form_waiting_others} 
and \ref{eq:closed_form_waiting_ambulance}, goes through all states in the set of 
accepting 
states of either class 1 or class 2 individuals respectively, where a wait 
incurs. 
In equation \ref{eq:closed_form_waiting_others} that includes all states \((u,v)\) 
in the set of accepting states of class 1 individuals such that \( v \geq C\); i.e. 
whenever an arrival occurs and the system is at a state where the number of individuals 
in the system is more than or equal to $C$. 
Consequently, for the states that are included in the summation the expression 
\( v-C+1 \) indicates the amount of people in service one would have to wait for 
upon arrival at the hospital.

Additionally, the minimisation function in equation 
\ref{eq:closed_form_waiting_ambulance} 
ensures that when a class 2 individual arrives at any state 
that is greater than the predetermined threshold, the wait that the individual will 
have to endure remains the same. 
In essence, the expression \(\min(v+1,T) - C\) returns the number of people in line 
in front of a particular individual upon arrival.


\subsubsection{Overall Waiting Time}

Consequently, the overall waiting time should can be estimated by a linear combination 
of the waiting times of class 1 and class 2 individuals. 
The overall waiting time can be then given by the following equation where \(c_1\) 
and \(c_2\) are the coefficients of each individual's type waiting time:

\begin{equation}\label{overall_waiting_time_coeff}
    W = c_1 W^{(1)} + c_2 W^{(2)}
\end{equation}

The two coefficients represent the proportion of individuals of each type that 
traversed through the model. 
Theoretically, getting these percentages should be as simple as looking at the arrival 
rates of each type but in practise if the service centre or the buffer centre 
is full, some individuals may be lost to the system. 
Thus, one should account for the probability that an individual is lost to the system. 
This probability can be easily calculated by using the two sets of accepting states 
\(S_A^{(2)}\) and \(S_A^{(1)}\) defined earlier in equations.
Let us define here the probability, for either class type, that an individual 
is not lost in the system by:

\begin{equation*}
    P(L'_1) = \sum_{(u,v) \, \in S_A^{(1)}} \pi(u,v) \hspace{2cm}
    P(L'_2) = \sum_{(u,v) \, \in S_A^{(2)}} \pi(u,v)
\end{equation*}

Having defined these probabilities one may combine them with the arrival rates of 
each class type in such a way to get the expected proportions of class 1 and 
class 2 individuals in the model. 
Thus, by using these values as the coefficient of equation 
\ref{overall_waiting_time_coeff} 
the resultant equation can be used to get the overall waiting time. 
Note here that the equation below gets the overall waiting time for both the recursive 
and the closed-form formula.

\begin{equation}\label{overall_waiting_time}
    W = \frac{\lambda_1 P(L'_1)}{\lambda_2 P(L'_2) + \lambda_1 P(L'_1)} W^{(1)} + 
    \frac{\lambda_2 P(L'_2)}{\lambda_2 P(L'_2) + \lambda_1 P(L'_1)} W^{(2)}
\end{equation}



\subsubsection{Mean blocking time}
Unlike the waiting time, the blocking time is only calculated for class 2 individuals.  
That is because class 1 individuals cannot be blocked. 
Thus, one only needs to consider the pathway of class 2 individuals to get the 
mean blocking time of the system. 
Blocking occurs at states \((u,v)\) where \(u > 0 \). 
Thus, the set of blocking states can be defined as:

\begin{equation*}
    S_b = \{(u,v) \in S \; | \; u > 0\}
\end{equation*}
 
In order to not consider individuals that will be lost to the system, the set of 
accepting states needs to be taken into account. The set of accepting states is given by:

\begin{equation*}
    S_A^{(2)}=
    \begin{cases}
        \{(u, v) \in S \; | \; u < M \} & \textbf{if } T \leq N\\
        \{(u, v) \in S \; | \; v < N \} & \textbf{otherwise}
    \end{cases}
\end{equation*}

For the waiting time formula,
the mean sojourn time for each state was considered,
ignoring any arrivals. Here, the same approach is used but ignoring only class 2
arrivals. That is because for the waiting time formula, once an individual enters 
the service centre (i.e. starts waiting) any individual arriving after them will 
not affect their
pathway. That is not the case for blocking time. When a class 2 individual is 
blocked, 
any class 1 individual that arrives will cause the blocked individual to remain 
blocked for more time. Therefore, class 1 arrivals are considered here:

\begin{equation}\label{eq:time_in_state_blocking_time}
    c(u,v) = 
    \begin{cases}
        \frac{1}{\min(v,C) \mu}, & \text{if } v = C\\
        \frac{1}{\min(v,C) \mu + \lambda_1}, & \text{otherwise}
    \end{cases}
\end{equation}
 
In equation \ref{eq:time_in_state_blocking_time}, both service completions and 
class 1 arrivals are considered. 
Thus, from a blocked individual's perspective whenever the system moves from one 
state \((u,v)\)
to another state it can either:

\begin{itemize}
    \item be because of a service being completed: we will denote the probability 
    of this happening by \(p_s(u,v)\). 
    \item be because of an arrival of an individual of class 1: denoting such 
    probability by \(p_o(u,v)\).
\end{itemize}
The probabilities are given by:

\begin{equation*}
    p_s(u,v) = \frac{\min(v,C)\mu}{\lambda_1 + \min(v,C)\mu}, \qquad
    p_o(u,v) = \frac{\lambda_1}{\lambda_1 + \min(v,C)\mu}
\end{equation*}


Having defined \(c(u,v)\) and \(S_b\) a formula for the blocking time that is
expected to occur at each state can be given by:

\begin{equation}\label{eq:blocking-time-at-each-state}
    b(u,v) = 
    \begin{cases} 
        0, & \textbf{if } (u,v) \notin S_b \\
        c(u,v) + b(u - 1, v), & \textbf{if } v = N = T\\
        c(u,v) + b(u, v-1), & \textbf{if } v = N \neq T \\
        c(u,v) + p_s(u,v) b(u-1, v) + p_o(u,v) b(u, v+1), & \textbf{if } u > 0 
        \textbf{ and } v = T \\
        c(u,v) + p_s(u,v) b(u, v-1) + p_o(u,v) b(u, v+1), & \textbf{otherwise} \\
    \end{cases}
\end{equation}

Equation 
(\ref{eq:blocking-time-at-each-state}) will not be solved recursively. 
A direct approach will be used to solve this equation here. 
By enumerating all equations of (\ref{eq:blocking-time-at-each-state}) for all 
states \((u,v)\) that belong in \(S_b\) 
a system of linear equations arises where the unknown variables are all the \(b(u,v)\)
terms.
For instance, let us consider a Markov model where \(C=2, T=3, N=6, M=2\). 
The Markov model is shown in Figure \ref{fig:example-algeb-blocking}
and the equivalent equations are 
(\ref{eq:first_eq_of_blocking_example})-(\ref{eq:last_eq_of_blocking_example}).
The equations considered here are only the ones that correspond to the blocking 
states.

\begin{multicols*}{2}
    \begin{figure}[H]
        \scalebox{0.50}{\documentclass{article}

\usepackage{amsmath}
\usepackage{amsfonts} 
\usepackage{geometry}
\usepackage{multicol}
\usepackage{float}
% \usepackage{mathtools}
% \usepackage{graphicx}
% \usepackage{soul}
% \usepackage{indentfirst}
\usepackage{tikz}
\usetikzlibrary{calc, automata, chains, arrows.meta, math}
\setcounter{MaxMatrixCols}{20}


\title{A game theoretic model of the behavioural gaming that takes place at the EMS - ED interface}

\author{
    Michalis Panayides, 
    Paul Harper, 
    Vince Knight
}

\begin{document}

\maketitle

\input{Abstract/main.tex}


\newpage
\tableofcontents

\newpage
\input{Introduction/main.tex}

\newpage
\input{Game_theory_component/main.tex}

\newpage
\input{MarkovChain/markov_chain_model/main.tex}
\input{MarkovChain/expressions_from_pi/main.tex}
\input{MarkovChain/markov_example/main.tex}

\newpage
\input{BehaviouralMethodology/main.tex}

\newpage
\input{Application_EMS_ED/main.tex}

\newpage
\input{Conclusion/main.tex}


\end{document}}
        \caption{Example of Markov chain}
        \label{fig:example-algeb-blocking}
    \end{figure}
    \columnbreak
    \begin{align}
        b(1,2) &= c(1,2) + p_o b(1,3) \label{eq:first_eq_of_blocking_example} \\
        b(1,3) &= c(1,3) + p_s b(1,2) + p_o b(1,4) \\
        b(1,4) &= c(1,4) + b(1,3) \\
        b(2,2) &= c(2,2) + p_s b(1,2) + p_o b(2,3) \\
        b(2,3) &= c(2,3) + p_s b(2,2) + p_o b(1,4) \\
        b(2,4) &= c(2,4) + b(2,3)\label{eq:last_eq_of_blocking_example}
    \end{align}
\end{multicols*}

Additionally, the above equations can be transformed into a linear system of the 
form \(Zx=y\) where:

\begin{equation}\label{eq:example-algebaric-approach-blocking-time}
    Z=
    \begin{pmatrix}
        -1 & p_o & 0 & 0 & 0 & 0 \\ %(1,2)
        p_s & -1 & p_o & 0 & 0 & 0 \\ %(1,3)
        0 & 1 & -1 & 0 & 0 & 0 \\ %(1,4)
        p_s & 0 & 0 & -1 & p_o & 0\\ %(2,2)
        0 & 0 & 0 & p_s & -1 & p_o \\ %(2,3)
        0 & 0 & 0 & 0 & 1 & -1 \\ %(2,4)
    \end{pmatrix},
    x=
    \begin{pmatrix}
        b(1,2) \\
        b(1,3) \\
        b(1,4) \\
        b(2,2) \\
        b(2,3) \\
        b(2,4) \\
    \end{pmatrix}, 
    y=
    \begin{pmatrix}
        -c(1,2) \\
        -c(1,3) \\
        -c(1,4) \\
        -c(2,2) \\
        -c(2,3) \\
        -c(2,4) \\
    \end{pmatrix}
\end{equation}

A more generalised form of the equations in 
(\ref{eq:example-algebaric-approach-blocking-time})
can thus be given for any value of \(C,T,N,M\) by:

\begin{align}
    b(1,T) =& c(1, T) + p_o b(1, T + 1) \label{eq:first_eq_of_blocking_general}\\
    b(1,T + 1) =& c(1, T + 1) + p_s(1, T) + p_o b(1, T + 1) \\
    b(1,T + 2) =& c(1, T + 2) + p_s(1, T + 1) + p_o b(1, T + 3) \\
    & \vdots \nonumber \\
    b(1, N) =& c(1, N) + b(1, N - 1) \\
    b(2, T) =& c(2, T) + p_s b(1, T) + p_o b(2, T + 1) \\
    b(2, T + 1) =& c(2, T + 1) + p_s b(2, T) + p_o b(2, T + 2) \\
    & \vdots \nonumber \\
    b(M, T) =& c(M, T) + b(M, T-1) \label{eq:last_eq_of_blocking_general}
\end{align}

The equivalent matrix form of the linear system of equations 
(\ref{eq:first_eq_of_blocking_general}) - (\ref{eq:last_eq_of_blocking_general})
is given by \(Zx=y\), where:
\begin{equation}\label{eq:general-algebaric-approach-blocking-time}
    \scalebox{0.9}{
        \(
        Z = 
        \begin{pmatrix}
            -1 & p_o & 0 & \dots & 0 & 0 & 0 & 0 & 0 & \dots & 0 & 0 \\ %(1,T)
            p_s & -1 & p_o & \dots & 0 & 0 & 0 & 0 & 0 & \dots & 0 & 0 \\ %(1,T+1)
            0 & p_s & -1 & \dots & 0 & 0 & 0 & 0 & 0 & \dots & 0 & 0 \\ %(1,T+2)
            \vdots & \vdots & \vdots & \ddots & \vdots & \vdots & \vdots & \vdots & 
            \vdots & \ddots & \vdots & \vdots \\ 
            0 & 0 & 0 & \dots & 1 & -1 & 0 & 0 & 0 & \dots & 0 & 0 \\ %(1,N)
            p_s & 0 & 0 & \dots & 0 & 0 & -1 & p_o & 0 & \dots & 0 & 0 \\ %(2,T)
            0 & 0 & 0 & \dots & 0 & 0 & p_s & -1 & p_o & \dots & 0 & 0 \\ %(2,T+1)
            \vdots & \vdots & \vdots & \ddots & \vdots & \vdots & \vdots & \vdots & 
            \vdots & \ddots & \vdots & \vdots \\ 
            0 & 0 & 0 & \dots & 0 & 0 & 0 & 0 & 0 & \dots & 1 & -1 \\ %(M,T)
        \end{pmatrix},
        x = 
        \begin{pmatrix}
            b(1,T) \\
            b(1,T+1) \\
            b(1,T+2) \\
            \vdots \\
            b(1,N) \\
            b(2,T) \\
            b(2,T+1) \\
            \vdots \\
            b(M,T) \\
        \end{pmatrix}, 
        y= 
        \begin{pmatrix}
            -c(1,T) \\
            -c(1,T+1) \\
            -c(1,T+2) \\
            \vdots \\
            -c(1,N) \\
            -c(2,T) \\
            -c(2,T+1) \\
            \vdots \\
            -c(M,T) \\
        \end{pmatrix}
        \)
    }
\end{equation}

Thus, having calculated the mean blocking time for all blocking states \(b(u,v)\), 
it only remains to put them together in a formula.
The resultant blocking time formula is given by:

\begin{equation}\label{eq:algebraic-blocking-time}
    B = \frac{\sum_{(u,v) \in S_A} \pi_{(u,v)} \; b(u,v)}{\sum_{(u,v) \in S_A} 
    \pi_{(u,v)}}
\end{equation}

\documentclass{article}

\usepackage{amsmath}
\usepackage{amsfonts} 
\usepackage{geometry}
\usepackage{multicol}
\usepackage{float}
% \usepackage{mathtools}
% \usepackage{graphicx}
% \usepackage{soul}
% \usepackage{indentfirst}
\usepackage{tikz}
\usetikzlibrary{calc, automata, chains, arrows.meta, math}
\setcounter{MaxMatrixCols}{20}


\title{A game theoretic model of the behavioural gaming that takes place at the EMS - ED interface}

\author{
    Michalis Panayides, 
    Paul Harper, 
    Vince Knight
}

\begin{document}

\maketitle

\documentclass{article}

\usepackage{amsmath}
\usepackage{amsfonts} 
\usepackage{geometry}
\usepackage{multicol}
\usepackage{float}
% \usepackage{mathtools}
% \usepackage{graphicx}
% \usepackage{soul}
% \usepackage{indentfirst}
\usepackage{tikz}
\usetikzlibrary{calc, automata, chains, arrows.meta, math}
\setcounter{MaxMatrixCols}{20}


\title{A game theoretic model of the behavioural gaming that takes place at the EMS - ED interface}

\author{
    Michalis Panayides, 
    Paul Harper, 
    Vince Knight
}

\begin{document}

\maketitle

\input{Abstract/main.tex}


\newpage
\tableofcontents

\newpage
\input{Introduction/main.tex}

\newpage
\input{Game_theory_component/main.tex}

\newpage
\input{MarkovChain/markov_chain_model/main.tex}
\input{MarkovChain/expressions_from_pi/main.tex}
\input{MarkovChain/markov_example/main.tex}

\newpage
\input{BehaviouralMethodology/main.tex}

\newpage
\input{Application_EMS_ED/main.tex}

\newpage
\input{Conclusion/main.tex}


\end{document}


\newpage
\tableofcontents

\newpage
\documentclass{article}

\usepackage{amsmath}
\usepackage{amsfonts} 
\usepackage{geometry}
\usepackage{multicol}
\usepackage{float}
% \usepackage{mathtools}
% \usepackage{graphicx}
% \usepackage{soul}
% \usepackage{indentfirst}
\usepackage{tikz}
\usetikzlibrary{calc, automata, chains, arrows.meta, math}
\setcounter{MaxMatrixCols}{20}


\title{A game theoretic model of the behavioural gaming that takes place at the EMS - ED interface}

\author{
    Michalis Panayides, 
    Paul Harper, 
    Vince Knight
}

\begin{document}

\maketitle

\input{Abstract/main.tex}


\newpage
\tableofcontents

\newpage
\input{Introduction/main.tex}

\newpage
\input{Game_theory_component/main.tex}

\newpage
\input{MarkovChain/markov_chain_model/main.tex}
\input{MarkovChain/expressions_from_pi/main.tex}
\input{MarkovChain/markov_example/main.tex}

\newpage
\input{BehaviouralMethodology/main.tex}

\newpage
\input{Application_EMS_ED/main.tex}

\newpage
\input{Conclusion/main.tex}


\end{document}

\newpage
\documentclass{article}

\usepackage{amsmath}
\usepackage{amsfonts} 
\usepackage{geometry}
\usepackage{multicol}
\usepackage{float}
% \usepackage{mathtools}
% \usepackage{graphicx}
% \usepackage{soul}
% \usepackage{indentfirst}
\usepackage{tikz}
\usetikzlibrary{calc, automata, chains, arrows.meta, math}
\setcounter{MaxMatrixCols}{20}


\title{A game theoretic model of the behavioural gaming that takes place at the EMS - ED interface}

\author{
    Michalis Panayides, 
    Paul Harper, 
    Vince Knight
}

\begin{document}

\maketitle

\input{Abstract/main.tex}


\newpage
\tableofcontents

\newpage
\input{Introduction/main.tex}

\newpage
\input{Game_theory_component/main.tex}

\newpage
\input{MarkovChain/markov_chain_model/main.tex}
\input{MarkovChain/expressions_from_pi/main.tex}
\input{MarkovChain/markov_example/main.tex}

\newpage
\input{BehaviouralMethodology/main.tex}

\newpage
\input{Application_EMS_ED/main.tex}

\newpage
\input{Conclusion/main.tex}


\end{document}

\newpage
\documentclass{article}

\usepackage{amsmath}
\usepackage{amsfonts} 
\usepackage{geometry}
\usepackage{multicol}
\usepackage{float}
% \usepackage{mathtools}
% \usepackage{graphicx}
% \usepackage{soul}
% \usepackage{indentfirst}
\usepackage{tikz}
\usetikzlibrary{calc, automata, chains, arrows.meta, math}
\setcounter{MaxMatrixCols}{20}


\title{A game theoretic model of the behavioural gaming that takes place at the EMS - ED interface}

\author{
    Michalis Panayides, 
    Paul Harper, 
    Vince Knight
}

\begin{document}

\maketitle

\input{Abstract/main.tex}


\newpage
\tableofcontents

\newpage
\input{Introduction/main.tex}

\newpage
\input{Game_theory_component/main.tex}

\newpage
\input{MarkovChain/markov_chain_model/main.tex}
\input{MarkovChain/expressions_from_pi/main.tex}
\input{MarkovChain/markov_example/main.tex}

\newpage
\input{BehaviouralMethodology/main.tex}

\newpage
\input{Application_EMS_ED/main.tex}

\newpage
\input{Conclusion/main.tex}


\end{document}
\subsection{Performance Measures}
One may easily derive the average number of individuals that are at any given state 
using \( pi \). 
The average number of individuals in state \( i \) can be calculated by multiplying 
the number of individuals that are present in state \( i \) with the probability 
of being at that particular state (i.e \(\pi_i (u_i + v_i)\)). 
Using this logic it is possible to calculate any performance measures that are related 
to the mean number of individuals in the system.


Average number of people in the system: 
\begin{equation}
    L = \sum_{i=1}^{|\pi|} \pi_i (u_i + v_i)
\end{equation} 

Average number of people in the service centre: 
\begin{equation}
    L_H = \sum_{i=1}^{|\pi|} \pi_i v_i
\end{equation}

Average number of people in the buffer centre:
\begin{equation}
    L_A = \sum_{i=1}^{|\pi|} \pi_i u_i
\end{equation}

Consequently getting the performance measures that are related to the duration of 
time is not as straightforward. 
Such performance measures are the mean waiting time in the system and the mean time 
blocked in the system. 
Under the scope of this study three approaches have been considered to calculate these 
performance measures; a direct approach, a recursive algorithm and consequently a
closed-form formula.

The research question that needs to be answered here is: ``When a class 1/2 
individuals enters the system, what is the expected time that they will have to 
wait?''. 
In order to formulate the answer to that question one needs to consider all possible 
scenarios of what state the system can be in when an individual arrives. 
Furthermore, different formulas arises for class 1 individuals 
and a different one for class 2 individuals.

\subsubsection{Mean waiting time} 
Upon closer inspection of the recursive formula a more compact formula can arise. 
The equivalent closed-form formula eliminates the need for recursion and thus makes 
the computation of waiting times much more efficient. 
Just like in the recursive part there are two formulas; one for \textit{class 1} 
and one for class 2 individuals. 
The formulas are given by:

\begin{equation} \label{eq:closed_form_waiting_others}
    W^{(1)} = \frac{\sum_{\substack{(u,v) \, \in S_A^{(1)} \\ v \geq C}} 
    \frac{1}{C \mu} \times (v-C+1) \times \pi(u,v)}{\sum_{(u,v) \, 
    \in S_A^{(1)}} \pi(u,v)}
\end{equation}
    
\begin{equation}\label{eq:closed_form_waiting_ambulance}
    W^{(2)} = \frac{\sum_{\substack{(u,v) \, \in S_A^{(2)} \\ min(v,T) \geq C}} 
    \frac{1}{C \mu} \times (\min(v+1,T)-C) \times \pi(u,v)}{\sum_{(u,v) \, 
    \in S_A^{(2)}} \pi(u,v)}
\end{equation}

Note here that the summation, in both equations \ref{eq:closed_form_waiting_others} 
and \ref{eq:closed_form_waiting_ambulance}, goes through all states in the set of 
accepting 
states of either class 1 or class 2 individuals respectively, where a wait 
incurs. 
In equation \ref{eq:closed_form_waiting_others} that includes all states \((u,v)\) 
in the set of accepting states of class 1 individuals such that \( v \geq C\); i.e. 
whenever an arrival occurs and the system is at a state where the number of individuals 
in the system is more than or equal to $C$. 
Consequently, for the states that are included in the summation the expression 
\( v-C+1 \) indicates the amount of people in service one would have to wait for 
upon arrival at the hospital.

Additionally, the minimisation function in equation 
\ref{eq:closed_form_waiting_ambulance} 
ensures that when a class 2 individual arrives at any state 
that is greater than the predetermined threshold, the wait that the individual will 
have to endure remains the same. 
In essence, the expression \(\min(v+1,T) - C\) returns the number of people in line 
in front of a particular individual upon arrival.


\subsubsection{Overall Waiting Time}

Consequently, the overall waiting time should can be estimated by a linear combination 
of the waiting times of class 1 and class 2 individuals. 
The overall waiting time can be then given by the following equation where \(c_1\) 
and \(c_2\) are the coefficients of each individual's type waiting time:

\begin{equation}\label{overall_waiting_time_coeff}
    W = c_1 W^{(1)} + c_2 W^{(2)}
\end{equation}

The two coefficients represent the proportion of individuals of each type that 
traversed through the model. 
Theoretically, getting these percentages should be as simple as looking at the arrival 
rates of each type but in practise if the service centre or the buffer centre 
is full, some individuals may be lost to the system. 
Thus, one should account for the probability that an individual is lost to the system. 
This probability can be easily calculated by using the two sets of accepting states 
\(S_A^{(2)}\) and \(S_A^{(1)}\) defined earlier in equations.
Let us define here the probability, for either class type, that an individual 
is not lost in the system by:

\begin{equation*}
    P(L'_1) = \sum_{(u,v) \, \in S_A^{(1)}} \pi(u,v) \hspace{2cm}
    P(L'_2) = \sum_{(u,v) \, \in S_A^{(2)}} \pi(u,v)
\end{equation*}

Having defined these probabilities one may combine them with the arrival rates of 
each class type in such a way to get the expected proportions of class 1 and 
class 2 individuals in the model. 
Thus, by using these values as the coefficient of equation 
\ref{overall_waiting_time_coeff} 
the resultant equation can be used to get the overall waiting time. 
Note here that the equation below gets the overall waiting time for both the recursive 
and the closed-form formula.

\begin{equation}\label{overall_waiting_time}
    W = \frac{\lambda_1 P(L'_1)}{\lambda_2 P(L'_2) + \lambda_1 P(L'_1)} W^{(1)} + 
    \frac{\lambda_2 P(L'_2)}{\lambda_2 P(L'_2) + \lambda_1 P(L'_1)} W^{(2)}
\end{equation}



\subsubsection{Mean blocking time}
Unlike the waiting time, the blocking time is only calculated for class 2 individuals.  
That is because class 1 individuals cannot be blocked. 
Thus, one only needs to consider the pathway of class 2 individuals to get the 
mean blocking time of the system. 
Blocking occurs at states \((u,v)\) where \(u > 0 \). 
Thus, the set of blocking states can be defined as:

\begin{equation*}
    S_b = \{(u,v) \in S \; | \; u > 0\}
\end{equation*}
 
In order to not consider individuals that will be lost to the system, the set of 
accepting states needs to be taken into account. The set of accepting states is given by:

\begin{equation*}
    S_A^{(2)}=
    \begin{cases}
        \{(u, v) \in S \; | \; u < M \} & \textbf{if } T \leq N\\
        \{(u, v) \in S \; | \; v < N \} & \textbf{otherwise}
    \end{cases}
\end{equation*}

For the waiting time formula,
the mean sojourn time for each state was considered,
ignoring any arrivals. Here, the same approach is used but ignoring only class 2
arrivals. That is because for the waiting time formula, once an individual enters 
the service centre (i.e. starts waiting) any individual arriving after them will 
not affect their
pathway. That is not the case for blocking time. When a class 2 individual is 
blocked, 
any class 1 individual that arrives will cause the blocked individual to remain 
blocked for more time. Therefore, class 1 arrivals are considered here:

\begin{equation}\label{eq:time_in_state_blocking_time}
    c(u,v) = 
    \begin{cases}
        \frac{1}{\min(v,C) \mu}, & \text{if } v = C\\
        \frac{1}{\min(v,C) \mu + \lambda_1}, & \text{otherwise}
    \end{cases}
\end{equation}
 
In equation \ref{eq:time_in_state_blocking_time}, both service completions and 
class 1 arrivals are considered. 
Thus, from a blocked individual's perspective whenever the system moves from one 
state \((u,v)\)
to another state it can either:

\begin{itemize}
    \item be because of a service being completed: we will denote the probability 
    of this happening by \(p_s(u,v)\). 
    \item be because of an arrival of an individual of class 1: denoting such 
    probability by \(p_o(u,v)\).
\end{itemize}
The probabilities are given by:

\begin{equation*}
    p_s(u,v) = \frac{\min(v,C)\mu}{\lambda_1 + \min(v,C)\mu}, \qquad
    p_o(u,v) = \frac{\lambda_1}{\lambda_1 + \min(v,C)\mu}
\end{equation*}


Having defined \(c(u,v)\) and \(S_b\) a formula for the blocking time that is
expected to occur at each state can be given by:

\begin{equation}\label{eq:blocking-time-at-each-state}
    b(u,v) = 
    \begin{cases} 
        0, & \textbf{if } (u,v) \notin S_b \\
        c(u,v) + b(u - 1, v), & \textbf{if } v = N = T\\
        c(u,v) + b(u, v-1), & \textbf{if } v = N \neq T \\
        c(u,v) + p_s(u,v) b(u-1, v) + p_o(u,v) b(u, v+1), & \textbf{if } u > 0 
        \textbf{ and } v = T \\
        c(u,v) + p_s(u,v) b(u, v-1) + p_o(u,v) b(u, v+1), & \textbf{otherwise} \\
    \end{cases}
\end{equation}

Equation 
(\ref{eq:blocking-time-at-each-state}) will not be solved recursively. 
A direct approach will be used to solve this equation here. 
By enumerating all equations of (\ref{eq:blocking-time-at-each-state}) for all 
states \((u,v)\) that belong in \(S_b\) 
a system of linear equations arises where the unknown variables are all the \(b(u,v)\)
terms.
For instance, let us consider a Markov model where \(C=2, T=3, N=6, M=2\). 
The Markov model is shown in Figure \ref{fig:example-algeb-blocking}
and the equivalent equations are 
(\ref{eq:first_eq_of_blocking_example})-(\ref{eq:last_eq_of_blocking_example}).
The equations considered here are only the ones that correspond to the blocking 
states.

\begin{multicols*}{2}
    \begin{figure}[H]
        \scalebox{0.50}{\input{MarkovChain/expressions_from_pi/example_model_2362/main.tex}}
        \caption{Example of Markov chain}
        \label{fig:example-algeb-blocking}
    \end{figure}
    \columnbreak
    \begin{align}
        b(1,2) &= c(1,2) + p_o b(1,3) \label{eq:first_eq_of_blocking_example} \\
        b(1,3) &= c(1,3) + p_s b(1,2) + p_o b(1,4) \\
        b(1,4) &= c(1,4) + b(1,3) \\
        b(2,2) &= c(2,2) + p_s b(1,2) + p_o b(2,3) \\
        b(2,3) &= c(2,3) + p_s b(2,2) + p_o b(1,4) \\
        b(2,4) &= c(2,4) + b(2,3)\label{eq:last_eq_of_blocking_example}
    \end{align}
\end{multicols*}

Additionally, the above equations can be transformed into a linear system of the 
form \(Zx=y\) where:

\begin{equation}\label{eq:example-algebaric-approach-blocking-time}
    Z=
    \begin{pmatrix}
        -1 & p_o & 0 & 0 & 0 & 0 \\ %(1,2)
        p_s & -1 & p_o & 0 & 0 & 0 \\ %(1,3)
        0 & 1 & -1 & 0 & 0 & 0 \\ %(1,4)
        p_s & 0 & 0 & -1 & p_o & 0\\ %(2,2)
        0 & 0 & 0 & p_s & -1 & p_o \\ %(2,3)
        0 & 0 & 0 & 0 & 1 & -1 \\ %(2,4)
    \end{pmatrix},
    x=
    \begin{pmatrix}
        b(1,2) \\
        b(1,3) \\
        b(1,4) \\
        b(2,2) \\
        b(2,3) \\
        b(2,4) \\
    \end{pmatrix}, 
    y=
    \begin{pmatrix}
        -c(1,2) \\
        -c(1,3) \\
        -c(1,4) \\
        -c(2,2) \\
        -c(2,3) \\
        -c(2,4) \\
    \end{pmatrix}
\end{equation}

A more generalised form of the equations in 
(\ref{eq:example-algebaric-approach-blocking-time})
can thus be given for any value of \(C,T,N,M\) by:

\begin{align}
    b(1,T) =& c(1, T) + p_o b(1, T + 1) \label{eq:first_eq_of_blocking_general}\\
    b(1,T + 1) =& c(1, T + 1) + p_s(1, T) + p_o b(1, T + 1) \\
    b(1,T + 2) =& c(1, T + 2) + p_s(1, T + 1) + p_o b(1, T + 3) \\
    & \vdots \nonumber \\
    b(1, N) =& c(1, N) + b(1, N - 1) \\
    b(2, T) =& c(2, T) + p_s b(1, T) + p_o b(2, T + 1) \\
    b(2, T + 1) =& c(2, T + 1) + p_s b(2, T) + p_o b(2, T + 2) \\
    & \vdots \nonumber \\
    b(M, T) =& c(M, T) + b(M, T-1) \label{eq:last_eq_of_blocking_general}
\end{align}

The equivalent matrix form of the linear system of equations 
(\ref{eq:first_eq_of_blocking_general}) - (\ref{eq:last_eq_of_blocking_general})
is given by \(Zx=y\), where:
\begin{equation}\label{eq:general-algebaric-approach-blocking-time}
    \scalebox{0.9}{
        \(
        Z = 
        \begin{pmatrix}
            -1 & p_o & 0 & \dots & 0 & 0 & 0 & 0 & 0 & \dots & 0 & 0 \\ %(1,T)
            p_s & -1 & p_o & \dots & 0 & 0 & 0 & 0 & 0 & \dots & 0 & 0 \\ %(1,T+1)
            0 & p_s & -1 & \dots & 0 & 0 & 0 & 0 & 0 & \dots & 0 & 0 \\ %(1,T+2)
            \vdots & \vdots & \vdots & \ddots & \vdots & \vdots & \vdots & \vdots & 
            \vdots & \ddots & \vdots & \vdots \\ 
            0 & 0 & 0 & \dots & 1 & -1 & 0 & 0 & 0 & \dots & 0 & 0 \\ %(1,N)
            p_s & 0 & 0 & \dots & 0 & 0 & -1 & p_o & 0 & \dots & 0 & 0 \\ %(2,T)
            0 & 0 & 0 & \dots & 0 & 0 & p_s & -1 & p_o & \dots & 0 & 0 \\ %(2,T+1)
            \vdots & \vdots & \vdots & \ddots & \vdots & \vdots & \vdots & \vdots & 
            \vdots & \ddots & \vdots & \vdots \\ 
            0 & 0 & 0 & \dots & 0 & 0 & 0 & 0 & 0 & \dots & 1 & -1 \\ %(M,T)
        \end{pmatrix},
        x = 
        \begin{pmatrix}
            b(1,T) \\
            b(1,T+1) \\
            b(1,T+2) \\
            \vdots \\
            b(1,N) \\
            b(2,T) \\
            b(2,T+1) \\
            \vdots \\
            b(M,T) \\
        \end{pmatrix}, 
        y= 
        \begin{pmatrix}
            -c(1,T) \\
            -c(1,T+1) \\
            -c(1,T+2) \\
            \vdots \\
            -c(1,N) \\
            -c(2,T) \\
            -c(2,T+1) \\
            \vdots \\
            -c(M,T) \\
        \end{pmatrix}
        \)
    }
\end{equation}

Thus, having calculated the mean blocking time for all blocking states \(b(u,v)\), 
it only remains to put them together in a formula.
The resultant blocking time formula is given by:

\begin{equation}\label{eq:algebraic-blocking-time}
    B = \frac{\sum_{(u,v) \in S_A} \pi_{(u,v)} \; b(u,v)}{\sum_{(u,v) \in S_A} 
    \pi_{(u,v)}}
\end{equation}

\documentclass{article}

\usepackage{amsmath}
\usepackage{amsfonts} 
\usepackage{geometry}
\usepackage{multicol}
\usepackage{float}
% \usepackage{mathtools}
% \usepackage{graphicx}
% \usepackage{soul}
% \usepackage{indentfirst}
\usepackage{tikz}
\usetikzlibrary{calc, automata, chains, arrows.meta, math}
\setcounter{MaxMatrixCols}{20}


\title{A game theoretic model of the behavioural gaming that takes place at the EMS - ED interface}

\author{
    Michalis Panayides, 
    Paul Harper, 
    Vince Knight
}

\begin{document}

\maketitle

\input{Abstract/main.tex}


\newpage
\tableofcontents

\newpage
\input{Introduction/main.tex}

\newpage
\input{Game_theory_component/main.tex}

\newpage
\input{MarkovChain/markov_chain_model/main.tex}
\input{MarkovChain/expressions_from_pi/main.tex}
\input{MarkovChain/markov_example/main.tex}

\newpage
\input{BehaviouralMethodology/main.tex}

\newpage
\input{Application_EMS_ED/main.tex}

\newpage
\input{Conclusion/main.tex}


\end{document}

\newpage
\documentclass{article}

\usepackage{amsmath}
\usepackage{amsfonts} 
\usepackage{geometry}
\usepackage{multicol}
\usepackage{float}
% \usepackage{mathtools}
% \usepackage{graphicx}
% \usepackage{soul}
% \usepackage{indentfirst}
\usepackage{tikz}
\usetikzlibrary{calc, automata, chains, arrows.meta, math}
\setcounter{MaxMatrixCols}{20}


\title{A game theoretic model of the behavioural gaming that takes place at the EMS - ED interface}

\author{
    Michalis Panayides, 
    Paul Harper, 
    Vince Knight
}

\begin{document}

\maketitle

\input{Abstract/main.tex}


\newpage
\tableofcontents

\newpage
\input{Introduction/main.tex}

\newpage
\input{Game_theory_component/main.tex}

\newpage
\input{MarkovChain/markov_chain_model/main.tex}
\input{MarkovChain/expressions_from_pi/main.tex}
\input{MarkovChain/markov_example/main.tex}

\newpage
\input{BehaviouralMethodology/main.tex}

\newpage
\input{Application_EMS_ED/main.tex}

\newpage
\input{Conclusion/main.tex}


\end{document}

\newpage
\section{EMS-ED application}

\subsection{Application}

\subsection{Data analysis of generated problem}

\newpage
\documentclass{article}

\usepackage{amsmath}
\usepackage{amsfonts} 
\usepackage{geometry}
\usepackage{multicol}
\usepackage{float}
% \usepackage{mathtools}
% \usepackage{graphicx}
% \usepackage{soul}
% \usepackage{indentfirst}
\usepackage{tikz}
\usetikzlibrary{calc, automata, chains, arrows.meta, math}
\setcounter{MaxMatrixCols}{20}


\title{A game theoretic model of the behavioural gaming that takes place at the EMS - ED interface}

\author{
    Michalis Panayides, 
    Paul Harper, 
    Vince Knight
}

\begin{document}

\maketitle

\input{Abstract/main.tex}


\newpage
\tableofcontents

\newpage
\input{Introduction/main.tex}

\newpage
\input{Game_theory_component/main.tex}

\newpage
\input{MarkovChain/markov_chain_model/main.tex}
\input{MarkovChain/expressions_from_pi/main.tex}
\input{MarkovChain/markov_example/main.tex}

\newpage
\input{BehaviouralMethodology/main.tex}

\newpage
\input{Application_EMS_ED/main.tex}

\newpage
\input{Conclusion/main.tex}


\end{document}


\end{document}

\newpage
\documentclass{article}

\usepackage{amsmath}
\usepackage{amsfonts} 
\usepackage{geometry}
\usepackage{multicol}
\usepackage{float}
% \usepackage{mathtools}
% \usepackage{graphicx}
% \usepackage{soul}
% \usepackage{indentfirst}
\usepackage{tikz}
\usetikzlibrary{calc, automata, chains, arrows.meta, math}
\setcounter{MaxMatrixCols}{20}


\title{A game theoretic model of the behavioural gaming that takes place at the EMS - ED interface}

\author{
    Michalis Panayides, 
    Paul Harper, 
    Vince Knight
}

\begin{document}

\maketitle

\documentclass{article}

\usepackage{amsmath}
\usepackage{amsfonts} 
\usepackage{geometry}
\usepackage{multicol}
\usepackage{float}
% \usepackage{mathtools}
% \usepackage{graphicx}
% \usepackage{soul}
% \usepackage{indentfirst}
\usepackage{tikz}
\usetikzlibrary{calc, automata, chains, arrows.meta, math}
\setcounter{MaxMatrixCols}{20}


\title{A game theoretic model of the behavioural gaming that takes place at the EMS - ED interface}

\author{
    Michalis Panayides, 
    Paul Harper, 
    Vince Knight
}

\begin{document}

\maketitle

\input{Abstract/main.tex}


\newpage
\tableofcontents

\newpage
\input{Introduction/main.tex}

\newpage
\input{Game_theory_component/main.tex}

\newpage
\input{MarkovChain/markov_chain_model/main.tex}
\input{MarkovChain/expressions_from_pi/main.tex}
\input{MarkovChain/markov_example/main.tex}

\newpage
\input{BehaviouralMethodology/main.tex}

\newpage
\input{Application_EMS_ED/main.tex}

\newpage
\input{Conclusion/main.tex}


\end{document}


\newpage
\tableofcontents

\newpage
\documentclass{article}

\usepackage{amsmath}
\usepackage{amsfonts} 
\usepackage{geometry}
\usepackage{multicol}
\usepackage{float}
% \usepackage{mathtools}
% \usepackage{graphicx}
% \usepackage{soul}
% \usepackage{indentfirst}
\usepackage{tikz}
\usetikzlibrary{calc, automata, chains, arrows.meta, math}
\setcounter{MaxMatrixCols}{20}


\title{A game theoretic model of the behavioural gaming that takes place at the EMS - ED interface}

\author{
    Michalis Panayides, 
    Paul Harper, 
    Vince Knight
}

\begin{document}

\maketitle

\input{Abstract/main.tex}


\newpage
\tableofcontents

\newpage
\input{Introduction/main.tex}

\newpage
\input{Game_theory_component/main.tex}

\newpage
\input{MarkovChain/markov_chain_model/main.tex}
\input{MarkovChain/expressions_from_pi/main.tex}
\input{MarkovChain/markov_example/main.tex}

\newpage
\input{BehaviouralMethodology/main.tex}

\newpage
\input{Application_EMS_ED/main.tex}

\newpage
\input{Conclusion/main.tex}


\end{document}

\newpage
\documentclass{article}

\usepackage{amsmath}
\usepackage{amsfonts} 
\usepackage{geometry}
\usepackage{multicol}
\usepackage{float}
% \usepackage{mathtools}
% \usepackage{graphicx}
% \usepackage{soul}
% \usepackage{indentfirst}
\usepackage{tikz}
\usetikzlibrary{calc, automata, chains, arrows.meta, math}
\setcounter{MaxMatrixCols}{20}


\title{A game theoretic model of the behavioural gaming that takes place at the EMS - ED interface}

\author{
    Michalis Panayides, 
    Paul Harper, 
    Vince Knight
}

\begin{document}

\maketitle

\input{Abstract/main.tex}


\newpage
\tableofcontents

\newpage
\input{Introduction/main.tex}

\newpage
\input{Game_theory_component/main.tex}

\newpage
\input{MarkovChain/markov_chain_model/main.tex}
\input{MarkovChain/expressions_from_pi/main.tex}
\input{MarkovChain/markov_example/main.tex}

\newpage
\input{BehaviouralMethodology/main.tex}

\newpage
\input{Application_EMS_ED/main.tex}

\newpage
\input{Conclusion/main.tex}


\end{document}

\newpage
\documentclass{article}

\usepackage{amsmath}
\usepackage{amsfonts} 
\usepackage{geometry}
\usepackage{multicol}
\usepackage{float}
% \usepackage{mathtools}
% \usepackage{graphicx}
% \usepackage{soul}
% \usepackage{indentfirst}
\usepackage{tikz}
\usetikzlibrary{calc, automata, chains, arrows.meta, math}
\setcounter{MaxMatrixCols}{20}


\title{A game theoretic model of the behavioural gaming that takes place at the EMS - ED interface}

\author{
    Michalis Panayides, 
    Paul Harper, 
    Vince Knight
}

\begin{document}

\maketitle

\input{Abstract/main.tex}


\newpage
\tableofcontents

\newpage
\input{Introduction/main.tex}

\newpage
\input{Game_theory_component/main.tex}

\newpage
\input{MarkovChain/markov_chain_model/main.tex}
\input{MarkovChain/expressions_from_pi/main.tex}
\input{MarkovChain/markov_example/main.tex}

\newpage
\input{BehaviouralMethodology/main.tex}

\newpage
\input{Application_EMS_ED/main.tex}

\newpage
\input{Conclusion/main.tex}


\end{document}
\subsection{Performance Measures}
One may easily derive the average number of individuals that are at any given state 
using \( pi \). 
The average number of individuals in state \( i \) can be calculated by multiplying 
the number of individuals that are present in state \( i \) with the probability 
of being at that particular state (i.e \(\pi_i (u_i + v_i)\)). 
Using this logic it is possible to calculate any performance measures that are related 
to the mean number of individuals in the system.


Average number of people in the system: 
\begin{equation}
    L = \sum_{i=1}^{|\pi|} \pi_i (u_i + v_i)
\end{equation} 

Average number of people in the service centre: 
\begin{equation}
    L_H = \sum_{i=1}^{|\pi|} \pi_i v_i
\end{equation}

Average number of people in the buffer centre:
\begin{equation}
    L_A = \sum_{i=1}^{|\pi|} \pi_i u_i
\end{equation}

Consequently getting the performance measures that are related to the duration of 
time is not as straightforward. 
Such performance measures are the mean waiting time in the system and the mean time 
blocked in the system. 
Under the scope of this study three approaches have been considered to calculate these 
performance measures; a direct approach, a recursive algorithm and consequently a
closed-form formula.

The research question that needs to be answered here is: ``When a class 1/2 
individuals enters the system, what is the expected time that they will have to 
wait?''. 
In order to formulate the answer to that question one needs to consider all possible 
scenarios of what state the system can be in when an individual arrives. 
Furthermore, different formulas arises for class 1 individuals 
and a different one for class 2 individuals.

\subsubsection{Mean waiting time} 
Upon closer inspection of the recursive formula a more compact formula can arise. 
The equivalent closed-form formula eliminates the need for recursion and thus makes 
the computation of waiting times much more efficient. 
Just like in the recursive part there are two formulas; one for \textit{class 1} 
and one for class 2 individuals. 
The formulas are given by:

\begin{equation} \label{eq:closed_form_waiting_others}
    W^{(1)} = \frac{\sum_{\substack{(u,v) \, \in S_A^{(1)} \\ v \geq C}} 
    \frac{1}{C \mu} \times (v-C+1) \times \pi(u,v)}{\sum_{(u,v) \, 
    \in S_A^{(1)}} \pi(u,v)}
\end{equation}
    
\begin{equation}\label{eq:closed_form_waiting_ambulance}
    W^{(2)} = \frac{\sum_{\substack{(u,v) \, \in S_A^{(2)} \\ min(v,T) \geq C}} 
    \frac{1}{C \mu} \times (\min(v+1,T)-C) \times \pi(u,v)}{\sum_{(u,v) \, 
    \in S_A^{(2)}} \pi(u,v)}
\end{equation}

Note here that the summation, in both equations \ref{eq:closed_form_waiting_others} 
and \ref{eq:closed_form_waiting_ambulance}, goes through all states in the set of 
accepting 
states of either class 1 or class 2 individuals respectively, where a wait 
incurs. 
In equation \ref{eq:closed_form_waiting_others} that includes all states \((u,v)\) 
in the set of accepting states of class 1 individuals such that \( v \geq C\); i.e. 
whenever an arrival occurs and the system is at a state where the number of individuals 
in the system is more than or equal to $C$. 
Consequently, for the states that are included in the summation the expression 
\( v-C+1 \) indicates the amount of people in service one would have to wait for 
upon arrival at the hospital.

Additionally, the minimisation function in equation 
\ref{eq:closed_form_waiting_ambulance} 
ensures that when a class 2 individual arrives at any state 
that is greater than the predetermined threshold, the wait that the individual will 
have to endure remains the same. 
In essence, the expression \(\min(v+1,T) - C\) returns the number of people in line 
in front of a particular individual upon arrival.


\subsubsection{Overall Waiting Time}

Consequently, the overall waiting time should can be estimated by a linear combination 
of the waiting times of class 1 and class 2 individuals. 
The overall waiting time can be then given by the following equation where \(c_1\) 
and \(c_2\) are the coefficients of each individual's type waiting time:

\begin{equation}\label{overall_waiting_time_coeff}
    W = c_1 W^{(1)} + c_2 W^{(2)}
\end{equation}

The two coefficients represent the proportion of individuals of each type that 
traversed through the model. 
Theoretically, getting these percentages should be as simple as looking at the arrival 
rates of each type but in practise if the service centre or the buffer centre 
is full, some individuals may be lost to the system. 
Thus, one should account for the probability that an individual is lost to the system. 
This probability can be easily calculated by using the two sets of accepting states 
\(S_A^{(2)}\) and \(S_A^{(1)}\) defined earlier in equations.
Let us define here the probability, for either class type, that an individual 
is not lost in the system by:

\begin{equation*}
    P(L'_1) = \sum_{(u,v) \, \in S_A^{(1)}} \pi(u,v) \hspace{2cm}
    P(L'_2) = \sum_{(u,v) \, \in S_A^{(2)}} \pi(u,v)
\end{equation*}

Having defined these probabilities one may combine them with the arrival rates of 
each class type in such a way to get the expected proportions of class 1 and 
class 2 individuals in the model. 
Thus, by using these values as the coefficient of equation 
\ref{overall_waiting_time_coeff} 
the resultant equation can be used to get the overall waiting time. 
Note here that the equation below gets the overall waiting time for both the recursive 
and the closed-form formula.

\begin{equation}\label{overall_waiting_time}
    W = \frac{\lambda_1 P(L'_1)}{\lambda_2 P(L'_2) + \lambda_1 P(L'_1)} W^{(1)} + 
    \frac{\lambda_2 P(L'_2)}{\lambda_2 P(L'_2) + \lambda_1 P(L'_1)} W^{(2)}
\end{equation}



\subsubsection{Mean blocking time}
Unlike the waiting time, the blocking time is only calculated for class 2 individuals.  
That is because class 1 individuals cannot be blocked. 
Thus, one only needs to consider the pathway of class 2 individuals to get the 
mean blocking time of the system. 
Blocking occurs at states \((u,v)\) where \(u > 0 \). 
Thus, the set of blocking states can be defined as:

\begin{equation*}
    S_b = \{(u,v) \in S \; | \; u > 0\}
\end{equation*}
 
In order to not consider individuals that will be lost to the system, the set of 
accepting states needs to be taken into account. The set of accepting states is given by:

\begin{equation*}
    S_A^{(2)}=
    \begin{cases}
        \{(u, v) \in S \; | \; u < M \} & \textbf{if } T \leq N\\
        \{(u, v) \in S \; | \; v < N \} & \textbf{otherwise}
    \end{cases}
\end{equation*}

For the waiting time formula,
the mean sojourn time for each state was considered,
ignoring any arrivals. Here, the same approach is used but ignoring only class 2
arrivals. That is because for the waiting time formula, once an individual enters 
the service centre (i.e. starts waiting) any individual arriving after them will 
not affect their
pathway. That is not the case for blocking time. When a class 2 individual is 
blocked, 
any class 1 individual that arrives will cause the blocked individual to remain 
blocked for more time. Therefore, class 1 arrivals are considered here:

\begin{equation}\label{eq:time_in_state_blocking_time}
    c(u,v) = 
    \begin{cases}
        \frac{1}{\min(v,C) \mu}, & \text{if } v = C\\
        \frac{1}{\min(v,C) \mu + \lambda_1}, & \text{otherwise}
    \end{cases}
\end{equation}
 
In equation \ref{eq:time_in_state_blocking_time}, both service completions and 
class 1 arrivals are considered. 
Thus, from a blocked individual's perspective whenever the system moves from one 
state \((u,v)\)
to another state it can either:

\begin{itemize}
    \item be because of a service being completed: we will denote the probability 
    of this happening by \(p_s(u,v)\). 
    \item be because of an arrival of an individual of class 1: denoting such 
    probability by \(p_o(u,v)\).
\end{itemize}
The probabilities are given by:

\begin{equation*}
    p_s(u,v) = \frac{\min(v,C)\mu}{\lambda_1 + \min(v,C)\mu}, \qquad
    p_o(u,v) = \frac{\lambda_1}{\lambda_1 + \min(v,C)\mu}
\end{equation*}


Having defined \(c(u,v)\) and \(S_b\) a formula for the blocking time that is
expected to occur at each state can be given by:

\begin{equation}\label{eq:blocking-time-at-each-state}
    b(u,v) = 
    \begin{cases} 
        0, & \textbf{if } (u,v) \notin S_b \\
        c(u,v) + b(u - 1, v), & \textbf{if } v = N = T\\
        c(u,v) + b(u, v-1), & \textbf{if } v = N \neq T \\
        c(u,v) + p_s(u,v) b(u-1, v) + p_o(u,v) b(u, v+1), & \textbf{if } u > 0 
        \textbf{ and } v = T \\
        c(u,v) + p_s(u,v) b(u, v-1) + p_o(u,v) b(u, v+1), & \textbf{otherwise} \\
    \end{cases}
\end{equation}

Equation 
(\ref{eq:blocking-time-at-each-state}) will not be solved recursively. 
A direct approach will be used to solve this equation here. 
By enumerating all equations of (\ref{eq:blocking-time-at-each-state}) for all 
states \((u,v)\) that belong in \(S_b\) 
a system of linear equations arises where the unknown variables are all the \(b(u,v)\)
terms.
For instance, let us consider a Markov model where \(C=2, T=3, N=6, M=2\). 
The Markov model is shown in Figure \ref{fig:example-algeb-blocking}
and the equivalent equations are 
(\ref{eq:first_eq_of_blocking_example})-(\ref{eq:last_eq_of_blocking_example}).
The equations considered here are only the ones that correspond to the blocking 
states.

\begin{multicols*}{2}
    \begin{figure}[H]
        \scalebox{0.50}{\input{MarkovChain/expressions_from_pi/example_model_2362/main.tex}}
        \caption{Example of Markov chain}
        \label{fig:example-algeb-blocking}
    \end{figure}
    \columnbreak
    \begin{align}
        b(1,2) &= c(1,2) + p_o b(1,3) \label{eq:first_eq_of_blocking_example} \\
        b(1,3) &= c(1,3) + p_s b(1,2) + p_o b(1,4) \\
        b(1,4) &= c(1,4) + b(1,3) \\
        b(2,2) &= c(2,2) + p_s b(1,2) + p_o b(2,3) \\
        b(2,3) &= c(2,3) + p_s b(2,2) + p_o b(1,4) \\
        b(2,4) &= c(2,4) + b(2,3)\label{eq:last_eq_of_blocking_example}
    \end{align}
\end{multicols*}

Additionally, the above equations can be transformed into a linear system of the 
form \(Zx=y\) where:

\begin{equation}\label{eq:example-algebaric-approach-blocking-time}
    Z=
    \begin{pmatrix}
        -1 & p_o & 0 & 0 & 0 & 0 \\ %(1,2)
        p_s & -1 & p_o & 0 & 0 & 0 \\ %(1,3)
        0 & 1 & -1 & 0 & 0 & 0 \\ %(1,4)
        p_s & 0 & 0 & -1 & p_o & 0\\ %(2,2)
        0 & 0 & 0 & p_s & -1 & p_o \\ %(2,3)
        0 & 0 & 0 & 0 & 1 & -1 \\ %(2,4)
    \end{pmatrix},
    x=
    \begin{pmatrix}
        b(1,2) \\
        b(1,3) \\
        b(1,4) \\
        b(2,2) \\
        b(2,3) \\
        b(2,4) \\
    \end{pmatrix}, 
    y=
    \begin{pmatrix}
        -c(1,2) \\
        -c(1,3) \\
        -c(1,4) \\
        -c(2,2) \\
        -c(2,3) \\
        -c(2,4) \\
    \end{pmatrix}
\end{equation}

A more generalised form of the equations in 
(\ref{eq:example-algebaric-approach-blocking-time})
can thus be given for any value of \(C,T,N,M\) by:

\begin{align}
    b(1,T) =& c(1, T) + p_o b(1, T + 1) \label{eq:first_eq_of_blocking_general}\\
    b(1,T + 1) =& c(1, T + 1) + p_s(1, T) + p_o b(1, T + 1) \\
    b(1,T + 2) =& c(1, T + 2) + p_s(1, T + 1) + p_o b(1, T + 3) \\
    & \vdots \nonumber \\
    b(1, N) =& c(1, N) + b(1, N - 1) \\
    b(2, T) =& c(2, T) + p_s b(1, T) + p_o b(2, T + 1) \\
    b(2, T + 1) =& c(2, T + 1) + p_s b(2, T) + p_o b(2, T + 2) \\
    & \vdots \nonumber \\
    b(M, T) =& c(M, T) + b(M, T-1) \label{eq:last_eq_of_blocking_general}
\end{align}

The equivalent matrix form of the linear system of equations 
(\ref{eq:first_eq_of_blocking_general}) - (\ref{eq:last_eq_of_blocking_general})
is given by \(Zx=y\), where:
\begin{equation}\label{eq:general-algebaric-approach-blocking-time}
    \scalebox{0.9}{
        \(
        Z = 
        \begin{pmatrix}
            -1 & p_o & 0 & \dots & 0 & 0 & 0 & 0 & 0 & \dots & 0 & 0 \\ %(1,T)
            p_s & -1 & p_o & \dots & 0 & 0 & 0 & 0 & 0 & \dots & 0 & 0 \\ %(1,T+1)
            0 & p_s & -1 & \dots & 0 & 0 & 0 & 0 & 0 & \dots & 0 & 0 \\ %(1,T+2)
            \vdots & \vdots & \vdots & \ddots & \vdots & \vdots & \vdots & \vdots & 
            \vdots & \ddots & \vdots & \vdots \\ 
            0 & 0 & 0 & \dots & 1 & -1 & 0 & 0 & 0 & \dots & 0 & 0 \\ %(1,N)
            p_s & 0 & 0 & \dots & 0 & 0 & -1 & p_o & 0 & \dots & 0 & 0 \\ %(2,T)
            0 & 0 & 0 & \dots & 0 & 0 & p_s & -1 & p_o & \dots & 0 & 0 \\ %(2,T+1)
            \vdots & \vdots & \vdots & \ddots & \vdots & \vdots & \vdots & \vdots & 
            \vdots & \ddots & \vdots & \vdots \\ 
            0 & 0 & 0 & \dots & 0 & 0 & 0 & 0 & 0 & \dots & 1 & -1 \\ %(M,T)
        \end{pmatrix},
        x = 
        \begin{pmatrix}
            b(1,T) \\
            b(1,T+1) \\
            b(1,T+2) \\
            \vdots \\
            b(1,N) \\
            b(2,T) \\
            b(2,T+1) \\
            \vdots \\
            b(M,T) \\
        \end{pmatrix}, 
        y= 
        \begin{pmatrix}
            -c(1,T) \\
            -c(1,T+1) \\
            -c(1,T+2) \\
            \vdots \\
            -c(1,N) \\
            -c(2,T) \\
            -c(2,T+1) \\
            \vdots \\
            -c(M,T) \\
        \end{pmatrix}
        \)
    }
\end{equation}

Thus, having calculated the mean blocking time for all blocking states \(b(u,v)\), 
it only remains to put them together in a formula.
The resultant blocking time formula is given by:

\begin{equation}\label{eq:algebraic-blocking-time}
    B = \frac{\sum_{(u,v) \in S_A} \pi_{(u,v)} \; b(u,v)}{\sum_{(u,v) \in S_A} 
    \pi_{(u,v)}}
\end{equation}

\documentclass{article}

\usepackage{amsmath}
\usepackage{amsfonts} 
\usepackage{geometry}
\usepackage{multicol}
\usepackage{float}
% \usepackage{mathtools}
% \usepackage{graphicx}
% \usepackage{soul}
% \usepackage{indentfirst}
\usepackage{tikz}
\usetikzlibrary{calc, automata, chains, arrows.meta, math}
\setcounter{MaxMatrixCols}{20}


\title{A game theoretic model of the behavioural gaming that takes place at the EMS - ED interface}

\author{
    Michalis Panayides, 
    Paul Harper, 
    Vince Knight
}

\begin{document}

\maketitle

\input{Abstract/main.tex}


\newpage
\tableofcontents

\newpage
\input{Introduction/main.tex}

\newpage
\input{Game_theory_component/main.tex}

\newpage
\input{MarkovChain/markov_chain_model/main.tex}
\input{MarkovChain/expressions_from_pi/main.tex}
\input{MarkovChain/markov_example/main.tex}

\newpage
\input{BehaviouralMethodology/main.tex}

\newpage
\input{Application_EMS_ED/main.tex}

\newpage
\input{Conclusion/main.tex}


\end{document}

\newpage
\documentclass{article}

\usepackage{amsmath}
\usepackage{amsfonts} 
\usepackage{geometry}
\usepackage{multicol}
\usepackage{float}
% \usepackage{mathtools}
% \usepackage{graphicx}
% \usepackage{soul}
% \usepackage{indentfirst}
\usepackage{tikz}
\usetikzlibrary{calc, automata, chains, arrows.meta, math}
\setcounter{MaxMatrixCols}{20}


\title{A game theoretic model of the behavioural gaming that takes place at the EMS - ED interface}

\author{
    Michalis Panayides, 
    Paul Harper, 
    Vince Knight
}

\begin{document}

\maketitle

\input{Abstract/main.tex}


\newpage
\tableofcontents

\newpage
\input{Introduction/main.tex}

\newpage
\input{Game_theory_component/main.tex}

\newpage
\input{MarkovChain/markov_chain_model/main.tex}
\input{MarkovChain/expressions_from_pi/main.tex}
\input{MarkovChain/markov_example/main.tex}

\newpage
\input{BehaviouralMethodology/main.tex}

\newpage
\input{Application_EMS_ED/main.tex}

\newpage
\input{Conclusion/main.tex}


\end{document}

\newpage
\section{EMS-ED application}

\subsection{Application}

\subsection{Data analysis of generated problem}

\newpage
\documentclass{article}

\usepackage{amsmath}
\usepackage{amsfonts} 
\usepackage{geometry}
\usepackage{multicol}
\usepackage{float}
% \usepackage{mathtools}
% \usepackage{graphicx}
% \usepackage{soul}
% \usepackage{indentfirst}
\usepackage{tikz}
\usetikzlibrary{calc, automata, chains, arrows.meta, math}
\setcounter{MaxMatrixCols}{20}


\title{A game theoretic model of the behavioural gaming that takes place at the EMS - ED interface}

\author{
    Michalis Panayides, 
    Paul Harper, 
    Vince Knight
}

\begin{document}

\maketitle

\input{Abstract/main.tex}


\newpage
\tableofcontents

\newpage
\input{Introduction/main.tex}

\newpage
\input{Game_theory_component/main.tex}

\newpage
\input{MarkovChain/markov_chain_model/main.tex}
\input{MarkovChain/expressions_from_pi/main.tex}
\input{MarkovChain/markov_example/main.tex}

\newpage
\input{BehaviouralMethodology/main.tex}

\newpage
\input{Application_EMS_ED/main.tex}

\newpage
\input{Conclusion/main.tex}


\end{document}


\end{document}

\newpage
\section{EMS-ED application}

\subsection{Application}

\subsection{Data analysis of generated problem}

\newpage
\documentclass{article}

\usepackage{amsmath}
\usepackage{amsfonts} 
\usepackage{geometry}
\usepackage{multicol}
\usepackage{float}
% \usepackage{mathtools}
% \usepackage{graphicx}
% \usepackage{soul}
% \usepackage{indentfirst}
\usepackage{tikz}
\usetikzlibrary{calc, automata, chains, arrows.meta, math}
\setcounter{MaxMatrixCols}{20}


\title{A game theoretic model of the behavioural gaming that takes place at the EMS - ED interface}

\author{
    Michalis Panayides, 
    Paul Harper, 
    Vince Knight
}

\begin{document}

\maketitle

\documentclass{article}

\usepackage{amsmath}
\usepackage{amsfonts} 
\usepackage{geometry}
\usepackage{multicol}
\usepackage{float}
% \usepackage{mathtools}
% \usepackage{graphicx}
% \usepackage{soul}
% \usepackage{indentfirst}
\usepackage{tikz}
\usetikzlibrary{calc, automata, chains, arrows.meta, math}
\setcounter{MaxMatrixCols}{20}


\title{A game theoretic model of the behavioural gaming that takes place at the EMS - ED interface}

\author{
    Michalis Panayides, 
    Paul Harper, 
    Vince Knight
}

\begin{document}

\maketitle

\input{Abstract/main.tex}


\newpage
\tableofcontents

\newpage
\input{Introduction/main.tex}

\newpage
\input{Game_theory_component/main.tex}

\newpage
\input{MarkovChain/markov_chain_model/main.tex}
\input{MarkovChain/expressions_from_pi/main.tex}
\input{MarkovChain/markov_example/main.tex}

\newpage
\input{BehaviouralMethodology/main.tex}

\newpage
\input{Application_EMS_ED/main.tex}

\newpage
\input{Conclusion/main.tex}


\end{document}


\newpage
\tableofcontents

\newpage
\documentclass{article}

\usepackage{amsmath}
\usepackage{amsfonts} 
\usepackage{geometry}
\usepackage{multicol}
\usepackage{float}
% \usepackage{mathtools}
% \usepackage{graphicx}
% \usepackage{soul}
% \usepackage{indentfirst}
\usepackage{tikz}
\usetikzlibrary{calc, automata, chains, arrows.meta, math}
\setcounter{MaxMatrixCols}{20}


\title{A game theoretic model of the behavioural gaming that takes place at the EMS - ED interface}

\author{
    Michalis Panayides, 
    Paul Harper, 
    Vince Knight
}

\begin{document}

\maketitle

\input{Abstract/main.tex}


\newpage
\tableofcontents

\newpage
\input{Introduction/main.tex}

\newpage
\input{Game_theory_component/main.tex}

\newpage
\input{MarkovChain/markov_chain_model/main.tex}
\input{MarkovChain/expressions_from_pi/main.tex}
\input{MarkovChain/markov_example/main.tex}

\newpage
\input{BehaviouralMethodology/main.tex}

\newpage
\input{Application_EMS_ED/main.tex}

\newpage
\input{Conclusion/main.tex}


\end{document}

\newpage
\documentclass{article}

\usepackage{amsmath}
\usepackage{amsfonts} 
\usepackage{geometry}
\usepackage{multicol}
\usepackage{float}
% \usepackage{mathtools}
% \usepackage{graphicx}
% \usepackage{soul}
% \usepackage{indentfirst}
\usepackage{tikz}
\usetikzlibrary{calc, automata, chains, arrows.meta, math}
\setcounter{MaxMatrixCols}{20}


\title{A game theoretic model of the behavioural gaming that takes place at the EMS - ED interface}

\author{
    Michalis Panayides, 
    Paul Harper, 
    Vince Knight
}

\begin{document}

\maketitle

\input{Abstract/main.tex}


\newpage
\tableofcontents

\newpage
\input{Introduction/main.tex}

\newpage
\input{Game_theory_component/main.tex}

\newpage
\input{MarkovChain/markov_chain_model/main.tex}
\input{MarkovChain/expressions_from_pi/main.tex}
\input{MarkovChain/markov_example/main.tex}

\newpage
\input{BehaviouralMethodology/main.tex}

\newpage
\input{Application_EMS_ED/main.tex}

\newpage
\input{Conclusion/main.tex}


\end{document}

\newpage
\documentclass{article}

\usepackage{amsmath}
\usepackage{amsfonts} 
\usepackage{geometry}
\usepackage{multicol}
\usepackage{float}
% \usepackage{mathtools}
% \usepackage{graphicx}
% \usepackage{soul}
% \usepackage{indentfirst}
\usepackage{tikz}
\usetikzlibrary{calc, automata, chains, arrows.meta, math}
\setcounter{MaxMatrixCols}{20}


\title{A game theoretic model of the behavioural gaming that takes place at the EMS - ED interface}

\author{
    Michalis Panayides, 
    Paul Harper, 
    Vince Knight
}

\begin{document}

\maketitle

\input{Abstract/main.tex}


\newpage
\tableofcontents

\newpage
\input{Introduction/main.tex}

\newpage
\input{Game_theory_component/main.tex}

\newpage
\input{MarkovChain/markov_chain_model/main.tex}
\input{MarkovChain/expressions_from_pi/main.tex}
\input{MarkovChain/markov_example/main.tex}

\newpage
\input{BehaviouralMethodology/main.tex}

\newpage
\input{Application_EMS_ED/main.tex}

\newpage
\input{Conclusion/main.tex}


\end{document}
\subsection{Performance Measures}
One may easily derive the average number of individuals that are at any given state 
using \( pi \). 
The average number of individuals in state \( i \) can be calculated by multiplying 
the number of individuals that are present in state \( i \) with the probability 
of being at that particular state (i.e \(\pi_i (u_i + v_i)\)). 
Using this logic it is possible to calculate any performance measures that are related 
to the mean number of individuals in the system.


Average number of people in the system: 
\begin{equation}
    L = \sum_{i=1}^{|\pi|} \pi_i (u_i + v_i)
\end{equation} 

Average number of people in the service centre: 
\begin{equation}
    L_H = \sum_{i=1}^{|\pi|} \pi_i v_i
\end{equation}

Average number of people in the buffer centre:
\begin{equation}
    L_A = \sum_{i=1}^{|\pi|} \pi_i u_i
\end{equation}

Consequently getting the performance measures that are related to the duration of 
time is not as straightforward. 
Such performance measures are the mean waiting time in the system and the mean time 
blocked in the system. 
Under the scope of this study three approaches have been considered to calculate these 
performance measures; a direct approach, a recursive algorithm and consequently a
closed-form formula.

The research question that needs to be answered here is: ``When a class 1/2 
individuals enters the system, what is the expected time that they will have to 
wait?''. 
In order to formulate the answer to that question one needs to consider all possible 
scenarios of what state the system can be in when an individual arrives. 
Furthermore, different formulas arises for class 1 individuals 
and a different one for class 2 individuals.

\subsubsection{Mean waiting time} 
Upon closer inspection of the recursive formula a more compact formula can arise. 
The equivalent closed-form formula eliminates the need for recursion and thus makes 
the computation of waiting times much more efficient. 
Just like in the recursive part there are two formulas; one for \textit{class 1} 
and one for class 2 individuals. 
The formulas are given by:

\begin{equation} \label{eq:closed_form_waiting_others}
    W^{(1)} = \frac{\sum_{\substack{(u,v) \, \in S_A^{(1)} \\ v \geq C}} 
    \frac{1}{C \mu} \times (v-C+1) \times \pi(u,v)}{\sum_{(u,v) \, 
    \in S_A^{(1)}} \pi(u,v)}
\end{equation}
    
\begin{equation}\label{eq:closed_form_waiting_ambulance}
    W^{(2)} = \frac{\sum_{\substack{(u,v) \, \in S_A^{(2)} \\ min(v,T) \geq C}} 
    \frac{1}{C \mu} \times (\min(v+1,T)-C) \times \pi(u,v)}{\sum_{(u,v) \, 
    \in S_A^{(2)}} \pi(u,v)}
\end{equation}

Note here that the summation, in both equations \ref{eq:closed_form_waiting_others} 
and \ref{eq:closed_form_waiting_ambulance}, goes through all states in the set of 
accepting 
states of either class 1 or class 2 individuals respectively, where a wait 
incurs. 
In equation \ref{eq:closed_form_waiting_others} that includes all states \((u,v)\) 
in the set of accepting states of class 1 individuals such that \( v \geq C\); i.e. 
whenever an arrival occurs and the system is at a state where the number of individuals 
in the system is more than or equal to $C$. 
Consequently, for the states that are included in the summation the expression 
\( v-C+1 \) indicates the amount of people in service one would have to wait for 
upon arrival at the hospital.

Additionally, the minimisation function in equation 
\ref{eq:closed_form_waiting_ambulance} 
ensures that when a class 2 individual arrives at any state 
that is greater than the predetermined threshold, the wait that the individual will 
have to endure remains the same. 
In essence, the expression \(\min(v+1,T) - C\) returns the number of people in line 
in front of a particular individual upon arrival.


\subsubsection{Overall Waiting Time}

Consequently, the overall waiting time should can be estimated by a linear combination 
of the waiting times of class 1 and class 2 individuals. 
The overall waiting time can be then given by the following equation where \(c_1\) 
and \(c_2\) are the coefficients of each individual's type waiting time:

\begin{equation}\label{overall_waiting_time_coeff}
    W = c_1 W^{(1)} + c_2 W^{(2)}
\end{equation}

The two coefficients represent the proportion of individuals of each type that 
traversed through the model. 
Theoretically, getting these percentages should be as simple as looking at the arrival 
rates of each type but in practise if the service centre or the buffer centre 
is full, some individuals may be lost to the system. 
Thus, one should account for the probability that an individual is lost to the system. 
This probability can be easily calculated by using the two sets of accepting states 
\(S_A^{(2)}\) and \(S_A^{(1)}\) defined earlier in equations.
Let us define here the probability, for either class type, that an individual 
is not lost in the system by:

\begin{equation*}
    P(L'_1) = \sum_{(u,v) \, \in S_A^{(1)}} \pi(u,v) \hspace{2cm}
    P(L'_2) = \sum_{(u,v) \, \in S_A^{(2)}} \pi(u,v)
\end{equation*}

Having defined these probabilities one may combine them with the arrival rates of 
each class type in such a way to get the expected proportions of class 1 and 
class 2 individuals in the model. 
Thus, by using these values as the coefficient of equation 
\ref{overall_waiting_time_coeff} 
the resultant equation can be used to get the overall waiting time. 
Note here that the equation below gets the overall waiting time for both the recursive 
and the closed-form formula.

\begin{equation}\label{overall_waiting_time}
    W = \frac{\lambda_1 P(L'_1)}{\lambda_2 P(L'_2) + \lambda_1 P(L'_1)} W^{(1)} + 
    \frac{\lambda_2 P(L'_2)}{\lambda_2 P(L'_2) + \lambda_1 P(L'_1)} W^{(2)}
\end{equation}



\subsubsection{Mean blocking time}
Unlike the waiting time, the blocking time is only calculated for class 2 individuals.  
That is because class 1 individuals cannot be blocked. 
Thus, one only needs to consider the pathway of class 2 individuals to get the 
mean blocking time of the system. 
Blocking occurs at states \((u,v)\) where \(u > 0 \). 
Thus, the set of blocking states can be defined as:

\begin{equation*}
    S_b = \{(u,v) \in S \; | \; u > 0\}
\end{equation*}
 
In order to not consider individuals that will be lost to the system, the set of 
accepting states needs to be taken into account. The set of accepting states is given by:

\begin{equation*}
    S_A^{(2)}=
    \begin{cases}
        \{(u, v) \in S \; | \; u < M \} & \textbf{if } T \leq N\\
        \{(u, v) \in S \; | \; v < N \} & \textbf{otherwise}
    \end{cases}
\end{equation*}

For the waiting time formula,
the mean sojourn time for each state was considered,
ignoring any arrivals. Here, the same approach is used but ignoring only class 2
arrivals. That is because for the waiting time formula, once an individual enters 
the service centre (i.e. starts waiting) any individual arriving after them will 
not affect their
pathway. That is not the case for blocking time. When a class 2 individual is 
blocked, 
any class 1 individual that arrives will cause the blocked individual to remain 
blocked for more time. Therefore, class 1 arrivals are considered here:

\begin{equation}\label{eq:time_in_state_blocking_time}
    c(u,v) = 
    \begin{cases}
        \frac{1}{\min(v,C) \mu}, & \text{if } v = C\\
        \frac{1}{\min(v,C) \mu + \lambda_1}, & \text{otherwise}
    \end{cases}
\end{equation}
 
In equation \ref{eq:time_in_state_blocking_time}, both service completions and 
class 1 arrivals are considered. 
Thus, from a blocked individual's perspective whenever the system moves from one 
state \((u,v)\)
to another state it can either:

\begin{itemize}
    \item be because of a service being completed: we will denote the probability 
    of this happening by \(p_s(u,v)\). 
    \item be because of an arrival of an individual of class 1: denoting such 
    probability by \(p_o(u,v)\).
\end{itemize}
The probabilities are given by:

\begin{equation*}
    p_s(u,v) = \frac{\min(v,C)\mu}{\lambda_1 + \min(v,C)\mu}, \qquad
    p_o(u,v) = \frac{\lambda_1}{\lambda_1 + \min(v,C)\mu}
\end{equation*}


Having defined \(c(u,v)\) and \(S_b\) a formula for the blocking time that is
expected to occur at each state can be given by:

\begin{equation}\label{eq:blocking-time-at-each-state}
    b(u,v) = 
    \begin{cases} 
        0, & \textbf{if } (u,v) \notin S_b \\
        c(u,v) + b(u - 1, v), & \textbf{if } v = N = T\\
        c(u,v) + b(u, v-1), & \textbf{if } v = N \neq T \\
        c(u,v) + p_s(u,v) b(u-1, v) + p_o(u,v) b(u, v+1), & \textbf{if } u > 0 
        \textbf{ and } v = T \\
        c(u,v) + p_s(u,v) b(u, v-1) + p_o(u,v) b(u, v+1), & \textbf{otherwise} \\
    \end{cases}
\end{equation}

Equation 
(\ref{eq:blocking-time-at-each-state}) will not be solved recursively. 
A direct approach will be used to solve this equation here. 
By enumerating all equations of (\ref{eq:blocking-time-at-each-state}) for all 
states \((u,v)\) that belong in \(S_b\) 
a system of linear equations arises where the unknown variables are all the \(b(u,v)\)
terms.
For instance, let us consider a Markov model where \(C=2, T=3, N=6, M=2\). 
The Markov model is shown in Figure \ref{fig:example-algeb-blocking}
and the equivalent equations are 
(\ref{eq:first_eq_of_blocking_example})-(\ref{eq:last_eq_of_blocking_example}).
The equations considered here are only the ones that correspond to the blocking 
states.

\begin{multicols*}{2}
    \begin{figure}[H]
        \scalebox{0.50}{\input{MarkovChain/expressions_from_pi/example_model_2362/main.tex}}
        \caption{Example of Markov chain}
        \label{fig:example-algeb-blocking}
    \end{figure}
    \columnbreak
    \begin{align}
        b(1,2) &= c(1,2) + p_o b(1,3) \label{eq:first_eq_of_blocking_example} \\
        b(1,3) &= c(1,3) + p_s b(1,2) + p_o b(1,4) \\
        b(1,4) &= c(1,4) + b(1,3) \\
        b(2,2) &= c(2,2) + p_s b(1,2) + p_o b(2,3) \\
        b(2,3) &= c(2,3) + p_s b(2,2) + p_o b(1,4) \\
        b(2,4) &= c(2,4) + b(2,3)\label{eq:last_eq_of_blocking_example}
    \end{align}
\end{multicols*}

Additionally, the above equations can be transformed into a linear system of the 
form \(Zx=y\) where:

\begin{equation}\label{eq:example-algebaric-approach-blocking-time}
    Z=
    \begin{pmatrix}
        -1 & p_o & 0 & 0 & 0 & 0 \\ %(1,2)
        p_s & -1 & p_o & 0 & 0 & 0 \\ %(1,3)
        0 & 1 & -1 & 0 & 0 & 0 \\ %(1,4)
        p_s & 0 & 0 & -1 & p_o & 0\\ %(2,2)
        0 & 0 & 0 & p_s & -1 & p_o \\ %(2,3)
        0 & 0 & 0 & 0 & 1 & -1 \\ %(2,4)
    \end{pmatrix},
    x=
    \begin{pmatrix}
        b(1,2) \\
        b(1,3) \\
        b(1,4) \\
        b(2,2) \\
        b(2,3) \\
        b(2,4) \\
    \end{pmatrix}, 
    y=
    \begin{pmatrix}
        -c(1,2) \\
        -c(1,3) \\
        -c(1,4) \\
        -c(2,2) \\
        -c(2,3) \\
        -c(2,4) \\
    \end{pmatrix}
\end{equation}

A more generalised form of the equations in 
(\ref{eq:example-algebaric-approach-blocking-time})
can thus be given for any value of \(C,T,N,M\) by:

\begin{align}
    b(1,T) =& c(1, T) + p_o b(1, T + 1) \label{eq:first_eq_of_blocking_general}\\
    b(1,T + 1) =& c(1, T + 1) + p_s(1, T) + p_o b(1, T + 1) \\
    b(1,T + 2) =& c(1, T + 2) + p_s(1, T + 1) + p_o b(1, T + 3) \\
    & \vdots \nonumber \\
    b(1, N) =& c(1, N) + b(1, N - 1) \\
    b(2, T) =& c(2, T) + p_s b(1, T) + p_o b(2, T + 1) \\
    b(2, T + 1) =& c(2, T + 1) + p_s b(2, T) + p_o b(2, T + 2) \\
    & \vdots \nonumber \\
    b(M, T) =& c(M, T) + b(M, T-1) \label{eq:last_eq_of_blocking_general}
\end{align}

The equivalent matrix form of the linear system of equations 
(\ref{eq:first_eq_of_blocking_general}) - (\ref{eq:last_eq_of_blocking_general})
is given by \(Zx=y\), where:
\begin{equation}\label{eq:general-algebaric-approach-blocking-time}
    \scalebox{0.9}{
        \(
        Z = 
        \begin{pmatrix}
            -1 & p_o & 0 & \dots & 0 & 0 & 0 & 0 & 0 & \dots & 0 & 0 \\ %(1,T)
            p_s & -1 & p_o & \dots & 0 & 0 & 0 & 0 & 0 & \dots & 0 & 0 \\ %(1,T+1)
            0 & p_s & -1 & \dots & 0 & 0 & 0 & 0 & 0 & \dots & 0 & 0 \\ %(1,T+2)
            \vdots & \vdots & \vdots & \ddots & \vdots & \vdots & \vdots & \vdots & 
            \vdots & \ddots & \vdots & \vdots \\ 
            0 & 0 & 0 & \dots & 1 & -1 & 0 & 0 & 0 & \dots & 0 & 0 \\ %(1,N)
            p_s & 0 & 0 & \dots & 0 & 0 & -1 & p_o & 0 & \dots & 0 & 0 \\ %(2,T)
            0 & 0 & 0 & \dots & 0 & 0 & p_s & -1 & p_o & \dots & 0 & 0 \\ %(2,T+1)
            \vdots & \vdots & \vdots & \ddots & \vdots & \vdots & \vdots & \vdots & 
            \vdots & \ddots & \vdots & \vdots \\ 
            0 & 0 & 0 & \dots & 0 & 0 & 0 & 0 & 0 & \dots & 1 & -1 \\ %(M,T)
        \end{pmatrix},
        x = 
        \begin{pmatrix}
            b(1,T) \\
            b(1,T+1) \\
            b(1,T+2) \\
            \vdots \\
            b(1,N) \\
            b(2,T) \\
            b(2,T+1) \\
            \vdots \\
            b(M,T) \\
        \end{pmatrix}, 
        y= 
        \begin{pmatrix}
            -c(1,T) \\
            -c(1,T+1) \\
            -c(1,T+2) \\
            \vdots \\
            -c(1,N) \\
            -c(2,T) \\
            -c(2,T+1) \\
            \vdots \\
            -c(M,T) \\
        \end{pmatrix}
        \)
    }
\end{equation}

Thus, having calculated the mean blocking time for all blocking states \(b(u,v)\), 
it only remains to put them together in a formula.
The resultant blocking time formula is given by:

\begin{equation}\label{eq:algebraic-blocking-time}
    B = \frac{\sum_{(u,v) \in S_A} \pi_{(u,v)} \; b(u,v)}{\sum_{(u,v) \in S_A} 
    \pi_{(u,v)}}
\end{equation}

\documentclass{article}

\usepackage{amsmath}
\usepackage{amsfonts} 
\usepackage{geometry}
\usepackage{multicol}
\usepackage{float}
% \usepackage{mathtools}
% \usepackage{graphicx}
% \usepackage{soul}
% \usepackage{indentfirst}
\usepackage{tikz}
\usetikzlibrary{calc, automata, chains, arrows.meta, math}
\setcounter{MaxMatrixCols}{20}


\title{A game theoretic model of the behavioural gaming that takes place at the EMS - ED interface}

\author{
    Michalis Panayides, 
    Paul Harper, 
    Vince Knight
}

\begin{document}

\maketitle

\input{Abstract/main.tex}


\newpage
\tableofcontents

\newpage
\input{Introduction/main.tex}

\newpage
\input{Game_theory_component/main.tex}

\newpage
\input{MarkovChain/markov_chain_model/main.tex}
\input{MarkovChain/expressions_from_pi/main.tex}
\input{MarkovChain/markov_example/main.tex}

\newpage
\input{BehaviouralMethodology/main.tex}

\newpage
\input{Application_EMS_ED/main.tex}

\newpage
\input{Conclusion/main.tex}


\end{document}

\newpage
\documentclass{article}

\usepackage{amsmath}
\usepackage{amsfonts} 
\usepackage{geometry}
\usepackage{multicol}
\usepackage{float}
% \usepackage{mathtools}
% \usepackage{graphicx}
% \usepackage{soul}
% \usepackage{indentfirst}
\usepackage{tikz}
\usetikzlibrary{calc, automata, chains, arrows.meta, math}
\setcounter{MaxMatrixCols}{20}


\title{A game theoretic model of the behavioural gaming that takes place at the EMS - ED interface}

\author{
    Michalis Panayides, 
    Paul Harper, 
    Vince Knight
}

\begin{document}

\maketitle

\input{Abstract/main.tex}


\newpage
\tableofcontents

\newpage
\input{Introduction/main.tex}

\newpage
\input{Game_theory_component/main.tex}

\newpage
\input{MarkovChain/markov_chain_model/main.tex}
\input{MarkovChain/expressions_from_pi/main.tex}
\input{MarkovChain/markov_example/main.tex}

\newpage
\input{BehaviouralMethodology/main.tex}

\newpage
\input{Application_EMS_ED/main.tex}

\newpage
\input{Conclusion/main.tex}


\end{document}

\newpage
\section{EMS-ED application}

\subsection{Application}

\subsection{Data analysis of generated problem}

\newpage
\documentclass{article}

\usepackage{amsmath}
\usepackage{amsfonts} 
\usepackage{geometry}
\usepackage{multicol}
\usepackage{float}
% \usepackage{mathtools}
% \usepackage{graphicx}
% \usepackage{soul}
% \usepackage{indentfirst}
\usepackage{tikz}
\usetikzlibrary{calc, automata, chains, arrows.meta, math}
\setcounter{MaxMatrixCols}{20}


\title{A game theoretic model of the behavioural gaming that takes place at the EMS - ED interface}

\author{
    Michalis Panayides, 
    Paul Harper, 
    Vince Knight
}

\begin{document}

\maketitle

\input{Abstract/main.tex}


\newpage
\tableofcontents

\newpage
\input{Introduction/main.tex}

\newpage
\input{Game_theory_component/main.tex}

\newpage
\input{MarkovChain/markov_chain_model/main.tex}
\input{MarkovChain/expressions_from_pi/main.tex}
\input{MarkovChain/markov_example/main.tex}

\newpage
\input{BehaviouralMethodology/main.tex}

\newpage
\input{Application_EMS_ED/main.tex}

\newpage
\input{Conclusion/main.tex}


\end{document}


\end{document}


\end{document}}
    \caption{Example Markov model \(C=2, T=2, N=4, M=2\)}
    \label{fig:distribution_of_time_at_specific_state_2_servers}
\end{figure}

Figure \ref{fig:distribution_of_time_at_specific_state_2_servers} represents the 
same Markov model as figure 
\ref{fig:distribution_of_time_at_specific_state_1_server} with the only 
exception that there are 2 servers here. 
By applying the same logic, assuming that an individual arrives at state 
\((0,4)\), the sum of the following random variables arises.

\begin{align}
    (0,4) \Rightarrow \quad & X_2 \sim Exp(2\mu) \nonumber \\
    (0,3) \Rightarrow \quad & X_1 \sim Exp(2\mu) \\
    (0,2) \Rightarrow \quad & X_0 \sim Exp(\mu) \nonumber
\end{align}

Since these exponentially distributed random variables do not share the same 
parameter, an erlang distribution cannot be used. 
In fact, the problem can now be viewed either as the sum of exponentially 
distributed random variables with different parameters or as the sum of 
erlang distributed random variables.
The sum of erlang distributed random variables is said to follow the 
hypoexponential distribution. 
The hypoexponential distribution is defined with two vectors of size equal
to the number of Erlang random variables \cite{Akkouchi2008}, \cite{Smaili2013}. 
The vector \(\vec{r}\) contains all the \(k\)-values of the erlang distributions 
and \(\vec{\lambda}\) is a vector of the distinct parameters as illustrated in
equation (\ref{eq:connection_between_Hypoexponential_Erlang}).

\begin{equation}\label{eq:connection_between_Hypoexponential_Erlang}
    \begin{rcases}
        Erlang(k_1, \lambda_1) \\
        Erlang(k_2, \lambda_2) \\
        \hspace{1cm} \vdots \\
        Erlang(k_n, \lambda_n)
    \end{rcases}
    Hypo(
        \underbrace{(k_1, k_2, \dots k_n)}_{\vec{k}}, 
        \underbrace{(\lambda_1, \lambda_2, \dots \lambda_n)}_{\vec{\lambda}}
    )
\end{equation}

Equivalently, for this particular example:
\begin{align} \label{eq:multiple_servers_distribution_example}
    \begin{rcases}
        \begin{rcases}
            \scriptstyle{X_2 \sim Exp(2\mu)} \\
            \scriptstyle{X_1 \sim Exp(2\mu)}
        \end{rcases}
        \scriptstyle{X_1 + X_2 = S_1 \sim Erlang(2, 2\mu)} \\
        \scriptstyle{X_0 \sim Exp(\mu) \Rightarrow 
        \hspace{1cm} X_0 = S_2 \sim Erlang(1, \mu)}
    \end{rcases}
    \scriptstyle{S_1 + S_2 = H \sim Hypo((2,1), (2\mu, \mu))}
\end{align}

Therefore, the CDF of this distribution can be used to get the probability of 
the time in spent in the system being less than a given target.
The general CDF of the hypoexponential distribution \(Hypo(\vec{r}, 
\vec{\lambda})\), is given by the following expression \cite{Favaro2010}:

\begin{align} \label{eq:general_cdf_hypoexponential}
    & P(H < t) = 1 - \left( \prod_{j=1}^{\mid \vec{r} \mid} \lambda_j^{r_j} \right) 
    \sum_{k=1}^{\mid \vec{r} \mid} \sum_{l=1}^{r_k} \frac{\Psi_{k,l}(-\lambda_k)t^{r_k - l} 
    e^{-\lambda_k t}}
    {(r_k - l)! (l - 1)!} \nonumber \\ 
    & \textbf{where} \qquad \Psi_{k,l}(t) = - \frac{\partial^{l - 1}}
    {\partial t ^{l - 1}} \left( \prod_{j = 0, j \neq k}^{\mid \vec{r} \mid} 
    (\lambda_j + t)^{-r_j} \right) \nonumber \\
    & \textbf{and} \quad \qquad \lambda_0 = 0, r_0 = 1
\end{align}


The computation of the derivative makes equation 
(\ref{eq:general_cdf_hypoexponential}) computationally expensive. 
In \cite{Legros2015} an alternative linear version of that CDF is explored via 
matrix analysis, and is given by the following formula:

\begin{equation} \label{eq:linear_general_cdf_hypoexponential}
    \begin{split}
        F(x) = &1 - \sum_{k=1}^{n} \sum_{l=0}^{k-1} (-1)^{k-1} \binom{n}{k} 
            \binom{k-1}{l} \sum_{j=1}^{n} \sum_{s=1}^{j-1} e^{-x \lambda_s} 
            \prod_{l=1}^{s-1} \left( \frac{\lambda_l}{\lambda_l - \lambda_s} \right)
            ^ {k_s} \\
        & \times \sum_{s < a_1 < \dots < a_{l-1} < j} 
            \left( \frac{\lambda_s}{\lambda_s - \lambda_{a_1}} \right) ^ {k_s}
            \prod_{m=s+1}^{a_1-1} \left( \frac{\lambda_m}{\lambda_m - 
            \lambda_{a_1}}\right) ^ {k_m} \\  
        & \times \prod_{n=a_1}^{a_2-1} \left( \frac{\lambda_n}{\lambda_n - 
            \lambda_{a_2}}\right) ^ {k_n} \dots 
            \prod_{r=a_l-1}^{j-1} \left( \frac{\lambda_r}{\lambda_r - 
            \lambda_{a_j}}\right) ^ {k_r}  
            \sum_{q=0}^{k_s - 1} \frac{((\lambda_s - \lambda_{a_1})x)^q}{q!}, \\
        & \text{for } x \geq 0
    \end{split}
\end{equation}


\paragraph{Specific CDF of hypoexponential distribution}
Equations (\ref{eq:general_cdf_hypoexponential}) and 
(\ref{eq:linear_general_cdf_hypoexponential}) refers to the general CDF of the
hypoexponential distribution where the size of the vector parameters can be of
any size \cite{Favaro2010}.
In the Markov chain models described in figures 
\ref{fig:distribution_of_time_at_specific_state_1_server} and 
\ref{fig:distribution_of_time_at_specific_state_2_servers} the parameter vectors 
of the hypoexponential distribution are of size two, and in fact, for any 
possible version of the investigated Markov chain model the vectors can only be 
of size two.
This is true since for any dimensions of this Markov chain model there will 
always be at most two distinct exponential parameters; the parameter for 
finishing a service (\(\mu\)) and the parameter for moving forward in the queue 
(\(C \mu\)). 
For the special case of \(C=1\) the hypoexponential distribution will not be 
used as this is equivalent to an erlang distribution.
Therefore, by fixing the sizes of \(\vec{r}\) and \(\vec{\lambda}\) to 2, the 
following specific expression for the CDF of the hypoexponential distribution
arises, where the derivative is removed:


\begin{align} \label{eq:specific_cdf_hypoexponential}
    & P(H < t) = 1 - \left( \prod_{j=1}^{\mid \vec{r} \mid} \lambda_j^{r_j} \right) 
    \sum_{k=1}^{\mid \vec{r} \mid} \sum_{l=1}^{r_k} \frac{\Psi_{k,l}(-\lambda_k)t^{r_k - l} 
    e^{-\lambda_k t}}{(r_k - l)! (l - 1)!} \nonumber \\ 
    & \textbf{where} \qquad \Psi_{k,l}(t) = 
    \begin{cases} 
        \frac{(-1)^{l} (l-1)!}{\lambda_2} \left[\frac{1}{t^l} - \frac{1}
        {(t + \lambda_2)^l}\right] , & k=1 \\
        - \frac{1}{t (t + \lambda_1)^{r_1}}, & k=2
    \end{cases} \nonumber \\
    & \textbf{and} \quad \qquad \lambda_0 = 0, r_0 = 1
\end{align}

Note here that the only difference between equations
(\ref{eq:general_cdf_hypoexponential}) and (\ref{eq:specific_cdf_hypoexponential}) 
is the \(\Psi\) function. 
The next part proves that the expression for \(\Psi\) can be simplified for the 
cases of \(k = 1,2\). 
Equation (\ref{eq:hypoexponential_expression_to_proof}) shows the expression to 
be proved.

\begin{equation} \label{eq:hypoexponential_expression_to_proof}
    \Psi_{(k,l)}(t) = - \frac{\partial^{l - 1}}{\partial t ^{l - 1}} 
    \left( \prod_{j = 0, j \neq k}^{\mid \vec{r} \mid} (\lambda_j + t)^{-r_j} \right) = 
    \begin{cases} 
        \frac{(-1)^{l} (l-1)!}{\lambda_2} \left[\frac{1}{t^l} - \frac{1}
        {(t + \lambda_2)^l}\right] , & k=1 \\
        - \frac{1}{t (t + \lambda_1)^{r_1}}, & k=2
    \end{cases}
\end{equation}



\paragraph{Proof of equation (\ref{eq:hypoexponential_expression_to_proof})}
 
This section aims to show that there exists a simplified version of equation 
(\ref{eq:general_cdf_hypoexponential}) that is specific to the proposed Markov 
model.
Function \(\Psi\) is defined using the parameter \(t\) and the variables \(k\) 
and \(l\).
Given the Markov model, the range of values that \(k\) and \(l\) can take can be
bounded.
First, from the range of the double summation in equation 
(\ref{eq:general_cdf_hypoexponential}), it can be seen that 
\(k = 1, 2, \dots, \mid \vec{r} \mid\).
Now, \(\mid \vec{r} \mid\) represents the size of the parameter vectors that, 
for the Markov model, will always be 2. 
That is because, for all the exponentially distributed random variables that are
added together to form the new distribution, there only two distinct parameters,
thus forming two erlang distributions. Therefore:

\begin{equation*}
    k = 1, 2
\end{equation*}

By observing equation (\ref{eq:general_cdf_hypoexponential}) once more, the range
of values that \(l\) takes are \(l = 1, 2, \dots, r_k\), where \(r_1\) is 
subject to the individual's position in the queue and \(r_2 = 1\).
In essence, the hypoexponential distribution will be used with these bounds:

\begin{align}
    k = 1 & \qquad \Rightarrow \qquad l = 1, 2, \dots, r_1 \nonumber \\
    k = 2 & \qquad \Rightarrow \qquad l = 1
\end{align}

Thus the left hand side of equation (\ref{eq:hypoexponential_expression_to_proof}) 
needs only to be defined for these bounds. 
The specific hypoexponential distribution investigated here is of the form
\(Hypo((r_1, 1)(\lambda_1, \lambda_2))\).
Note the initial conditions \(\lambda_0=0, r_0=1\) defined in equation 
(\ref{eq:general_cdf_hypoexponential}) also hold here.
Thus the proof is split into two parts, for \(k=1\) and \(k=2\).



\begin{itemize}
    \item \(k = 2, l = 1\)
    \begin{equation*}
        \begin{split}
            LHS &= - \frac{\partial^{1-1}}{\partial t^{1-1}} 
            \left( \prod_{j=0, j \neq 2}^{2} (\lambda_j + t)^{-r_j} \right) \\
            &=-\left( (\lambda_0 + t)^{-r_0} \times (\lambda_1 + t)^{-r_1} \right) \\
            &=-\left( t^{-1} \times (\lambda_1 + t)^{-r_1} \right) \\ 
            &= - \frac{1}{t(t + \lambda_1)^{r_1}} \\
            & \hspace{7cm} \square
        \end{split}
    \end{equation*}
    \item \(k = 1, l = 1, \dots, r_1\)
    \begin{equation*}
        \begin{split}
            LHS &= -\frac{\partial^{l-1}}{\partial t^{l-1}} 
            \left( \prod_{j=0, j \neq 1}^{2} (\lambda_j + t)^{-r_j} \right) \\
            &= -\frac{\partial^{l-1}}{\partial t^{l-1}}
            \left( (\lambda_o + t)^{-r_0} \times (\lambda_2 + t)^{-r_2} \right) \\
            &= -\frac{\partial^{l-1}}{\partial t^{l-1}}
            \left( \frac{1}{t(t + \lambda_2)}\right)
        \end{split}
    \end{equation*}
    In essence, it only remains to show that:
    \[-\frac{\partial^{l-1}}{\partial t^{l-1}} 
    \left( \frac{1}{t(t + \lambda_2)}\right) = \frac{(-1)^{l} (l-1)!}{\lambda_2}
    \left[\frac{1}{t^l} - \frac{1}{(t + \lambda_2)^l}\right]\]
    
    \textbf{Proof by Induction:}
    \begin{enumerate}
        \item Base case (\(l=1\)):
        \begin{equation*}
            \begin{split}
                LHS &= -\frac{\partial^{1-1}}{\partial t^{1-1}} 
                \left( \frac{1}{t(t + \lambda_2)}\right) = 
                - \frac{1}{t(t + \lambda_2)} \\
                RHS &= \frac{(-1)^{1} (1-1)!}{\lambda_2}
                \left[\frac{1}{t^1} - \frac{1}{(t + \lambda_2)^1}\right] \\
                &= - \frac{t + \lambda_2 - t}{\lambda_2 t (t + \lambda_2)} \\
                &= - \frac{1}{t (t + \lambda_2)} \\
                LHS &= RHS
            \end{split}
        \end{equation*}
        \item Assume true for \(l = x\):
        \begin{equation*}
            -\frac{\partial^{x-1}}{\partial t^{x-1}} 
            \left( \frac{1}{t(t + \lambda_2)}\right) = 
            \frac{(-1)^{x} (x-1)!}{\lambda_2}
            \left[\frac{1}{t^x} - \frac{1}{(t + \lambda_2)^x}\right]
        \end{equation*}
        \item Prove true for \(l = x + 1\). Need to show that:
        \[ 
            \frac{\partial^x}{\partial t ^ x} 
            \left( \frac{-1}{t (t + \lambda_2)} \right) = 
            \frac{(-1)^{x + 1} (x)!}{\lambda_2}
            \left[ \frac{1}{t^{x+1}}-\frac{1}{(t + \lambda_2)^{x+1}} \right] 
        \]
        \begin{equation*}
            \begin{split}
                LHS &= \frac{\partial}{\partial t}
                \left[ \frac{\partial^{x-1}}{\partial t ^ {x-1}} 
                \left( \frac{-1}{t (t + \lambda_2)} \right) \right] \\
                &= \frac{\partial}{\partial t} \left[
                    \frac{(-1)^x (x-1)!}{\lambda_2} \left(
                        \frac{1}{t^x} - \frac{1}{(t + \lambda_2)^x}
                    \right)
                \right] \\
                &= \frac{(-1)^x (x-1)!}{\lambda_2} \left(
                    \frac{(-x)}{t^{x+1}} - \frac{(-x)}{(t + \lambda_2)^x}
                \right) \\
                &= \frac{(-1)^x (x-1)! (-x)}{\lambda_2} \left(
                    \frac{1}{t^{x+1}} - \frac{1}{(t + \lambda_2)^x}
                \right) \\
                &= \frac{(-1)^{x+1} (x)!}{\lambda_2} \left(
                    \frac{1}{t^{x+1}} - \frac{1}{(t + \lambda_2)^x}
                \right) \\
                & = RHS \\
                & \hspace{7cm} \square
            \end{split}
        \end{equation*}
    \end{enumerate}
\end{itemize}

\paragraph{Proportion within target for both types of individuals}

Given the two CDFs of the Erlang and Hypoexponential distributions a new 
function has to be defined to decide which one to use among the two.
Based on the state of the model, there can be three scenarios when an individual
arrives.
\begin{enumerate}
    \item There is a free server and the individual does not have to wait
    \begin{equation*}
        X_{(u,v)} \sim Erlang(1, \mu) 
    \end{equation*}
    \item The individual arrives at a queue at the \(n^{th}\) position and the 
    model has \(C > 1\) servers
    \begin{equation*}
        X_{(u,v)} \sim Hypo((n, 1), (C \mu, \mu)) 
    \end{equation*}
    \item The individual arrives at a queue at the \(n^{th}\) position and the 
    model has \(C = 1\) servers
    \begin{equation*}
        X_{(u,v)} \sim Erlang(n + 1, \mu) 
    \end{equation*}
\end{enumerate}

Note here that for the first case \(Erlang(1, \mu)\) is equivalent to 
\(Exp(\mu)\). 
Consider \(X_{(u,v)}^{(1)}\) to be the distribution of type 1 individuals and
\(X_{(u,v)}^{(2)}\) the distribution of type 2 individuals, when arriving at 
state \((u,v)\) of the model.

\begin{equation}
    X_{(u,v)}^{(1)} \sim 
    \begin{cases}
        \textbf{Erlang}(v, \mu), & \textbf{if } C = 1 \textbf{ and } v>1 \\
        \textbf{Hypo} \left(
            \left[v - C, 1\right], \left[C \mu, \mu \right]
        \right), & \textbf{if } C > 1 \textbf{ and } v>C \\
        \textbf{Erlang}(1, \mu), & \textbf{if } v \leq C
    \end{cases}
\end{equation}

\begin{equation}
    X_{(u,v)}^{(2)} \sim 
    \begin{cases}
        \textbf{Erlang}(\min(v, T), \mu), & \textbf{if } C = 1
            \textbf{ and } v, T > 1 \\
        \textbf{Hypo}\left(
            \left[ \min(v, T) - C, 1 \right], \left[ C \mu, \mu \right]
        \right), & \textbf{if } C > 1 \textbf{ and } v, T  > C \\
        \textbf{Erlang}(1, \mu), & \textbf{if } v \leq C \textbf{ or } T \leq C
    \end{cases}
\end{equation}


Thus, the CDF of the random variables \(X_{(u,v)}^{(1)}\) and 
\(X_{(u,v)}^{(2)}\) can be calculated using equations (\ref{eq:cdf_erlang}) and 
(\ref{eq:specific_cdf_hypoexponential}):

\begin{equation}
    P(X_{(u,v)}^{(1)} < t) = 
    \begin{cases}
        1 - \sum_{i=0}^{v-1} \frac{1}{i!} e^{-\mu t} (\mu t)^i, 
            & \textbf{if } C = 1 \\
            & \textbf{and } v>1 \\
        & \\
        1 - (\mu C)^{v-C} \mu  
            \sum_{k=1}^{\mid \vec{r} \mid} \sum_{l=1}^{r_k}
            \frac{\Psi_{k,l}(-\lambda_k)t^{r_k - l} 
            e^{-\lambda_k t}}{(r_k - l)! (l - 1)!},
            & \textbf{if } C > 1 \\
        \textbf{where } \vec{r}=(v - C, 1) \textbf{ and } 
            \vec{\lambda}=(C \mu, \mu) & \textbf{and } v > C \\
        & \\
        1 - e^{-\mu t},  & \textbf{if } v \leq C
    \end{cases}
    \tag{\ref{eq:proportion_within_target_type_1} revisited}
\end{equation}


\begin{equation}
    P(X_{(u,v)}^{(2)} < t) = 
    \begin{cases}
        1 - \sum_{i=0}^{\min(v,T)-1} \frac{1}{i!} e^{-\mu t} (\mu t)^i,  
            & \textbf{if } C = 1 \\ 
            & \textbf{and } v, T > 1 \\
            & \\
        1 - (\mu C) ^ {\min(v,T) - C} \mu  & \textbf{if } C > 1 \\
        \qquad \times \sum_{k=1}^{\mid \vec{r} \mid} \sum_{l=1}^{r_k} 
            \frac{\Psi_{k,l}(-\lambda_k)t^{r_k - l} 
            e^{-\lambda_k t}}{(r_k - l)! (l - 1)!}, 
            & \textbf{and } v, T  > C \\
        \textbf{where } \vec{r}=(\min(v, T) - C, 1) \\
        \hspace{1.15cm} \vec{\lambda}=(C \mu, \mu) \\
        & \\
        1 - e^{-\mu t}, & \textbf{if } v \leq C \\ 
        & \textbf{or } T \leq C \\
    \end{cases}
    \tag{\ref{eq:proportion_within_target_type_2} revisited}
\end{equation}


In addition, the set of accepting states for type 1 \(S_A^{(1)}\) and type 2 
\(S_A^{(2)}\) individuals defined in (\ref{eq:accepting_states_type_1}) and 
(\ref{eq:accepting_states_type_2}) are also needed here.
Note here that, \(S\) denotes the set of all states of the Markov chain model. 

\begin{align*}
    S_A^{(1)} &= \{(u, v) \in S \; | \; v < N \} \\
    S_A^{(2)} &=
    \begin{cases}
        \{(u, v) \in S \; | \; u < M \}, & \textbf{if } T \leq N \\
        \{(u, v) \in S \; | \; v < N \}, & \textbf{otherwise}
    \end{cases}
\end{align*}

The following formula uses the state probability vector \(\pi\) to get the 
weighted average of the probability below target of all states in the Markov
model.

\begin{equation}
    P(X^{(1)} < t) = \frac{\sum_{(u,v) \in S_A^{(1)}} P(X_{u,v}^{(1)} < t) 
    \pi_{u,v} }{\sum_{(u,v) \in S_A^{(1)}} \pi_{u,v}}
\end{equation}

\begin{equation}
    P(X^{(2)} < t) = \frac{\sum_{(u,v) \in S_A^{(2)}} P(X_{u,v}^{(2)} < t) 
    \pi_{u,v} }{\sum_{(u,v) \in S_A^{(2)}} \pi_{u,v}}
\end{equation}


\paragraph{Overall proportion within target}

The overall proportion of individuals for both types of individuals is given by 
the equivalent formula of equation (\ref{eq:overall_waiting_time}).
The following formula uses the probability of lost individuals from both types
to get the weighted sum of the two probabilities.

\begin{equation*}
    P_{L'_1} = \sum_{(u,v) \, \in S_A^{(1)}} \pi(u,v), \hspace{1.5cm}
    P_{L'_2} = \sum_{(u,v) \, \in S_A^{(2)}} \pi(u,v)
\end{equation*}

\begin{align}
    P(X < t) &= \frac{\lambda_1 P_{L'_1}}{\lambda_2 P_{L'_2}+\lambda_1 P_{L'_1}} 
    P(X^{(1)} < t) \nonumber \\
    &+ \frac{\lambda_2 P_{L'_2}}{\lambda_2 P_{L'_2} + \lambda_1 
    P_{L'_1}} P(X^{(2)} < t) 
    \tag{\ref{eq:overall_proportion_within_target} revisited}
\end{align}
