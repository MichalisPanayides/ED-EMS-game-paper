\section{Mean blocking time} \label{sec:appendix_mean_blocking}

The set of states where individuals can be blocked is defined as:
\begin{equation*}
    S_b = \{(u,v) \in S \; | \; u > 0\} 
    \tag{\ref{eq:set_of_blocking_states} revisited}
\end{equation*}

The mean sojourn time for each state is given by the inverse of the out-flow of
that state ~\cite{Stewart2019}.
However, whenever a type 2 individual arrives at the system, no subsequent 
arrival of another type 2 individual can affect its pathway or total time in 
the system.
Therefore, looking at the mean time in the system from the perspective of an 
individual of the second type, all such type 2 arrivals need to be ignored.
Note here that this is not the case for individuals of the first type.
Whenever a type 2 individual is blocked and a type 1 individual arrives the type
2 individuals will stay blocked for some additional amount of time.
Thus, the mean time that a type 2 individual spends at each state is given by:

\begin{equation*}
    c(u,v) = 
    \begin{cases}
        \frac{1}{\min(v,C) \mu}, & \text{if } v = N\\
        \frac{1}{\lambda_1 + \min(v,C) \mu}, & \text{otherwise}
    \end{cases} 
    \tag{\ref{eq:sojourn_blocking_time} revisited}
\end{equation*}

In equation (\ref{eq:sojourn_blocking_time}), both service completions and 
type 1 arrivals are considered. 
Thus, from a blocked individual's perspective whenever the system moves from one 
state \((u,v)\)
to another state it can either:

\begin{itemize}
    \item be because of a service being completed: we will denote the probability 
    of this happening by \(p_s(u,v)\). 
    \item be because of an arrival of an individual of type 1: denoting such 
    probability by \(p_a(u,v)\).
\end{itemize}
The probabilities are given by:

\begin{equation*}
    p_s(u,v) = \frac{\min(v,C)\mu}{\lambda_1 + \min(v,C)\mu}, \qquad
    p_a(u,v) = \frac{\lambda_1}{\lambda_1 + \min(v,C)\mu}
    \tag{\ref{eq:probs_of_service_and_arrival} revisited} 
\end{equation*}


Having defined \(c(u,v)\) and \(S_b\) a formula for the blocking time that is
expected to occur at each state can be given by:

    \begin{equation*}
    b(u,v) = 
    \begin{cases} 
        0, & \textbf{if } (u,v) \notin S_b \\
        c(u,v) + b(u - 1, v), & \textbf{if } v = N = T\\
        c(u,v) + b(u, v-1), & \textbf{if } v = N \neq T \\
        c(u,v) + p_s(u,v) b(u-1, v) + p_a(u,v) b(u, v+1), & \textbf{if } u > 0 
        \textbf{ and } \vspace{-0.2cm} \\ 
        & \quad v = T \\
        c(u,v) + p_s(u,v) b(u, v-1) + p_a(u,v) b(u, v+1), & \textbf{otherwise}\\
    \end{cases}
    \tag{\ref{eq:general_blocking_time_at_each_state} revisited}
\end{equation*}

A direct approach will be used to solve this equation here. 
By enumerating all equations of (\ref{eq:general_blocking_time_at_each_state}) 
for all states \((u,v)\) that belong in \(S_b\) 
a system of linear equations arises where the unknown variables are all the 
\(b(u,v)\) terms. 
Note here that these equations correspond to all blocking states as defined in
(\ref{eq:set_of_blocking_states}). 
Equations that correspond to non-blocking states have a value of \(0\) as 
defined in (\ref{eq:general_blocking_time_at_each_state})
The general form of the equation in terms of \(C,T,N \text{ and } M\) is given by: 

\begin{align}
    b(1,T) \quad &= \quad c(1, T) + p_a b(1, T + 1) \label{eq:first_eq_of_blocking_general}\\
    b(1,T + 1) \quad &= \quad c(1, T + 1) + p_s b(1, T) + p_a b(1, T + 1) \\
    b(1,T + 2) \quad &= \quad c(1, T + 2) + p_s b(1, T + 1) + p_a b(1, T + 3) \\
    & \ \, \vdots \nonumber \\
    b(1, N) \quad &= \quad c(1, N) + b(1, N - 1) \\
    b(2, T) \quad &= \quad c(2, T) + p_s b(1, T) + p_a b(2, T + 1) \\
    b(2, T + 1) \quad &= \quad c(2, T + 1) + p_s b(2, T) + p_a b(2, T + 2) \\
    & \ \, \vdots \nonumber \\
    b(M - 1, N) \quad &= \quad c(M, N - 1) + b(M, N-1) \\ 
    b(M, T) \quad &= \quad c(T, N) + p_s b(T-1, N) + p_a b(T, N+1) \\
    & \ \, \vdots \nonumber \\
    b(M, N) \quad &= \quad c(M, N) + b(M, N-1) \label{eq:last_eq_of_blocking_general}
\end{align}

The equivalent matrix notation of the linear system of equations 
(\ref{eq:first_eq_of_blocking_general}) - (\ref{eq:last_eq_of_blocking_general})
is given by \(Zx=y\), where:
\begin{equation}
    \scalebox{0.73}{\(
        Z = 
        \begin{pmatrix}
            -1 & p_a & 0 & \dots & 0 & 0 & 0 & 0 & 0 & \dots & 0 & 0 \\ %(1,T)
            p_s & -1 & p_a & \dots & 0 & 0 & 0 & 0 & 0 & \dots & 0 & 0 \\ %(1,T+1)
            0 & p_s & -1 & \dots & 0 & 0 & 0 & 0 & 0 & \dots & 0 & 0 \\ %(1,T+2)
            \vdots & \vdots & \vdots & \ddots & \vdots & \vdots & \vdots & 
            \vdots & \vdots & \ddots & \vdots & \vdots \\ 
            0 & 0 & 0 & \dots & 1 & -1 & 0 & 0 & 0 & \dots & 0 & 0 \\ %(1,N)
            p_s & 0 & 0 & \dots & 0 & 0 & -1 & p_a & 0 & \dots & 0 & 0 \\ %(2,T)
            0 & 0 & 0 & \dots & 0 & 0 & p_s & -1 & p_a & \dots & 0 & 0 \\ %(2,T+1)
            \vdots & \vdots & \vdots & \ddots & \vdots & \vdots & \vdots & 
            \vdots & \vdots & \ddots & \vdots & \vdots \\ 
            0 & 0 & 0 & \dots & 0 & 0 & 0 & 0 & 0 & \dots & 1 & -1 \\ %(M,N)
        \end{pmatrix},
        x = 
        \begin{pmatrix}
            b(1,T) \\
            b(1,T+1) \\
            b(1,T+2) \\
            \vdots \\
            b(1,N) \\
            b(2,T) \\
            b(2,T+1) \\
            \vdots \\
            b(M,N) \\
        \end{pmatrix}, 
        y= 
        \begin{pmatrix}
            -c(1,T) \\
            -c(1,T+1) \\
            -c(1,T+2) \\
            \vdots \\
            -c(1,N) \\
            -c(2,T) \\
            -c(2,T+1) \\
            \vdots \\
            -c(M,N) \\
        \end{pmatrix}
    \)} \tag{\ref{eq:general_algebaric_approach_blocking_time} revisited}
    \end{equation}

    The elements of the matrix \(Z\) can be acquired using \(Z_{ij}\) defined in 
    equation (\ref{eq:general_mapping_function_of_blocking_matrix}) where \(i\) 
    and \(j\) are states \((u_i, v_i), (u_j, v_j) \in S_b\) 
    (\ref{eq:set_of_blocking_states}).

    \begin{equation}
    Z_{ij} = 
    \begin{cases}
        p_a, & \textbf{if } j = i + 1 \textbf{ and } v_i \neq N \\
        p_s, & \textbf{if } j = i - 1 \textbf{ and } v_i \neq N, v_i \neq T \\
            & \textbf{or } j = i - N + T \textbf{ and } u_i \geq 2,\,v_i = T \\
        1, & \textbf{if } j = i - 1 \textbf{ and } v_i = N \\
        -1, & \textbf{if } i = j \\
        0, & \textbf{otherwise} \\
    \end{cases}
    \tag{\ref{eq:general_mapping_function_of_blocking_matrix} revisited}
\end{equation}


Thus, having calculated the mean blocking time for all blocking states 
\(b(u,v)\), they can be combined together in a formula.
Using the arriving states \(\mathcal{A}_2\) defined in section 
\ref{sec:blocking_time} by equation \ref{eq:arriving_state_class_2} 
the resultant formula for the mean blocking time is given by:

\begin{equation}
    B = \frac{\sum_{(u,v) \in S_A} \pi_{(u,v)} \; b(\mathcal{A}_2(u,v))}
    {\sum_{(u,v) \in S_A} \pi_{(u,v)}} 
    \tag{\ref{eq:algebraic_blocking_time} revisited}
\end{equation}

To illustrate how the described formula works consider a Markov model where 
\(C=2, T=2, N=4, M=2\) (figure \ref{fig:example_algeb_blocking}). 
The equations that correspond to such a model are shown in 
(\ref{eq:first_eq_of_blocking_example})-(\ref{eq:last_eq_of_blocking_example}) 
and their equivalent matrix notation form is shown in 
(\ref{eq:example_algebaric_approach_blocking_time}).

\begin{minipage}{.5\textwidth}
    \begin{figure}[H]
        \scalebox{0.6}{\section{Conclusion}

The motivation behind this study has been that emergency departments are 
under a lot of pressure to satisfy some regulations. 
This paper shows how this can negatively impact the pathway of both the 
ambulance patients and the ambulance service itself.
Due to some managerial decision making that takes place at the ED, ambulances 
stay blocked outside of the ED at the hospital's parking zone in an attempt
to satisfy these regulations.

This study explores a generic 3-player game theoretic model between the 
decision makers of two queueing systems and a service that distributes 
individuals to these two systems (section \ref{sec:model_overview}).
It also describes the construction of the underlying queueing theoretic model 
that has a tandem buffer and a single service centre (section 
\ref{sec:queueing_model}).
Furthermore, the formulas for the performance measures of the queueing model 
are also derived (sections \ref{sec:waiting_time}, \ref{sec:blocking_time}, 
\ref{sec:proportion_within_target}). 
This novel queuing model is the first contribution of the paper.
The game theoretic model is then applied to a healthcare scenario by looking at
the interface between the EDs and the EMS (section 
\ref{sec:ems_ed_application}).
The inefficiencies that emerge from the perspective of the EMS were explored 
along with ways to apply some incentive mechanisms to improve them.
The key findings from this paper that were observed when playing the game
between two EDs and the EMS (section \ref{sec:application}) are:
\begin{itemize}
    \item Inefficiencies can be learned and emerge naturally
    \item Targeted incentivisation of behaviours can help escape inefficiencies
\end{itemize}
The former relates to the results of asymmetric replicator dynamics that showed 
that inefficient scenarios can arise by playing the game while the latter 
implies that the learned inefficiencies can be escaped by carefully applying 
certain incentives to the players.
This applied game theoretic model is the second main contribution of this paper.

The model presented here assumes the presence of two players that can receive 
individuals. 
However, in a realistic healthcare scenario an ambulance may have to decide 
among multiple EDs.
An immediate extension of this work would be to consider a multiplayer system
that could represent a group of hospitals in a concentrated area.
Additionally, the game theoretic model that was created uses a discrete 
strategy space for the EDs (something that is also present in various related 
literature~\cite{deo2011centralized, knight2017measuring}).
The single threshold parameter that is used for the ED's decision may not be 
the best way to describe the model.
In reality ED managers could have far more complex parameters for their 
decision making process.
Finally this work assumes that the EMS and EDs act in a selfish and rational
way by only aiming to satisfy their own objectives.
In fact, modelling behaviour in a healthcare setting is a much more complex 
task where some cooperation may often be observed.
Another extension would be to explore the behaviour of the ED staff via an 
agent-based model. 
This in turn can be used to model ED staff as agents each with their own 
behavioural traits.
}
        \caption{
            \centering{Example of Markov chain with \(C=2, T=2, N=4, M=2\)}
        }
        \label{fig:example_algeb_blocking}
    \end{figure}
    \end{minipage}
    \begin{minipage}{.43\textwidth}
    \begin{align}
        b(1,2) &= c(1,2) + p_a b(1,3) \label{eq:first_eq_of_blocking_example} \\
        b(1,3) &= c(1,3) + p_s b(1,2) \nonumber \\ &+ p_a b(1,4) \\
        b(1,4) &= c(1,4) + b(1,3) \\
        b(2,2) &= c(2,2) + p_s b(1,2) \nonumber \\ &+ p_a b(2,3) \\
        b(2,3) &= c(2,3) + p_s b(2,2) \nonumber \\ &+ p_a b(1,4) \\
        b(2,4) &= c(2,4) + b(2,3) \label{eq:last_eq_of_blocking_example}
    \end{align}
\end{minipage}

\begin{equation}\label{eq:example_algebaric_approach_blocking_time}
    Z=
    \begin{pmatrix}
        -1 & p_a & 0 & 0 & 0 & 0 \\ %(1,2)
        p_s & -1 & p_a & 0 & 0 & 0 \\ %(1,3)
        0 & 1 & -1 & 0 & 0 & 0 \\ %(1,4)
        p_s & 0 & 0 & -1 & p_a & 0\\ %(2,2)
        0 & 0 & 0 & p_s & -1 & p_a \\ %(2,3)
        0 & 0 & 0 & 0 & 1 & -1 \\ %(2,4)
    \end{pmatrix},
    x=
    \begin{pmatrix}
        b(1,2) \\
        b(1,3) \\
        b(1,4) \\
        b(2,2) \\
        b(2,3) \\
        b(2,4) \\
    \end{pmatrix}, 
    y=
    \begin{pmatrix}
        -c(1,2) \\
        -c(1,3) \\
        -c(1,4) \\
        -c(2,2) \\
        -c(2,3) \\
        -c(2,4) \\
    \end{pmatrix}
\end{equation}
