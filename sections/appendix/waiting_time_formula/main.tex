\section{Mean waiting time}\label{sec:appendix_mean_waiting}

The recursive formula described here is the origin of the closed-form formula 
described in section~\ref{sec:waiting_time}.

To calculate the mean waiting time one must first identify the set of states 
\((u, v)\) where a wait will occur. 
For this particular Markov chain, this points to all states that satisfy 
\(v > C\) i.e. all states where the number of individuals in the service centre 
exceed the number of servers. 
The set of such states is defined as \textit{waiting states} and can be denoted 
as a subset of all the states, where:

\begin{equation} \label{eq:waiting_states}
    S_w = \{(u, v) \in S \; | \; v > C \}    
\end{equation}

Additionally, there are certain states in the model where arrivals cannot occur. 
A type 1 individual cannot arrive whenever the model is at any state 
\((u, N)\) 
for all \(u\) where \(N\) is the system capacity. 
Equivalently, a type 2 individual cannot arrive in the model when the model is 
at any state \((M, v)\) for all \(v \geq T\).
Therefore the set of all such states that an arrival may occur are defined as 
\textit{accepting states} and are denoted as:

\begin{equation*}
    S_A^{(1)} = \{(u, v) \in S \; | \; v < N \}
    \tag{\ref{eq:accepting_states_type_1} revisited}
\end{equation*}

\begin{equation}
    S_A^{(2)}=
    \begin{cases}
        \{(u, v) \in S \; | \; u < M \} & \textbf{if } T \leq N\\
        \{(u, v) \in S \; | \; v < N \} & \textbf{otherwise}
    \end{cases}
    \tag{\ref{eq:accepting_states_type_2} revisited}
\end{equation}

Moreover, another element that needs to be considered is the expected waiting time 
% in each state \( c(u,v) \), otherwise known as sojourn time \cite{Raghunandanan}. 
In order to do so a variation of the Markov model has to be considered where when 
the individual arrives at any of the states of the model no more arrivals can 
occur after that. 


Thus, one may acquire the desired time by calculating the inverse of the sum of 
the out-flow rate of that state. 
Since arrivals are ignored though the only way to exit the state will only be 
via a service. 
In essence this notion can be expressed as:

\begin{equation} \label{eq:sojourn_type_1}
    c^{(1)}(u,v) = 
    \begin{cases}
        0, & \textbf{if } u > 0 \textbf{ and } v = T \\
        \frac{1}{\text{min}(v,C)\mu}, & \textbf{otherwise}
    \end{cases}
\end{equation}

Now, like in the type 1 individuals case, the sojourn time is needed. 
For type 2 individuals whenever individuals are at any row apart from the 
first one they automatically get a wait time of \(0\) since they are essentially 
blocked.

\begin{equation} \label{eq:sojourn_type_2}
    c^{(2)}(u,v) = 
    \begin{cases}
        0, & \textbf{if } u > 0 \\
        \frac{1}{\text{min}(v,C)\mu}, & \textbf{otherwise}
    \end{cases}
\end{equation}

Note that whenever any type 1 individual is at a state \((u,v)\) where 
\(u > 0\) 
and \(v = T\) (i.e. all states \((1,T), (2,T) \dots, (M,T)\)) the sojourn time is 
set to \(0\). 
This is done to capture the trip thorough the Markov chain from the perspective 
of individuals. 
Meaning that they will visit all states of the threshold column but only the one 
in the first row will return a non-zero sojourn time.
Thus, the expected waiting time of type 1 and type 2 individuals when upon
arriving at state \( (u,v) \) can be given by the following recursive formulas:

\begin{equation} \label{eq:waiting_time_at_each_state_type_1}
    w^{(1)}(u,v) = 
    \begin{cases} 
        0, \hspace{4.85cm} & \textbf{if } (u,v) \notin S_w \\
        c^{(1)}(u,v) + w^{(1)}(u-1, v), & \textbf{if } u > 0 \textbf{ and } v = T \\
        c^{(1)}(u,v) + w^{(1)}(u, v-1), & \textbf{otherwise}
    \end{cases}
\end{equation}

\begin{equation} \label{eq:waiting_time_at_each_state_type_2}
    w^{(2)}(u,v) = 
    \begin{cases} 
        0, \hspace{4.85cm} & \textbf{if } (u,v) \notin S_w \\
        c^{(2)}(u,v) + w^{(2)}(u-1, v), & \textbf{if } u > 0 \textbf{ and } v = T \\
        c^{(2)}(u,v) + w^{(2)}(u, v-1), & \textbf{otherwise}
    \end{cases}
\end{equation}

Finally, the mean waiting time can be calculated by summing over all expected 
waiting times of accepting states multiplied by the probability of being at that 
state and dividing by the probability of being in any accepting state.

\begin{equation} \label{eq:recursive_waiting_time_type_1}
    W^{(1)} = \frac{\sum_{(u,v) \in S_A^{(1)}} w^{(1)}(u,v) 
    \pi_{(u,v)}}{\sum_{(u,v) \in S_A^{(1)}} \pi_{(u,v)}}
\end{equation}

\begin{equation}\label{eq:recursive_waiting_time_type_2}
    W^{(2)} = \frac{\sum_{(u,v) \in S_A^{(2)}} w^{(2)}(u,v) \pi_{(u,v)}}
    {\sum_{(u,v) \in S_A^{(2)}} \pi_{(u,v)}}
\end{equation}
