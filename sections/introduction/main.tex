\section{Introduction}

% Introduction + motivation
Emergency departments (EDs) are under increasing pressure to meet patient
waiting time targets and satisfy 
regulations~\cite{EmergencyDepartmentWinterPressures}.
It is widely reported (e.g.~\cite{mirror, thenews, bmj}) that ED congestion 
severely impacts not only patients in the ED but also Emergency Medical 
Services (EMS).
A major concern for ambulances is that they are held waiting parked outside the
ED to offload (dispatch) their patient when the ED is particularly 
busy~\cite{clarey2014ambulance}. 
Since the patient waiting time in ED is measured from the time they enter the
ED itself, there is no incentive, should the patient be stable in the
ambulance, to offload them from EMS to ED services.
As a result, ambulance blocking not only impacts on patients waiting for ED
service, but has a major knock-on effect to delaying the ability of ambulances
to respond to new EMS calls, thus placing lives at risk~\cite{eastanglia}.

There are numerous news articles that focus on the complexity that arises when
ambulances stay blocked outside of the hospital for a long amount of 
time~\cite{dailyrecords, staffordshirelive}.
Some news reports comment on the long idle time of ambulances when
they are not in use~\cite{herefordtimes} and there are several reports of 
examples where this became an issue for new patients~\cite{southwalesargus}
and paramedics~\cite{bbcwales}.

% ED + EMS application
This paper aims to describe the EMS-ED interface using a game theoretic model
informed by an underlying queueing model.
The model describes the situation where an ambulance service would
have to distribute its patients between two EDs.
The two EDs can be thought of as two queueing systems and the EMS as a 
distributor that distributes patients to them, aiming to minimise some
performance measure.
The patients that are distributed by the EMS arrive at the hospital via an 
ambulance and are then either offloaded at the ED or stay blocked outside in 
the ambulance.
Whether or not the ambulance and its patient stay blocked is determined by 
the threshold that the given ED chooses to play.
A high threshold indicates that the ED accepts ambulance patients even if it is 
relatively full, while a low threshold means that the ED blocks ambulances more 
frequently.

In the United Kingdom, the National Heath Services (NHS) sets some regulations 
on ED performance.
One of these regulations is that 95\% of patients that arrive at the ED should 
be admitted, transferred or discharged within four hours.
This is where gaming behaviour might be observed between the EDs and the EMS.
An assumption of this work is that some managerial decision making is involved
in choosing when to start blocking ambulances.
This is similar to~\cite{deo2011centralized}.

% Major contributions
The major contributions of this paper are:
\begin{itemize}
    \item A queueing model with two consecutive waiting spaces where one would 
    serve as a parking space for the ambulances.
    \item Analytic performance measure formulas for the queueing model.
    \item A 3-player game theoretic model between the EMS and two EDs.
    \item Numerical experiments showing emergent behaviour of gaming between
    EDs and the EMS.
\end{itemize}
Specifically, our focus is on the construction of a 3-player game theoretic 
model between two queueing systems and a service that distributes individuals
to them. 
The resultant model is then used to explore the emergent dynamics between 
the three players.
This study explores two new concepts: getting performance measures for a new
queueing theoretic model with a parking space and a service centre, and
using a learning algorithm to model the emergence of behaviour.
The developed theoretical model is illustrated through the application to 
a healthcare system of two EDs and the EMS, exploring the inefficiencies that 
emerge and ways to apply some incentive mechanisms to improve them.
The EDs are modelled as two queueing systems each with a tandem buffer and a 
service centre.
The performance measures are then used as the utilities of the game.
The novelty of the queueing model here is a contribution not only the game 
theoretic literature but also to the queueing theoretic literature.
To the authors knowledge, no such model of a tandem queueing model with a pair 
of parameters for the buffer has been previously considered.

% Sign posting
This paper consists of two main sections.
Section \ref{sec:queueing_model} presents a novel queueing model for a hospital
with two types of patients and two waiting zones.
A detailed description of how to acquire the performance measure formulas of 
such queueing system is given.
Section \ref{sec:model_overview} gives an overview of the game theoretic model
and several
theoretic results pertaining to the performance measures of this model which 
are used to build the utilities of the game.