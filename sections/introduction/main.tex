\section{Introduction}

% TODO: If this is one paper remove the queueing model part

Emergency departments (EDs) are under increasing pressure to meet patient
waiting time targets and satisfy 
regulations~\cite{EmergencyDepartmentWinterPressures}.
It is widely reported (e.g.~\cite{mirror, thenews, bmj}) that ED congestion 
severely impacts not only patients in the ED but also Emergency Medical 
Services (EMS).
A major concern for ambulances is that they are held waiting parked outside the
ED to offload (dispatch) their patient when the ED is particularly 
busy~\cite{clarey2014ambulance}. 
Since the patient waiting time in ED is measured from the time they enter the
ED itself, there is no incentive, should the patient be stable in the
ambulance, to offload them from EMS to ED services.
As a result, ambulance blocking not only impacts on patients waiting for ED
service, but has a major knock-on effect to delaying the ability of ambulances
to respond to new EMS calls, thus placing lives at risk~\cite{eastanglia}.
This paper aims to describe the EMS-ED interface using a game theoretic model
informed by an underlying queueing model. 

A number of papers have been published that touch upon the use of 
queueing models together with game theoretic concepts.
In~\cite{FirmCompetition} the authors study a simultaneous price competition 
between two firms that are modelled as two distinct queueing systems with a 
fixed capacity and a combined arrival rate.
They calculate the Nash equilibrium both for identical and heterogeneous firms
and show that for the former a pure Nash equilibrium always exist and for the 
latter a unique equilibrium exists where only one firm operates.
The authors have also extended their model in~\cite{FirmCompetition2} by 
allowing the players (firms) to choose capacities. 
A main result from this paper was that when both firms operate independently as
a monopoly, the equilibria are socially optimal, but this is not the case when
the firms operate together.
Another extension of~\cite{FirmCompetition} was introduced 
in~\cite{FirmCompetitionExtension} where a long-run version of the competition 
was considered that also had capacity as a decision variable.
In~\cite{knight2017measuring} a normal form game is built that is informed by a 
two-dimensional Markov chain in order to model interactions between critical
care units.
An additional paper that focuses on competition is~\cite{fan2009short} where
the authors created a competition between two sellers where seller 1 supplies 
a product instantly and seller 2 is modelled as a make-to-order M/M/1 queue.
The game that is played requires the two sellers to make a choice on the price 
of the product and then seller 2 to set a capacity that guarantees a maximum 
expected delay.

In the above models, the players are attempting to increase their share of 
individuals choosing to queue.
In public healthcare type settings~\cite{knight2013selfish}, this is not 
necessarily the case. 
Players might not necessarily aim to maximise the so called `market share'.
Rather, the overall service needs to be considered as players aim to minimise
their experienced congestion.
In~\cite{sadat2015can} a healthcare application was studied where patients 
could choose between two hospitals, where a utility function is derived that is
based on patients' perceived quality of life.
How individual behaviour based on congestion impacts the efficiency of public 
services is investigated in~\cite{knight2013selfish}.
The authors place the individuals' choices between different public services 
within the formulation of routing games and measure inefficiencies using a 
concept known as the price of anarchy (PoA)~\cite{koutsoupias1999worst}.
They show that the price of anarchy increases with worth of service and that is
low for systems with insufficient capacities.
In~\cite{deo2011centralized} the authors study the network effect of ambulance 
diversion by proposing a non-cooperative game between two EDs that are modelled
as a queueing network.
Each ED's objective is to minimise its own waiting time and chooses a diversion
threshold based on the patients it has.
In equilibrium both EDs choose to divert ambulances in order to avoid getting
arrivals from the other ED.
In this paper this concept is extended by allowing the ambulance service to 
decide how to distribute its patients among the two EDs.

Another specific part of our research, as described later in the paper, is the 
construction of a queueing system with a tandem buffer and a single service
centre.
In~\cite{d2015pure} the authors explore threshold joining strategies in a 
Markov model that has two tandem queues.
Another example is the one described in~\cite{burnetas2013customer}
where they investigated a network of multiple tandem queues where customers 
decide which queue to attend before joining.
Similarly, in~\cite{bacsar2002stackelberg} the authors examine a network of 
\(N\) tandem M/M/1 queues and with multi-type customers. 
The customers in this paper react to a price \(p\) by picking demand rates that 
maximise utility.
In~\cite{veltman2005equilibrium} a profit maximisation problem is studied that
has two servers; an M/M/1 queue and a parking service providing complementary 
service while the customer is in the first service. 
The providers gain a reward when customers complete both services and no reward 
when they finish one of them.
One of the main conclusions of this study is that by increasing the general 
demand both providers lower their prices to compensate for the increase in wait.
The problem was later extended by~\cite{sun2009equilibrium} where they 
considered arrivals of batches that can share the parking service.
Finally,~\cite{afeche2007decentralized} examines a tandem network of two M/M/1 
queues that are ran by two different profit-maximising service providers.
The network receives three types of customers; those requiring both services, 
customers requiring the first service and customers requiring the second service.
The authors showed that optimal prices also maximise social utility and that
removing two types of customers that don't need both services leads to higher 
profit and lower demand rate.

EDs in the United Kingdom have to follow a set of regulations imposed to 
them by the National Health Service (NHS).
One of these regulations is that 95\% of patients that arrive at the ED should 
be admitted, transferred or discharged within four hours.
This is where gaming behaviour might be observed between the EDs and the EMS.
An assumption of this work is that some managerial decision making is involved
in choosing when to start blocking ambulances.
This is similar to~\cite{deo2011centralized}.

The novelty of our research contribution in this paper is to consider:
\begin{itemize}
    \item The emergence of gaming behaviour between EDs and the EMS.
    \item A queueing model with 2 consecutive waiting spaces where one would 
    serve as a parking space for the ambulances
\end{itemize}
Specifically, our focus is on the construction of a 3-player game theoretic 
model between two queueing systems and a service that distributes individuals
to them. 
The resultant model is then used to explore the emergent dynamics between 
the three players.
Whilst giving some consideration to equilibrium behaviour similarly to the 
work of~\cite{FirmCompetition, FirmCompetition2, deo2011centralized, 
veltman2005equilibrium}, emergent behaviour is more precisely addressed by 
considering a number of learning algorithms.
This study explores two new concepts: getting performance measures for a new
queueing theoretic model with a tandem buffer and a single service centre and 
using a learning algorithm to model the emergence of behaviour.
The developed theoretical model is then illustrated through the application to 
a healthcare system of two EDs and the EMS, exploring the inefficiencies that 
emerge and ways to apply some incentive mechanisms to improve them.
The EDs are modelled as two queueing systems each with a tandem buffer and a 
service centre. 
The performance measures are then used as the utilities of the game.
The novelty of the queueing model here is a contribution not only the game 
theoretic literature but also to the queueing theoretic literature.
To the authors knowledge, no such model of a tandem queueing model with a pair 
of parameters for the buffer has been considered.
% TODO Be sure if this is correct
Although our research is motivated by the particular EMS-ED example, our 
developed modelling framework and behavioural insights has application to 
similar systems across a range of sectors and settings. 

This paper first gives an overview of the game theoretic model, then goes on 
to describe a novel queueing model, and then follows with a number of theoretic 
results pertaining to the performance measures of this model. 
These are used to build the utilities of the game.
Finally, there is a section that describes how this game theoretic framework
can be applied at the interface between the EMS and EDs together with some 
analysis on the behaviour that is observed.
