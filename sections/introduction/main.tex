\section{Introduction}


% 1. Introduce the topic (Write a hook) [X]
% 2. Describe the background [\]
%   - Literature review []
%       - Strategic gaming in healthcare []
%       - Wider search of news reports -> BMJ []
%       - Pressures in hospital system?? []
%   - Gap in the literature []
%   - Problem driven AND Methodology driven []
% 3. Establish the research problem [\]
%   - Bullet points about why paper is important []
% 4. Specify the objective(s) [\]
%   - What is the objective? []
%   - So what? []
% 5. Map out the paper []

% 1.
A huge issue for ambulances in the United Kingdom is that they stay 
parked outside hospitals for a huge amount of time. 
This is a problem for the ambulance service, since the longer ambulances are
blocked in the hospital's parking space the longer the patients in them will
have to wait to receive their treatment but also the longer future patients 
will be stuck waiting for an ambulance.
This paper aims to describe a game theoretic model that is informed by a 
queueing model motivated by this scenario.

% 2.1 Problem driven Literature
There are various news reports that indicate that this is in fact an ongoing 
issue that affects both the patients and the ambulance service.
% TODO: Reference papers/news reports

% 2.2 Methodology driven Literature
In the past a number of papers have been published that touch upon the use of 
queueing models together with game theoretic concepts.
In \cite{FirmCompetition} the authors study a simultaneous price competition 
between two firms and calculate the Nash equilibrium both for identical and 
heterogeneous firms. 
The authors have also extended their model in \cite{FirmCompetition2} by 
allowing the players (firms) to choose capacities. 
A main result from this paper was that equilibria, where both firms operate
together, are not socially optimal while when both firms operate as a monopoly 
they are.
Another extension of \cite{FirmCompetition} was introduced in 
\cite{FirmCompetitionExtension} where a long-run version of the competition was 
considered that also had capacity as a decision variable.
% Moreover, \cite{touati2004performance} analyse the case where there are two 
% M/M/1 queues with capacities \(\mu_1, \mu_2\) and prices \(p_1 > p_2\).
An additional paper that focuses on competition is \cite{fan2009short} where
the authors created a competition between 2 sellers with products having 
different implementation costs.
In \cite{sadat2015can} a healthcare application was studied where patients 
could choose between two hospitals, where a utility function is derived that is
based on patients' perceived quality of life.

Another specific part of this research is the construction of a queueing system
with two tandem queues.
In \cite{d2015pure} the authors explore threshold joining strategies in a 
Markov model that has two tandem queues.
Another great example is the one described in \cite{burnetas2013customer}
where they investigated a network of multiple tandem queues where customers 
decide which queue to attend before joining.
Similarly, in \cite{bacsar2002stackelberg} the authors examine a network of 
\(N\) tandem M/M/1 queues and with multi-type customers. 
The customers in this paper react to a price \(p\) by picking demand rates that 
maximise utility.
In \cite{veltman2005equilibrium} a profit maximisation problem is studied that
has an M/M/1 queue and a parking service providing complementary service while
the customer is in service. 
The problem was later extended by \cite{sun2009equilibrium} where they 
considered arrivals of batches that can share the parking service.
Finally, \cite{afeche2007decentralized} examines a tandem network of two M/M/1 
queues that are ran by two different profit-maximising service providers and 
receive three different types of customers.


% 3.
The emergency departments (EDs) in the United Kingdom have to follow some set 
of regulations imposed to them by the NHS.
One of these regulations is that 95\% of patients that arrive at the ED should 
be admitted, transferred or discharged within four hours.
This is where gaming behaviour might be observed between the EDs and the EMS.
An assumption of this work is that some managerial decision making is involved
in choosing when to start blocking ambulances.
This paper touches upon the following new aspects:
\begin{itemize}
    \item The emergence of gaming behaviour in the emergency department
    \item A queueing model with 2 consecutive waiting spaces
\end{itemize}

% 4.
The main focus of this research is the construction of a 3-player game 
theoretic model between two queueing systems and a service that distributes 
individuals to them. 
The resultant model will then be used to explore the emergent dynamics between 
the three players.

% 5.
This paper first gives an overview of the game theoretic model that was built, 
then goes on to describe how the queueing models were constructed and then 
follows with how the methodology that was used to build the game.
Finally, there is a section that describes how this game theoretic framework
can be applied at the interface between the Emergency Medical Services (EMS) 
and Emergency Department (EDs).

% TODO: If this is one paper remove the queueing model part




% Paper suggested by Paul:https://www.magonlinelibrary.com/doi/pdf/10.12968/prma.2021.31.4.28?casa_token=cuKE80im0tsAAAAA%3AFYuYJsk0Qd9XLvPAaDmZVUW-BPU8xJmHGeDRRPbfdCld4cJiBs3-mmoBxewtCLtSXndrZWTbqUbe1g&
% Paper suggested by Vince: Vince's and Izabella