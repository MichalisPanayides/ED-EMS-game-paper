\section{Conclusion}

The motivation behind this study has been that emergency departments are 
constantly under a lot of pressure to satisfy some regulations. 
This paper shows how this can negatively impact the pathway of both the 
ambulance patients and the ambulance service itself.
Due to some managerial decision making that takes place at the ED, ambulances 
stay blocked outside of the ED at the hospital's parking space in an attempt
to satisfy these regulations.

This study explores a generic 3-player game theoretic model between the 
decision makers of two queueing systems and a service that distributes 
individuals to these two systems (section \ref{sec:model_overview}).
It also describes the construction of the underlying queueing theoretic model 
that has a tandem buffer and a single service centre (section 
\ref{sec:queueing_model}).
Furthermore, the formulas for the performance measures of the queueing model 
are also derived (sections \ref{sec:waiting_time}, \ref{sec:blocking_time}, 
\ref{sec:proportion_within_target}). 
This novel queuing model is the first contribution of the paper.
The game theoretic model is then applied to a healthcare scenario by looking at
the interface between the EDs and the EMS (section 
\ref{sec:ems_ed_application}).
The inefficiencies that emerge from the perspective of the EMS were explored 
along with ways to apply some incentive mechanisms to improve them.
The main takeaways from this paper that were observed when playing the game
between two EDs and the EMS (section \ref{sec:application}) are:
\begin{itemize}
    \item Inefficiencies can be learned and emerge naturally
    \item Targeted incentivisation of behaviours can help escape inefficiencies
\end{itemize}
The former relates to the results of asymmetric replicator dynamics that showed 
that inefficient scenarios can arise by playing the game while the latter 
implies that the learned inefficiencies can be escaped by carefully applying 
certain incentives to the players.
This applied game theoretic model is the second main contribution of this paper.

The model presented here assumes the presence of two players that can receive 
individuals. 
However, in a realistic healthcare scenario an ambulance may have to decide 
among multiple EDs.
An immediate extension of this work would be to consider a multiplayer system
that could represent a group of hospitals in a concentrated area.
Additionally, the game theoretic model that was created uses a discrete 
strategy space for the EDs (something that is also present in various pieces of 
literature~\cite{deo2011centralized, knight2017measuring}).
The single threshold parameter that is used for the ED's decision may not be 
the best way to describe the model.
In reality ED managers could have far more complex parameters for their 
decision making process.
Finally this work assumes that ambulances and EDs act in a selfish and rational
way by only aiming to satisfy their own objectives.
In fact, modelling behaviour in a healthcare setting is a much more complex 
task where some cooperation is often observed.
Another extension would be to explore the behaviour of the ED staff via an 
agent-based model. 
This in turn can be used to model ED staff as agents each with their own 
behavioural traits.
