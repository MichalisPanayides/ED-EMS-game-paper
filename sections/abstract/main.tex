
\begin{abstract}
This research describes the development and application of a 3-player game 
theoretic model between two queueing systems and a service that distributes 
individuals to them. 
The resultant model is used to explore dynamics between all players.
The first aspect of this work is the development of a queueing system with two
consecutive waiting spaces where
the strategic managerial behaviour corresponds to how individuals use these 
waiting spaces. 
Two modelling techniques are deployed: discrete event simulation and Markov 
chains.
The state probabilities of the Markov chain system are used to extract 
the performance measures of the queueing model (e.g. mean time in each waiting
room, mean number of individuals in each room, etc.).
A 3-player game theoretic model is subsequently proposed between the two 
queueing systems and the service that distributes individuals to them.
In particular this can be viewed as a 2-player normal-form game where the 
utilities are determined by a third player with its own strategies and 
objectives. 
A backwards induction technique is used to get the utilities of the normal-form
game between the two queueing systems.
This particular system has many applications, including those in healthcare
where it captures the emergent behaviour between the Emergency Medical 
Service (EMS) and the Emergency Department (ED). 
The impact of time-target measures on patient well-being is explored in this 
paper.
\end{abstract}

\begin{keyword}
    OR in health services \sep Game theory \sep Queueing theory 
    \sep Behavioural modelling
\end{keyword}
