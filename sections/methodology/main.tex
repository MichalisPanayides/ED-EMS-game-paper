\section{Methodology}\label{sec:methodology}

Th problem defined in section \ref{sec:model_overview} describes a normal-form
game between the decision makers of two queueing systems and a third player that 
decides how to distribute individuals to the systems.
The strategy space of the two queueing systems is defined as the possible values
that the threshold parameter can take \(T_i \in (1, N_i)\).
Then, the distributor has to decide on the proportion of individuals to send to 
each queueing system \(p_A \text{ and } p_B\), where \(p_A, p_B \in [0, 1] \)
and \(p_A + p_B = 1\).
Figure \ref{fig:imperfect_info_game} form section \ref{sec:model_overview} 
shows a diagrammatic representation of the game to be played and the decisions 
to be made.
As described in section \ref{sec:model_overview} queueing system \(A\) decides
on a strategy, at the same time, queueing system \(B\) chooses its own threshold, unaware 
of the first queueing system's choice.
Finally, the distributor makes its choice based on the strategies that the 
queueing systems chose to play. 

The utilities to each player can be represented by 3 matrices; the two payoff matrices of the 
normal form game and the routing matrix.
The payoff matrices and their utilities are defined by equations 
(\ref{eq:payoff_entry}) and (\ref{eq:payoff_matrices}) in section 
\ref{sec:model_overview}.

The routing matrix holds the values \((p_A, p_B)\) which are the proportion 
of class 2 individuals to send to queueing system \(A\) and \(B\).
Each pair \((p_A, p_B)\) can be calculated using equation 
(\ref{eq:obj_distributor}) and is essentially
a best response to the actions of the queueing systems.
Equation \ref{eq:obj_distributor} is defined and explained in section 
\ref{sec:model_overview}.
Thus, using equation \ref{eq:obj_distributor} for all possible sets of 
thresholds we can get the full routing matrix that consists of the proportions
to send to queueing system A (\(p_A\)) and to queueing system B (\(p_B\)).

\begin{equation}\label{eq:routing_matrix}
    R = 
    \begin{pmatrix}
        (p_{1,1}^A, p_{1,1}^B) & (p_{1,2}^A, p_{1,2}^B) & \dots & 
        (p_{1,N_B}^A, p_{1,N_B}^B) \\
        (p_{2,1}^A, p_{2,1}^B) & (p_{2,2}^A, p_{2,2}^B) & \dots & 
        (p_{2,N_B}^A, p_{2,N_B}^B) \\
        \vdots & \vdots & \ddots & \vdots \\
        (p_{N_A,1}^A, p_{N_A,1}^B) & (p_{N_A,2}^A, p_{N_A,2}^B) & \dots & 
        (p_{N_A,N_B}^A, p_{N_A,N_B}^B) \\
    \end{pmatrix}
\end{equation}

Note that since \(p_{i,j}^A + p_{i,j}^B = 1\) the routing matrix needs only to
store one of the two values; either \(p_{i,j}^A\) or \(p_{i,j}^B\).
Thus, the routing matrix \(R\) can be simplified to:

\begin{equation}\label{eq:routing_matrix_simplified}
    R = 
    \begin{pmatrix}
        p_{1,1}^A & p_{1,2}^A & \dots & p_{1,N_B}^A \\
        p_{2,1}^A & p_{2,2}^A & \dots & p_{2,N_B}^A \\
        \vdots & \vdots & \ddots & \vdots \\
        p_{N_A,1}^A & p_{N_A,2}^A & \dots & p_{N_A,N_B}^A \\
    \end{pmatrix}
\end{equation}

The game can thus be partitioned into a normal form game between the
two queueing systems and then finding the distributor's best strategy. 

\subsection{Backwards Induction}

In order to populate the payoff matrices and the routing matrix the method
of backwards induction is used.
Consider figure \ref{fig:imperfect_info_game} and the flow of the game that was
described (i.e. \(Q_A, Q_B \rightarrow D\)).
Due to the fact that the payoff matrices \(A\) and \(B\) depend on the routing 
matrix \(R\) the entries of the matrices are calculated in a backwards way 
(\(D \rightarrow Q_A, Q_B\)). 
Thus for every pair of strategies \(T_A, T_B\), the proportion of individuals 
that are to be transported to each queueing system is calculated first. 

Consider a game where the capacities of the two systems are \(N_A = 4\) and 
\(N_B = 3\).
The strategy space of players \(Q_A\) and \(Q_B\) would then be 
\(T_A = \{1, 2, 3, 4\}\) and \(T_B = \{1, 2, 3\}\) respectively.
Now, starting from an arbitrary starting point of \(T_A=1\) and \(T_B=1\), the
corresponding entry of the routing matrix the routing matrix can be calculated.
Using equation (\ref{eq:obj_distributor}) for \(T_A=1\) and \(T_B=1\):
\begin{align}
    & \alpha L_A(p_A;T_A=1;T_B=1) + (1 - \alpha) B_A(p_A;T_A=1;T_B=1) = 
    \nonumber \\
    & \alpha L_B(p_B;T_A=1;T_B=1) + (1 - \alpha) B_B(p_B;T_A=1;T_B=1) 
    \label{eq:obj_distributor_1_1}
\end{align}

The values of \(p_A\) and \(p_B\) that satisfy equation 
(\ref{eq:obj_distributor_1_1}) are here found numerically using Brent's bisection
algorithm~\cite{brent1973algorithms}.
The calculated value of \(p_A\) corresponds to the entry on the first row and 
first column of the routing matrix:

\begin{equation}\label{eq:routing_matrix_1_1}
    R = 
    \begin{pmatrix}
        p_{1,1}^A & - & - \\
        - & - & - \\
        - & - & - \\
        - & - & - \\
    \end{pmatrix}
\end{equation}

In fact all remaining values of the routing matrix can be calculated by
solving \(N \times M\) similar equations using Brent's bisection algorithm.
Therefore, having calculated the entire routing matrix the payoff matrices can
be calculated.
Consider once again the case of \(T_A=1, T_B=1\). 


\begin{align}
    U_{1, 1}^A = 1 -\left( 
        \hat{P} - P(W_A(T_A=1, T_B=1) < t) 
    \right)^2 \nonumber
    \\
    U_{1, 1}^B = 1 -\left( 
        \hat{P} - P(W_B(T_A=1, T_B=1) < t) 
    \right)^2 \label{eq:payoff_entry_1_1}
\end{align}

These are the utilities of players \(A\) and \(B\) when both players choose a
strategy of \(T = 1\).
Here \(R\) is the predefined waiting time target to be met by a percentage of 
individuals \( \hat{P} \) and \(P(W_i < t)\) is properly defined in section
\ref{sec:proportion_within_target}.

\begin{equation}\label{eq:payoff_matrices_1_1}
    A = 
    \begin{pmatrix}
        U_{1, 1}^A & - & - \\
        - & - & - \\
        - & - & - \\
        - & - & - \\
    \end{pmatrix}, \quad
    B = 
    \begin{pmatrix}
        U_{1, 1}^B & - & - \\
        - & - & - \\
        - & - & - \\
        - & - & - \\
    \end{pmatrix}
\end{equation}

Similar to the routing matrix, the above procedure can be repeated for all 
possible values of \(T_A\) and \(T_B\) to populate all entries of the payoff 
matrices. 

\subsection{Nash Equilibrium}\label{sec:methodology_nash_equilibrium}

Having calculated the payoff matrices \(A\) and \(B\), several algorithms can 
be used to measure some form of emergent behaviour.
One possibility would be to compute the Nash equilibrium
which is the point of the game were both players have no 
incentive to deviate from their played strategies \cite{kreps1989nash}.
In other words their chosen strategies are a best response to each other.

Computation of Nash equilibria can be done relatively efficiently using the Lemke Howson algorithm
\cite{LemkeHowson}.
Lemke-Howson uses best response polytopes to get one of the Nash equilibrium of
the game. 
Other algorithms exists that will compute all Nash equilibria but 
for large games the computational complexity becomes problematic.
All game theoretic calculations were done in Python using the Nashpy library 
\cite{axelrodproject}.

\subsection{Learning Algorithms}\label{sec:methodology_learning_algorithms}

Another approach to measuring emergent behaviour is to consider 
the emergence itself and not just stable end points. 
Indeed, some Nash equilibria might not arise naturally. Thus, in order to analyse the strategies played by the two EDs, the learning algorithm:
asymmetric replicator dynamics is used \cite{asymmetricreplicatordynamics}.
The two players are modelled as two separate populations where each 
individual in the population is assigned a strategy.
As the game progresses the proportion of each player playing each strategy 
changes based on their previous interactions.
The fitness of each strategy is defined as:

\begin{equation}\label{eq:asymmetric_fitness}
    f_x = Ay, \quad f_y = x^T B
\end{equation}

Here, \(x \in \mathbb{R}^{m \times 1} \) and
\(y \in \mathbb{R}^{n \times 1}\) correspond to the proportion of individuals 
that play each strategy in each population.
Similarly, the average fitness of each strategy is given by:
\begin{equation}
    \phi_x = f_x x^T, \quad \phi_y = f_y y
\end{equation}

The rate of change of strategy \( i \) of both types of individuals is captured 
by:
\begin{equation}
    \frac{dx}{dt}_i = x_i((f_x)_i - \phi_x), \quad \text{ for all }i
\end{equation}

\begin{equation}
    \frac{dy}{dt}_i = y_i((f_y)_i - \phi_y), \quad \text{ for all }i
\end{equation}
 
In addition to asymmetric replicator dynamics, the learning algorithms
fictitious play and stochastic fictitious play \cite{fudenberg1998theory}
were used.