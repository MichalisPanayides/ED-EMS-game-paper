\section{Literature review}

% Literature review
A number of papers have been published that touch upon the use of 
queueing models together with game theoretic concepts.
In~\cite{FirmCompetition} the authors study a simultaneous price competition 
between two firms that are modelled as two distinct queueing systems with a 
fixed capacity and a combined arrival rate.
They calculate the Nash equilibrium both for identical and heterogeneous firms
and show that for the former a pure Nash equilibrium always exist and for the 
latter a unique equilibrium exists where only one firm operates.
The authors have also extended their model in~\cite{FirmCompetition2} by 
allowing the players (firms) to choose capacities. 
A main result from this paper was that when both firms operate independently as
a monopoly, the equilibria are socially optimal, but this is not the case when
the firms operate together.
Another extension of~\cite{FirmCompetition} was introduced 
in~\cite{FirmCompetitionExtension} where a long-run version of the competition 
was considered that also had capacity as a decision variable.
An additional paper that focuses on competition is~\cite{fan2009short} where
the authors created a competition between two sellers where seller 1 supplies 
a product instantly and seller 2 is modelled as a make-to-order M/M/1 queue.
The game that is played requires the two sellers to make a choice on the price 
of the product and then seller 2 to set a capacity that guarantees a maximum 
expected delay.
In our work, while giving some consideration to equilibrium behaviour,
similar to the work of~\cite{FirmCompetition, FirmCompetition2}, emergent
behaviour is more precisely addressed by considering learning algorithms like
asymmetric replicator dynamics~\cite{fudenberg1998theory}.

In the above models, the players are attempting to increase their share of 
individuals choosing to queue.
In public healthcare type settings, this is not 
necessarily the case. 
Rational usage of public services will not necessarily lead to a socially
optimal outcome.
Rather, the overall service needs to be considered as players aim to minimise
their experienced congestion.
In~\cite{sadat2015can} a healthcare application was studied where patients 
could choose between two hospitals, where a utility function is derived that is
based on patients' perceived quality of life.
In~\cite{knight2013selfish} the authors place the individuals' choices between
different public services 
within the formulation of routing games and measure inefficiencies using a 
concept known as the price of anarchy (PoA)~\cite{koutsoupias1999worst}.
They show that the price of anarchy increases with worth of service and that is
low for systems with insufficient capacities.
In~\cite{knight2017measuring} a normal form game is built that is informed by a 
two-dimensional Markov chain in order to model interactions between critical
care units.
In~\cite{deo2011centralized} the authors study the network effect of ambulance 
diversion by proposing a non-cooperative game between two EDs that are modelled
as a queueing network.
Each ED's objective is to minimise its own waiting time and chooses a diversion
threshold based on the patients it has.
In equilibrium both EDs choose to divert ambulances in order to avoid getting
arrivals from the other ED.
In this paper this concept is extended by allowing the ambulance service to 
decide how to distribute its patients among the two EDs.
The players of the game are both the hospitals and the customers of the
hospitals, as opposed to the previous models which are one or the other.
Thus, the novelty of our work is combining both these aspects.

Another specific part of our research, as described later in the paper, is the 
construction of a queueing system with a tandem buffer and a single service
centre.
There are several examples from literature that touch upon queueing models
with tandem queues.
In~\cite{d2015pure} the authors explore threshold joining strategies in a 
Markov model that has two tandem queues.
Another example is the one described in~\cite{burnetas2013customer}
where they investigated a network of multiple tandem queues where customers 
decide which queue to attend before joining.
Similarly, in~\cite{bacsar2002stackelberg} the authors examine a network of 
\(N\) tandem M/M/1 queues and with multi-type customers. 
The customers in this paper react to a price \(p\) by picking demand rates that 
maximise utility.
In~\cite{veltman2005equilibrium} a profit maximisation problem is studied that
has two servers; an M/M/1 queue and a parking service providing complementary 
service while the customer is in the first service. 
The providers gain a reward when customers complete both services and no reward 
when they finish one of them.
One of the main conclusions of this study is that by increasing the general 
demand both providers lower their prices to compensate for the increase in wait.
The problem was later extended by~\cite{sun2009equilibrium} where they 
considered arrivals of batches that can share the parking service.
Finally,~\cite{afeche2007decentralized} examines a tandem network of two M/M/1 
queues that are ran by two different profit-maximising service providers.
The network receives three types of customers; those requiring both services, 
customers requiring the first service and customers requiring the second service.
The authors showed that optimal prices also maximise social utility and that
removing two types of customers that don't need both services leads to higher 
profit and lower demand rate.
In our work, the concepts described in~\cite{d2015pure, burnetas2013customer,
bacsar2002stackelberg} are extended by introducing a threshold parameter that 
determines when individuals can progress from one queue to the other.