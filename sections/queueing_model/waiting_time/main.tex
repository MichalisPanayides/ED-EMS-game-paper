\subsection{Waiting time} \label{sec:waiting_time}

Waiting time is the amount of time that patients from either type wait in 
the hospital's waiting space before they can receive their service. 
For a given set of parameters there are three different performance measures 
around the mean waiting time that can be calculated. The mean waiting time of
type 1 individuals:

\begin{equation} \label{eq:closed_form_waiting_type_1}
    W^{(1)} = \frac{\sum_{\substack{(u,v) \, \in S_A^{(1)} \\ v \geq C}} 
    \frac{1}{C \mu} \times (v-C+1) \times \pi(u,v)}{\sum_{(u,v) \, 
    \in S_A^{(1)}} \pi(u,v)}
\end{equation}

The mean waiting time of type 2 individuals:

\begin{equation}\label{eq:closed_form_waiting_type_2}
    W^{(2)} = \frac{\sum_{\substack{(u,v) \, \in S_A^{(2)} \\ min(v,T) \geq C}} 
    \frac{1}{C \mu} \times (\min(v+1,T)-C) \times \pi(u,v)}{\sum_{(u,v) \, 
    \in S_A^{(2)}} \pi(u,v)}
\end{equation} 

The overall mean waiting time:

\begin{equation}\label{eq:overall_waiting_time}
    W = \frac{\lambda_1 P_{L'_1}}{\lambda_2 P_{L'_2} + \lambda_1 P_{L'_1}} W^{(1)} 
    + \frac{\lambda_2 P_{L'_2}}{\lambda_2 P_{L'_2} + \lambda_1 P_{L'_1}} W^{(2)}
\end{equation}
 
Here \(S_A^{(1)}\) and \(S_A^{(2)}\) are the set of accepting states for type
1 and type 2 individuals. These are the set of states that the model is able
to accept a specific type of individuals.

\begin{equation}\label{eq:accepting_states_type_1}
    S_A^{(1)} = \{(u, v) \in S \; | \; v < N \}
\end{equation}

\begin{equation}\label{eq:accepting_states_type_2}
    S_A^{(2)}=
    \begin{cases}
        \{(u, v) \in S \; | \; u < M \}, & \textbf{if } T \leq N\\
        \{(u, v) \in S \; | \; v < N \}, & \textbf{otherwise}
    \end{cases}
\end{equation}

Equation \ref{eq:overall_waiting_time} makes use of the proportion of type 1 
and type 2 individuals that are not lost to the system. These probabilities are 
given by \(P_{L'_1}\) and \(P_{L'_2}\) where:

\begin{equation}\label{eq:proportion_of_accepting_individuals}
    P_{L'_1} = \sum_{(u,v) \, \in S_A^{(1)}} \pi(u,v) \hspace{2cm}
    P_{L'_2} = \sum_{(u,v) \, \in S_A^{(2)}} \pi(u,v)
\end{equation}
 
Appendix \ref{sec:appendix_mean_waiting} gives more details on the recursive
formula that equations (\ref{eq:closed_form_waiting_type_1}),
(\ref{eq:closed_form_waiting_type_2}) and (\ref{eq:overall_waiting_time})
originate from. 

Figure \ref{fig:markov_vs_des_waiting_time_comparison_overall} shows a 
comparison between the calculated mean waiting time using Markov chains and the
simulated waiting time using discrete event simulation over a range of values of 
\(\lambda_2\) (details of the discrete event simulation model are given in 
appendix~\ref{sec:appendix_des}).
The simulation was ran 100 times and the recorded mean waiting time at each 
iteration is used to populate the violin plots.
The waiting times generated by the simulation match the ones generated by the 
Markov chains model.
Note that this comparison includes both type 1 and type 2 individuals.
A separate comparison of only type 1 and only type 2 individuals can be found 
in appendix~\ref{sec:appendix_additional_figures}.

\begin{figure}[H]
    \centering
    \includegraphics[width=.8\textwidth]{imgs/waiting_time_comparison/waiting_overall.pdf}
    \caption{
        Comparison of mean waiting time for both types of individuals between 
        values obtained from the Markov chain formulas and values obtained from 
        simulation.
    }
    \label{fig:markov_vs_des_waiting_time_comparison_overall}
\end{figure}
