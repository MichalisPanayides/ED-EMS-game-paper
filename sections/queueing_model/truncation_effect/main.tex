\subsection{Truncation effect timings} \label{sec:truncation_effect}

The choice of the artificial parameters \(N\) and \(M\) is an important 
decision of the model.
In the untruncated simulation these values are set to be infinite. 
This is more computationally expensive to run.
Table \ref{tab:truncation_effect_timings} shows the relative timings of the
different approaches used to get the performance measures.

\begin{table}[h]
    \centering
    \begin{tabular}{c|cc|ccc}
        & \multicolumn{2}{c}{\textbf{Simulation timings}} & 
        \multicolumn{3}{c}{\textbf{Markov chain timings}} \\
        \textbf{Value of} & \textbf{Single} & \textbf{A hundred} & 
        \textbf{Waiting time} & \textbf{Blocking time} & 
        \textbf{Proportion within} \\
        \textbf{N and M} & \textbf{trial} & \textbf{trials} & 
        \textbf{formula} & \textbf{formula} & \textbf{time formula} \\
        \hline
        \(10\) & 1 & 144.3 & 0.015 & 0.014 & 0.014 \\
        \hline
        \(30\) & 1 & 143.4 & 3.731 & 3.828 & 3.649 \\
        \hline
        \(50\) & 1 & 139.8 & 31.57 & 38.39 & 31.98 \\
        \hline
        \(\infty\) & 1 & 142.1 & N/A & N/A & N/A \\
    \end{tabular}
    \caption{Relative timings of the simulation and Markov chain model}
    \label{tab:truncation_effect_timings}
\end{table}
