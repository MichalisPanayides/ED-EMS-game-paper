\section{A queueing model with 2 consecutive buffer centres}

% TODO: If this is to be one paper maybe the first two paragraphs do not need to
% include so much details or even better omit some of the details in the game
% overview section.
In this section, a more in-depth explanation of the queueing model shown in 
figure \ref{fig:diagram_of_queueing_system} will be given.
This is a queuing model that consists of two waiting spaces, one for each type
of individual.

The model consists of two types of individuals; class 1 and class 2.
Class 1 individuals arrive instantly at waiting zone 1 and proceed to wait to
receive their service. 
Class 2 individuals arrive at waiting zone 2 and wait there until they are 
allowed to move to waiting zone 1. 
They are allowed to proceed only when the number of 
individuals in waiting zone 1 \textbf{and} in service is less than a 
pre-determined threshold \(T\).
When the number of individuals is equal to or exceeds this threshold, all second type individuals that arrive will remain 
\textit{``blocked''} in waiting zone 1 until the number of people in the 
system is reduced below \(T\). 
This is shown diagrammatically in figure \ref{fig:diagram_of_queueing_system}.
The parameters of the described queueing model are:

\begin{itemize}
    \item \(\lambda_i\): The arrival rate of individuals of type \(i\in\{1, 2\}\)
    \item \(\mu\): The service rate for individuals receiving service
    \item \(C\): The number of servers
    \item \(T\): The threshold at which individuals of the second type are blocked
\end{itemize}

Under the assumption that all rates (arrival and service) are Markovian the
queuing system corresponds to a Markov chain~\cite{kemeny1976markov}.
The states of the Markov chain are denoted by \((u,v)\) where:

\begin{itemize}
    \item \(u\) is the number of individuals blocked
    \item \(v\) is the number of individuals either in waiting zone 1 or in the
    service centre
\end{itemize}

We denote the state space of the Markov chain as  \(S=S(T)\) which can be 
written as the disjoint union (\ref{eq:definition_of_S_as_disjoint_union}).

\begin{align}
    S(T) =& S_1(T) \cup S_2(T) \text{ where:} \nonumber \\
    S_1(T) =& \left\{(0, v)\in\mathbb{N}_0^2 \; | \; v < T \right\} 
    \label{eq:definition_of_S_as_disjoint_union} \\
    S_2(T) =& \{(u, v)\in\mathbb{N}_0^2 \; | \; v \geq T \} \nonumber
\end{align}

The transition matrix \(Q\) of the Markov chain consists of the transition rates
between the numerous states of the model. Every entry \( Q_{ij} = 
Q_{(u_i, v_i),(u_j, v_j)} \) represents the transition rate from state 
\( i = (u_i, v_i) \) to state \( j = (u_j , v_j) \) for all 
\( (u_i, v_i), (u_j, v_j) \in S \).
The entries of \(Q\) can be calculated using the state-mapping function 
described in (\ref{eq:markov_transition_rate}): 

\begin{equation} \label{eq:markov_transition_rate}
    Q_{ij} = 
    \begin{cases}
        \Lambda, & \textbf{if } (u_i, v_i) - (u_j, v_j) = (0,-1) \textbf{ and } 
        v_i < \text{t} \\
        \lambda_1, & \textbf{if } (u_i, v_i) - (u_j, v_j) = (0,-1) 
        \textbf{ and } v_i \geq \text{t} \\
        \lambda_2, & \textbf{if } (u_i, v_i) - (u_j, v_j) = (-1,0) \\
        v_i \mu, & \textbf{if } (u_i, v_i) - (u_j, v_j) = (0,1) \textbf{ and } 
        v_i \leq C \textbf{ or} \\ & \hspace{0.37cm}(u_i, v_i) - (u_j, v_j) = 
        (1,0) \textbf{ and } v_i = T \leq C \\
        C \mu, & \textbf{if } (u_i, v_i) - (u_j, v_j) = (0,1) \textbf{ and } 
        v_i > C 
        \textbf{ or} \\ & \hspace{0.37cm}(u_i, v_i) - (u_j, v_j) = (1,0) 
        \textbf{ and } v_i = T > C\\
        -\sum_{j=1}^{|Q|}{Q_{ij}} & \textbf{if } i = j \\
        0, & \textbf{otherwise}
    \end{cases}
\end{equation}

Note that \(\Lambda\) here denotes the overall arrival rate in the model by both 
classes of individuals (i.e. \(\Lambda = \lambda_1 + \lambda_2\)). 
A visualisation of how the transition rates relate to the states of the model 
can be seen in the general Markov chain model shown in figure 
\ref{fig:general-markov-model}.

\documentclass{article}

\usepackage{amsmath}
\usepackage{amsfonts} 
\usepackage{geometry}
\usepackage{multicol}
\usepackage{float}
% \usepackage{mathtools}
% \usepackage{graphicx}
% \usepackage{soul}
% \usepackage{indentfirst}
\usepackage{tikz}
\usetikzlibrary{calc, automata, chains, arrows.meta, math}
\setcounter{MaxMatrixCols}{20}


\title{A game theoretic model of the behavioural gaming that takes place at the EMS - ED interface}

\author{
    Michalis Panayides, 
    Paul Harper, 
    Vince Knight
}

\begin{document}

\maketitle

\documentclass{article}

\usepackage{amsmath}
\usepackage{amsfonts} 
\usepackage{geometry}
\usepackage{multicol}
\usepackage{float}
% \usepackage{mathtools}
% \usepackage{graphicx}
% \usepackage{soul}
% \usepackage{indentfirst}
\usepackage{tikz}
\usetikzlibrary{calc, automata, chains, arrows.meta, math}
\setcounter{MaxMatrixCols}{20}


\title{A game theoretic model of the behavioural gaming that takes place at the EMS - ED interface}

\author{
    Michalis Panayides, 
    Paul Harper, 
    Vince Knight
}

\begin{document}

\maketitle

\documentclass{article}

\usepackage{amsmath}
\usepackage{amsfonts} 
\usepackage{geometry}
\usepackage{multicol}
\usepackage{float}
% \usepackage{mathtools}
% \usepackage{graphicx}
% \usepackage{soul}
% \usepackage{indentfirst}
\usepackage{tikz}
\usetikzlibrary{calc, automata, chains, arrows.meta, math}
\setcounter{MaxMatrixCols}{20}


\title{A game theoretic model of the behavioural gaming that takes place at the EMS - ED interface}

\author{
    Michalis Panayides, 
    Paul Harper, 
    Vince Knight
}

\begin{document}

\maketitle

\input{Abstract/main.tex}


\newpage
\tableofcontents

\newpage
\input{Introduction/main.tex}

\newpage
\input{Game_theory_component/main.tex}

\newpage
\input{MarkovChain/markov_chain_model/main.tex}
\input{MarkovChain/expressions_from_pi/main.tex}
\input{MarkovChain/markov_example/main.tex}

\newpage
\input{BehaviouralMethodology/main.tex}

\newpage
\input{Application_EMS_ED/main.tex}

\newpage
\input{Conclusion/main.tex}


\end{document}


\newpage
\tableofcontents

\newpage
\documentclass{article}

\usepackage{amsmath}
\usepackage{amsfonts} 
\usepackage{geometry}
\usepackage{multicol}
\usepackage{float}
% \usepackage{mathtools}
% \usepackage{graphicx}
% \usepackage{soul}
% \usepackage{indentfirst}
\usepackage{tikz}
\usetikzlibrary{calc, automata, chains, arrows.meta, math}
\setcounter{MaxMatrixCols}{20}


\title{A game theoretic model of the behavioural gaming that takes place at the EMS - ED interface}

\author{
    Michalis Panayides, 
    Paul Harper, 
    Vince Knight
}

\begin{document}

\maketitle

\input{Abstract/main.tex}


\newpage
\tableofcontents

\newpage
\input{Introduction/main.tex}

\newpage
\input{Game_theory_component/main.tex}

\newpage
\input{MarkovChain/markov_chain_model/main.tex}
\input{MarkovChain/expressions_from_pi/main.tex}
\input{MarkovChain/markov_example/main.tex}

\newpage
\input{BehaviouralMethodology/main.tex}

\newpage
\input{Application_EMS_ED/main.tex}

\newpage
\input{Conclusion/main.tex}


\end{document}

\newpage
\documentclass{article}

\usepackage{amsmath}
\usepackage{amsfonts} 
\usepackage{geometry}
\usepackage{multicol}
\usepackage{float}
% \usepackage{mathtools}
% \usepackage{graphicx}
% \usepackage{soul}
% \usepackage{indentfirst}
\usepackage{tikz}
\usetikzlibrary{calc, automata, chains, arrows.meta, math}
\setcounter{MaxMatrixCols}{20}


\title{A game theoretic model of the behavioural gaming that takes place at the EMS - ED interface}

\author{
    Michalis Panayides, 
    Paul Harper, 
    Vince Knight
}

\begin{document}

\maketitle

\input{Abstract/main.tex}


\newpage
\tableofcontents

\newpage
\input{Introduction/main.tex}

\newpage
\input{Game_theory_component/main.tex}

\newpage
\input{MarkovChain/markov_chain_model/main.tex}
\input{MarkovChain/expressions_from_pi/main.tex}
\input{MarkovChain/markov_example/main.tex}

\newpage
\input{BehaviouralMethodology/main.tex}

\newpage
\input{Application_EMS_ED/main.tex}

\newpage
\input{Conclusion/main.tex}


\end{document}

\newpage
\documentclass{article}

\usepackage{amsmath}
\usepackage{amsfonts} 
\usepackage{geometry}
\usepackage{multicol}
\usepackage{float}
% \usepackage{mathtools}
% \usepackage{graphicx}
% \usepackage{soul}
% \usepackage{indentfirst}
\usepackage{tikz}
\usetikzlibrary{calc, automata, chains, arrows.meta, math}
\setcounter{MaxMatrixCols}{20}


\title{A game theoretic model of the behavioural gaming that takes place at the EMS - ED interface}

\author{
    Michalis Panayides, 
    Paul Harper, 
    Vince Knight
}

\begin{document}

\maketitle

\input{Abstract/main.tex}


\newpage
\tableofcontents

\newpage
\input{Introduction/main.tex}

\newpage
\input{Game_theory_component/main.tex}

\newpage
\input{MarkovChain/markov_chain_model/main.tex}
\input{MarkovChain/expressions_from_pi/main.tex}
\input{MarkovChain/markov_example/main.tex}

\newpage
\input{BehaviouralMethodology/main.tex}

\newpage
\input{Application_EMS_ED/main.tex}

\newpage
\input{Conclusion/main.tex}


\end{document}
\subsection{Performance Measures}
One may easily derive the average number of individuals that are at any given state 
using \( pi \). 
The average number of individuals in state \( i \) can be calculated by multiplying 
the number of individuals that are present in state \( i \) with the probability 
of being at that particular state (i.e \(\pi_i (u_i + v_i)\)). 
Using this logic it is possible to calculate any performance measures that are related 
to the mean number of individuals in the system.


Average number of people in the system: 
\begin{equation}
    L = \sum_{i=1}^{|\pi|} \pi_i (u_i + v_i)
\end{equation} 

Average number of people in the service centre: 
\begin{equation}
    L_H = \sum_{i=1}^{|\pi|} \pi_i v_i
\end{equation}

Average number of people in the buffer centre:
\begin{equation}
    L_A = \sum_{i=1}^{|\pi|} \pi_i u_i
\end{equation}

Consequently getting the performance measures that are related to the duration of 
time is not as straightforward. 
Such performance measures are the mean waiting time in the system and the mean time 
blocked in the system. 
Under the scope of this study three approaches have been considered to calculate these 
performance measures; a direct approach, a recursive algorithm and consequently a
closed-form formula.

The research question that needs to be answered here is: ``When a class 1/2 
individuals enters the system, what is the expected time that they will have to 
wait?''. 
In order to formulate the answer to that question one needs to consider all possible 
scenarios of what state the system can be in when an individual arrives. 
Furthermore, different formulas arises for class 1 individuals 
and a different one for class 2 individuals.

\subsubsection{Mean waiting time} 
Upon closer inspection of the recursive formula a more compact formula can arise. 
The equivalent closed-form formula eliminates the need for recursion and thus makes 
the computation of waiting times much more efficient. 
Just like in the recursive part there are two formulas; one for \textit{class 1} 
and one for class 2 individuals. 
The formulas are given by:

\begin{equation} \label{eq:closed_form_waiting_others}
    W^{(1)} = \frac{\sum_{\substack{(u,v) \, \in S_A^{(1)} \\ v \geq C}} 
    \frac{1}{C \mu} \times (v-C+1) \times \pi(u,v)}{\sum_{(u,v) \, 
    \in S_A^{(1)}} \pi(u,v)}
\end{equation}
    
\begin{equation}\label{eq:closed_form_waiting_ambulance}
    W^{(2)} = \frac{\sum_{\substack{(u,v) \, \in S_A^{(2)} \\ min(v,T) \geq C}} 
    \frac{1}{C \mu} \times (\min(v+1,T)-C) \times \pi(u,v)}{\sum_{(u,v) \, 
    \in S_A^{(2)}} \pi(u,v)}
\end{equation}

Note here that the summation, in both equations \ref{eq:closed_form_waiting_others} 
and \ref{eq:closed_form_waiting_ambulance}, goes through all states in the set of 
accepting 
states of either class 1 or class 2 individuals respectively, where a wait 
incurs. 
In equation \ref{eq:closed_form_waiting_others} that includes all states \((u,v)\) 
in the set of accepting states of class 1 individuals such that \( v \geq C\); i.e. 
whenever an arrival occurs and the system is at a state where the number of individuals 
in the system is more than or equal to $C$. 
Consequently, for the states that are included in the summation the expression 
\( v-C+1 \) indicates the amount of people in service one would have to wait for 
upon arrival at the hospital.

Additionally, the minimisation function in equation 
\ref{eq:closed_form_waiting_ambulance} 
ensures that when a class 2 individual arrives at any state 
that is greater than the predetermined threshold, the wait that the individual will 
have to endure remains the same. 
In essence, the expression \(\min(v+1,T) - C\) returns the number of people in line 
in front of a particular individual upon arrival.


\subsubsection{Overall Waiting Time}

Consequently, the overall waiting time should can be estimated by a linear combination 
of the waiting times of class 1 and class 2 individuals. 
The overall waiting time can be then given by the following equation where \(c_1\) 
and \(c_2\) are the coefficients of each individual's type waiting time:

\begin{equation}\label{overall_waiting_time_coeff}
    W = c_1 W^{(1)} + c_2 W^{(2)}
\end{equation}

The two coefficients represent the proportion of individuals of each type that 
traversed through the model. 
Theoretically, getting these percentages should be as simple as looking at the arrival 
rates of each type but in practise if the service centre or the buffer centre 
is full, some individuals may be lost to the system. 
Thus, one should account for the probability that an individual is lost to the system. 
This probability can be easily calculated by using the two sets of accepting states 
\(S_A^{(2)}\) and \(S_A^{(1)}\) defined earlier in equations.
Let us define here the probability, for either class type, that an individual 
is not lost in the system by:

\begin{equation*}
    P(L'_1) = \sum_{(u,v) \, \in S_A^{(1)}} \pi(u,v) \hspace{2cm}
    P(L'_2) = \sum_{(u,v) \, \in S_A^{(2)}} \pi(u,v)
\end{equation*}

Having defined these probabilities one may combine them with the arrival rates of 
each class type in such a way to get the expected proportions of class 1 and 
class 2 individuals in the model. 
Thus, by using these values as the coefficient of equation 
\ref{overall_waiting_time_coeff} 
the resultant equation can be used to get the overall waiting time. 
Note here that the equation below gets the overall waiting time for both the recursive 
and the closed-form formula.

\begin{equation}\label{overall_waiting_time}
    W = \frac{\lambda_1 P(L'_1)}{\lambda_2 P(L'_2) + \lambda_1 P(L'_1)} W^{(1)} + 
    \frac{\lambda_2 P(L'_2)}{\lambda_2 P(L'_2) + \lambda_1 P(L'_1)} W^{(2)}
\end{equation}



\subsubsection{Mean blocking time}
Unlike the waiting time, the blocking time is only calculated for class 2 individuals.  
That is because class 1 individuals cannot be blocked. 
Thus, one only needs to consider the pathway of class 2 individuals to get the 
mean blocking time of the system. 
Blocking occurs at states \((u,v)\) where \(u > 0 \). 
Thus, the set of blocking states can be defined as:

\begin{equation*}
    S_b = \{(u,v) \in S \; | \; u > 0\}
\end{equation*}
 
In order to not consider individuals that will be lost to the system, the set of 
accepting states needs to be taken into account. The set of accepting states is given by:

\begin{equation*}
    S_A^{(2)}=
    \begin{cases}
        \{(u, v) \in S \; | \; u < M \} & \textbf{if } T \leq N\\
        \{(u, v) \in S \; | \; v < N \} & \textbf{otherwise}
    \end{cases}
\end{equation*}

For the waiting time formula,
the mean sojourn time for each state was considered,
ignoring any arrivals. Here, the same approach is used but ignoring only class 2
arrivals. That is because for the waiting time formula, once an individual enters 
the service centre (i.e. starts waiting) any individual arriving after them will 
not affect their
pathway. That is not the case for blocking time. When a class 2 individual is 
blocked, 
any class 1 individual that arrives will cause the blocked individual to remain 
blocked for more time. Therefore, class 1 arrivals are considered here:

\begin{equation}\label{eq:time_in_state_blocking_time}
    c(u,v) = 
    \begin{cases}
        \frac{1}{\min(v,C) \mu}, & \text{if } v = C\\
        \frac{1}{\min(v,C) \mu + \lambda_1}, & \text{otherwise}
    \end{cases}
\end{equation}
 
In equation \ref{eq:time_in_state_blocking_time}, both service completions and 
class 1 arrivals are considered. 
Thus, from a blocked individual's perspective whenever the system moves from one 
state \((u,v)\)
to another state it can either:

\begin{itemize}
    \item be because of a service being completed: we will denote the probability 
    of this happening by \(p_s(u,v)\). 
    \item be because of an arrival of an individual of class 1: denoting such 
    probability by \(p_o(u,v)\).
\end{itemize}
The probabilities are given by:

\begin{equation*}
    p_s(u,v) = \frac{\min(v,C)\mu}{\lambda_1 + \min(v,C)\mu}, \qquad
    p_o(u,v) = \frac{\lambda_1}{\lambda_1 + \min(v,C)\mu}
\end{equation*}


Having defined \(c(u,v)\) and \(S_b\) a formula for the blocking time that is
expected to occur at each state can be given by:

\begin{equation}\label{eq:blocking-time-at-each-state}
    b(u,v) = 
    \begin{cases} 
        0, & \textbf{if } (u,v) \notin S_b \\
        c(u,v) + b(u - 1, v), & \textbf{if } v = N = T\\
        c(u,v) + b(u, v-1), & \textbf{if } v = N \neq T \\
        c(u,v) + p_s(u,v) b(u-1, v) + p_o(u,v) b(u, v+1), & \textbf{if } u > 0 
        \textbf{ and } v = T \\
        c(u,v) + p_s(u,v) b(u, v-1) + p_o(u,v) b(u, v+1), & \textbf{otherwise} \\
    \end{cases}
\end{equation}

Equation 
(\ref{eq:blocking-time-at-each-state}) will not be solved recursively. 
A direct approach will be used to solve this equation here. 
By enumerating all equations of (\ref{eq:blocking-time-at-each-state}) for all 
states \((u,v)\) that belong in \(S_b\) 
a system of linear equations arises where the unknown variables are all the \(b(u,v)\)
terms.
For instance, let us consider a Markov model where \(C=2, T=3, N=6, M=2\). 
The Markov model is shown in Figure \ref{fig:example-algeb-blocking}
and the equivalent equations are 
(\ref{eq:first_eq_of_blocking_example})-(\ref{eq:last_eq_of_blocking_example}).
The equations considered here are only the ones that correspond to the blocking 
states.

\begin{multicols*}{2}
    \begin{figure}[H]
        \scalebox{0.50}{\input{MarkovChain/expressions_from_pi/example_model_2362/main.tex}}
        \caption{Example of Markov chain}
        \label{fig:example-algeb-blocking}
    \end{figure}
    \columnbreak
    \begin{align}
        b(1,2) &= c(1,2) + p_o b(1,3) \label{eq:first_eq_of_blocking_example} \\
        b(1,3) &= c(1,3) + p_s b(1,2) + p_o b(1,4) \\
        b(1,4) &= c(1,4) + b(1,3) \\
        b(2,2) &= c(2,2) + p_s b(1,2) + p_o b(2,3) \\
        b(2,3) &= c(2,3) + p_s b(2,2) + p_o b(1,4) \\
        b(2,4) &= c(2,4) + b(2,3)\label{eq:last_eq_of_blocking_example}
    \end{align}
\end{multicols*}

Additionally, the above equations can be transformed into a linear system of the 
form \(Zx=y\) where:

\begin{equation}\label{eq:example-algebaric-approach-blocking-time}
    Z=
    \begin{pmatrix}
        -1 & p_o & 0 & 0 & 0 & 0 \\ %(1,2)
        p_s & -1 & p_o & 0 & 0 & 0 \\ %(1,3)
        0 & 1 & -1 & 0 & 0 & 0 \\ %(1,4)
        p_s & 0 & 0 & -1 & p_o & 0\\ %(2,2)
        0 & 0 & 0 & p_s & -1 & p_o \\ %(2,3)
        0 & 0 & 0 & 0 & 1 & -1 \\ %(2,4)
    \end{pmatrix},
    x=
    \begin{pmatrix}
        b(1,2) \\
        b(1,3) \\
        b(1,4) \\
        b(2,2) \\
        b(2,3) \\
        b(2,4) \\
    \end{pmatrix}, 
    y=
    \begin{pmatrix}
        -c(1,2) \\
        -c(1,3) \\
        -c(1,4) \\
        -c(2,2) \\
        -c(2,3) \\
        -c(2,4) \\
    \end{pmatrix}
\end{equation}

A more generalised form of the equations in 
(\ref{eq:example-algebaric-approach-blocking-time})
can thus be given for any value of \(C,T,N,M\) by:

\begin{align}
    b(1,T) =& c(1, T) + p_o b(1, T + 1) \label{eq:first_eq_of_blocking_general}\\
    b(1,T + 1) =& c(1, T + 1) + p_s(1, T) + p_o b(1, T + 1) \\
    b(1,T + 2) =& c(1, T + 2) + p_s(1, T + 1) + p_o b(1, T + 3) \\
    & \vdots \nonumber \\
    b(1, N) =& c(1, N) + b(1, N - 1) \\
    b(2, T) =& c(2, T) + p_s b(1, T) + p_o b(2, T + 1) \\
    b(2, T + 1) =& c(2, T + 1) + p_s b(2, T) + p_o b(2, T + 2) \\
    & \vdots \nonumber \\
    b(M, T) =& c(M, T) + b(M, T-1) \label{eq:last_eq_of_blocking_general}
\end{align}

The equivalent matrix form of the linear system of equations 
(\ref{eq:first_eq_of_blocking_general}) - (\ref{eq:last_eq_of_blocking_general})
is given by \(Zx=y\), where:
\begin{equation}\label{eq:general-algebaric-approach-blocking-time}
    \scalebox{0.9}{
        \(
        Z = 
        \begin{pmatrix}
            -1 & p_o & 0 & \dots & 0 & 0 & 0 & 0 & 0 & \dots & 0 & 0 \\ %(1,T)
            p_s & -1 & p_o & \dots & 0 & 0 & 0 & 0 & 0 & \dots & 0 & 0 \\ %(1,T+1)
            0 & p_s & -1 & \dots & 0 & 0 & 0 & 0 & 0 & \dots & 0 & 0 \\ %(1,T+2)
            \vdots & \vdots & \vdots & \ddots & \vdots & \vdots & \vdots & \vdots & 
            \vdots & \ddots & \vdots & \vdots \\ 
            0 & 0 & 0 & \dots & 1 & -1 & 0 & 0 & 0 & \dots & 0 & 0 \\ %(1,N)
            p_s & 0 & 0 & \dots & 0 & 0 & -1 & p_o & 0 & \dots & 0 & 0 \\ %(2,T)
            0 & 0 & 0 & \dots & 0 & 0 & p_s & -1 & p_o & \dots & 0 & 0 \\ %(2,T+1)
            \vdots & \vdots & \vdots & \ddots & \vdots & \vdots & \vdots & \vdots & 
            \vdots & \ddots & \vdots & \vdots \\ 
            0 & 0 & 0 & \dots & 0 & 0 & 0 & 0 & 0 & \dots & 1 & -1 \\ %(M,T)
        \end{pmatrix},
        x = 
        \begin{pmatrix}
            b(1,T) \\
            b(1,T+1) \\
            b(1,T+2) \\
            \vdots \\
            b(1,N) \\
            b(2,T) \\
            b(2,T+1) \\
            \vdots \\
            b(M,T) \\
        \end{pmatrix}, 
        y= 
        \begin{pmatrix}
            -c(1,T) \\
            -c(1,T+1) \\
            -c(1,T+2) \\
            \vdots \\
            -c(1,N) \\
            -c(2,T) \\
            -c(2,T+1) \\
            \vdots \\
            -c(M,T) \\
        \end{pmatrix}
        \)
    }
\end{equation}

Thus, having calculated the mean blocking time for all blocking states \(b(u,v)\), 
it only remains to put them together in a formula.
The resultant blocking time formula is given by:

\begin{equation}\label{eq:algebraic-blocking-time}
    B = \frac{\sum_{(u,v) \in S_A} \pi_{(u,v)} \; b(u,v)}{\sum_{(u,v) \in S_A} 
    \pi_{(u,v)}}
\end{equation}

\documentclass{article}

\usepackage{amsmath}
\usepackage{amsfonts} 
\usepackage{geometry}
\usepackage{multicol}
\usepackage{float}
% \usepackage{mathtools}
% \usepackage{graphicx}
% \usepackage{soul}
% \usepackage{indentfirst}
\usepackage{tikz}
\usetikzlibrary{calc, automata, chains, arrows.meta, math}
\setcounter{MaxMatrixCols}{20}


\title{A game theoretic model of the behavioural gaming that takes place at the EMS - ED interface}

\author{
    Michalis Panayides, 
    Paul Harper, 
    Vince Knight
}

\begin{document}

\maketitle

\input{Abstract/main.tex}


\newpage
\tableofcontents

\newpage
\input{Introduction/main.tex}

\newpage
\input{Game_theory_component/main.tex}

\newpage
\input{MarkovChain/markov_chain_model/main.tex}
\input{MarkovChain/expressions_from_pi/main.tex}
\input{MarkovChain/markov_example/main.tex}

\newpage
\input{BehaviouralMethodology/main.tex}

\newpage
\input{Application_EMS_ED/main.tex}

\newpage
\input{Conclusion/main.tex}


\end{document}

\newpage
\documentclass{article}

\usepackage{amsmath}
\usepackage{amsfonts} 
\usepackage{geometry}
\usepackage{multicol}
\usepackage{float}
% \usepackage{mathtools}
% \usepackage{graphicx}
% \usepackage{soul}
% \usepackage{indentfirst}
\usepackage{tikz}
\usetikzlibrary{calc, automata, chains, arrows.meta, math}
\setcounter{MaxMatrixCols}{20}


\title{A game theoretic model of the behavioural gaming that takes place at the EMS - ED interface}

\author{
    Michalis Panayides, 
    Paul Harper, 
    Vince Knight
}

\begin{document}

\maketitle

\input{Abstract/main.tex}


\newpage
\tableofcontents

\newpage
\input{Introduction/main.tex}

\newpage
\input{Game_theory_component/main.tex}

\newpage
\input{MarkovChain/markov_chain_model/main.tex}
\input{MarkovChain/expressions_from_pi/main.tex}
\input{MarkovChain/markov_example/main.tex}

\newpage
\input{BehaviouralMethodology/main.tex}

\newpage
\input{Application_EMS_ED/main.tex}

\newpage
\input{Conclusion/main.tex}


\end{document}

\newpage
\section{EMS-ED application}

\subsection{Application}

\subsection{Data analysis of generated problem}

\newpage
\documentclass{article}

\usepackage{amsmath}
\usepackage{amsfonts} 
\usepackage{geometry}
\usepackage{multicol}
\usepackage{float}
% \usepackage{mathtools}
% \usepackage{graphicx}
% \usepackage{soul}
% \usepackage{indentfirst}
\usepackage{tikz}
\usetikzlibrary{calc, automata, chains, arrows.meta, math}
\setcounter{MaxMatrixCols}{20}


\title{A game theoretic model of the behavioural gaming that takes place at the EMS - ED interface}

\author{
    Michalis Panayides, 
    Paul Harper, 
    Vince Knight
}

\begin{document}

\maketitle

\input{Abstract/main.tex}


\newpage
\tableofcontents

\newpage
\input{Introduction/main.tex}

\newpage
\input{Game_theory_component/main.tex}

\newpage
\input{MarkovChain/markov_chain_model/main.tex}
\input{MarkovChain/expressions_from_pi/main.tex}
\input{MarkovChain/markov_example/main.tex}

\newpage
\input{BehaviouralMethodology/main.tex}

\newpage
\input{Application_EMS_ED/main.tex}

\newpage
\input{Conclusion/main.tex}


\end{document}


\end{document}


\newpage
\tableofcontents

\newpage
\documentclass{article}

\usepackage{amsmath}
\usepackage{amsfonts} 
\usepackage{geometry}
\usepackage{multicol}
\usepackage{float}
% \usepackage{mathtools}
% \usepackage{graphicx}
% \usepackage{soul}
% \usepackage{indentfirst}
\usepackage{tikz}
\usetikzlibrary{calc, automata, chains, arrows.meta, math}
\setcounter{MaxMatrixCols}{20}


\title{A game theoretic model of the behavioural gaming that takes place at the EMS - ED interface}

\author{
    Michalis Panayides, 
    Paul Harper, 
    Vince Knight
}

\begin{document}

\maketitle

\documentclass{article}

\usepackage{amsmath}
\usepackage{amsfonts} 
\usepackage{geometry}
\usepackage{multicol}
\usepackage{float}
% \usepackage{mathtools}
% \usepackage{graphicx}
% \usepackage{soul}
% \usepackage{indentfirst}
\usepackage{tikz}
\usetikzlibrary{calc, automata, chains, arrows.meta, math}
\setcounter{MaxMatrixCols}{20}


\title{A game theoretic model of the behavioural gaming that takes place at the EMS - ED interface}

\author{
    Michalis Panayides, 
    Paul Harper, 
    Vince Knight
}

\begin{document}

\maketitle

\input{Abstract/main.tex}


\newpage
\tableofcontents

\newpage
\input{Introduction/main.tex}

\newpage
\input{Game_theory_component/main.tex}

\newpage
\input{MarkovChain/markov_chain_model/main.tex}
\input{MarkovChain/expressions_from_pi/main.tex}
\input{MarkovChain/markov_example/main.tex}

\newpage
\input{BehaviouralMethodology/main.tex}

\newpage
\input{Application_EMS_ED/main.tex}

\newpage
\input{Conclusion/main.tex}


\end{document}


\newpage
\tableofcontents

\newpage
\documentclass{article}

\usepackage{amsmath}
\usepackage{amsfonts} 
\usepackage{geometry}
\usepackage{multicol}
\usepackage{float}
% \usepackage{mathtools}
% \usepackage{graphicx}
% \usepackage{soul}
% \usepackage{indentfirst}
\usepackage{tikz}
\usetikzlibrary{calc, automata, chains, arrows.meta, math}
\setcounter{MaxMatrixCols}{20}


\title{A game theoretic model of the behavioural gaming that takes place at the EMS - ED interface}

\author{
    Michalis Panayides, 
    Paul Harper, 
    Vince Knight
}

\begin{document}

\maketitle

\input{Abstract/main.tex}


\newpage
\tableofcontents

\newpage
\input{Introduction/main.tex}

\newpage
\input{Game_theory_component/main.tex}

\newpage
\input{MarkovChain/markov_chain_model/main.tex}
\input{MarkovChain/expressions_from_pi/main.tex}
\input{MarkovChain/markov_example/main.tex}

\newpage
\input{BehaviouralMethodology/main.tex}

\newpage
\input{Application_EMS_ED/main.tex}

\newpage
\input{Conclusion/main.tex}


\end{document}

\newpage
\documentclass{article}

\usepackage{amsmath}
\usepackage{amsfonts} 
\usepackage{geometry}
\usepackage{multicol}
\usepackage{float}
% \usepackage{mathtools}
% \usepackage{graphicx}
% \usepackage{soul}
% \usepackage{indentfirst}
\usepackage{tikz}
\usetikzlibrary{calc, automata, chains, arrows.meta, math}
\setcounter{MaxMatrixCols}{20}


\title{A game theoretic model of the behavioural gaming that takes place at the EMS - ED interface}

\author{
    Michalis Panayides, 
    Paul Harper, 
    Vince Knight
}

\begin{document}

\maketitle

\input{Abstract/main.tex}


\newpage
\tableofcontents

\newpage
\input{Introduction/main.tex}

\newpage
\input{Game_theory_component/main.tex}

\newpage
\input{MarkovChain/markov_chain_model/main.tex}
\input{MarkovChain/expressions_from_pi/main.tex}
\input{MarkovChain/markov_example/main.tex}

\newpage
\input{BehaviouralMethodology/main.tex}

\newpage
\input{Application_EMS_ED/main.tex}

\newpage
\input{Conclusion/main.tex}


\end{document}

\newpage
\documentclass{article}

\usepackage{amsmath}
\usepackage{amsfonts} 
\usepackage{geometry}
\usepackage{multicol}
\usepackage{float}
% \usepackage{mathtools}
% \usepackage{graphicx}
% \usepackage{soul}
% \usepackage{indentfirst}
\usepackage{tikz}
\usetikzlibrary{calc, automata, chains, arrows.meta, math}
\setcounter{MaxMatrixCols}{20}


\title{A game theoretic model of the behavioural gaming that takes place at the EMS - ED interface}

\author{
    Michalis Panayides, 
    Paul Harper, 
    Vince Knight
}

\begin{document}

\maketitle

\input{Abstract/main.tex}


\newpage
\tableofcontents

\newpage
\input{Introduction/main.tex}

\newpage
\input{Game_theory_component/main.tex}

\newpage
\input{MarkovChain/markov_chain_model/main.tex}
\input{MarkovChain/expressions_from_pi/main.tex}
\input{MarkovChain/markov_example/main.tex}

\newpage
\input{BehaviouralMethodology/main.tex}

\newpage
\input{Application_EMS_ED/main.tex}

\newpage
\input{Conclusion/main.tex}


\end{document}
\subsection{Performance Measures}
One may easily derive the average number of individuals that are at any given state 
using \( pi \). 
The average number of individuals in state \( i \) can be calculated by multiplying 
the number of individuals that are present in state \( i \) with the probability 
of being at that particular state (i.e \(\pi_i (u_i + v_i)\)). 
Using this logic it is possible to calculate any performance measures that are related 
to the mean number of individuals in the system.


Average number of people in the system: 
\begin{equation}
    L = \sum_{i=1}^{|\pi|} \pi_i (u_i + v_i)
\end{equation} 

Average number of people in the service centre: 
\begin{equation}
    L_H = \sum_{i=1}^{|\pi|} \pi_i v_i
\end{equation}

Average number of people in the buffer centre:
\begin{equation}
    L_A = \sum_{i=1}^{|\pi|} \pi_i u_i
\end{equation}

Consequently getting the performance measures that are related to the duration of 
time is not as straightforward. 
Such performance measures are the mean waiting time in the system and the mean time 
blocked in the system. 
Under the scope of this study three approaches have been considered to calculate these 
performance measures; a direct approach, a recursive algorithm and consequently a
closed-form formula.

The research question that needs to be answered here is: ``When a class 1/2 
individuals enters the system, what is the expected time that they will have to 
wait?''. 
In order to formulate the answer to that question one needs to consider all possible 
scenarios of what state the system can be in when an individual arrives. 
Furthermore, different formulas arises for class 1 individuals 
and a different one for class 2 individuals.

\subsubsection{Mean waiting time} 
Upon closer inspection of the recursive formula a more compact formula can arise. 
The equivalent closed-form formula eliminates the need for recursion and thus makes 
the computation of waiting times much more efficient. 
Just like in the recursive part there are two formulas; one for \textit{class 1} 
and one for class 2 individuals. 
The formulas are given by:

\begin{equation} \label{eq:closed_form_waiting_others}
    W^{(1)} = \frac{\sum_{\substack{(u,v) \, \in S_A^{(1)} \\ v \geq C}} 
    \frac{1}{C \mu} \times (v-C+1) \times \pi(u,v)}{\sum_{(u,v) \, 
    \in S_A^{(1)}} \pi(u,v)}
\end{equation}
    
\begin{equation}\label{eq:closed_form_waiting_ambulance}
    W^{(2)} = \frac{\sum_{\substack{(u,v) \, \in S_A^{(2)} \\ min(v,T) \geq C}} 
    \frac{1}{C \mu} \times (\min(v+1,T)-C) \times \pi(u,v)}{\sum_{(u,v) \, 
    \in S_A^{(2)}} \pi(u,v)}
\end{equation}

Note here that the summation, in both equations \ref{eq:closed_form_waiting_others} 
and \ref{eq:closed_form_waiting_ambulance}, goes through all states in the set of 
accepting 
states of either class 1 or class 2 individuals respectively, where a wait 
incurs. 
In equation \ref{eq:closed_form_waiting_others} that includes all states \((u,v)\) 
in the set of accepting states of class 1 individuals such that \( v \geq C\); i.e. 
whenever an arrival occurs and the system is at a state where the number of individuals 
in the system is more than or equal to $C$. 
Consequently, for the states that are included in the summation the expression 
\( v-C+1 \) indicates the amount of people in service one would have to wait for 
upon arrival at the hospital.

Additionally, the minimisation function in equation 
\ref{eq:closed_form_waiting_ambulance} 
ensures that when a class 2 individual arrives at any state 
that is greater than the predetermined threshold, the wait that the individual will 
have to endure remains the same. 
In essence, the expression \(\min(v+1,T) - C\) returns the number of people in line 
in front of a particular individual upon arrival.


\subsubsection{Overall Waiting Time}

Consequently, the overall waiting time should can be estimated by a linear combination 
of the waiting times of class 1 and class 2 individuals. 
The overall waiting time can be then given by the following equation where \(c_1\) 
and \(c_2\) are the coefficients of each individual's type waiting time:

\begin{equation}\label{overall_waiting_time_coeff}
    W = c_1 W^{(1)} + c_2 W^{(2)}
\end{equation}

The two coefficients represent the proportion of individuals of each type that 
traversed through the model. 
Theoretically, getting these percentages should be as simple as looking at the arrival 
rates of each type but in practise if the service centre or the buffer centre 
is full, some individuals may be lost to the system. 
Thus, one should account for the probability that an individual is lost to the system. 
This probability can be easily calculated by using the two sets of accepting states 
\(S_A^{(2)}\) and \(S_A^{(1)}\) defined earlier in equations.
Let us define here the probability, for either class type, that an individual 
is not lost in the system by:

\begin{equation*}
    P(L'_1) = \sum_{(u,v) \, \in S_A^{(1)}} \pi(u,v) \hspace{2cm}
    P(L'_2) = \sum_{(u,v) \, \in S_A^{(2)}} \pi(u,v)
\end{equation*}

Having defined these probabilities one may combine them with the arrival rates of 
each class type in such a way to get the expected proportions of class 1 and 
class 2 individuals in the model. 
Thus, by using these values as the coefficient of equation 
\ref{overall_waiting_time_coeff} 
the resultant equation can be used to get the overall waiting time. 
Note here that the equation below gets the overall waiting time for both the recursive 
and the closed-form formula.

\begin{equation}\label{overall_waiting_time}
    W = \frac{\lambda_1 P(L'_1)}{\lambda_2 P(L'_2) + \lambda_1 P(L'_1)} W^{(1)} + 
    \frac{\lambda_2 P(L'_2)}{\lambda_2 P(L'_2) + \lambda_1 P(L'_1)} W^{(2)}
\end{equation}



\subsubsection{Mean blocking time}
Unlike the waiting time, the blocking time is only calculated for class 2 individuals.  
That is because class 1 individuals cannot be blocked. 
Thus, one only needs to consider the pathway of class 2 individuals to get the 
mean blocking time of the system. 
Blocking occurs at states \((u,v)\) where \(u > 0 \). 
Thus, the set of blocking states can be defined as:

\begin{equation*}
    S_b = \{(u,v) \in S \; | \; u > 0\}
\end{equation*}
 
In order to not consider individuals that will be lost to the system, the set of 
accepting states needs to be taken into account. The set of accepting states is given by:

\begin{equation*}
    S_A^{(2)}=
    \begin{cases}
        \{(u, v) \in S \; | \; u < M \} & \textbf{if } T \leq N\\
        \{(u, v) \in S \; | \; v < N \} & \textbf{otherwise}
    \end{cases}
\end{equation*}

For the waiting time formula,
the mean sojourn time for each state was considered,
ignoring any arrivals. Here, the same approach is used but ignoring only class 2
arrivals. That is because for the waiting time formula, once an individual enters 
the service centre (i.e. starts waiting) any individual arriving after them will 
not affect their
pathway. That is not the case for blocking time. When a class 2 individual is 
blocked, 
any class 1 individual that arrives will cause the blocked individual to remain 
blocked for more time. Therefore, class 1 arrivals are considered here:

\begin{equation}\label{eq:time_in_state_blocking_time}
    c(u,v) = 
    \begin{cases}
        \frac{1}{\min(v,C) \mu}, & \text{if } v = C\\
        \frac{1}{\min(v,C) \mu + \lambda_1}, & \text{otherwise}
    \end{cases}
\end{equation}
 
In equation \ref{eq:time_in_state_blocking_time}, both service completions and 
class 1 arrivals are considered. 
Thus, from a blocked individual's perspective whenever the system moves from one 
state \((u,v)\)
to another state it can either:

\begin{itemize}
    \item be because of a service being completed: we will denote the probability 
    of this happening by \(p_s(u,v)\). 
    \item be because of an arrival of an individual of class 1: denoting such 
    probability by \(p_o(u,v)\).
\end{itemize}
The probabilities are given by:

\begin{equation*}
    p_s(u,v) = \frac{\min(v,C)\mu}{\lambda_1 + \min(v,C)\mu}, \qquad
    p_o(u,v) = \frac{\lambda_1}{\lambda_1 + \min(v,C)\mu}
\end{equation*}


Having defined \(c(u,v)\) and \(S_b\) a formula for the blocking time that is
expected to occur at each state can be given by:

\begin{equation}\label{eq:blocking-time-at-each-state}
    b(u,v) = 
    \begin{cases} 
        0, & \textbf{if } (u,v) \notin S_b \\
        c(u,v) + b(u - 1, v), & \textbf{if } v = N = T\\
        c(u,v) + b(u, v-1), & \textbf{if } v = N \neq T \\
        c(u,v) + p_s(u,v) b(u-1, v) + p_o(u,v) b(u, v+1), & \textbf{if } u > 0 
        \textbf{ and } v = T \\
        c(u,v) + p_s(u,v) b(u, v-1) + p_o(u,v) b(u, v+1), & \textbf{otherwise} \\
    \end{cases}
\end{equation}

Equation 
(\ref{eq:blocking-time-at-each-state}) will not be solved recursively. 
A direct approach will be used to solve this equation here. 
By enumerating all equations of (\ref{eq:blocking-time-at-each-state}) for all 
states \((u,v)\) that belong in \(S_b\) 
a system of linear equations arises where the unknown variables are all the \(b(u,v)\)
terms.
For instance, let us consider a Markov model where \(C=2, T=3, N=6, M=2\). 
The Markov model is shown in Figure \ref{fig:example-algeb-blocking}
and the equivalent equations are 
(\ref{eq:first_eq_of_blocking_example})-(\ref{eq:last_eq_of_blocking_example}).
The equations considered here are only the ones that correspond to the blocking 
states.

\begin{multicols*}{2}
    \begin{figure}[H]
        \scalebox{0.50}{\input{MarkovChain/expressions_from_pi/example_model_2362/main.tex}}
        \caption{Example of Markov chain}
        \label{fig:example-algeb-blocking}
    \end{figure}
    \columnbreak
    \begin{align}
        b(1,2) &= c(1,2) + p_o b(1,3) \label{eq:first_eq_of_blocking_example} \\
        b(1,3) &= c(1,3) + p_s b(1,2) + p_o b(1,4) \\
        b(1,4) &= c(1,4) + b(1,3) \\
        b(2,2) &= c(2,2) + p_s b(1,2) + p_o b(2,3) \\
        b(2,3) &= c(2,3) + p_s b(2,2) + p_o b(1,4) \\
        b(2,4) &= c(2,4) + b(2,3)\label{eq:last_eq_of_blocking_example}
    \end{align}
\end{multicols*}

Additionally, the above equations can be transformed into a linear system of the 
form \(Zx=y\) where:

\begin{equation}\label{eq:example-algebaric-approach-blocking-time}
    Z=
    \begin{pmatrix}
        -1 & p_o & 0 & 0 & 0 & 0 \\ %(1,2)
        p_s & -1 & p_o & 0 & 0 & 0 \\ %(1,3)
        0 & 1 & -1 & 0 & 0 & 0 \\ %(1,4)
        p_s & 0 & 0 & -1 & p_o & 0\\ %(2,2)
        0 & 0 & 0 & p_s & -1 & p_o \\ %(2,3)
        0 & 0 & 0 & 0 & 1 & -1 \\ %(2,4)
    \end{pmatrix},
    x=
    \begin{pmatrix}
        b(1,2) \\
        b(1,3) \\
        b(1,4) \\
        b(2,2) \\
        b(2,3) \\
        b(2,4) \\
    \end{pmatrix}, 
    y=
    \begin{pmatrix}
        -c(1,2) \\
        -c(1,3) \\
        -c(1,4) \\
        -c(2,2) \\
        -c(2,3) \\
        -c(2,4) \\
    \end{pmatrix}
\end{equation}

A more generalised form of the equations in 
(\ref{eq:example-algebaric-approach-blocking-time})
can thus be given for any value of \(C,T,N,M\) by:

\begin{align}
    b(1,T) =& c(1, T) + p_o b(1, T + 1) \label{eq:first_eq_of_blocking_general}\\
    b(1,T + 1) =& c(1, T + 1) + p_s(1, T) + p_o b(1, T + 1) \\
    b(1,T + 2) =& c(1, T + 2) + p_s(1, T + 1) + p_o b(1, T + 3) \\
    & \vdots \nonumber \\
    b(1, N) =& c(1, N) + b(1, N - 1) \\
    b(2, T) =& c(2, T) + p_s b(1, T) + p_o b(2, T + 1) \\
    b(2, T + 1) =& c(2, T + 1) + p_s b(2, T) + p_o b(2, T + 2) \\
    & \vdots \nonumber \\
    b(M, T) =& c(M, T) + b(M, T-1) \label{eq:last_eq_of_blocking_general}
\end{align}

The equivalent matrix form of the linear system of equations 
(\ref{eq:first_eq_of_blocking_general}) - (\ref{eq:last_eq_of_blocking_general})
is given by \(Zx=y\), where:
\begin{equation}\label{eq:general-algebaric-approach-blocking-time}
    \scalebox{0.9}{
        \(
        Z = 
        \begin{pmatrix}
            -1 & p_o & 0 & \dots & 0 & 0 & 0 & 0 & 0 & \dots & 0 & 0 \\ %(1,T)
            p_s & -1 & p_o & \dots & 0 & 0 & 0 & 0 & 0 & \dots & 0 & 0 \\ %(1,T+1)
            0 & p_s & -1 & \dots & 0 & 0 & 0 & 0 & 0 & \dots & 0 & 0 \\ %(1,T+2)
            \vdots & \vdots & \vdots & \ddots & \vdots & \vdots & \vdots & \vdots & 
            \vdots & \ddots & \vdots & \vdots \\ 
            0 & 0 & 0 & \dots & 1 & -1 & 0 & 0 & 0 & \dots & 0 & 0 \\ %(1,N)
            p_s & 0 & 0 & \dots & 0 & 0 & -1 & p_o & 0 & \dots & 0 & 0 \\ %(2,T)
            0 & 0 & 0 & \dots & 0 & 0 & p_s & -1 & p_o & \dots & 0 & 0 \\ %(2,T+1)
            \vdots & \vdots & \vdots & \ddots & \vdots & \vdots & \vdots & \vdots & 
            \vdots & \ddots & \vdots & \vdots \\ 
            0 & 0 & 0 & \dots & 0 & 0 & 0 & 0 & 0 & \dots & 1 & -1 \\ %(M,T)
        \end{pmatrix},
        x = 
        \begin{pmatrix}
            b(1,T) \\
            b(1,T+1) \\
            b(1,T+2) \\
            \vdots \\
            b(1,N) \\
            b(2,T) \\
            b(2,T+1) \\
            \vdots \\
            b(M,T) \\
        \end{pmatrix}, 
        y= 
        \begin{pmatrix}
            -c(1,T) \\
            -c(1,T+1) \\
            -c(1,T+2) \\
            \vdots \\
            -c(1,N) \\
            -c(2,T) \\
            -c(2,T+1) \\
            \vdots \\
            -c(M,T) \\
        \end{pmatrix}
        \)
    }
\end{equation}

Thus, having calculated the mean blocking time for all blocking states \(b(u,v)\), 
it only remains to put them together in a formula.
The resultant blocking time formula is given by:

\begin{equation}\label{eq:algebraic-blocking-time}
    B = \frac{\sum_{(u,v) \in S_A} \pi_{(u,v)} \; b(u,v)}{\sum_{(u,v) \in S_A} 
    \pi_{(u,v)}}
\end{equation}

\documentclass{article}

\usepackage{amsmath}
\usepackage{amsfonts} 
\usepackage{geometry}
\usepackage{multicol}
\usepackage{float}
% \usepackage{mathtools}
% \usepackage{graphicx}
% \usepackage{soul}
% \usepackage{indentfirst}
\usepackage{tikz}
\usetikzlibrary{calc, automata, chains, arrows.meta, math}
\setcounter{MaxMatrixCols}{20}


\title{A game theoretic model of the behavioural gaming that takes place at the EMS - ED interface}

\author{
    Michalis Panayides, 
    Paul Harper, 
    Vince Knight
}

\begin{document}

\maketitle

\input{Abstract/main.tex}


\newpage
\tableofcontents

\newpage
\input{Introduction/main.tex}

\newpage
\input{Game_theory_component/main.tex}

\newpage
\input{MarkovChain/markov_chain_model/main.tex}
\input{MarkovChain/expressions_from_pi/main.tex}
\input{MarkovChain/markov_example/main.tex}

\newpage
\input{BehaviouralMethodology/main.tex}

\newpage
\input{Application_EMS_ED/main.tex}

\newpage
\input{Conclusion/main.tex}


\end{document}

\newpage
\documentclass{article}

\usepackage{amsmath}
\usepackage{amsfonts} 
\usepackage{geometry}
\usepackage{multicol}
\usepackage{float}
% \usepackage{mathtools}
% \usepackage{graphicx}
% \usepackage{soul}
% \usepackage{indentfirst}
\usepackage{tikz}
\usetikzlibrary{calc, automata, chains, arrows.meta, math}
\setcounter{MaxMatrixCols}{20}


\title{A game theoretic model of the behavioural gaming that takes place at the EMS - ED interface}

\author{
    Michalis Panayides, 
    Paul Harper, 
    Vince Knight
}

\begin{document}

\maketitle

\input{Abstract/main.tex}


\newpage
\tableofcontents

\newpage
\input{Introduction/main.tex}

\newpage
\input{Game_theory_component/main.tex}

\newpage
\input{MarkovChain/markov_chain_model/main.tex}
\input{MarkovChain/expressions_from_pi/main.tex}
\input{MarkovChain/markov_example/main.tex}

\newpage
\input{BehaviouralMethodology/main.tex}

\newpage
\input{Application_EMS_ED/main.tex}

\newpage
\input{Conclusion/main.tex}


\end{document}

\newpage
\section{EMS-ED application}

\subsection{Application}

\subsection{Data analysis of generated problem}

\newpage
\documentclass{article}

\usepackage{amsmath}
\usepackage{amsfonts} 
\usepackage{geometry}
\usepackage{multicol}
\usepackage{float}
% \usepackage{mathtools}
% \usepackage{graphicx}
% \usepackage{soul}
% \usepackage{indentfirst}
\usepackage{tikz}
\usetikzlibrary{calc, automata, chains, arrows.meta, math}
\setcounter{MaxMatrixCols}{20}


\title{A game theoretic model of the behavioural gaming that takes place at the EMS - ED interface}

\author{
    Michalis Panayides, 
    Paul Harper, 
    Vince Knight
}

\begin{document}

\maketitle

\input{Abstract/main.tex}


\newpage
\tableofcontents

\newpage
\input{Introduction/main.tex}

\newpage
\input{Game_theory_component/main.tex}

\newpage
\input{MarkovChain/markov_chain_model/main.tex}
\input{MarkovChain/expressions_from_pi/main.tex}
\input{MarkovChain/markov_example/main.tex}

\newpage
\input{BehaviouralMethodology/main.tex}

\newpage
\input{Application_EMS_ED/main.tex}

\newpage
\input{Conclusion/main.tex}


\end{document}


\end{document}

\newpage
\documentclass{article}

\usepackage{amsmath}
\usepackage{amsfonts} 
\usepackage{geometry}
\usepackage{multicol}
\usepackage{float}
% \usepackage{mathtools}
% \usepackage{graphicx}
% \usepackage{soul}
% \usepackage{indentfirst}
\usepackage{tikz}
\usetikzlibrary{calc, automata, chains, arrows.meta, math}
\setcounter{MaxMatrixCols}{20}


\title{A game theoretic model of the behavioural gaming that takes place at the EMS - ED interface}

\author{
    Michalis Panayides, 
    Paul Harper, 
    Vince Knight
}

\begin{document}

\maketitle

\documentclass{article}

\usepackage{amsmath}
\usepackage{amsfonts} 
\usepackage{geometry}
\usepackage{multicol}
\usepackage{float}
% \usepackage{mathtools}
% \usepackage{graphicx}
% \usepackage{soul}
% \usepackage{indentfirst}
\usepackage{tikz}
\usetikzlibrary{calc, automata, chains, arrows.meta, math}
\setcounter{MaxMatrixCols}{20}


\title{A game theoretic model of the behavioural gaming that takes place at the EMS - ED interface}

\author{
    Michalis Panayides, 
    Paul Harper, 
    Vince Knight
}

\begin{document}

\maketitle

\input{Abstract/main.tex}


\newpage
\tableofcontents

\newpage
\input{Introduction/main.tex}

\newpage
\input{Game_theory_component/main.tex}

\newpage
\input{MarkovChain/markov_chain_model/main.tex}
\input{MarkovChain/expressions_from_pi/main.tex}
\input{MarkovChain/markov_example/main.tex}

\newpage
\input{BehaviouralMethodology/main.tex}

\newpage
\input{Application_EMS_ED/main.tex}

\newpage
\input{Conclusion/main.tex}


\end{document}


\newpage
\tableofcontents

\newpage
\documentclass{article}

\usepackage{amsmath}
\usepackage{amsfonts} 
\usepackage{geometry}
\usepackage{multicol}
\usepackage{float}
% \usepackage{mathtools}
% \usepackage{graphicx}
% \usepackage{soul}
% \usepackage{indentfirst}
\usepackage{tikz}
\usetikzlibrary{calc, automata, chains, arrows.meta, math}
\setcounter{MaxMatrixCols}{20}


\title{A game theoretic model of the behavioural gaming that takes place at the EMS - ED interface}

\author{
    Michalis Panayides, 
    Paul Harper, 
    Vince Knight
}

\begin{document}

\maketitle

\input{Abstract/main.tex}


\newpage
\tableofcontents

\newpage
\input{Introduction/main.tex}

\newpage
\input{Game_theory_component/main.tex}

\newpage
\input{MarkovChain/markov_chain_model/main.tex}
\input{MarkovChain/expressions_from_pi/main.tex}
\input{MarkovChain/markov_example/main.tex}

\newpage
\input{BehaviouralMethodology/main.tex}

\newpage
\input{Application_EMS_ED/main.tex}

\newpage
\input{Conclusion/main.tex}


\end{document}

\newpage
\documentclass{article}

\usepackage{amsmath}
\usepackage{amsfonts} 
\usepackage{geometry}
\usepackage{multicol}
\usepackage{float}
% \usepackage{mathtools}
% \usepackage{graphicx}
% \usepackage{soul}
% \usepackage{indentfirst}
\usepackage{tikz}
\usetikzlibrary{calc, automata, chains, arrows.meta, math}
\setcounter{MaxMatrixCols}{20}


\title{A game theoretic model of the behavioural gaming that takes place at the EMS - ED interface}

\author{
    Michalis Panayides, 
    Paul Harper, 
    Vince Knight
}

\begin{document}

\maketitle

\input{Abstract/main.tex}


\newpage
\tableofcontents

\newpage
\input{Introduction/main.tex}

\newpage
\input{Game_theory_component/main.tex}

\newpage
\input{MarkovChain/markov_chain_model/main.tex}
\input{MarkovChain/expressions_from_pi/main.tex}
\input{MarkovChain/markov_example/main.tex}

\newpage
\input{BehaviouralMethodology/main.tex}

\newpage
\input{Application_EMS_ED/main.tex}

\newpage
\input{Conclusion/main.tex}


\end{document}

\newpage
\documentclass{article}

\usepackage{amsmath}
\usepackage{amsfonts} 
\usepackage{geometry}
\usepackage{multicol}
\usepackage{float}
% \usepackage{mathtools}
% \usepackage{graphicx}
% \usepackage{soul}
% \usepackage{indentfirst}
\usepackage{tikz}
\usetikzlibrary{calc, automata, chains, arrows.meta, math}
\setcounter{MaxMatrixCols}{20}


\title{A game theoretic model of the behavioural gaming that takes place at the EMS - ED interface}

\author{
    Michalis Panayides, 
    Paul Harper, 
    Vince Knight
}

\begin{document}

\maketitle

\input{Abstract/main.tex}


\newpage
\tableofcontents

\newpage
\input{Introduction/main.tex}

\newpage
\input{Game_theory_component/main.tex}

\newpage
\input{MarkovChain/markov_chain_model/main.tex}
\input{MarkovChain/expressions_from_pi/main.tex}
\input{MarkovChain/markov_example/main.tex}

\newpage
\input{BehaviouralMethodology/main.tex}

\newpage
\input{Application_EMS_ED/main.tex}

\newpage
\input{Conclusion/main.tex}


\end{document}
\subsection{Performance Measures}
One may easily derive the average number of individuals that are at any given state 
using \( pi \). 
The average number of individuals in state \( i \) can be calculated by multiplying 
the number of individuals that are present in state \( i \) with the probability 
of being at that particular state (i.e \(\pi_i (u_i + v_i)\)). 
Using this logic it is possible to calculate any performance measures that are related 
to the mean number of individuals in the system.


Average number of people in the system: 
\begin{equation}
    L = \sum_{i=1}^{|\pi|} \pi_i (u_i + v_i)
\end{equation} 

Average number of people in the service centre: 
\begin{equation}
    L_H = \sum_{i=1}^{|\pi|} \pi_i v_i
\end{equation}

Average number of people in the buffer centre:
\begin{equation}
    L_A = \sum_{i=1}^{|\pi|} \pi_i u_i
\end{equation}

Consequently getting the performance measures that are related to the duration of 
time is not as straightforward. 
Such performance measures are the mean waiting time in the system and the mean time 
blocked in the system. 
Under the scope of this study three approaches have been considered to calculate these 
performance measures; a direct approach, a recursive algorithm and consequently a
closed-form formula.

The research question that needs to be answered here is: ``When a class 1/2 
individuals enters the system, what is the expected time that they will have to 
wait?''. 
In order to formulate the answer to that question one needs to consider all possible 
scenarios of what state the system can be in when an individual arrives. 
Furthermore, different formulas arises for class 1 individuals 
and a different one for class 2 individuals.

\subsubsection{Mean waiting time} 
Upon closer inspection of the recursive formula a more compact formula can arise. 
The equivalent closed-form formula eliminates the need for recursion and thus makes 
the computation of waiting times much more efficient. 
Just like in the recursive part there are two formulas; one for \textit{class 1} 
and one for class 2 individuals. 
The formulas are given by:

\begin{equation} \label{eq:closed_form_waiting_others}
    W^{(1)} = \frac{\sum_{\substack{(u,v) \, \in S_A^{(1)} \\ v \geq C}} 
    \frac{1}{C \mu} \times (v-C+1) \times \pi(u,v)}{\sum_{(u,v) \, 
    \in S_A^{(1)}} \pi(u,v)}
\end{equation}
    
\begin{equation}\label{eq:closed_form_waiting_ambulance}
    W^{(2)} = \frac{\sum_{\substack{(u,v) \, \in S_A^{(2)} \\ min(v,T) \geq C}} 
    \frac{1}{C \mu} \times (\min(v+1,T)-C) \times \pi(u,v)}{\sum_{(u,v) \, 
    \in S_A^{(2)}} \pi(u,v)}
\end{equation}

Note here that the summation, in both equations \ref{eq:closed_form_waiting_others} 
and \ref{eq:closed_form_waiting_ambulance}, goes through all states in the set of 
accepting 
states of either class 1 or class 2 individuals respectively, where a wait 
incurs. 
In equation \ref{eq:closed_form_waiting_others} that includes all states \((u,v)\) 
in the set of accepting states of class 1 individuals such that \( v \geq C\); i.e. 
whenever an arrival occurs and the system is at a state where the number of individuals 
in the system is more than or equal to $C$. 
Consequently, for the states that are included in the summation the expression 
\( v-C+1 \) indicates the amount of people in service one would have to wait for 
upon arrival at the hospital.

Additionally, the minimisation function in equation 
\ref{eq:closed_form_waiting_ambulance} 
ensures that when a class 2 individual arrives at any state 
that is greater than the predetermined threshold, the wait that the individual will 
have to endure remains the same. 
In essence, the expression \(\min(v+1,T) - C\) returns the number of people in line 
in front of a particular individual upon arrival.


\subsubsection{Overall Waiting Time}

Consequently, the overall waiting time should can be estimated by a linear combination 
of the waiting times of class 1 and class 2 individuals. 
The overall waiting time can be then given by the following equation where \(c_1\) 
and \(c_2\) are the coefficients of each individual's type waiting time:

\begin{equation}\label{overall_waiting_time_coeff}
    W = c_1 W^{(1)} + c_2 W^{(2)}
\end{equation}

The two coefficients represent the proportion of individuals of each type that 
traversed through the model. 
Theoretically, getting these percentages should be as simple as looking at the arrival 
rates of each type but in practise if the service centre or the buffer centre 
is full, some individuals may be lost to the system. 
Thus, one should account for the probability that an individual is lost to the system. 
This probability can be easily calculated by using the two sets of accepting states 
\(S_A^{(2)}\) and \(S_A^{(1)}\) defined earlier in equations.
Let us define here the probability, for either class type, that an individual 
is not lost in the system by:

\begin{equation*}
    P(L'_1) = \sum_{(u,v) \, \in S_A^{(1)}} \pi(u,v) \hspace{2cm}
    P(L'_2) = \sum_{(u,v) \, \in S_A^{(2)}} \pi(u,v)
\end{equation*}

Having defined these probabilities one may combine them with the arrival rates of 
each class type in such a way to get the expected proportions of class 1 and 
class 2 individuals in the model. 
Thus, by using these values as the coefficient of equation 
\ref{overall_waiting_time_coeff} 
the resultant equation can be used to get the overall waiting time. 
Note here that the equation below gets the overall waiting time for both the recursive 
and the closed-form formula.

\begin{equation}\label{overall_waiting_time}
    W = \frac{\lambda_1 P(L'_1)}{\lambda_2 P(L'_2) + \lambda_1 P(L'_1)} W^{(1)} + 
    \frac{\lambda_2 P(L'_2)}{\lambda_2 P(L'_2) + \lambda_1 P(L'_1)} W^{(2)}
\end{equation}



\subsubsection{Mean blocking time}
Unlike the waiting time, the blocking time is only calculated for class 2 individuals.  
That is because class 1 individuals cannot be blocked. 
Thus, one only needs to consider the pathway of class 2 individuals to get the 
mean blocking time of the system. 
Blocking occurs at states \((u,v)\) where \(u > 0 \). 
Thus, the set of blocking states can be defined as:

\begin{equation*}
    S_b = \{(u,v) \in S \; | \; u > 0\}
\end{equation*}
 
In order to not consider individuals that will be lost to the system, the set of 
accepting states needs to be taken into account. The set of accepting states is given by:

\begin{equation*}
    S_A^{(2)}=
    \begin{cases}
        \{(u, v) \in S \; | \; u < M \} & \textbf{if } T \leq N\\
        \{(u, v) \in S \; | \; v < N \} & \textbf{otherwise}
    \end{cases}
\end{equation*}

For the waiting time formula,
the mean sojourn time for each state was considered,
ignoring any arrivals. Here, the same approach is used but ignoring only class 2
arrivals. That is because for the waiting time formula, once an individual enters 
the service centre (i.e. starts waiting) any individual arriving after them will 
not affect their
pathway. That is not the case for blocking time. When a class 2 individual is 
blocked, 
any class 1 individual that arrives will cause the blocked individual to remain 
blocked for more time. Therefore, class 1 arrivals are considered here:

\begin{equation}\label{eq:time_in_state_blocking_time}
    c(u,v) = 
    \begin{cases}
        \frac{1}{\min(v,C) \mu}, & \text{if } v = C\\
        \frac{1}{\min(v,C) \mu + \lambda_1}, & \text{otherwise}
    \end{cases}
\end{equation}
 
In equation \ref{eq:time_in_state_blocking_time}, both service completions and 
class 1 arrivals are considered. 
Thus, from a blocked individual's perspective whenever the system moves from one 
state \((u,v)\)
to another state it can either:

\begin{itemize}
    \item be because of a service being completed: we will denote the probability 
    of this happening by \(p_s(u,v)\). 
    \item be because of an arrival of an individual of class 1: denoting such 
    probability by \(p_o(u,v)\).
\end{itemize}
The probabilities are given by:

\begin{equation*}
    p_s(u,v) = \frac{\min(v,C)\mu}{\lambda_1 + \min(v,C)\mu}, \qquad
    p_o(u,v) = \frac{\lambda_1}{\lambda_1 + \min(v,C)\mu}
\end{equation*}


Having defined \(c(u,v)\) and \(S_b\) a formula for the blocking time that is
expected to occur at each state can be given by:

\begin{equation}\label{eq:blocking-time-at-each-state}
    b(u,v) = 
    \begin{cases} 
        0, & \textbf{if } (u,v) \notin S_b \\
        c(u,v) + b(u - 1, v), & \textbf{if } v = N = T\\
        c(u,v) + b(u, v-1), & \textbf{if } v = N \neq T \\
        c(u,v) + p_s(u,v) b(u-1, v) + p_o(u,v) b(u, v+1), & \textbf{if } u > 0 
        \textbf{ and } v = T \\
        c(u,v) + p_s(u,v) b(u, v-1) + p_o(u,v) b(u, v+1), & \textbf{otherwise} \\
    \end{cases}
\end{equation}

Equation 
(\ref{eq:blocking-time-at-each-state}) will not be solved recursively. 
A direct approach will be used to solve this equation here. 
By enumerating all equations of (\ref{eq:blocking-time-at-each-state}) for all 
states \((u,v)\) that belong in \(S_b\) 
a system of linear equations arises where the unknown variables are all the \(b(u,v)\)
terms.
For instance, let us consider a Markov model where \(C=2, T=3, N=6, M=2\). 
The Markov model is shown in Figure \ref{fig:example-algeb-blocking}
and the equivalent equations are 
(\ref{eq:first_eq_of_blocking_example})-(\ref{eq:last_eq_of_blocking_example}).
The equations considered here are only the ones that correspond to the blocking 
states.

\begin{multicols*}{2}
    \begin{figure}[H]
        \scalebox{0.50}{\input{MarkovChain/expressions_from_pi/example_model_2362/main.tex}}
        \caption{Example of Markov chain}
        \label{fig:example-algeb-blocking}
    \end{figure}
    \columnbreak
    \begin{align}
        b(1,2) &= c(1,2) + p_o b(1,3) \label{eq:first_eq_of_blocking_example} \\
        b(1,3) &= c(1,3) + p_s b(1,2) + p_o b(1,4) \\
        b(1,4) &= c(1,4) + b(1,3) \\
        b(2,2) &= c(2,2) + p_s b(1,2) + p_o b(2,3) \\
        b(2,3) &= c(2,3) + p_s b(2,2) + p_o b(1,4) \\
        b(2,4) &= c(2,4) + b(2,3)\label{eq:last_eq_of_blocking_example}
    \end{align}
\end{multicols*}

Additionally, the above equations can be transformed into a linear system of the 
form \(Zx=y\) where:

\begin{equation}\label{eq:example-algebaric-approach-blocking-time}
    Z=
    \begin{pmatrix}
        -1 & p_o & 0 & 0 & 0 & 0 \\ %(1,2)
        p_s & -1 & p_o & 0 & 0 & 0 \\ %(1,3)
        0 & 1 & -1 & 0 & 0 & 0 \\ %(1,4)
        p_s & 0 & 0 & -1 & p_o & 0\\ %(2,2)
        0 & 0 & 0 & p_s & -1 & p_o \\ %(2,3)
        0 & 0 & 0 & 0 & 1 & -1 \\ %(2,4)
    \end{pmatrix},
    x=
    \begin{pmatrix}
        b(1,2) \\
        b(1,3) \\
        b(1,4) \\
        b(2,2) \\
        b(2,3) \\
        b(2,4) \\
    \end{pmatrix}, 
    y=
    \begin{pmatrix}
        -c(1,2) \\
        -c(1,3) \\
        -c(1,4) \\
        -c(2,2) \\
        -c(2,3) \\
        -c(2,4) \\
    \end{pmatrix}
\end{equation}

A more generalised form of the equations in 
(\ref{eq:example-algebaric-approach-blocking-time})
can thus be given for any value of \(C,T,N,M\) by:

\begin{align}
    b(1,T) =& c(1, T) + p_o b(1, T + 1) \label{eq:first_eq_of_blocking_general}\\
    b(1,T + 1) =& c(1, T + 1) + p_s(1, T) + p_o b(1, T + 1) \\
    b(1,T + 2) =& c(1, T + 2) + p_s(1, T + 1) + p_o b(1, T + 3) \\
    & \vdots \nonumber \\
    b(1, N) =& c(1, N) + b(1, N - 1) \\
    b(2, T) =& c(2, T) + p_s b(1, T) + p_o b(2, T + 1) \\
    b(2, T + 1) =& c(2, T + 1) + p_s b(2, T) + p_o b(2, T + 2) \\
    & \vdots \nonumber \\
    b(M, T) =& c(M, T) + b(M, T-1) \label{eq:last_eq_of_blocking_general}
\end{align}

The equivalent matrix form of the linear system of equations 
(\ref{eq:first_eq_of_blocking_general}) - (\ref{eq:last_eq_of_blocking_general})
is given by \(Zx=y\), where:
\begin{equation}\label{eq:general-algebaric-approach-blocking-time}
    \scalebox{0.9}{
        \(
        Z = 
        \begin{pmatrix}
            -1 & p_o & 0 & \dots & 0 & 0 & 0 & 0 & 0 & \dots & 0 & 0 \\ %(1,T)
            p_s & -1 & p_o & \dots & 0 & 0 & 0 & 0 & 0 & \dots & 0 & 0 \\ %(1,T+1)
            0 & p_s & -1 & \dots & 0 & 0 & 0 & 0 & 0 & \dots & 0 & 0 \\ %(1,T+2)
            \vdots & \vdots & \vdots & \ddots & \vdots & \vdots & \vdots & \vdots & 
            \vdots & \ddots & \vdots & \vdots \\ 
            0 & 0 & 0 & \dots & 1 & -1 & 0 & 0 & 0 & \dots & 0 & 0 \\ %(1,N)
            p_s & 0 & 0 & \dots & 0 & 0 & -1 & p_o & 0 & \dots & 0 & 0 \\ %(2,T)
            0 & 0 & 0 & \dots & 0 & 0 & p_s & -1 & p_o & \dots & 0 & 0 \\ %(2,T+1)
            \vdots & \vdots & \vdots & \ddots & \vdots & \vdots & \vdots & \vdots & 
            \vdots & \ddots & \vdots & \vdots \\ 
            0 & 0 & 0 & \dots & 0 & 0 & 0 & 0 & 0 & \dots & 1 & -1 \\ %(M,T)
        \end{pmatrix},
        x = 
        \begin{pmatrix}
            b(1,T) \\
            b(1,T+1) \\
            b(1,T+2) \\
            \vdots \\
            b(1,N) \\
            b(2,T) \\
            b(2,T+1) \\
            \vdots \\
            b(M,T) \\
        \end{pmatrix}, 
        y= 
        \begin{pmatrix}
            -c(1,T) \\
            -c(1,T+1) \\
            -c(1,T+2) \\
            \vdots \\
            -c(1,N) \\
            -c(2,T) \\
            -c(2,T+1) \\
            \vdots \\
            -c(M,T) \\
        \end{pmatrix}
        \)
    }
\end{equation}

Thus, having calculated the mean blocking time for all blocking states \(b(u,v)\), 
it only remains to put them together in a formula.
The resultant blocking time formula is given by:

\begin{equation}\label{eq:algebraic-blocking-time}
    B = \frac{\sum_{(u,v) \in S_A} \pi_{(u,v)} \; b(u,v)}{\sum_{(u,v) \in S_A} 
    \pi_{(u,v)}}
\end{equation}

\documentclass{article}

\usepackage{amsmath}
\usepackage{amsfonts} 
\usepackage{geometry}
\usepackage{multicol}
\usepackage{float}
% \usepackage{mathtools}
% \usepackage{graphicx}
% \usepackage{soul}
% \usepackage{indentfirst}
\usepackage{tikz}
\usetikzlibrary{calc, automata, chains, arrows.meta, math}
\setcounter{MaxMatrixCols}{20}


\title{A game theoretic model of the behavioural gaming that takes place at the EMS - ED interface}

\author{
    Michalis Panayides, 
    Paul Harper, 
    Vince Knight
}

\begin{document}

\maketitle

\input{Abstract/main.tex}


\newpage
\tableofcontents

\newpage
\input{Introduction/main.tex}

\newpage
\input{Game_theory_component/main.tex}

\newpage
\input{MarkovChain/markov_chain_model/main.tex}
\input{MarkovChain/expressions_from_pi/main.tex}
\input{MarkovChain/markov_example/main.tex}

\newpage
\input{BehaviouralMethodology/main.tex}

\newpage
\input{Application_EMS_ED/main.tex}

\newpage
\input{Conclusion/main.tex}


\end{document}

\newpage
\documentclass{article}

\usepackage{amsmath}
\usepackage{amsfonts} 
\usepackage{geometry}
\usepackage{multicol}
\usepackage{float}
% \usepackage{mathtools}
% \usepackage{graphicx}
% \usepackage{soul}
% \usepackage{indentfirst}
\usepackage{tikz}
\usetikzlibrary{calc, automata, chains, arrows.meta, math}
\setcounter{MaxMatrixCols}{20}


\title{A game theoretic model of the behavioural gaming that takes place at the EMS - ED interface}

\author{
    Michalis Panayides, 
    Paul Harper, 
    Vince Knight
}

\begin{document}

\maketitle

\input{Abstract/main.tex}


\newpage
\tableofcontents

\newpage
\input{Introduction/main.tex}

\newpage
\input{Game_theory_component/main.tex}

\newpage
\input{MarkovChain/markov_chain_model/main.tex}
\input{MarkovChain/expressions_from_pi/main.tex}
\input{MarkovChain/markov_example/main.tex}

\newpage
\input{BehaviouralMethodology/main.tex}

\newpage
\input{Application_EMS_ED/main.tex}

\newpage
\input{Conclusion/main.tex}


\end{document}

\newpage
\section{EMS-ED application}

\subsection{Application}

\subsection{Data analysis of generated problem}

\newpage
\documentclass{article}

\usepackage{amsmath}
\usepackage{amsfonts} 
\usepackage{geometry}
\usepackage{multicol}
\usepackage{float}
% \usepackage{mathtools}
% \usepackage{graphicx}
% \usepackage{soul}
% \usepackage{indentfirst}
\usepackage{tikz}
\usetikzlibrary{calc, automata, chains, arrows.meta, math}
\setcounter{MaxMatrixCols}{20}


\title{A game theoretic model of the behavioural gaming that takes place at the EMS - ED interface}

\author{
    Michalis Panayides, 
    Paul Harper, 
    Vince Knight
}

\begin{document}

\maketitle

\input{Abstract/main.tex}


\newpage
\tableofcontents

\newpage
\input{Introduction/main.tex}

\newpage
\input{Game_theory_component/main.tex}

\newpage
\input{MarkovChain/markov_chain_model/main.tex}
\input{MarkovChain/expressions_from_pi/main.tex}
\input{MarkovChain/markov_example/main.tex}

\newpage
\input{BehaviouralMethodology/main.tex}

\newpage
\input{Application_EMS_ED/main.tex}

\newpage
\input{Conclusion/main.tex}


\end{document}


\end{document}

\newpage
\documentclass{article}

\usepackage{amsmath}
\usepackage{amsfonts} 
\usepackage{geometry}
\usepackage{multicol}
\usepackage{float}
% \usepackage{mathtools}
% \usepackage{graphicx}
% \usepackage{soul}
% \usepackage{indentfirst}
\usepackage{tikz}
\usetikzlibrary{calc, automata, chains, arrows.meta, math}
\setcounter{MaxMatrixCols}{20}


\title{A game theoretic model of the behavioural gaming that takes place at the EMS - ED interface}

\author{
    Michalis Panayides, 
    Paul Harper, 
    Vince Knight
}

\begin{document}

\maketitle

\documentclass{article}

\usepackage{amsmath}
\usepackage{amsfonts} 
\usepackage{geometry}
\usepackage{multicol}
\usepackage{float}
% \usepackage{mathtools}
% \usepackage{graphicx}
% \usepackage{soul}
% \usepackage{indentfirst}
\usepackage{tikz}
\usetikzlibrary{calc, automata, chains, arrows.meta, math}
\setcounter{MaxMatrixCols}{20}


\title{A game theoretic model of the behavioural gaming that takes place at the EMS - ED interface}

\author{
    Michalis Panayides, 
    Paul Harper, 
    Vince Knight
}

\begin{document}

\maketitle

\input{Abstract/main.tex}


\newpage
\tableofcontents

\newpage
\input{Introduction/main.tex}

\newpage
\input{Game_theory_component/main.tex}

\newpage
\input{MarkovChain/markov_chain_model/main.tex}
\input{MarkovChain/expressions_from_pi/main.tex}
\input{MarkovChain/markov_example/main.tex}

\newpage
\input{BehaviouralMethodology/main.tex}

\newpage
\input{Application_EMS_ED/main.tex}

\newpage
\input{Conclusion/main.tex}


\end{document}


\newpage
\tableofcontents

\newpage
\documentclass{article}

\usepackage{amsmath}
\usepackage{amsfonts} 
\usepackage{geometry}
\usepackage{multicol}
\usepackage{float}
% \usepackage{mathtools}
% \usepackage{graphicx}
% \usepackage{soul}
% \usepackage{indentfirst}
\usepackage{tikz}
\usetikzlibrary{calc, automata, chains, arrows.meta, math}
\setcounter{MaxMatrixCols}{20}


\title{A game theoretic model of the behavioural gaming that takes place at the EMS - ED interface}

\author{
    Michalis Panayides, 
    Paul Harper, 
    Vince Knight
}

\begin{document}

\maketitle

\input{Abstract/main.tex}


\newpage
\tableofcontents

\newpage
\input{Introduction/main.tex}

\newpage
\input{Game_theory_component/main.tex}

\newpage
\input{MarkovChain/markov_chain_model/main.tex}
\input{MarkovChain/expressions_from_pi/main.tex}
\input{MarkovChain/markov_example/main.tex}

\newpage
\input{BehaviouralMethodology/main.tex}

\newpage
\input{Application_EMS_ED/main.tex}

\newpage
\input{Conclusion/main.tex}


\end{document}

\newpage
\documentclass{article}

\usepackage{amsmath}
\usepackage{amsfonts} 
\usepackage{geometry}
\usepackage{multicol}
\usepackage{float}
% \usepackage{mathtools}
% \usepackage{graphicx}
% \usepackage{soul}
% \usepackage{indentfirst}
\usepackage{tikz}
\usetikzlibrary{calc, automata, chains, arrows.meta, math}
\setcounter{MaxMatrixCols}{20}


\title{A game theoretic model of the behavioural gaming that takes place at the EMS - ED interface}

\author{
    Michalis Panayides, 
    Paul Harper, 
    Vince Knight
}

\begin{document}

\maketitle

\input{Abstract/main.tex}


\newpage
\tableofcontents

\newpage
\input{Introduction/main.tex}

\newpage
\input{Game_theory_component/main.tex}

\newpage
\input{MarkovChain/markov_chain_model/main.tex}
\input{MarkovChain/expressions_from_pi/main.tex}
\input{MarkovChain/markov_example/main.tex}

\newpage
\input{BehaviouralMethodology/main.tex}

\newpage
\input{Application_EMS_ED/main.tex}

\newpage
\input{Conclusion/main.tex}


\end{document}

\newpage
\documentclass{article}

\usepackage{amsmath}
\usepackage{amsfonts} 
\usepackage{geometry}
\usepackage{multicol}
\usepackage{float}
% \usepackage{mathtools}
% \usepackage{graphicx}
% \usepackage{soul}
% \usepackage{indentfirst}
\usepackage{tikz}
\usetikzlibrary{calc, automata, chains, arrows.meta, math}
\setcounter{MaxMatrixCols}{20}


\title{A game theoretic model of the behavioural gaming that takes place at the EMS - ED interface}

\author{
    Michalis Panayides, 
    Paul Harper, 
    Vince Knight
}

\begin{document}

\maketitle

\input{Abstract/main.tex}


\newpage
\tableofcontents

\newpage
\input{Introduction/main.tex}

\newpage
\input{Game_theory_component/main.tex}

\newpage
\input{MarkovChain/markov_chain_model/main.tex}
\input{MarkovChain/expressions_from_pi/main.tex}
\input{MarkovChain/markov_example/main.tex}

\newpage
\input{BehaviouralMethodology/main.tex}

\newpage
\input{Application_EMS_ED/main.tex}

\newpage
\input{Conclusion/main.tex}


\end{document}
\subsection{Performance Measures}
One may easily derive the average number of individuals that are at any given state 
using \( pi \). 
The average number of individuals in state \( i \) can be calculated by multiplying 
the number of individuals that are present in state \( i \) with the probability 
of being at that particular state (i.e \(\pi_i (u_i + v_i)\)). 
Using this logic it is possible to calculate any performance measures that are related 
to the mean number of individuals in the system.


Average number of people in the system: 
\begin{equation}
    L = \sum_{i=1}^{|\pi|} \pi_i (u_i + v_i)
\end{equation} 

Average number of people in the service centre: 
\begin{equation}
    L_H = \sum_{i=1}^{|\pi|} \pi_i v_i
\end{equation}

Average number of people in the buffer centre:
\begin{equation}
    L_A = \sum_{i=1}^{|\pi|} \pi_i u_i
\end{equation}

Consequently getting the performance measures that are related to the duration of 
time is not as straightforward. 
Such performance measures are the mean waiting time in the system and the mean time 
blocked in the system. 
Under the scope of this study three approaches have been considered to calculate these 
performance measures; a direct approach, a recursive algorithm and consequently a
closed-form formula.

The research question that needs to be answered here is: ``When a class 1/2 
individuals enters the system, what is the expected time that they will have to 
wait?''. 
In order to formulate the answer to that question one needs to consider all possible 
scenarios of what state the system can be in when an individual arrives. 
Furthermore, different formulas arises for class 1 individuals 
and a different one for class 2 individuals.

\subsubsection{Mean waiting time} 
Upon closer inspection of the recursive formula a more compact formula can arise. 
The equivalent closed-form formula eliminates the need for recursion and thus makes 
the computation of waiting times much more efficient. 
Just like in the recursive part there are two formulas; one for \textit{class 1} 
and one for class 2 individuals. 
The formulas are given by:

\begin{equation} \label{eq:closed_form_waiting_others}
    W^{(1)} = \frac{\sum_{\substack{(u,v) \, \in S_A^{(1)} \\ v \geq C}} 
    \frac{1}{C \mu} \times (v-C+1) \times \pi(u,v)}{\sum_{(u,v) \, 
    \in S_A^{(1)}} \pi(u,v)}
\end{equation}
    
\begin{equation}\label{eq:closed_form_waiting_ambulance}
    W^{(2)} = \frac{\sum_{\substack{(u,v) \, \in S_A^{(2)} \\ min(v,T) \geq C}} 
    \frac{1}{C \mu} \times (\min(v+1,T)-C) \times \pi(u,v)}{\sum_{(u,v) \, 
    \in S_A^{(2)}} \pi(u,v)}
\end{equation}

Note here that the summation, in both equations \ref{eq:closed_form_waiting_others} 
and \ref{eq:closed_form_waiting_ambulance}, goes through all states in the set of 
accepting 
states of either class 1 or class 2 individuals respectively, where a wait 
incurs. 
In equation \ref{eq:closed_form_waiting_others} that includes all states \((u,v)\) 
in the set of accepting states of class 1 individuals such that \( v \geq C\); i.e. 
whenever an arrival occurs and the system is at a state where the number of individuals 
in the system is more than or equal to $C$. 
Consequently, for the states that are included in the summation the expression 
\( v-C+1 \) indicates the amount of people in service one would have to wait for 
upon arrival at the hospital.

Additionally, the minimisation function in equation 
\ref{eq:closed_form_waiting_ambulance} 
ensures that when a class 2 individual arrives at any state 
that is greater than the predetermined threshold, the wait that the individual will 
have to endure remains the same. 
In essence, the expression \(\min(v+1,T) - C\) returns the number of people in line 
in front of a particular individual upon arrival.


\subsubsection{Overall Waiting Time}

Consequently, the overall waiting time should can be estimated by a linear combination 
of the waiting times of class 1 and class 2 individuals. 
The overall waiting time can be then given by the following equation where \(c_1\) 
and \(c_2\) are the coefficients of each individual's type waiting time:

\begin{equation}\label{overall_waiting_time_coeff}
    W = c_1 W^{(1)} + c_2 W^{(2)}
\end{equation}

The two coefficients represent the proportion of individuals of each type that 
traversed through the model. 
Theoretically, getting these percentages should be as simple as looking at the arrival 
rates of each type but in practise if the service centre or the buffer centre 
is full, some individuals may be lost to the system. 
Thus, one should account for the probability that an individual is lost to the system. 
This probability can be easily calculated by using the two sets of accepting states 
\(S_A^{(2)}\) and \(S_A^{(1)}\) defined earlier in equations.
Let us define here the probability, for either class type, that an individual 
is not lost in the system by:

\begin{equation*}
    P(L'_1) = \sum_{(u,v) \, \in S_A^{(1)}} \pi(u,v) \hspace{2cm}
    P(L'_2) = \sum_{(u,v) \, \in S_A^{(2)}} \pi(u,v)
\end{equation*}

Having defined these probabilities one may combine them with the arrival rates of 
each class type in such a way to get the expected proportions of class 1 and 
class 2 individuals in the model. 
Thus, by using these values as the coefficient of equation 
\ref{overall_waiting_time_coeff} 
the resultant equation can be used to get the overall waiting time. 
Note here that the equation below gets the overall waiting time for both the recursive 
and the closed-form formula.

\begin{equation}\label{overall_waiting_time}
    W = \frac{\lambda_1 P(L'_1)}{\lambda_2 P(L'_2) + \lambda_1 P(L'_1)} W^{(1)} + 
    \frac{\lambda_2 P(L'_2)}{\lambda_2 P(L'_2) + \lambda_1 P(L'_1)} W^{(2)}
\end{equation}



\subsubsection{Mean blocking time}
Unlike the waiting time, the blocking time is only calculated for class 2 individuals.  
That is because class 1 individuals cannot be blocked. 
Thus, one only needs to consider the pathway of class 2 individuals to get the 
mean blocking time of the system. 
Blocking occurs at states \((u,v)\) where \(u > 0 \). 
Thus, the set of blocking states can be defined as:

\begin{equation*}
    S_b = \{(u,v) \in S \; | \; u > 0\}
\end{equation*}
 
In order to not consider individuals that will be lost to the system, the set of 
accepting states needs to be taken into account. The set of accepting states is given by:

\begin{equation*}
    S_A^{(2)}=
    \begin{cases}
        \{(u, v) \in S \; | \; u < M \} & \textbf{if } T \leq N\\
        \{(u, v) \in S \; | \; v < N \} & \textbf{otherwise}
    \end{cases}
\end{equation*}

For the waiting time formula,
the mean sojourn time for each state was considered,
ignoring any arrivals. Here, the same approach is used but ignoring only class 2
arrivals. That is because for the waiting time formula, once an individual enters 
the service centre (i.e. starts waiting) any individual arriving after them will 
not affect their
pathway. That is not the case for blocking time. When a class 2 individual is 
blocked, 
any class 1 individual that arrives will cause the blocked individual to remain 
blocked for more time. Therefore, class 1 arrivals are considered here:

\begin{equation}\label{eq:time_in_state_blocking_time}
    c(u,v) = 
    \begin{cases}
        \frac{1}{\min(v,C) \mu}, & \text{if } v = C\\
        \frac{1}{\min(v,C) \mu + \lambda_1}, & \text{otherwise}
    \end{cases}
\end{equation}
 
In equation \ref{eq:time_in_state_blocking_time}, both service completions and 
class 1 arrivals are considered. 
Thus, from a blocked individual's perspective whenever the system moves from one 
state \((u,v)\)
to another state it can either:

\begin{itemize}
    \item be because of a service being completed: we will denote the probability 
    of this happening by \(p_s(u,v)\). 
    \item be because of an arrival of an individual of class 1: denoting such 
    probability by \(p_o(u,v)\).
\end{itemize}
The probabilities are given by:

\begin{equation*}
    p_s(u,v) = \frac{\min(v,C)\mu}{\lambda_1 + \min(v,C)\mu}, \qquad
    p_o(u,v) = \frac{\lambda_1}{\lambda_1 + \min(v,C)\mu}
\end{equation*}


Having defined \(c(u,v)\) and \(S_b\) a formula for the blocking time that is
expected to occur at each state can be given by:

\begin{equation}\label{eq:blocking-time-at-each-state}
    b(u,v) = 
    \begin{cases} 
        0, & \textbf{if } (u,v) \notin S_b \\
        c(u,v) + b(u - 1, v), & \textbf{if } v = N = T\\
        c(u,v) + b(u, v-1), & \textbf{if } v = N \neq T \\
        c(u,v) + p_s(u,v) b(u-1, v) + p_o(u,v) b(u, v+1), & \textbf{if } u > 0 
        \textbf{ and } v = T \\
        c(u,v) + p_s(u,v) b(u, v-1) + p_o(u,v) b(u, v+1), & \textbf{otherwise} \\
    \end{cases}
\end{equation}

Equation 
(\ref{eq:blocking-time-at-each-state}) will not be solved recursively. 
A direct approach will be used to solve this equation here. 
By enumerating all equations of (\ref{eq:blocking-time-at-each-state}) for all 
states \((u,v)\) that belong in \(S_b\) 
a system of linear equations arises where the unknown variables are all the \(b(u,v)\)
terms.
For instance, let us consider a Markov model where \(C=2, T=3, N=6, M=2\). 
The Markov model is shown in Figure \ref{fig:example-algeb-blocking}
and the equivalent equations are 
(\ref{eq:first_eq_of_blocking_example})-(\ref{eq:last_eq_of_blocking_example}).
The equations considered here are only the ones that correspond to the blocking 
states.

\begin{multicols*}{2}
    \begin{figure}[H]
        \scalebox{0.50}{\input{MarkovChain/expressions_from_pi/example_model_2362/main.tex}}
        \caption{Example of Markov chain}
        \label{fig:example-algeb-blocking}
    \end{figure}
    \columnbreak
    \begin{align}
        b(1,2) &= c(1,2) + p_o b(1,3) \label{eq:first_eq_of_blocking_example} \\
        b(1,3) &= c(1,3) + p_s b(1,2) + p_o b(1,4) \\
        b(1,4) &= c(1,4) + b(1,3) \\
        b(2,2) &= c(2,2) + p_s b(1,2) + p_o b(2,3) \\
        b(2,3) &= c(2,3) + p_s b(2,2) + p_o b(1,4) \\
        b(2,4) &= c(2,4) + b(2,3)\label{eq:last_eq_of_blocking_example}
    \end{align}
\end{multicols*}

Additionally, the above equations can be transformed into a linear system of the 
form \(Zx=y\) where:

\begin{equation}\label{eq:example-algebaric-approach-blocking-time}
    Z=
    \begin{pmatrix}
        -1 & p_o & 0 & 0 & 0 & 0 \\ %(1,2)
        p_s & -1 & p_o & 0 & 0 & 0 \\ %(1,3)
        0 & 1 & -1 & 0 & 0 & 0 \\ %(1,4)
        p_s & 0 & 0 & -1 & p_o & 0\\ %(2,2)
        0 & 0 & 0 & p_s & -1 & p_o \\ %(2,3)
        0 & 0 & 0 & 0 & 1 & -1 \\ %(2,4)
    \end{pmatrix},
    x=
    \begin{pmatrix}
        b(1,2) \\
        b(1,3) \\
        b(1,4) \\
        b(2,2) \\
        b(2,3) \\
        b(2,4) \\
    \end{pmatrix}, 
    y=
    \begin{pmatrix}
        -c(1,2) \\
        -c(1,3) \\
        -c(1,4) \\
        -c(2,2) \\
        -c(2,3) \\
        -c(2,4) \\
    \end{pmatrix}
\end{equation}

A more generalised form of the equations in 
(\ref{eq:example-algebaric-approach-blocking-time})
can thus be given for any value of \(C,T,N,M\) by:

\begin{align}
    b(1,T) =& c(1, T) + p_o b(1, T + 1) \label{eq:first_eq_of_blocking_general}\\
    b(1,T + 1) =& c(1, T + 1) + p_s(1, T) + p_o b(1, T + 1) \\
    b(1,T + 2) =& c(1, T + 2) + p_s(1, T + 1) + p_o b(1, T + 3) \\
    & \vdots \nonumber \\
    b(1, N) =& c(1, N) + b(1, N - 1) \\
    b(2, T) =& c(2, T) + p_s b(1, T) + p_o b(2, T + 1) \\
    b(2, T + 1) =& c(2, T + 1) + p_s b(2, T) + p_o b(2, T + 2) \\
    & \vdots \nonumber \\
    b(M, T) =& c(M, T) + b(M, T-1) \label{eq:last_eq_of_blocking_general}
\end{align}

The equivalent matrix form of the linear system of equations 
(\ref{eq:first_eq_of_blocking_general}) - (\ref{eq:last_eq_of_blocking_general})
is given by \(Zx=y\), where:
\begin{equation}\label{eq:general-algebaric-approach-blocking-time}
    \scalebox{0.9}{
        \(
        Z = 
        \begin{pmatrix}
            -1 & p_o & 0 & \dots & 0 & 0 & 0 & 0 & 0 & \dots & 0 & 0 \\ %(1,T)
            p_s & -1 & p_o & \dots & 0 & 0 & 0 & 0 & 0 & \dots & 0 & 0 \\ %(1,T+1)
            0 & p_s & -1 & \dots & 0 & 0 & 0 & 0 & 0 & \dots & 0 & 0 \\ %(1,T+2)
            \vdots & \vdots & \vdots & \ddots & \vdots & \vdots & \vdots & \vdots & 
            \vdots & \ddots & \vdots & \vdots \\ 
            0 & 0 & 0 & \dots & 1 & -1 & 0 & 0 & 0 & \dots & 0 & 0 \\ %(1,N)
            p_s & 0 & 0 & \dots & 0 & 0 & -1 & p_o & 0 & \dots & 0 & 0 \\ %(2,T)
            0 & 0 & 0 & \dots & 0 & 0 & p_s & -1 & p_o & \dots & 0 & 0 \\ %(2,T+1)
            \vdots & \vdots & \vdots & \ddots & \vdots & \vdots & \vdots & \vdots & 
            \vdots & \ddots & \vdots & \vdots \\ 
            0 & 0 & 0 & \dots & 0 & 0 & 0 & 0 & 0 & \dots & 1 & -1 \\ %(M,T)
        \end{pmatrix},
        x = 
        \begin{pmatrix}
            b(1,T) \\
            b(1,T+1) \\
            b(1,T+2) \\
            \vdots \\
            b(1,N) \\
            b(2,T) \\
            b(2,T+1) \\
            \vdots \\
            b(M,T) \\
        \end{pmatrix}, 
        y= 
        \begin{pmatrix}
            -c(1,T) \\
            -c(1,T+1) \\
            -c(1,T+2) \\
            \vdots \\
            -c(1,N) \\
            -c(2,T) \\
            -c(2,T+1) \\
            \vdots \\
            -c(M,T) \\
        \end{pmatrix}
        \)
    }
\end{equation}

Thus, having calculated the mean blocking time for all blocking states \(b(u,v)\), 
it only remains to put them together in a formula.
The resultant blocking time formula is given by:

\begin{equation}\label{eq:algebraic-blocking-time}
    B = \frac{\sum_{(u,v) \in S_A} \pi_{(u,v)} \; b(u,v)}{\sum_{(u,v) \in S_A} 
    \pi_{(u,v)}}
\end{equation}

\documentclass{article}

\usepackage{amsmath}
\usepackage{amsfonts} 
\usepackage{geometry}
\usepackage{multicol}
\usepackage{float}
% \usepackage{mathtools}
% \usepackage{graphicx}
% \usepackage{soul}
% \usepackage{indentfirst}
\usepackage{tikz}
\usetikzlibrary{calc, automata, chains, arrows.meta, math}
\setcounter{MaxMatrixCols}{20}


\title{A game theoretic model of the behavioural gaming that takes place at the EMS - ED interface}

\author{
    Michalis Panayides, 
    Paul Harper, 
    Vince Knight
}

\begin{document}

\maketitle

\input{Abstract/main.tex}


\newpage
\tableofcontents

\newpage
\input{Introduction/main.tex}

\newpage
\input{Game_theory_component/main.tex}

\newpage
\input{MarkovChain/markov_chain_model/main.tex}
\input{MarkovChain/expressions_from_pi/main.tex}
\input{MarkovChain/markov_example/main.tex}

\newpage
\input{BehaviouralMethodology/main.tex}

\newpage
\input{Application_EMS_ED/main.tex}

\newpage
\input{Conclusion/main.tex}


\end{document}

\newpage
\documentclass{article}

\usepackage{amsmath}
\usepackage{amsfonts} 
\usepackage{geometry}
\usepackage{multicol}
\usepackage{float}
% \usepackage{mathtools}
% \usepackage{graphicx}
% \usepackage{soul}
% \usepackage{indentfirst}
\usepackage{tikz}
\usetikzlibrary{calc, automata, chains, arrows.meta, math}
\setcounter{MaxMatrixCols}{20}


\title{A game theoretic model of the behavioural gaming that takes place at the EMS - ED interface}

\author{
    Michalis Panayides, 
    Paul Harper, 
    Vince Knight
}

\begin{document}

\maketitle

\input{Abstract/main.tex}


\newpage
\tableofcontents

\newpage
\input{Introduction/main.tex}

\newpage
\input{Game_theory_component/main.tex}

\newpage
\input{MarkovChain/markov_chain_model/main.tex}
\input{MarkovChain/expressions_from_pi/main.tex}
\input{MarkovChain/markov_example/main.tex}

\newpage
\input{BehaviouralMethodology/main.tex}

\newpage
\input{Application_EMS_ED/main.tex}

\newpage
\input{Conclusion/main.tex}


\end{document}

\newpage
\section{EMS-ED application}

\subsection{Application}

\subsection{Data analysis of generated problem}

\newpage
\documentclass{article}

\usepackage{amsmath}
\usepackage{amsfonts} 
\usepackage{geometry}
\usepackage{multicol}
\usepackage{float}
% \usepackage{mathtools}
% \usepackage{graphicx}
% \usepackage{soul}
% \usepackage{indentfirst}
\usepackage{tikz}
\usetikzlibrary{calc, automata, chains, arrows.meta, math}
\setcounter{MaxMatrixCols}{20}


\title{A game theoretic model of the behavioural gaming that takes place at the EMS - ED interface}

\author{
    Michalis Panayides, 
    Paul Harper, 
    Vince Knight
}

\begin{document}

\maketitle

\input{Abstract/main.tex}


\newpage
\tableofcontents

\newpage
\input{Introduction/main.tex}

\newpage
\input{Game_theory_component/main.tex}

\newpage
\input{MarkovChain/markov_chain_model/main.tex}
\input{MarkovChain/expressions_from_pi/main.tex}
\input{MarkovChain/markov_example/main.tex}

\newpage
\input{BehaviouralMethodology/main.tex}

\newpage
\input{Application_EMS_ED/main.tex}

\newpage
\input{Conclusion/main.tex}


\end{document}


\end{document}
\subsection{Performance Measures}
One may easily derive the average number of individuals that are at any given state 
using \( pi \). 
The average number of individuals in state \( i \) can be calculated by multiplying 
the number of individuals that are present in state \( i \) with the probability 
of being at that particular state (i.e \(\pi_i (u_i + v_i)\)). 
Using this logic it is possible to calculate any performance measures that are related 
to the mean number of individuals in the system.


Average number of people in the system: 
\begin{equation}
    L = \sum_{i=1}^{|\pi|} \pi_i (u_i + v_i)
\end{equation} 

Average number of people in the service centre: 
\begin{equation}
    L_H = \sum_{i=1}^{|\pi|} \pi_i v_i
\end{equation}

Average number of people in the buffer centre:
\begin{equation}
    L_A = \sum_{i=1}^{|\pi|} \pi_i u_i
\end{equation}

Consequently getting the performance measures that are related to the duration of 
time is not as straightforward. 
Such performance measures are the mean waiting time in the system and the mean time 
blocked in the system. 
Under the scope of this study three approaches have been considered to calculate these 
performance measures; a direct approach, a recursive algorithm and consequently a
closed-form formula.

The research question that needs to be answered here is: ``When a class 1/2 
individuals enters the system, what is the expected time that they will have to 
wait?''. 
In order to formulate the answer to that question one needs to consider all possible 
scenarios of what state the system can be in when an individual arrives. 
Furthermore, different formulas arises for class 1 individuals 
and a different one for class 2 individuals.

\subsubsection{Mean waiting time} 
Upon closer inspection of the recursive formula a more compact formula can arise. 
The equivalent closed-form formula eliminates the need for recursion and thus makes 
the computation of waiting times much more efficient. 
Just like in the recursive part there are two formulas; one for \textit{class 1} 
and one for class 2 individuals. 
The formulas are given by:

\begin{equation} \label{eq:closed_form_waiting_others}
    W^{(1)} = \frac{\sum_{\substack{(u,v) \, \in S_A^{(1)} \\ v \geq C}} 
    \frac{1}{C \mu} \times (v-C+1) \times \pi(u,v)}{\sum_{(u,v) \, 
    \in S_A^{(1)}} \pi(u,v)}
\end{equation}
    
\begin{equation}\label{eq:closed_form_waiting_ambulance}
    W^{(2)} = \frac{\sum_{\substack{(u,v) \, \in S_A^{(2)} \\ min(v,T) \geq C}} 
    \frac{1}{C \mu} \times (\min(v+1,T)-C) \times \pi(u,v)}{\sum_{(u,v) \, 
    \in S_A^{(2)}} \pi(u,v)}
\end{equation}

Note here that the summation, in both equations \ref{eq:closed_form_waiting_others} 
and \ref{eq:closed_form_waiting_ambulance}, goes through all states in the set of 
accepting 
states of either class 1 or class 2 individuals respectively, where a wait 
incurs. 
In equation \ref{eq:closed_form_waiting_others} that includes all states \((u,v)\) 
in the set of accepting states of class 1 individuals such that \( v \geq C\); i.e. 
whenever an arrival occurs and the system is at a state where the number of individuals 
in the system is more than or equal to $C$. 
Consequently, for the states that are included in the summation the expression 
\( v-C+1 \) indicates the amount of people in service one would have to wait for 
upon arrival at the hospital.

Additionally, the minimisation function in equation 
\ref{eq:closed_form_waiting_ambulance} 
ensures that when a class 2 individual arrives at any state 
that is greater than the predetermined threshold, the wait that the individual will 
have to endure remains the same. 
In essence, the expression \(\min(v+1,T) - C\) returns the number of people in line 
in front of a particular individual upon arrival.


\subsubsection{Overall Waiting Time}

Consequently, the overall waiting time should can be estimated by a linear combination 
of the waiting times of class 1 and class 2 individuals. 
The overall waiting time can be then given by the following equation where \(c_1\) 
and \(c_2\) are the coefficients of each individual's type waiting time:

\begin{equation}\label{overall_waiting_time_coeff}
    W = c_1 W^{(1)} + c_2 W^{(2)}
\end{equation}

The two coefficients represent the proportion of individuals of each type that 
traversed through the model. 
Theoretically, getting these percentages should be as simple as looking at the arrival 
rates of each type but in practise if the service centre or the buffer centre 
is full, some individuals may be lost to the system. 
Thus, one should account for the probability that an individual is lost to the system. 
This probability can be easily calculated by using the two sets of accepting states 
\(S_A^{(2)}\) and \(S_A^{(1)}\) defined earlier in equations.
Let us define here the probability, for either class type, that an individual 
is not lost in the system by:

\begin{equation*}
    P(L'_1) = \sum_{(u,v) \, \in S_A^{(1)}} \pi(u,v) \hspace{2cm}
    P(L'_2) = \sum_{(u,v) \, \in S_A^{(2)}} \pi(u,v)
\end{equation*}

Having defined these probabilities one may combine them with the arrival rates of 
each class type in such a way to get the expected proportions of class 1 and 
class 2 individuals in the model. 
Thus, by using these values as the coefficient of equation 
\ref{overall_waiting_time_coeff} 
the resultant equation can be used to get the overall waiting time. 
Note here that the equation below gets the overall waiting time for both the recursive 
and the closed-form formula.

\begin{equation}\label{overall_waiting_time}
    W = \frac{\lambda_1 P(L'_1)}{\lambda_2 P(L'_2) + \lambda_1 P(L'_1)} W^{(1)} + 
    \frac{\lambda_2 P(L'_2)}{\lambda_2 P(L'_2) + \lambda_1 P(L'_1)} W^{(2)}
\end{equation}



\subsubsection{Mean blocking time}
Unlike the waiting time, the blocking time is only calculated for class 2 individuals.  
That is because class 1 individuals cannot be blocked. 
Thus, one only needs to consider the pathway of class 2 individuals to get the 
mean blocking time of the system. 
Blocking occurs at states \((u,v)\) where \(u > 0 \). 
Thus, the set of blocking states can be defined as:

\begin{equation*}
    S_b = \{(u,v) \in S \; | \; u > 0\}
\end{equation*}
 
In order to not consider individuals that will be lost to the system, the set of 
accepting states needs to be taken into account. The set of accepting states is given by:

\begin{equation*}
    S_A^{(2)}=
    \begin{cases}
        \{(u, v) \in S \; | \; u < M \} & \textbf{if } T \leq N\\
        \{(u, v) \in S \; | \; v < N \} & \textbf{otherwise}
    \end{cases}
\end{equation*}

For the waiting time formula,
the mean sojourn time for each state was considered,
ignoring any arrivals. Here, the same approach is used but ignoring only class 2
arrivals. That is because for the waiting time formula, once an individual enters 
the service centre (i.e. starts waiting) any individual arriving after them will 
not affect their
pathway. That is not the case for blocking time. When a class 2 individual is 
blocked, 
any class 1 individual that arrives will cause the blocked individual to remain 
blocked for more time. Therefore, class 1 arrivals are considered here:

\begin{equation}\label{eq:time_in_state_blocking_time}
    c(u,v) = 
    \begin{cases}
        \frac{1}{\min(v,C) \mu}, & \text{if } v = C\\
        \frac{1}{\min(v,C) \mu + \lambda_1}, & \text{otherwise}
    \end{cases}
\end{equation}
 
In equation \ref{eq:time_in_state_blocking_time}, both service completions and 
class 1 arrivals are considered. 
Thus, from a blocked individual's perspective whenever the system moves from one 
state \((u,v)\)
to another state it can either:

\begin{itemize}
    \item be because of a service being completed: we will denote the probability 
    of this happening by \(p_s(u,v)\). 
    \item be because of an arrival of an individual of class 1: denoting such 
    probability by \(p_o(u,v)\).
\end{itemize}
The probabilities are given by:

\begin{equation*}
    p_s(u,v) = \frac{\min(v,C)\mu}{\lambda_1 + \min(v,C)\mu}, \qquad
    p_o(u,v) = \frac{\lambda_1}{\lambda_1 + \min(v,C)\mu}
\end{equation*}


Having defined \(c(u,v)\) and \(S_b\) a formula for the blocking time that is
expected to occur at each state can be given by:

\begin{equation}\label{eq:blocking-time-at-each-state}
    b(u,v) = 
    \begin{cases} 
        0, & \textbf{if } (u,v) \notin S_b \\
        c(u,v) + b(u - 1, v), & \textbf{if } v = N = T\\
        c(u,v) + b(u, v-1), & \textbf{if } v = N \neq T \\
        c(u,v) + p_s(u,v) b(u-1, v) + p_o(u,v) b(u, v+1), & \textbf{if } u > 0 
        \textbf{ and } v = T \\
        c(u,v) + p_s(u,v) b(u, v-1) + p_o(u,v) b(u, v+1), & \textbf{otherwise} \\
    \end{cases}
\end{equation}

Equation 
(\ref{eq:blocking-time-at-each-state}) will not be solved recursively. 
A direct approach will be used to solve this equation here. 
By enumerating all equations of (\ref{eq:blocking-time-at-each-state}) for all 
states \((u,v)\) that belong in \(S_b\) 
a system of linear equations arises where the unknown variables are all the \(b(u,v)\)
terms.
For instance, let us consider a Markov model where \(C=2, T=3, N=6, M=2\). 
The Markov model is shown in Figure \ref{fig:example-algeb-blocking}
and the equivalent equations are 
(\ref{eq:first_eq_of_blocking_example})-(\ref{eq:last_eq_of_blocking_example}).
The equations considered here are only the ones that correspond to the blocking 
states.

\begin{multicols*}{2}
    \begin{figure}[H]
        \scalebox{0.50}{\documentclass{article}

\usepackage{amsmath}
\usepackage{amsfonts} 
\usepackage{geometry}
\usepackage{multicol}
\usepackage{float}
% \usepackage{mathtools}
% \usepackage{graphicx}
% \usepackage{soul}
% \usepackage{indentfirst}
\usepackage{tikz}
\usetikzlibrary{calc, automata, chains, arrows.meta, math}
\setcounter{MaxMatrixCols}{20}


\title{A game theoretic model of the behavioural gaming that takes place at the EMS - ED interface}

\author{
    Michalis Panayides, 
    Paul Harper, 
    Vince Knight
}

\begin{document}

\maketitle

\input{Abstract/main.tex}


\newpage
\tableofcontents

\newpage
\input{Introduction/main.tex}

\newpage
\input{Game_theory_component/main.tex}

\newpage
\input{MarkovChain/markov_chain_model/main.tex}
\input{MarkovChain/expressions_from_pi/main.tex}
\input{MarkovChain/markov_example/main.tex}

\newpage
\input{BehaviouralMethodology/main.tex}

\newpage
\input{Application_EMS_ED/main.tex}

\newpage
\input{Conclusion/main.tex}


\end{document}}
        \caption{Example of Markov chain}
        \label{fig:example-algeb-blocking}
    \end{figure}
    \columnbreak
    \begin{align}
        b(1,2) &= c(1,2) + p_o b(1,3) \label{eq:first_eq_of_blocking_example} \\
        b(1,3) &= c(1,3) + p_s b(1,2) + p_o b(1,4) \\
        b(1,4) &= c(1,4) + b(1,3) \\
        b(2,2) &= c(2,2) + p_s b(1,2) + p_o b(2,3) \\
        b(2,3) &= c(2,3) + p_s b(2,2) + p_o b(1,4) \\
        b(2,4) &= c(2,4) + b(2,3)\label{eq:last_eq_of_blocking_example}
    \end{align}
\end{multicols*}

Additionally, the above equations can be transformed into a linear system of the 
form \(Zx=y\) where:

\begin{equation}\label{eq:example-algebaric-approach-blocking-time}
    Z=
    \begin{pmatrix}
        -1 & p_o & 0 & 0 & 0 & 0 \\ %(1,2)
        p_s & -1 & p_o & 0 & 0 & 0 \\ %(1,3)
        0 & 1 & -1 & 0 & 0 & 0 \\ %(1,4)
        p_s & 0 & 0 & -1 & p_o & 0\\ %(2,2)
        0 & 0 & 0 & p_s & -1 & p_o \\ %(2,3)
        0 & 0 & 0 & 0 & 1 & -1 \\ %(2,4)
    \end{pmatrix},
    x=
    \begin{pmatrix}
        b(1,2) \\
        b(1,3) \\
        b(1,4) \\
        b(2,2) \\
        b(2,3) \\
        b(2,4) \\
    \end{pmatrix}, 
    y=
    \begin{pmatrix}
        -c(1,2) \\
        -c(1,3) \\
        -c(1,4) \\
        -c(2,2) \\
        -c(2,3) \\
        -c(2,4) \\
    \end{pmatrix}
\end{equation}

A more generalised form of the equations in 
(\ref{eq:example-algebaric-approach-blocking-time})
can thus be given for any value of \(C,T,N,M\) by:

\begin{align}
    b(1,T) =& c(1, T) + p_o b(1, T + 1) \label{eq:first_eq_of_blocking_general}\\
    b(1,T + 1) =& c(1, T + 1) + p_s(1, T) + p_o b(1, T + 1) \\
    b(1,T + 2) =& c(1, T + 2) + p_s(1, T + 1) + p_o b(1, T + 3) \\
    & \vdots \nonumber \\
    b(1, N) =& c(1, N) + b(1, N - 1) \\
    b(2, T) =& c(2, T) + p_s b(1, T) + p_o b(2, T + 1) \\
    b(2, T + 1) =& c(2, T + 1) + p_s b(2, T) + p_o b(2, T + 2) \\
    & \vdots \nonumber \\
    b(M, T) =& c(M, T) + b(M, T-1) \label{eq:last_eq_of_blocking_general}
\end{align}

The equivalent matrix form of the linear system of equations 
(\ref{eq:first_eq_of_blocking_general}) - (\ref{eq:last_eq_of_blocking_general})
is given by \(Zx=y\), where:
\begin{equation}\label{eq:general-algebaric-approach-blocking-time}
    \scalebox{0.9}{
        \(
        Z = 
        \begin{pmatrix}
            -1 & p_o & 0 & \dots & 0 & 0 & 0 & 0 & 0 & \dots & 0 & 0 \\ %(1,T)
            p_s & -1 & p_o & \dots & 0 & 0 & 0 & 0 & 0 & \dots & 0 & 0 \\ %(1,T+1)
            0 & p_s & -1 & \dots & 0 & 0 & 0 & 0 & 0 & \dots & 0 & 0 \\ %(1,T+2)
            \vdots & \vdots & \vdots & \ddots & \vdots & \vdots & \vdots & \vdots & 
            \vdots & \ddots & \vdots & \vdots \\ 
            0 & 0 & 0 & \dots & 1 & -1 & 0 & 0 & 0 & \dots & 0 & 0 \\ %(1,N)
            p_s & 0 & 0 & \dots & 0 & 0 & -1 & p_o & 0 & \dots & 0 & 0 \\ %(2,T)
            0 & 0 & 0 & \dots & 0 & 0 & p_s & -1 & p_o & \dots & 0 & 0 \\ %(2,T+1)
            \vdots & \vdots & \vdots & \ddots & \vdots & \vdots & \vdots & \vdots & 
            \vdots & \ddots & \vdots & \vdots \\ 
            0 & 0 & 0 & \dots & 0 & 0 & 0 & 0 & 0 & \dots & 1 & -1 \\ %(M,T)
        \end{pmatrix},
        x = 
        \begin{pmatrix}
            b(1,T) \\
            b(1,T+1) \\
            b(1,T+2) \\
            \vdots \\
            b(1,N) \\
            b(2,T) \\
            b(2,T+1) \\
            \vdots \\
            b(M,T) \\
        \end{pmatrix}, 
        y= 
        \begin{pmatrix}
            -c(1,T) \\
            -c(1,T+1) \\
            -c(1,T+2) \\
            \vdots \\
            -c(1,N) \\
            -c(2,T) \\
            -c(2,T+1) \\
            \vdots \\
            -c(M,T) \\
        \end{pmatrix}
        \)
    }
\end{equation}

Thus, having calculated the mean blocking time for all blocking states \(b(u,v)\), 
it only remains to put them together in a formula.
The resultant blocking time formula is given by:

\begin{equation}\label{eq:algebraic-blocking-time}
    B = \frac{\sum_{(u,v) \in S_A} \pi_{(u,v)} \; b(u,v)}{\sum_{(u,v) \in S_A} 
    \pi_{(u,v)}}
\end{equation}

\documentclass{article}

\usepackage{amsmath}
\usepackage{amsfonts} 
\usepackage{geometry}
\usepackage{multicol}
\usepackage{float}
% \usepackage{mathtools}
% \usepackage{graphicx}
% \usepackage{soul}
% \usepackage{indentfirst}
\usepackage{tikz}
\usetikzlibrary{calc, automata, chains, arrows.meta, math}
\setcounter{MaxMatrixCols}{20}


\title{A game theoretic model of the behavioural gaming that takes place at the EMS - ED interface}

\author{
    Michalis Panayides, 
    Paul Harper, 
    Vince Knight
}

\begin{document}

\maketitle

\documentclass{article}

\usepackage{amsmath}
\usepackage{amsfonts} 
\usepackage{geometry}
\usepackage{multicol}
\usepackage{float}
% \usepackage{mathtools}
% \usepackage{graphicx}
% \usepackage{soul}
% \usepackage{indentfirst}
\usepackage{tikz}
\usetikzlibrary{calc, automata, chains, arrows.meta, math}
\setcounter{MaxMatrixCols}{20}


\title{A game theoretic model of the behavioural gaming that takes place at the EMS - ED interface}

\author{
    Michalis Panayides, 
    Paul Harper, 
    Vince Knight
}

\begin{document}

\maketitle

\input{Abstract/main.tex}


\newpage
\tableofcontents

\newpage
\input{Introduction/main.tex}

\newpage
\input{Game_theory_component/main.tex}

\newpage
\input{MarkovChain/markov_chain_model/main.tex}
\input{MarkovChain/expressions_from_pi/main.tex}
\input{MarkovChain/markov_example/main.tex}

\newpage
\input{BehaviouralMethodology/main.tex}

\newpage
\input{Application_EMS_ED/main.tex}

\newpage
\input{Conclusion/main.tex}


\end{document}


\newpage
\tableofcontents

\newpage
\documentclass{article}

\usepackage{amsmath}
\usepackage{amsfonts} 
\usepackage{geometry}
\usepackage{multicol}
\usepackage{float}
% \usepackage{mathtools}
% \usepackage{graphicx}
% \usepackage{soul}
% \usepackage{indentfirst}
\usepackage{tikz}
\usetikzlibrary{calc, automata, chains, arrows.meta, math}
\setcounter{MaxMatrixCols}{20}


\title{A game theoretic model of the behavioural gaming that takes place at the EMS - ED interface}

\author{
    Michalis Panayides, 
    Paul Harper, 
    Vince Knight
}

\begin{document}

\maketitle

\input{Abstract/main.tex}


\newpage
\tableofcontents

\newpage
\input{Introduction/main.tex}

\newpage
\input{Game_theory_component/main.tex}

\newpage
\input{MarkovChain/markov_chain_model/main.tex}
\input{MarkovChain/expressions_from_pi/main.tex}
\input{MarkovChain/markov_example/main.tex}

\newpage
\input{BehaviouralMethodology/main.tex}

\newpage
\input{Application_EMS_ED/main.tex}

\newpage
\input{Conclusion/main.tex}


\end{document}

\newpage
\documentclass{article}

\usepackage{amsmath}
\usepackage{amsfonts} 
\usepackage{geometry}
\usepackage{multicol}
\usepackage{float}
% \usepackage{mathtools}
% \usepackage{graphicx}
% \usepackage{soul}
% \usepackage{indentfirst}
\usepackage{tikz}
\usetikzlibrary{calc, automata, chains, arrows.meta, math}
\setcounter{MaxMatrixCols}{20}


\title{A game theoretic model of the behavioural gaming that takes place at the EMS - ED interface}

\author{
    Michalis Panayides, 
    Paul Harper, 
    Vince Knight
}

\begin{document}

\maketitle

\input{Abstract/main.tex}


\newpage
\tableofcontents

\newpage
\input{Introduction/main.tex}

\newpage
\input{Game_theory_component/main.tex}

\newpage
\input{MarkovChain/markov_chain_model/main.tex}
\input{MarkovChain/expressions_from_pi/main.tex}
\input{MarkovChain/markov_example/main.tex}

\newpage
\input{BehaviouralMethodology/main.tex}

\newpage
\input{Application_EMS_ED/main.tex}

\newpage
\input{Conclusion/main.tex}


\end{document}

\newpage
\documentclass{article}

\usepackage{amsmath}
\usepackage{amsfonts} 
\usepackage{geometry}
\usepackage{multicol}
\usepackage{float}
% \usepackage{mathtools}
% \usepackage{graphicx}
% \usepackage{soul}
% \usepackage{indentfirst}
\usepackage{tikz}
\usetikzlibrary{calc, automata, chains, arrows.meta, math}
\setcounter{MaxMatrixCols}{20}


\title{A game theoretic model of the behavioural gaming that takes place at the EMS - ED interface}

\author{
    Michalis Panayides, 
    Paul Harper, 
    Vince Knight
}

\begin{document}

\maketitle

\input{Abstract/main.tex}


\newpage
\tableofcontents

\newpage
\input{Introduction/main.tex}

\newpage
\input{Game_theory_component/main.tex}

\newpage
\input{MarkovChain/markov_chain_model/main.tex}
\input{MarkovChain/expressions_from_pi/main.tex}
\input{MarkovChain/markov_example/main.tex}

\newpage
\input{BehaviouralMethodology/main.tex}

\newpage
\input{Application_EMS_ED/main.tex}

\newpage
\input{Conclusion/main.tex}


\end{document}
\subsection{Performance Measures}
One may easily derive the average number of individuals that are at any given state 
using \( pi \). 
The average number of individuals in state \( i \) can be calculated by multiplying 
the number of individuals that are present in state \( i \) with the probability 
of being at that particular state (i.e \(\pi_i (u_i + v_i)\)). 
Using this logic it is possible to calculate any performance measures that are related 
to the mean number of individuals in the system.


Average number of people in the system: 
\begin{equation}
    L = \sum_{i=1}^{|\pi|} \pi_i (u_i + v_i)
\end{equation} 

Average number of people in the service centre: 
\begin{equation}
    L_H = \sum_{i=1}^{|\pi|} \pi_i v_i
\end{equation}

Average number of people in the buffer centre:
\begin{equation}
    L_A = \sum_{i=1}^{|\pi|} \pi_i u_i
\end{equation}

Consequently getting the performance measures that are related to the duration of 
time is not as straightforward. 
Such performance measures are the mean waiting time in the system and the mean time 
blocked in the system. 
Under the scope of this study three approaches have been considered to calculate these 
performance measures; a direct approach, a recursive algorithm and consequently a
closed-form formula.

The research question that needs to be answered here is: ``When a class 1/2 
individuals enters the system, what is the expected time that they will have to 
wait?''. 
In order to formulate the answer to that question one needs to consider all possible 
scenarios of what state the system can be in when an individual arrives. 
Furthermore, different formulas arises for class 1 individuals 
and a different one for class 2 individuals.

\subsubsection{Mean waiting time} 
Upon closer inspection of the recursive formula a more compact formula can arise. 
The equivalent closed-form formula eliminates the need for recursion and thus makes 
the computation of waiting times much more efficient. 
Just like in the recursive part there are two formulas; one for \textit{class 1} 
and one for class 2 individuals. 
The formulas are given by:

\begin{equation} \label{eq:closed_form_waiting_others}
    W^{(1)} = \frac{\sum_{\substack{(u,v) \, \in S_A^{(1)} \\ v \geq C}} 
    \frac{1}{C \mu} \times (v-C+1) \times \pi(u,v)}{\sum_{(u,v) \, 
    \in S_A^{(1)}} \pi(u,v)}
\end{equation}
    
\begin{equation}\label{eq:closed_form_waiting_ambulance}
    W^{(2)} = \frac{\sum_{\substack{(u,v) \, \in S_A^{(2)} \\ min(v,T) \geq C}} 
    \frac{1}{C \mu} \times (\min(v+1,T)-C) \times \pi(u,v)}{\sum_{(u,v) \, 
    \in S_A^{(2)}} \pi(u,v)}
\end{equation}

Note here that the summation, in both equations \ref{eq:closed_form_waiting_others} 
and \ref{eq:closed_form_waiting_ambulance}, goes through all states in the set of 
accepting 
states of either class 1 or class 2 individuals respectively, where a wait 
incurs. 
In equation \ref{eq:closed_form_waiting_others} that includes all states \((u,v)\) 
in the set of accepting states of class 1 individuals such that \( v \geq C\); i.e. 
whenever an arrival occurs and the system is at a state where the number of individuals 
in the system is more than or equal to $C$. 
Consequently, for the states that are included in the summation the expression 
\( v-C+1 \) indicates the amount of people in service one would have to wait for 
upon arrival at the hospital.

Additionally, the minimisation function in equation 
\ref{eq:closed_form_waiting_ambulance} 
ensures that when a class 2 individual arrives at any state 
that is greater than the predetermined threshold, the wait that the individual will 
have to endure remains the same. 
In essence, the expression \(\min(v+1,T) - C\) returns the number of people in line 
in front of a particular individual upon arrival.


\subsubsection{Overall Waiting Time}

Consequently, the overall waiting time should can be estimated by a linear combination 
of the waiting times of class 1 and class 2 individuals. 
The overall waiting time can be then given by the following equation where \(c_1\) 
and \(c_2\) are the coefficients of each individual's type waiting time:

\begin{equation}\label{overall_waiting_time_coeff}
    W = c_1 W^{(1)} + c_2 W^{(2)}
\end{equation}

The two coefficients represent the proportion of individuals of each type that 
traversed through the model. 
Theoretically, getting these percentages should be as simple as looking at the arrival 
rates of each type but in practise if the service centre or the buffer centre 
is full, some individuals may be lost to the system. 
Thus, one should account for the probability that an individual is lost to the system. 
This probability can be easily calculated by using the two sets of accepting states 
\(S_A^{(2)}\) and \(S_A^{(1)}\) defined earlier in equations.
Let us define here the probability, for either class type, that an individual 
is not lost in the system by:

\begin{equation*}
    P(L'_1) = \sum_{(u,v) \, \in S_A^{(1)}} \pi(u,v) \hspace{2cm}
    P(L'_2) = \sum_{(u,v) \, \in S_A^{(2)}} \pi(u,v)
\end{equation*}

Having defined these probabilities one may combine them with the arrival rates of 
each class type in such a way to get the expected proportions of class 1 and 
class 2 individuals in the model. 
Thus, by using these values as the coefficient of equation 
\ref{overall_waiting_time_coeff} 
the resultant equation can be used to get the overall waiting time. 
Note here that the equation below gets the overall waiting time for both the recursive 
and the closed-form formula.

\begin{equation}\label{overall_waiting_time}
    W = \frac{\lambda_1 P(L'_1)}{\lambda_2 P(L'_2) + \lambda_1 P(L'_1)} W^{(1)} + 
    \frac{\lambda_2 P(L'_2)}{\lambda_2 P(L'_2) + \lambda_1 P(L'_1)} W^{(2)}
\end{equation}



\subsubsection{Mean blocking time}
Unlike the waiting time, the blocking time is only calculated for class 2 individuals.  
That is because class 1 individuals cannot be blocked. 
Thus, one only needs to consider the pathway of class 2 individuals to get the 
mean blocking time of the system. 
Blocking occurs at states \((u,v)\) where \(u > 0 \). 
Thus, the set of blocking states can be defined as:

\begin{equation*}
    S_b = \{(u,v) \in S \; | \; u > 0\}
\end{equation*}
 
In order to not consider individuals that will be lost to the system, the set of 
accepting states needs to be taken into account. The set of accepting states is given by:

\begin{equation*}
    S_A^{(2)}=
    \begin{cases}
        \{(u, v) \in S \; | \; u < M \} & \textbf{if } T \leq N\\
        \{(u, v) \in S \; | \; v < N \} & \textbf{otherwise}
    \end{cases}
\end{equation*}

For the waiting time formula,
the mean sojourn time for each state was considered,
ignoring any arrivals. Here, the same approach is used but ignoring only class 2
arrivals. That is because for the waiting time formula, once an individual enters 
the service centre (i.e. starts waiting) any individual arriving after them will 
not affect their
pathway. That is not the case for blocking time. When a class 2 individual is 
blocked, 
any class 1 individual that arrives will cause the blocked individual to remain 
blocked for more time. Therefore, class 1 arrivals are considered here:

\begin{equation}\label{eq:time_in_state_blocking_time}
    c(u,v) = 
    \begin{cases}
        \frac{1}{\min(v,C) \mu}, & \text{if } v = C\\
        \frac{1}{\min(v,C) \mu + \lambda_1}, & \text{otherwise}
    \end{cases}
\end{equation}
 
In equation \ref{eq:time_in_state_blocking_time}, both service completions and 
class 1 arrivals are considered. 
Thus, from a blocked individual's perspective whenever the system moves from one 
state \((u,v)\)
to another state it can either:

\begin{itemize}
    \item be because of a service being completed: we will denote the probability 
    of this happening by \(p_s(u,v)\). 
    \item be because of an arrival of an individual of class 1: denoting such 
    probability by \(p_o(u,v)\).
\end{itemize}
The probabilities are given by:

\begin{equation*}
    p_s(u,v) = \frac{\min(v,C)\mu}{\lambda_1 + \min(v,C)\mu}, \qquad
    p_o(u,v) = \frac{\lambda_1}{\lambda_1 + \min(v,C)\mu}
\end{equation*}


Having defined \(c(u,v)\) and \(S_b\) a formula for the blocking time that is
expected to occur at each state can be given by:

\begin{equation}\label{eq:blocking-time-at-each-state}
    b(u,v) = 
    \begin{cases} 
        0, & \textbf{if } (u,v) \notin S_b \\
        c(u,v) + b(u - 1, v), & \textbf{if } v = N = T\\
        c(u,v) + b(u, v-1), & \textbf{if } v = N \neq T \\
        c(u,v) + p_s(u,v) b(u-1, v) + p_o(u,v) b(u, v+1), & \textbf{if } u > 0 
        \textbf{ and } v = T \\
        c(u,v) + p_s(u,v) b(u, v-1) + p_o(u,v) b(u, v+1), & \textbf{otherwise} \\
    \end{cases}
\end{equation}

Equation 
(\ref{eq:blocking-time-at-each-state}) will not be solved recursively. 
A direct approach will be used to solve this equation here. 
By enumerating all equations of (\ref{eq:blocking-time-at-each-state}) for all 
states \((u,v)\) that belong in \(S_b\) 
a system of linear equations arises where the unknown variables are all the \(b(u,v)\)
terms.
For instance, let us consider a Markov model where \(C=2, T=3, N=6, M=2\). 
The Markov model is shown in Figure \ref{fig:example-algeb-blocking}
and the equivalent equations are 
(\ref{eq:first_eq_of_blocking_example})-(\ref{eq:last_eq_of_blocking_example}).
The equations considered here are only the ones that correspond to the blocking 
states.

\begin{multicols*}{2}
    \begin{figure}[H]
        \scalebox{0.50}{\input{MarkovChain/expressions_from_pi/example_model_2362/main.tex}}
        \caption{Example of Markov chain}
        \label{fig:example-algeb-blocking}
    \end{figure}
    \columnbreak
    \begin{align}
        b(1,2) &= c(1,2) + p_o b(1,3) \label{eq:first_eq_of_blocking_example} \\
        b(1,3) &= c(1,3) + p_s b(1,2) + p_o b(1,4) \\
        b(1,4) &= c(1,4) + b(1,3) \\
        b(2,2) &= c(2,2) + p_s b(1,2) + p_o b(2,3) \\
        b(2,3) &= c(2,3) + p_s b(2,2) + p_o b(1,4) \\
        b(2,4) &= c(2,4) + b(2,3)\label{eq:last_eq_of_blocking_example}
    \end{align}
\end{multicols*}

Additionally, the above equations can be transformed into a linear system of the 
form \(Zx=y\) where:

\begin{equation}\label{eq:example-algebaric-approach-blocking-time}
    Z=
    \begin{pmatrix}
        -1 & p_o & 0 & 0 & 0 & 0 \\ %(1,2)
        p_s & -1 & p_o & 0 & 0 & 0 \\ %(1,3)
        0 & 1 & -1 & 0 & 0 & 0 \\ %(1,4)
        p_s & 0 & 0 & -1 & p_o & 0\\ %(2,2)
        0 & 0 & 0 & p_s & -1 & p_o \\ %(2,3)
        0 & 0 & 0 & 0 & 1 & -1 \\ %(2,4)
    \end{pmatrix},
    x=
    \begin{pmatrix}
        b(1,2) \\
        b(1,3) \\
        b(1,4) \\
        b(2,2) \\
        b(2,3) \\
        b(2,4) \\
    \end{pmatrix}, 
    y=
    \begin{pmatrix}
        -c(1,2) \\
        -c(1,3) \\
        -c(1,4) \\
        -c(2,2) \\
        -c(2,3) \\
        -c(2,4) \\
    \end{pmatrix}
\end{equation}

A more generalised form of the equations in 
(\ref{eq:example-algebaric-approach-blocking-time})
can thus be given for any value of \(C,T,N,M\) by:

\begin{align}
    b(1,T) =& c(1, T) + p_o b(1, T + 1) \label{eq:first_eq_of_blocking_general}\\
    b(1,T + 1) =& c(1, T + 1) + p_s(1, T) + p_o b(1, T + 1) \\
    b(1,T + 2) =& c(1, T + 2) + p_s(1, T + 1) + p_o b(1, T + 3) \\
    & \vdots \nonumber \\
    b(1, N) =& c(1, N) + b(1, N - 1) \\
    b(2, T) =& c(2, T) + p_s b(1, T) + p_o b(2, T + 1) \\
    b(2, T + 1) =& c(2, T + 1) + p_s b(2, T) + p_o b(2, T + 2) \\
    & \vdots \nonumber \\
    b(M, T) =& c(M, T) + b(M, T-1) \label{eq:last_eq_of_blocking_general}
\end{align}

The equivalent matrix form of the linear system of equations 
(\ref{eq:first_eq_of_blocking_general}) - (\ref{eq:last_eq_of_blocking_general})
is given by \(Zx=y\), where:
\begin{equation}\label{eq:general-algebaric-approach-blocking-time}
    \scalebox{0.9}{
        \(
        Z = 
        \begin{pmatrix}
            -1 & p_o & 0 & \dots & 0 & 0 & 0 & 0 & 0 & \dots & 0 & 0 \\ %(1,T)
            p_s & -1 & p_o & \dots & 0 & 0 & 0 & 0 & 0 & \dots & 0 & 0 \\ %(1,T+1)
            0 & p_s & -1 & \dots & 0 & 0 & 0 & 0 & 0 & \dots & 0 & 0 \\ %(1,T+2)
            \vdots & \vdots & \vdots & \ddots & \vdots & \vdots & \vdots & \vdots & 
            \vdots & \ddots & \vdots & \vdots \\ 
            0 & 0 & 0 & \dots & 1 & -1 & 0 & 0 & 0 & \dots & 0 & 0 \\ %(1,N)
            p_s & 0 & 0 & \dots & 0 & 0 & -1 & p_o & 0 & \dots & 0 & 0 \\ %(2,T)
            0 & 0 & 0 & \dots & 0 & 0 & p_s & -1 & p_o & \dots & 0 & 0 \\ %(2,T+1)
            \vdots & \vdots & \vdots & \ddots & \vdots & \vdots & \vdots & \vdots & 
            \vdots & \ddots & \vdots & \vdots \\ 
            0 & 0 & 0 & \dots & 0 & 0 & 0 & 0 & 0 & \dots & 1 & -1 \\ %(M,T)
        \end{pmatrix},
        x = 
        \begin{pmatrix}
            b(1,T) \\
            b(1,T+1) \\
            b(1,T+2) \\
            \vdots \\
            b(1,N) \\
            b(2,T) \\
            b(2,T+1) \\
            \vdots \\
            b(M,T) \\
        \end{pmatrix}, 
        y= 
        \begin{pmatrix}
            -c(1,T) \\
            -c(1,T+1) \\
            -c(1,T+2) \\
            \vdots \\
            -c(1,N) \\
            -c(2,T) \\
            -c(2,T+1) \\
            \vdots \\
            -c(M,T) \\
        \end{pmatrix}
        \)
    }
\end{equation}

Thus, having calculated the mean blocking time for all blocking states \(b(u,v)\), 
it only remains to put them together in a formula.
The resultant blocking time formula is given by:

\begin{equation}\label{eq:algebraic-blocking-time}
    B = \frac{\sum_{(u,v) \in S_A} \pi_{(u,v)} \; b(u,v)}{\sum_{(u,v) \in S_A} 
    \pi_{(u,v)}}
\end{equation}

\documentclass{article}

\usepackage{amsmath}
\usepackage{amsfonts} 
\usepackage{geometry}
\usepackage{multicol}
\usepackage{float}
% \usepackage{mathtools}
% \usepackage{graphicx}
% \usepackage{soul}
% \usepackage{indentfirst}
\usepackage{tikz}
\usetikzlibrary{calc, automata, chains, arrows.meta, math}
\setcounter{MaxMatrixCols}{20}


\title{A game theoretic model of the behavioural gaming that takes place at the EMS - ED interface}

\author{
    Michalis Panayides, 
    Paul Harper, 
    Vince Knight
}

\begin{document}

\maketitle

\input{Abstract/main.tex}


\newpage
\tableofcontents

\newpage
\input{Introduction/main.tex}

\newpage
\input{Game_theory_component/main.tex}

\newpage
\input{MarkovChain/markov_chain_model/main.tex}
\input{MarkovChain/expressions_from_pi/main.tex}
\input{MarkovChain/markov_example/main.tex}

\newpage
\input{BehaviouralMethodology/main.tex}

\newpage
\input{Application_EMS_ED/main.tex}

\newpage
\input{Conclusion/main.tex}


\end{document}

\newpage
\documentclass{article}

\usepackage{amsmath}
\usepackage{amsfonts} 
\usepackage{geometry}
\usepackage{multicol}
\usepackage{float}
% \usepackage{mathtools}
% \usepackage{graphicx}
% \usepackage{soul}
% \usepackage{indentfirst}
\usepackage{tikz}
\usetikzlibrary{calc, automata, chains, arrows.meta, math}
\setcounter{MaxMatrixCols}{20}


\title{A game theoretic model of the behavioural gaming that takes place at the EMS - ED interface}

\author{
    Michalis Panayides, 
    Paul Harper, 
    Vince Knight
}

\begin{document}

\maketitle

\input{Abstract/main.tex}


\newpage
\tableofcontents

\newpage
\input{Introduction/main.tex}

\newpage
\input{Game_theory_component/main.tex}

\newpage
\input{MarkovChain/markov_chain_model/main.tex}
\input{MarkovChain/expressions_from_pi/main.tex}
\input{MarkovChain/markov_example/main.tex}

\newpage
\input{BehaviouralMethodology/main.tex}

\newpage
\input{Application_EMS_ED/main.tex}

\newpage
\input{Conclusion/main.tex}


\end{document}

\newpage
\section{EMS-ED application}

\subsection{Application}

\subsection{Data analysis of generated problem}

\newpage
\documentclass{article}

\usepackage{amsmath}
\usepackage{amsfonts} 
\usepackage{geometry}
\usepackage{multicol}
\usepackage{float}
% \usepackage{mathtools}
% \usepackage{graphicx}
% \usepackage{soul}
% \usepackage{indentfirst}
\usepackage{tikz}
\usetikzlibrary{calc, automata, chains, arrows.meta, math}
\setcounter{MaxMatrixCols}{20}


\title{A game theoretic model of the behavioural gaming that takes place at the EMS - ED interface}

\author{
    Michalis Panayides, 
    Paul Harper, 
    Vince Knight
}

\begin{document}

\maketitle

\input{Abstract/main.tex}


\newpage
\tableofcontents

\newpage
\input{Introduction/main.tex}

\newpage
\input{Game_theory_component/main.tex}

\newpage
\input{MarkovChain/markov_chain_model/main.tex}
\input{MarkovChain/expressions_from_pi/main.tex}
\input{MarkovChain/markov_example/main.tex}

\newpage
\input{BehaviouralMethodology/main.tex}

\newpage
\input{Application_EMS_ED/main.tex}

\newpage
\input{Conclusion/main.tex}


\end{document}


\end{document}

\newpage
\documentclass{article}

\usepackage{amsmath}
\usepackage{amsfonts} 
\usepackage{geometry}
\usepackage{multicol}
\usepackage{float}
% \usepackage{mathtools}
% \usepackage{graphicx}
% \usepackage{soul}
% \usepackage{indentfirst}
\usepackage{tikz}
\usetikzlibrary{calc, automata, chains, arrows.meta, math}
\setcounter{MaxMatrixCols}{20}


\title{A game theoretic model of the behavioural gaming that takes place at the EMS - ED interface}

\author{
    Michalis Panayides, 
    Paul Harper, 
    Vince Knight
}

\begin{document}

\maketitle

\documentclass{article}

\usepackage{amsmath}
\usepackage{amsfonts} 
\usepackage{geometry}
\usepackage{multicol}
\usepackage{float}
% \usepackage{mathtools}
% \usepackage{graphicx}
% \usepackage{soul}
% \usepackage{indentfirst}
\usepackage{tikz}
\usetikzlibrary{calc, automata, chains, arrows.meta, math}
\setcounter{MaxMatrixCols}{20}


\title{A game theoretic model of the behavioural gaming that takes place at the EMS - ED interface}

\author{
    Michalis Panayides, 
    Paul Harper, 
    Vince Knight
}

\begin{document}

\maketitle

\input{Abstract/main.tex}


\newpage
\tableofcontents

\newpage
\input{Introduction/main.tex}

\newpage
\input{Game_theory_component/main.tex}

\newpage
\input{MarkovChain/markov_chain_model/main.tex}
\input{MarkovChain/expressions_from_pi/main.tex}
\input{MarkovChain/markov_example/main.tex}

\newpage
\input{BehaviouralMethodology/main.tex}

\newpage
\input{Application_EMS_ED/main.tex}

\newpage
\input{Conclusion/main.tex}


\end{document}


\newpage
\tableofcontents

\newpage
\documentclass{article}

\usepackage{amsmath}
\usepackage{amsfonts} 
\usepackage{geometry}
\usepackage{multicol}
\usepackage{float}
% \usepackage{mathtools}
% \usepackage{graphicx}
% \usepackage{soul}
% \usepackage{indentfirst}
\usepackage{tikz}
\usetikzlibrary{calc, automata, chains, arrows.meta, math}
\setcounter{MaxMatrixCols}{20}


\title{A game theoretic model of the behavioural gaming that takes place at the EMS - ED interface}

\author{
    Michalis Panayides, 
    Paul Harper, 
    Vince Knight
}

\begin{document}

\maketitle

\input{Abstract/main.tex}


\newpage
\tableofcontents

\newpage
\input{Introduction/main.tex}

\newpage
\input{Game_theory_component/main.tex}

\newpage
\input{MarkovChain/markov_chain_model/main.tex}
\input{MarkovChain/expressions_from_pi/main.tex}
\input{MarkovChain/markov_example/main.tex}

\newpage
\input{BehaviouralMethodology/main.tex}

\newpage
\input{Application_EMS_ED/main.tex}

\newpage
\input{Conclusion/main.tex}


\end{document}

\newpage
\documentclass{article}

\usepackage{amsmath}
\usepackage{amsfonts} 
\usepackage{geometry}
\usepackage{multicol}
\usepackage{float}
% \usepackage{mathtools}
% \usepackage{graphicx}
% \usepackage{soul}
% \usepackage{indentfirst}
\usepackage{tikz}
\usetikzlibrary{calc, automata, chains, arrows.meta, math}
\setcounter{MaxMatrixCols}{20}


\title{A game theoretic model of the behavioural gaming that takes place at the EMS - ED interface}

\author{
    Michalis Panayides, 
    Paul Harper, 
    Vince Knight
}

\begin{document}

\maketitle

\input{Abstract/main.tex}


\newpage
\tableofcontents

\newpage
\input{Introduction/main.tex}

\newpage
\input{Game_theory_component/main.tex}

\newpage
\input{MarkovChain/markov_chain_model/main.tex}
\input{MarkovChain/expressions_from_pi/main.tex}
\input{MarkovChain/markov_example/main.tex}

\newpage
\input{BehaviouralMethodology/main.tex}

\newpage
\input{Application_EMS_ED/main.tex}

\newpage
\input{Conclusion/main.tex}


\end{document}

\newpage
\documentclass{article}

\usepackage{amsmath}
\usepackage{amsfonts} 
\usepackage{geometry}
\usepackage{multicol}
\usepackage{float}
% \usepackage{mathtools}
% \usepackage{graphicx}
% \usepackage{soul}
% \usepackage{indentfirst}
\usepackage{tikz}
\usetikzlibrary{calc, automata, chains, arrows.meta, math}
\setcounter{MaxMatrixCols}{20}


\title{A game theoretic model of the behavioural gaming that takes place at the EMS - ED interface}

\author{
    Michalis Panayides, 
    Paul Harper, 
    Vince Knight
}

\begin{document}

\maketitle

\input{Abstract/main.tex}


\newpage
\tableofcontents

\newpage
\input{Introduction/main.tex}

\newpage
\input{Game_theory_component/main.tex}

\newpage
\input{MarkovChain/markov_chain_model/main.tex}
\input{MarkovChain/expressions_from_pi/main.tex}
\input{MarkovChain/markov_example/main.tex}

\newpage
\input{BehaviouralMethodology/main.tex}

\newpage
\input{Application_EMS_ED/main.tex}

\newpage
\input{Conclusion/main.tex}


\end{document}
\subsection{Performance Measures}
One may easily derive the average number of individuals that are at any given state 
using \( pi \). 
The average number of individuals in state \( i \) can be calculated by multiplying 
the number of individuals that are present in state \( i \) with the probability 
of being at that particular state (i.e \(\pi_i (u_i + v_i)\)). 
Using this logic it is possible to calculate any performance measures that are related 
to the mean number of individuals in the system.


Average number of people in the system: 
\begin{equation}
    L = \sum_{i=1}^{|\pi|} \pi_i (u_i + v_i)
\end{equation} 

Average number of people in the service centre: 
\begin{equation}
    L_H = \sum_{i=1}^{|\pi|} \pi_i v_i
\end{equation}

Average number of people in the buffer centre:
\begin{equation}
    L_A = \sum_{i=1}^{|\pi|} \pi_i u_i
\end{equation}

Consequently getting the performance measures that are related to the duration of 
time is not as straightforward. 
Such performance measures are the mean waiting time in the system and the mean time 
blocked in the system. 
Under the scope of this study three approaches have been considered to calculate these 
performance measures; a direct approach, a recursive algorithm and consequently a
closed-form formula.

The research question that needs to be answered here is: ``When a class 1/2 
individuals enters the system, what is the expected time that they will have to 
wait?''. 
In order to formulate the answer to that question one needs to consider all possible 
scenarios of what state the system can be in when an individual arrives. 
Furthermore, different formulas arises for class 1 individuals 
and a different one for class 2 individuals.

\subsubsection{Mean waiting time} 
Upon closer inspection of the recursive formula a more compact formula can arise. 
The equivalent closed-form formula eliminates the need for recursion and thus makes 
the computation of waiting times much more efficient. 
Just like in the recursive part there are two formulas; one for \textit{class 1} 
and one for class 2 individuals. 
The formulas are given by:

\begin{equation} \label{eq:closed_form_waiting_others}
    W^{(1)} = \frac{\sum_{\substack{(u,v) \, \in S_A^{(1)} \\ v \geq C}} 
    \frac{1}{C \mu} \times (v-C+1) \times \pi(u,v)}{\sum_{(u,v) \, 
    \in S_A^{(1)}} \pi(u,v)}
\end{equation}
    
\begin{equation}\label{eq:closed_form_waiting_ambulance}
    W^{(2)} = \frac{\sum_{\substack{(u,v) \, \in S_A^{(2)} \\ min(v,T) \geq C}} 
    \frac{1}{C \mu} \times (\min(v+1,T)-C) \times \pi(u,v)}{\sum_{(u,v) \, 
    \in S_A^{(2)}} \pi(u,v)}
\end{equation}

Note here that the summation, in both equations \ref{eq:closed_form_waiting_others} 
and \ref{eq:closed_form_waiting_ambulance}, goes through all states in the set of 
accepting 
states of either class 1 or class 2 individuals respectively, where a wait 
incurs. 
In equation \ref{eq:closed_form_waiting_others} that includes all states \((u,v)\) 
in the set of accepting states of class 1 individuals such that \( v \geq C\); i.e. 
whenever an arrival occurs and the system is at a state where the number of individuals 
in the system is more than or equal to $C$. 
Consequently, for the states that are included in the summation the expression 
\( v-C+1 \) indicates the amount of people in service one would have to wait for 
upon arrival at the hospital.

Additionally, the minimisation function in equation 
\ref{eq:closed_form_waiting_ambulance} 
ensures that when a class 2 individual arrives at any state 
that is greater than the predetermined threshold, the wait that the individual will 
have to endure remains the same. 
In essence, the expression \(\min(v+1,T) - C\) returns the number of people in line 
in front of a particular individual upon arrival.


\subsubsection{Overall Waiting Time}

Consequently, the overall waiting time should can be estimated by a linear combination 
of the waiting times of class 1 and class 2 individuals. 
The overall waiting time can be then given by the following equation where \(c_1\) 
and \(c_2\) are the coefficients of each individual's type waiting time:

\begin{equation}\label{overall_waiting_time_coeff}
    W = c_1 W^{(1)} + c_2 W^{(2)}
\end{equation}

The two coefficients represent the proportion of individuals of each type that 
traversed through the model. 
Theoretically, getting these percentages should be as simple as looking at the arrival 
rates of each type but in practise if the service centre or the buffer centre 
is full, some individuals may be lost to the system. 
Thus, one should account for the probability that an individual is lost to the system. 
This probability can be easily calculated by using the two sets of accepting states 
\(S_A^{(2)}\) and \(S_A^{(1)}\) defined earlier in equations.
Let us define here the probability, for either class type, that an individual 
is not lost in the system by:

\begin{equation*}
    P(L'_1) = \sum_{(u,v) \, \in S_A^{(1)}} \pi(u,v) \hspace{2cm}
    P(L'_2) = \sum_{(u,v) \, \in S_A^{(2)}} \pi(u,v)
\end{equation*}

Having defined these probabilities one may combine them with the arrival rates of 
each class type in such a way to get the expected proportions of class 1 and 
class 2 individuals in the model. 
Thus, by using these values as the coefficient of equation 
\ref{overall_waiting_time_coeff} 
the resultant equation can be used to get the overall waiting time. 
Note here that the equation below gets the overall waiting time for both the recursive 
and the closed-form formula.

\begin{equation}\label{overall_waiting_time}
    W = \frac{\lambda_1 P(L'_1)}{\lambda_2 P(L'_2) + \lambda_1 P(L'_1)} W^{(1)} + 
    \frac{\lambda_2 P(L'_2)}{\lambda_2 P(L'_2) + \lambda_1 P(L'_1)} W^{(2)}
\end{equation}



\subsubsection{Mean blocking time}
Unlike the waiting time, the blocking time is only calculated for class 2 individuals.  
That is because class 1 individuals cannot be blocked. 
Thus, one only needs to consider the pathway of class 2 individuals to get the 
mean blocking time of the system. 
Blocking occurs at states \((u,v)\) where \(u > 0 \). 
Thus, the set of blocking states can be defined as:

\begin{equation*}
    S_b = \{(u,v) \in S \; | \; u > 0\}
\end{equation*}
 
In order to not consider individuals that will be lost to the system, the set of 
accepting states needs to be taken into account. The set of accepting states is given by:

\begin{equation*}
    S_A^{(2)}=
    \begin{cases}
        \{(u, v) \in S \; | \; u < M \} & \textbf{if } T \leq N\\
        \{(u, v) \in S \; | \; v < N \} & \textbf{otherwise}
    \end{cases}
\end{equation*}

For the waiting time formula,
the mean sojourn time for each state was considered,
ignoring any arrivals. Here, the same approach is used but ignoring only class 2
arrivals. That is because for the waiting time formula, once an individual enters 
the service centre (i.e. starts waiting) any individual arriving after them will 
not affect their
pathway. That is not the case for blocking time. When a class 2 individual is 
blocked, 
any class 1 individual that arrives will cause the blocked individual to remain 
blocked for more time. Therefore, class 1 arrivals are considered here:

\begin{equation}\label{eq:time_in_state_blocking_time}
    c(u,v) = 
    \begin{cases}
        \frac{1}{\min(v,C) \mu}, & \text{if } v = C\\
        \frac{1}{\min(v,C) \mu + \lambda_1}, & \text{otherwise}
    \end{cases}
\end{equation}
 
In equation \ref{eq:time_in_state_blocking_time}, both service completions and 
class 1 arrivals are considered. 
Thus, from a blocked individual's perspective whenever the system moves from one 
state \((u,v)\)
to another state it can either:

\begin{itemize}
    \item be because of a service being completed: we will denote the probability 
    of this happening by \(p_s(u,v)\). 
    \item be because of an arrival of an individual of class 1: denoting such 
    probability by \(p_o(u,v)\).
\end{itemize}
The probabilities are given by:

\begin{equation*}
    p_s(u,v) = \frac{\min(v,C)\mu}{\lambda_1 + \min(v,C)\mu}, \qquad
    p_o(u,v) = \frac{\lambda_1}{\lambda_1 + \min(v,C)\mu}
\end{equation*}


Having defined \(c(u,v)\) and \(S_b\) a formula for the blocking time that is
expected to occur at each state can be given by:

\begin{equation}\label{eq:blocking-time-at-each-state}
    b(u,v) = 
    \begin{cases} 
        0, & \textbf{if } (u,v) \notin S_b \\
        c(u,v) + b(u - 1, v), & \textbf{if } v = N = T\\
        c(u,v) + b(u, v-1), & \textbf{if } v = N \neq T \\
        c(u,v) + p_s(u,v) b(u-1, v) + p_o(u,v) b(u, v+1), & \textbf{if } u > 0 
        \textbf{ and } v = T \\
        c(u,v) + p_s(u,v) b(u, v-1) + p_o(u,v) b(u, v+1), & \textbf{otherwise} \\
    \end{cases}
\end{equation}

Equation 
(\ref{eq:blocking-time-at-each-state}) will not be solved recursively. 
A direct approach will be used to solve this equation here. 
By enumerating all equations of (\ref{eq:blocking-time-at-each-state}) for all 
states \((u,v)\) that belong in \(S_b\) 
a system of linear equations arises where the unknown variables are all the \(b(u,v)\)
terms.
For instance, let us consider a Markov model where \(C=2, T=3, N=6, M=2\). 
The Markov model is shown in Figure \ref{fig:example-algeb-blocking}
and the equivalent equations are 
(\ref{eq:first_eq_of_blocking_example})-(\ref{eq:last_eq_of_blocking_example}).
The equations considered here are only the ones that correspond to the blocking 
states.

\begin{multicols*}{2}
    \begin{figure}[H]
        \scalebox{0.50}{\input{MarkovChain/expressions_from_pi/example_model_2362/main.tex}}
        \caption{Example of Markov chain}
        \label{fig:example-algeb-blocking}
    \end{figure}
    \columnbreak
    \begin{align}
        b(1,2) &= c(1,2) + p_o b(1,3) \label{eq:first_eq_of_blocking_example} \\
        b(1,3) &= c(1,3) + p_s b(1,2) + p_o b(1,4) \\
        b(1,4) &= c(1,4) + b(1,3) \\
        b(2,2) &= c(2,2) + p_s b(1,2) + p_o b(2,3) \\
        b(2,3) &= c(2,3) + p_s b(2,2) + p_o b(1,4) \\
        b(2,4) &= c(2,4) + b(2,3)\label{eq:last_eq_of_blocking_example}
    \end{align}
\end{multicols*}

Additionally, the above equations can be transformed into a linear system of the 
form \(Zx=y\) where:

\begin{equation}\label{eq:example-algebaric-approach-blocking-time}
    Z=
    \begin{pmatrix}
        -1 & p_o & 0 & 0 & 0 & 0 \\ %(1,2)
        p_s & -1 & p_o & 0 & 0 & 0 \\ %(1,3)
        0 & 1 & -1 & 0 & 0 & 0 \\ %(1,4)
        p_s & 0 & 0 & -1 & p_o & 0\\ %(2,2)
        0 & 0 & 0 & p_s & -1 & p_o \\ %(2,3)
        0 & 0 & 0 & 0 & 1 & -1 \\ %(2,4)
    \end{pmatrix},
    x=
    \begin{pmatrix}
        b(1,2) \\
        b(1,3) \\
        b(1,4) \\
        b(2,2) \\
        b(2,3) \\
        b(2,4) \\
    \end{pmatrix}, 
    y=
    \begin{pmatrix}
        -c(1,2) \\
        -c(1,3) \\
        -c(1,4) \\
        -c(2,2) \\
        -c(2,3) \\
        -c(2,4) \\
    \end{pmatrix}
\end{equation}

A more generalised form of the equations in 
(\ref{eq:example-algebaric-approach-blocking-time})
can thus be given for any value of \(C,T,N,M\) by:

\begin{align}
    b(1,T) =& c(1, T) + p_o b(1, T + 1) \label{eq:first_eq_of_blocking_general}\\
    b(1,T + 1) =& c(1, T + 1) + p_s(1, T) + p_o b(1, T + 1) \\
    b(1,T + 2) =& c(1, T + 2) + p_s(1, T + 1) + p_o b(1, T + 3) \\
    & \vdots \nonumber \\
    b(1, N) =& c(1, N) + b(1, N - 1) \\
    b(2, T) =& c(2, T) + p_s b(1, T) + p_o b(2, T + 1) \\
    b(2, T + 1) =& c(2, T + 1) + p_s b(2, T) + p_o b(2, T + 2) \\
    & \vdots \nonumber \\
    b(M, T) =& c(M, T) + b(M, T-1) \label{eq:last_eq_of_blocking_general}
\end{align}

The equivalent matrix form of the linear system of equations 
(\ref{eq:first_eq_of_blocking_general}) - (\ref{eq:last_eq_of_blocking_general})
is given by \(Zx=y\), where:
\begin{equation}\label{eq:general-algebaric-approach-blocking-time}
    \scalebox{0.9}{
        \(
        Z = 
        \begin{pmatrix}
            -1 & p_o & 0 & \dots & 0 & 0 & 0 & 0 & 0 & \dots & 0 & 0 \\ %(1,T)
            p_s & -1 & p_o & \dots & 0 & 0 & 0 & 0 & 0 & \dots & 0 & 0 \\ %(1,T+1)
            0 & p_s & -1 & \dots & 0 & 0 & 0 & 0 & 0 & \dots & 0 & 0 \\ %(1,T+2)
            \vdots & \vdots & \vdots & \ddots & \vdots & \vdots & \vdots & \vdots & 
            \vdots & \ddots & \vdots & \vdots \\ 
            0 & 0 & 0 & \dots & 1 & -1 & 0 & 0 & 0 & \dots & 0 & 0 \\ %(1,N)
            p_s & 0 & 0 & \dots & 0 & 0 & -1 & p_o & 0 & \dots & 0 & 0 \\ %(2,T)
            0 & 0 & 0 & \dots & 0 & 0 & p_s & -1 & p_o & \dots & 0 & 0 \\ %(2,T+1)
            \vdots & \vdots & \vdots & \ddots & \vdots & \vdots & \vdots & \vdots & 
            \vdots & \ddots & \vdots & \vdots \\ 
            0 & 0 & 0 & \dots & 0 & 0 & 0 & 0 & 0 & \dots & 1 & -1 \\ %(M,T)
        \end{pmatrix},
        x = 
        \begin{pmatrix}
            b(1,T) \\
            b(1,T+1) \\
            b(1,T+2) \\
            \vdots \\
            b(1,N) \\
            b(2,T) \\
            b(2,T+1) \\
            \vdots \\
            b(M,T) \\
        \end{pmatrix}, 
        y= 
        \begin{pmatrix}
            -c(1,T) \\
            -c(1,T+1) \\
            -c(1,T+2) \\
            \vdots \\
            -c(1,N) \\
            -c(2,T) \\
            -c(2,T+1) \\
            \vdots \\
            -c(M,T) \\
        \end{pmatrix}
        \)
    }
\end{equation}

Thus, having calculated the mean blocking time for all blocking states \(b(u,v)\), 
it only remains to put them together in a formula.
The resultant blocking time formula is given by:

\begin{equation}\label{eq:algebraic-blocking-time}
    B = \frac{\sum_{(u,v) \in S_A} \pi_{(u,v)} \; b(u,v)}{\sum_{(u,v) \in S_A} 
    \pi_{(u,v)}}
\end{equation}

\documentclass{article}

\usepackage{amsmath}
\usepackage{amsfonts} 
\usepackage{geometry}
\usepackage{multicol}
\usepackage{float}
% \usepackage{mathtools}
% \usepackage{graphicx}
% \usepackage{soul}
% \usepackage{indentfirst}
\usepackage{tikz}
\usetikzlibrary{calc, automata, chains, arrows.meta, math}
\setcounter{MaxMatrixCols}{20}


\title{A game theoretic model of the behavioural gaming that takes place at the EMS - ED interface}

\author{
    Michalis Panayides, 
    Paul Harper, 
    Vince Knight
}

\begin{document}

\maketitle

\input{Abstract/main.tex}


\newpage
\tableofcontents

\newpage
\input{Introduction/main.tex}

\newpage
\input{Game_theory_component/main.tex}

\newpage
\input{MarkovChain/markov_chain_model/main.tex}
\input{MarkovChain/expressions_from_pi/main.tex}
\input{MarkovChain/markov_example/main.tex}

\newpage
\input{BehaviouralMethodology/main.tex}

\newpage
\input{Application_EMS_ED/main.tex}

\newpage
\input{Conclusion/main.tex}


\end{document}

\newpage
\documentclass{article}

\usepackage{amsmath}
\usepackage{amsfonts} 
\usepackage{geometry}
\usepackage{multicol}
\usepackage{float}
% \usepackage{mathtools}
% \usepackage{graphicx}
% \usepackage{soul}
% \usepackage{indentfirst}
\usepackage{tikz}
\usetikzlibrary{calc, automata, chains, arrows.meta, math}
\setcounter{MaxMatrixCols}{20}


\title{A game theoretic model of the behavioural gaming that takes place at the EMS - ED interface}

\author{
    Michalis Panayides, 
    Paul Harper, 
    Vince Knight
}

\begin{document}

\maketitle

\input{Abstract/main.tex}


\newpage
\tableofcontents

\newpage
\input{Introduction/main.tex}

\newpage
\input{Game_theory_component/main.tex}

\newpage
\input{MarkovChain/markov_chain_model/main.tex}
\input{MarkovChain/expressions_from_pi/main.tex}
\input{MarkovChain/markov_example/main.tex}

\newpage
\input{BehaviouralMethodology/main.tex}

\newpage
\input{Application_EMS_ED/main.tex}

\newpage
\input{Conclusion/main.tex}


\end{document}

\newpage
\section{EMS-ED application}

\subsection{Application}

\subsection{Data analysis of generated problem}

\newpage
\documentclass{article}

\usepackage{amsmath}
\usepackage{amsfonts} 
\usepackage{geometry}
\usepackage{multicol}
\usepackage{float}
% \usepackage{mathtools}
% \usepackage{graphicx}
% \usepackage{soul}
% \usepackage{indentfirst}
\usepackage{tikz}
\usetikzlibrary{calc, automata, chains, arrows.meta, math}
\setcounter{MaxMatrixCols}{20}


\title{A game theoretic model of the behavioural gaming that takes place at the EMS - ED interface}

\author{
    Michalis Panayides, 
    Paul Harper, 
    Vince Knight
}

\begin{document}

\maketitle

\input{Abstract/main.tex}


\newpage
\tableofcontents

\newpage
\input{Introduction/main.tex}

\newpage
\input{Game_theory_component/main.tex}

\newpage
\input{MarkovChain/markov_chain_model/main.tex}
\input{MarkovChain/expressions_from_pi/main.tex}
\input{MarkovChain/markov_example/main.tex}

\newpage
\input{BehaviouralMethodology/main.tex}

\newpage
\input{Application_EMS_ED/main.tex}

\newpage
\input{Conclusion/main.tex}


\end{document}


\end{document}

\newpage
\section{EMS-ED application}

\subsection{Application}

\subsection{Data analysis of generated problem}

\newpage
\documentclass{article}

\usepackage{amsmath}
\usepackage{amsfonts} 
\usepackage{geometry}
\usepackage{multicol}
\usepackage{float}
% \usepackage{mathtools}
% \usepackage{graphicx}
% \usepackage{soul}
% \usepackage{indentfirst}
\usepackage{tikz}
\usetikzlibrary{calc, automata, chains, arrows.meta, math}
\setcounter{MaxMatrixCols}{20}


\title{A game theoretic model of the behavioural gaming that takes place at the EMS - ED interface}

\author{
    Michalis Panayides, 
    Paul Harper, 
    Vince Knight
}

\begin{document}

\maketitle

\documentclass{article}

\usepackage{amsmath}
\usepackage{amsfonts} 
\usepackage{geometry}
\usepackage{multicol}
\usepackage{float}
% \usepackage{mathtools}
% \usepackage{graphicx}
% \usepackage{soul}
% \usepackage{indentfirst}
\usepackage{tikz}
\usetikzlibrary{calc, automata, chains, arrows.meta, math}
\setcounter{MaxMatrixCols}{20}


\title{A game theoretic model of the behavioural gaming that takes place at the EMS - ED interface}

\author{
    Michalis Panayides, 
    Paul Harper, 
    Vince Knight
}

\begin{document}

\maketitle

\input{Abstract/main.tex}


\newpage
\tableofcontents

\newpage
\input{Introduction/main.tex}

\newpage
\input{Game_theory_component/main.tex}

\newpage
\input{MarkovChain/markov_chain_model/main.tex}
\input{MarkovChain/expressions_from_pi/main.tex}
\input{MarkovChain/markov_example/main.tex}

\newpage
\input{BehaviouralMethodology/main.tex}

\newpage
\input{Application_EMS_ED/main.tex}

\newpage
\input{Conclusion/main.tex}


\end{document}


\newpage
\tableofcontents

\newpage
\documentclass{article}

\usepackage{amsmath}
\usepackage{amsfonts} 
\usepackage{geometry}
\usepackage{multicol}
\usepackage{float}
% \usepackage{mathtools}
% \usepackage{graphicx}
% \usepackage{soul}
% \usepackage{indentfirst}
\usepackage{tikz}
\usetikzlibrary{calc, automata, chains, arrows.meta, math}
\setcounter{MaxMatrixCols}{20}


\title{A game theoretic model of the behavioural gaming that takes place at the EMS - ED interface}

\author{
    Michalis Panayides, 
    Paul Harper, 
    Vince Knight
}

\begin{document}

\maketitle

\input{Abstract/main.tex}


\newpage
\tableofcontents

\newpage
\input{Introduction/main.tex}

\newpage
\input{Game_theory_component/main.tex}

\newpage
\input{MarkovChain/markov_chain_model/main.tex}
\input{MarkovChain/expressions_from_pi/main.tex}
\input{MarkovChain/markov_example/main.tex}

\newpage
\input{BehaviouralMethodology/main.tex}

\newpage
\input{Application_EMS_ED/main.tex}

\newpage
\input{Conclusion/main.tex}


\end{document}

\newpage
\documentclass{article}

\usepackage{amsmath}
\usepackage{amsfonts} 
\usepackage{geometry}
\usepackage{multicol}
\usepackage{float}
% \usepackage{mathtools}
% \usepackage{graphicx}
% \usepackage{soul}
% \usepackage{indentfirst}
\usepackage{tikz}
\usetikzlibrary{calc, automata, chains, arrows.meta, math}
\setcounter{MaxMatrixCols}{20}


\title{A game theoretic model of the behavioural gaming that takes place at the EMS - ED interface}

\author{
    Michalis Panayides, 
    Paul Harper, 
    Vince Knight
}

\begin{document}

\maketitle

\input{Abstract/main.tex}


\newpage
\tableofcontents

\newpage
\input{Introduction/main.tex}

\newpage
\input{Game_theory_component/main.tex}

\newpage
\input{MarkovChain/markov_chain_model/main.tex}
\input{MarkovChain/expressions_from_pi/main.tex}
\input{MarkovChain/markov_example/main.tex}

\newpage
\input{BehaviouralMethodology/main.tex}

\newpage
\input{Application_EMS_ED/main.tex}

\newpage
\input{Conclusion/main.tex}


\end{document}

\newpage
\documentclass{article}

\usepackage{amsmath}
\usepackage{amsfonts} 
\usepackage{geometry}
\usepackage{multicol}
\usepackage{float}
% \usepackage{mathtools}
% \usepackage{graphicx}
% \usepackage{soul}
% \usepackage{indentfirst}
\usepackage{tikz}
\usetikzlibrary{calc, automata, chains, arrows.meta, math}
\setcounter{MaxMatrixCols}{20}


\title{A game theoretic model of the behavioural gaming that takes place at the EMS - ED interface}

\author{
    Michalis Panayides, 
    Paul Harper, 
    Vince Knight
}

\begin{document}

\maketitle

\input{Abstract/main.tex}


\newpage
\tableofcontents

\newpage
\input{Introduction/main.tex}

\newpage
\input{Game_theory_component/main.tex}

\newpage
\input{MarkovChain/markov_chain_model/main.tex}
\input{MarkovChain/expressions_from_pi/main.tex}
\input{MarkovChain/markov_example/main.tex}

\newpage
\input{BehaviouralMethodology/main.tex}

\newpage
\input{Application_EMS_ED/main.tex}

\newpage
\input{Conclusion/main.tex}


\end{document}
\subsection{Performance Measures}
One may easily derive the average number of individuals that are at any given state 
using \( pi \). 
The average number of individuals in state \( i \) can be calculated by multiplying 
the number of individuals that are present in state \( i \) with the probability 
of being at that particular state (i.e \(\pi_i (u_i + v_i)\)). 
Using this logic it is possible to calculate any performance measures that are related 
to the mean number of individuals in the system.


Average number of people in the system: 
\begin{equation}
    L = \sum_{i=1}^{|\pi|} \pi_i (u_i + v_i)
\end{equation} 

Average number of people in the service centre: 
\begin{equation}
    L_H = \sum_{i=1}^{|\pi|} \pi_i v_i
\end{equation}

Average number of people in the buffer centre:
\begin{equation}
    L_A = \sum_{i=1}^{|\pi|} \pi_i u_i
\end{equation}

Consequently getting the performance measures that are related to the duration of 
time is not as straightforward. 
Such performance measures are the mean waiting time in the system and the mean time 
blocked in the system. 
Under the scope of this study three approaches have been considered to calculate these 
performance measures; a direct approach, a recursive algorithm and consequently a
closed-form formula.

The research question that needs to be answered here is: ``When a class 1/2 
individuals enters the system, what is the expected time that they will have to 
wait?''. 
In order to formulate the answer to that question one needs to consider all possible 
scenarios of what state the system can be in when an individual arrives. 
Furthermore, different formulas arises for class 1 individuals 
and a different one for class 2 individuals.

\subsubsection{Mean waiting time} 
Upon closer inspection of the recursive formula a more compact formula can arise. 
The equivalent closed-form formula eliminates the need for recursion and thus makes 
the computation of waiting times much more efficient. 
Just like in the recursive part there are two formulas; one for \textit{class 1} 
and one for class 2 individuals. 
The formulas are given by:

\begin{equation} \label{eq:closed_form_waiting_others}
    W^{(1)} = \frac{\sum_{\substack{(u,v) \, \in S_A^{(1)} \\ v \geq C}} 
    \frac{1}{C \mu} \times (v-C+1) \times \pi(u,v)}{\sum_{(u,v) \, 
    \in S_A^{(1)}} \pi(u,v)}
\end{equation}
    
\begin{equation}\label{eq:closed_form_waiting_ambulance}
    W^{(2)} = \frac{\sum_{\substack{(u,v) \, \in S_A^{(2)} \\ min(v,T) \geq C}} 
    \frac{1}{C \mu} \times (\min(v+1,T)-C) \times \pi(u,v)}{\sum_{(u,v) \, 
    \in S_A^{(2)}} \pi(u,v)}
\end{equation}

Note here that the summation, in both equations \ref{eq:closed_form_waiting_others} 
and \ref{eq:closed_form_waiting_ambulance}, goes through all states in the set of 
accepting 
states of either class 1 or class 2 individuals respectively, where a wait 
incurs. 
In equation \ref{eq:closed_form_waiting_others} that includes all states \((u,v)\) 
in the set of accepting states of class 1 individuals such that \( v \geq C\); i.e. 
whenever an arrival occurs and the system is at a state where the number of individuals 
in the system is more than or equal to $C$. 
Consequently, for the states that are included in the summation the expression 
\( v-C+1 \) indicates the amount of people in service one would have to wait for 
upon arrival at the hospital.

Additionally, the minimisation function in equation 
\ref{eq:closed_form_waiting_ambulance} 
ensures that when a class 2 individual arrives at any state 
that is greater than the predetermined threshold, the wait that the individual will 
have to endure remains the same. 
In essence, the expression \(\min(v+1,T) - C\) returns the number of people in line 
in front of a particular individual upon arrival.


\subsubsection{Overall Waiting Time}

Consequently, the overall waiting time should can be estimated by a linear combination 
of the waiting times of class 1 and class 2 individuals. 
The overall waiting time can be then given by the following equation where \(c_1\) 
and \(c_2\) are the coefficients of each individual's type waiting time:

\begin{equation}\label{overall_waiting_time_coeff}
    W = c_1 W^{(1)} + c_2 W^{(2)}
\end{equation}

The two coefficients represent the proportion of individuals of each type that 
traversed through the model. 
Theoretically, getting these percentages should be as simple as looking at the arrival 
rates of each type but in practise if the service centre or the buffer centre 
is full, some individuals may be lost to the system. 
Thus, one should account for the probability that an individual is lost to the system. 
This probability can be easily calculated by using the two sets of accepting states 
\(S_A^{(2)}\) and \(S_A^{(1)}\) defined earlier in equations.
Let us define here the probability, for either class type, that an individual 
is not lost in the system by:

\begin{equation*}
    P(L'_1) = \sum_{(u,v) \, \in S_A^{(1)}} \pi(u,v) \hspace{2cm}
    P(L'_2) = \sum_{(u,v) \, \in S_A^{(2)}} \pi(u,v)
\end{equation*}

Having defined these probabilities one may combine them with the arrival rates of 
each class type in such a way to get the expected proportions of class 1 and 
class 2 individuals in the model. 
Thus, by using these values as the coefficient of equation 
\ref{overall_waiting_time_coeff} 
the resultant equation can be used to get the overall waiting time. 
Note here that the equation below gets the overall waiting time for both the recursive 
and the closed-form formula.

\begin{equation}\label{overall_waiting_time}
    W = \frac{\lambda_1 P(L'_1)}{\lambda_2 P(L'_2) + \lambda_1 P(L'_1)} W^{(1)} + 
    \frac{\lambda_2 P(L'_2)}{\lambda_2 P(L'_2) + \lambda_1 P(L'_1)} W^{(2)}
\end{equation}



\subsubsection{Mean blocking time}
Unlike the waiting time, the blocking time is only calculated for class 2 individuals.  
That is because class 1 individuals cannot be blocked. 
Thus, one only needs to consider the pathway of class 2 individuals to get the 
mean blocking time of the system. 
Blocking occurs at states \((u,v)\) where \(u > 0 \). 
Thus, the set of blocking states can be defined as:

\begin{equation*}
    S_b = \{(u,v) \in S \; | \; u > 0\}
\end{equation*}
 
In order to not consider individuals that will be lost to the system, the set of 
accepting states needs to be taken into account. The set of accepting states is given by:

\begin{equation*}
    S_A^{(2)}=
    \begin{cases}
        \{(u, v) \in S \; | \; u < M \} & \textbf{if } T \leq N\\
        \{(u, v) \in S \; | \; v < N \} & \textbf{otherwise}
    \end{cases}
\end{equation*}

For the waiting time formula,
the mean sojourn time for each state was considered,
ignoring any arrivals. Here, the same approach is used but ignoring only class 2
arrivals. That is because for the waiting time formula, once an individual enters 
the service centre (i.e. starts waiting) any individual arriving after them will 
not affect their
pathway. That is not the case for blocking time. When a class 2 individual is 
blocked, 
any class 1 individual that arrives will cause the blocked individual to remain 
blocked for more time. Therefore, class 1 arrivals are considered here:

\begin{equation}\label{eq:time_in_state_blocking_time}
    c(u,v) = 
    \begin{cases}
        \frac{1}{\min(v,C) \mu}, & \text{if } v = C\\
        \frac{1}{\min(v,C) \mu + \lambda_1}, & \text{otherwise}
    \end{cases}
\end{equation}
 
In equation \ref{eq:time_in_state_blocking_time}, both service completions and 
class 1 arrivals are considered. 
Thus, from a blocked individual's perspective whenever the system moves from one 
state \((u,v)\)
to another state it can either:

\begin{itemize}
    \item be because of a service being completed: we will denote the probability 
    of this happening by \(p_s(u,v)\). 
    \item be because of an arrival of an individual of class 1: denoting such 
    probability by \(p_o(u,v)\).
\end{itemize}
The probabilities are given by:

\begin{equation*}
    p_s(u,v) = \frac{\min(v,C)\mu}{\lambda_1 + \min(v,C)\mu}, \qquad
    p_o(u,v) = \frac{\lambda_1}{\lambda_1 + \min(v,C)\mu}
\end{equation*}


Having defined \(c(u,v)\) and \(S_b\) a formula for the blocking time that is
expected to occur at each state can be given by:

\begin{equation}\label{eq:blocking-time-at-each-state}
    b(u,v) = 
    \begin{cases} 
        0, & \textbf{if } (u,v) \notin S_b \\
        c(u,v) + b(u - 1, v), & \textbf{if } v = N = T\\
        c(u,v) + b(u, v-1), & \textbf{if } v = N \neq T \\
        c(u,v) + p_s(u,v) b(u-1, v) + p_o(u,v) b(u, v+1), & \textbf{if } u > 0 
        \textbf{ and } v = T \\
        c(u,v) + p_s(u,v) b(u, v-1) + p_o(u,v) b(u, v+1), & \textbf{otherwise} \\
    \end{cases}
\end{equation}

Equation 
(\ref{eq:blocking-time-at-each-state}) will not be solved recursively. 
A direct approach will be used to solve this equation here. 
By enumerating all equations of (\ref{eq:blocking-time-at-each-state}) for all 
states \((u,v)\) that belong in \(S_b\) 
a system of linear equations arises where the unknown variables are all the \(b(u,v)\)
terms.
For instance, let us consider a Markov model where \(C=2, T=3, N=6, M=2\). 
The Markov model is shown in Figure \ref{fig:example-algeb-blocking}
and the equivalent equations are 
(\ref{eq:first_eq_of_blocking_example})-(\ref{eq:last_eq_of_blocking_example}).
The equations considered here are only the ones that correspond to the blocking 
states.

\begin{multicols*}{2}
    \begin{figure}[H]
        \scalebox{0.50}{\input{MarkovChain/expressions_from_pi/example_model_2362/main.tex}}
        \caption{Example of Markov chain}
        \label{fig:example-algeb-blocking}
    \end{figure}
    \columnbreak
    \begin{align}
        b(1,2) &= c(1,2) + p_o b(1,3) \label{eq:first_eq_of_blocking_example} \\
        b(1,3) &= c(1,3) + p_s b(1,2) + p_o b(1,4) \\
        b(1,4) &= c(1,4) + b(1,3) \\
        b(2,2) &= c(2,2) + p_s b(1,2) + p_o b(2,3) \\
        b(2,3) &= c(2,3) + p_s b(2,2) + p_o b(1,4) \\
        b(2,4) &= c(2,4) + b(2,3)\label{eq:last_eq_of_blocking_example}
    \end{align}
\end{multicols*}

Additionally, the above equations can be transformed into a linear system of the 
form \(Zx=y\) where:

\begin{equation}\label{eq:example-algebaric-approach-blocking-time}
    Z=
    \begin{pmatrix}
        -1 & p_o & 0 & 0 & 0 & 0 \\ %(1,2)
        p_s & -1 & p_o & 0 & 0 & 0 \\ %(1,3)
        0 & 1 & -1 & 0 & 0 & 0 \\ %(1,4)
        p_s & 0 & 0 & -1 & p_o & 0\\ %(2,2)
        0 & 0 & 0 & p_s & -1 & p_o \\ %(2,3)
        0 & 0 & 0 & 0 & 1 & -1 \\ %(2,4)
    \end{pmatrix},
    x=
    \begin{pmatrix}
        b(1,2) \\
        b(1,3) \\
        b(1,4) \\
        b(2,2) \\
        b(2,3) \\
        b(2,4) \\
    \end{pmatrix}, 
    y=
    \begin{pmatrix}
        -c(1,2) \\
        -c(1,3) \\
        -c(1,4) \\
        -c(2,2) \\
        -c(2,3) \\
        -c(2,4) \\
    \end{pmatrix}
\end{equation}

A more generalised form of the equations in 
(\ref{eq:example-algebaric-approach-blocking-time})
can thus be given for any value of \(C,T,N,M\) by:

\begin{align}
    b(1,T) =& c(1, T) + p_o b(1, T + 1) \label{eq:first_eq_of_blocking_general}\\
    b(1,T + 1) =& c(1, T + 1) + p_s(1, T) + p_o b(1, T + 1) \\
    b(1,T + 2) =& c(1, T + 2) + p_s(1, T + 1) + p_o b(1, T + 3) \\
    & \vdots \nonumber \\
    b(1, N) =& c(1, N) + b(1, N - 1) \\
    b(2, T) =& c(2, T) + p_s b(1, T) + p_o b(2, T + 1) \\
    b(2, T + 1) =& c(2, T + 1) + p_s b(2, T) + p_o b(2, T + 2) \\
    & \vdots \nonumber \\
    b(M, T) =& c(M, T) + b(M, T-1) \label{eq:last_eq_of_blocking_general}
\end{align}

The equivalent matrix form of the linear system of equations 
(\ref{eq:first_eq_of_blocking_general}) - (\ref{eq:last_eq_of_blocking_general})
is given by \(Zx=y\), where:
\begin{equation}\label{eq:general-algebaric-approach-blocking-time}
    \scalebox{0.9}{
        \(
        Z = 
        \begin{pmatrix}
            -1 & p_o & 0 & \dots & 0 & 0 & 0 & 0 & 0 & \dots & 0 & 0 \\ %(1,T)
            p_s & -1 & p_o & \dots & 0 & 0 & 0 & 0 & 0 & \dots & 0 & 0 \\ %(1,T+1)
            0 & p_s & -1 & \dots & 0 & 0 & 0 & 0 & 0 & \dots & 0 & 0 \\ %(1,T+2)
            \vdots & \vdots & \vdots & \ddots & \vdots & \vdots & \vdots & \vdots & 
            \vdots & \ddots & \vdots & \vdots \\ 
            0 & 0 & 0 & \dots & 1 & -1 & 0 & 0 & 0 & \dots & 0 & 0 \\ %(1,N)
            p_s & 0 & 0 & \dots & 0 & 0 & -1 & p_o & 0 & \dots & 0 & 0 \\ %(2,T)
            0 & 0 & 0 & \dots & 0 & 0 & p_s & -1 & p_o & \dots & 0 & 0 \\ %(2,T+1)
            \vdots & \vdots & \vdots & \ddots & \vdots & \vdots & \vdots & \vdots & 
            \vdots & \ddots & \vdots & \vdots \\ 
            0 & 0 & 0 & \dots & 0 & 0 & 0 & 0 & 0 & \dots & 1 & -1 \\ %(M,T)
        \end{pmatrix},
        x = 
        \begin{pmatrix}
            b(1,T) \\
            b(1,T+1) \\
            b(1,T+2) \\
            \vdots \\
            b(1,N) \\
            b(2,T) \\
            b(2,T+1) \\
            \vdots \\
            b(M,T) \\
        \end{pmatrix}, 
        y= 
        \begin{pmatrix}
            -c(1,T) \\
            -c(1,T+1) \\
            -c(1,T+2) \\
            \vdots \\
            -c(1,N) \\
            -c(2,T) \\
            -c(2,T+1) \\
            \vdots \\
            -c(M,T) \\
        \end{pmatrix}
        \)
    }
\end{equation}

Thus, having calculated the mean blocking time for all blocking states \(b(u,v)\), 
it only remains to put them together in a formula.
The resultant blocking time formula is given by:

\begin{equation}\label{eq:algebraic-blocking-time}
    B = \frac{\sum_{(u,v) \in S_A} \pi_{(u,v)} \; b(u,v)}{\sum_{(u,v) \in S_A} 
    \pi_{(u,v)}}
\end{equation}

\documentclass{article}

\usepackage{amsmath}
\usepackage{amsfonts} 
\usepackage{geometry}
\usepackage{multicol}
\usepackage{float}
% \usepackage{mathtools}
% \usepackage{graphicx}
% \usepackage{soul}
% \usepackage{indentfirst}
\usepackage{tikz}
\usetikzlibrary{calc, automata, chains, arrows.meta, math}
\setcounter{MaxMatrixCols}{20}


\title{A game theoretic model of the behavioural gaming that takes place at the EMS - ED interface}

\author{
    Michalis Panayides, 
    Paul Harper, 
    Vince Knight
}

\begin{document}

\maketitle

\input{Abstract/main.tex}


\newpage
\tableofcontents

\newpage
\input{Introduction/main.tex}

\newpage
\input{Game_theory_component/main.tex}

\newpage
\input{MarkovChain/markov_chain_model/main.tex}
\input{MarkovChain/expressions_from_pi/main.tex}
\input{MarkovChain/markov_example/main.tex}

\newpage
\input{BehaviouralMethodology/main.tex}

\newpage
\input{Application_EMS_ED/main.tex}

\newpage
\input{Conclusion/main.tex}


\end{document}

\newpage
\documentclass{article}

\usepackage{amsmath}
\usepackage{amsfonts} 
\usepackage{geometry}
\usepackage{multicol}
\usepackage{float}
% \usepackage{mathtools}
% \usepackage{graphicx}
% \usepackage{soul}
% \usepackage{indentfirst}
\usepackage{tikz}
\usetikzlibrary{calc, automata, chains, arrows.meta, math}
\setcounter{MaxMatrixCols}{20}


\title{A game theoretic model of the behavioural gaming that takes place at the EMS - ED interface}

\author{
    Michalis Panayides, 
    Paul Harper, 
    Vince Knight
}

\begin{document}

\maketitle

\input{Abstract/main.tex}


\newpage
\tableofcontents

\newpage
\input{Introduction/main.tex}

\newpage
\input{Game_theory_component/main.tex}

\newpage
\input{MarkovChain/markov_chain_model/main.tex}
\input{MarkovChain/expressions_from_pi/main.tex}
\input{MarkovChain/markov_example/main.tex}

\newpage
\input{BehaviouralMethodology/main.tex}

\newpage
\input{Application_EMS_ED/main.tex}

\newpage
\input{Conclusion/main.tex}


\end{document}

\newpage
\section{EMS-ED application}

\subsection{Application}

\subsection{Data analysis of generated problem}

\newpage
\documentclass{article}

\usepackage{amsmath}
\usepackage{amsfonts} 
\usepackage{geometry}
\usepackage{multicol}
\usepackage{float}
% \usepackage{mathtools}
% \usepackage{graphicx}
% \usepackage{soul}
% \usepackage{indentfirst}
\usepackage{tikz}
\usetikzlibrary{calc, automata, chains, arrows.meta, math}
\setcounter{MaxMatrixCols}{20}


\title{A game theoretic model of the behavioural gaming that takes place at the EMS - ED interface}

\author{
    Michalis Panayides, 
    Paul Harper, 
    Vince Knight
}

\begin{document}

\maketitle

\input{Abstract/main.tex}


\newpage
\tableofcontents

\newpage
\input{Introduction/main.tex}

\newpage
\input{Game_theory_component/main.tex}

\newpage
\input{MarkovChain/markov_chain_model/main.tex}
\input{MarkovChain/expressions_from_pi/main.tex}
\input{MarkovChain/markov_example/main.tex}

\newpage
\input{BehaviouralMethodology/main.tex}

\newpage
\input{Application_EMS_ED/main.tex}

\newpage
\input{Conclusion/main.tex}


\end{document}


\end{document}


\end{document}


In order to consider this model numerically an adjustment needs to be made. 
The problem defined above assumes no upper boundary to the number of individuals 
that can wait for service or for the ones that are blocked in the buffer centre. 
Therefore, a different state space \( \tilde S \) is constructed where 
\( \tilde S \subseteq S \) and there is a maximum allowed number of individuals 
\(N\) that can be in the system and a maximum allowed number of individuals 
\(M\) that can be blocked in the buffer centre:

\begin{equation}
    \tilde S = \left\{ (u, v) \in S\;| u \leq M, v\leq N \right\}
\end{equation}


\subsection{Performance Measures}


The transition matrix \( Q \) defined in \ref{eq:markov_transition_rate} can be 
used to get the probability vector \( \pi \).
The vector \( \pi \) is commonly used to study stochastic systems and it's main
purpose is to keep track of the probability of being at any given state of 
the system.
The term \textit{steady state} refers to the instance of the vector \( \pi \) 
where the probabilities of being at any state become stable over time. 
Thus, by considering the steady state vector \( \pi \) the relationship between 
it and \( Q \) is given by:

\[
    \frac{d\pi}{dt} = \pi Q = 0
\]

Using vector \(\pi\) there are numerous performance measures of the model that 
can be calculated. 
The following equations utilise \(\pi\) to get performance measures on the 
average number of people at certain sets of state.

\begin{itemize}
    \item Average number of people in the system: 
        \[L = \sum_{i=1}^{|\pi|} \pi_i (u_i + v_i)\]
    \item Average number of people in the service centre: 
        \[L_H = \sum_{i=1}^{|\pi|} \pi_i v_i\]
    \item Average number of people in waiting zone 2:
        \[L_A = \sum_{i=1}^{|\pi|} \pi_i u_i\] 
\end{itemize}

Consequently, there are some additional performance measures of interest that
are not as straightforward to calculate.
Such performance measures are the mean waiting time in the system (for both 
class 1 and class 2 individuals), the mean time blocked in waiting zone 2 (only 
valid for class 2 individuals) and the proportion of individuals that wait in 
waiting zone 1 within a predefined time target.

\subsubsection{Waiting time} \label{sec:waiting_time}

Waiting time is the amount of time that individuals from either class wait in 
waiting zone 1 so that they can receive their service. 
For a given set of parameters there are three different performance measures 
around the mean waiting time that can be calculated; the mean waiting time of
class 1 individuals, the mean waiting time of class 2 individuals and the 
overall mean waiting time. 

Since some of the individuals can be lost to the model, a new set of states 
needs to be defined; the set of \textit{accepting states}. 
That is the set of states that the model is able to accept a certain type of
individual. 
The set of accepting states for class 1 individuals is defined as:

\begin{equation}\label{eq:accepting_states_class_1}
    S_A^{(1)} = \{(u, v) \in S \; | \; v < N \}
\end{equation}

In essence, for class 1 individuals, this is the set of states that are not on 
the last column of states in the Markov chain.
Equivalently, the set of accepting states for class 2 individuals is defined as:

\begin{equation}\label{eq:accepting_states_class_2}
    S_A^{(2)}=
    \begin{cases}
        \{(u, v) \in S \; | \; u < M \}, & \textbf{if } T \leq N\\
        \{(u, v) \in S \; | \; v < N \}, & \textbf{otherwise}
    \end{cases}
\end{equation}

Note here that if the threshold is less than or equal the total capacity of the
system the set includes all states that are not on the last column of the 
Markov chain.
Otherwise, the set of accepting state is identical to 
\ref{eq:accepting_states_class_1}. Thus, the expressions for the waiting times 
for class 1 and class 2 individuals are given by:

\begin{equation} \label{eq:closed_form_waiting_class_1}
    W^{(1)} = \frac{\sum_{\substack{(u,v) \, \in S_A^{(1)} \\ v \geq C}} 
    \frac{1}{C \mu} \times (v-C+1) \times \pi(u,v)}{\sum_{(u,v) \, 
    \in S_A^{(1)}} \pi(u,v)}
\end{equation}
    
\begin{equation}\label{eq:closed_form_waiting_class_2}
    W^{(2)} = \frac{\sum_{\substack{(u,v) \, \in S_A^{(2)} \\ min(v,T) \geq C}} 
    \frac{1}{C \mu} \times (\min(v+1,T)-C) \times \pi(u,v)}{\sum_{(u,v) \, 
    \in S_A^{(2)}} \pi(u,v)}
\end{equation}

Consequently, the overall waiting time can be estimated by a linear combination 
of \(W_1\) and \(W_2\). 
Thus, the overall waiting time can calculated by the following equation where 
\(c_1\) and \(c_2\) are the coefficients of the terms:

\begin{equation}\label{eq:overall_waiting_time_coeff}
    W = c_1 W^{(1)} + c_2 W^{(2)}
\end{equation}

The two coefficients represent the proportion of individuals of each type that 
did not get lost and traversed through the model.
Thus, one should account for the probability that an individual is lost to the 
system. 
This probability can be easily calculated by using the two sets of accepting 
states \(S_A^{(2)}\) and \(S_A^{(1)}\) defined in equations 
\ref{eq:accepting_states_class_1} and \ref{eq:accepting_states_class_2}. 
Using these equations the probability, for either class type, that an individual 
is not lost in the system is given by:

\begin{equation*}
    P(L'_1) = \sum_{(u,v) \, \in S_A^{(1)}} \pi(u,v) \hspace{2cm}
    P(L'_2) = \sum_{(u,v) \, \in S_A^{(2)}} \pi(u,v)
\end{equation*}
 
Thus, by using these values as the coefficient of equation 
\ref{eq:overall_waiting_time_coeff} the resultant equation can be used to get 
the overall waiting time. 

\begin{equation}\label{eq:overall_waiting_time}
    W = \frac{\lambda_1 P(L'_1)}{\lambda_2 P(L'_2) + \lambda_1 P(L'_1)} W^{(1)} + 
    \frac{\lambda_2 P(L'_2)}{\lambda_2 P(L'_2) + \lambda_1 P(L'_1)} W^{(2)}
\end{equation}


\subsubsection{Blocking time}

% TODO: Possibly replace the contents of this section with the ones for the 
% closed form formula of the blocking time
% Currently the direct approach one is shown

Unlike the waiting time, the blocking time is only calculated for individuals of
the second type.  
That is because individuals of the first type cannot be blocked. 
Thus, one only needs to consider the pathway of type 2 individuals to get the 
mean blocking time of the system. 
The set of states where individuals can be blocked is defined as:

\begin{equation} \label{eq:set_of_blocking_states}
    S_b = \{(u,v) \in S \; | \; u > 0\}
\end{equation}
 
In order to not consider individuals that will be lost to the system, the set of 
accepting states needs to be taken into consideration. 
As defined in section \ref{sec:waiting_time}, the set of accepting states is 
given by (\ref{eq:accepting_states_type_2}):

\begin{equation*}
    S_A^{(2)}=
    \begin{cases}
        \{(u, v) \in S \; | \; u < M \} & \textbf{if } T \leq N\\
        \{(u, v) \in S \; | \; v < N \} & \textbf{otherwise}
    \end{cases}
\end{equation*}

The mean sojourn time for each state is given by the inverse of the out-flow of
that state.
However, whenever a type 2 individual arrives at the system, no subsequent 
arrival of another type 2 individual can affect its pathway or total time in 
the system.
Therefore, looking at the mean time in the system from the perspective of an 
individual of the second type, all such type 2 arrivals need to be ignored.
Note here that this is not the case for individuals of the first type.
Whenever a type 2 individual is blocked and a type 1 individual arrives the type
2 individuals will remain blocked for some additional amount of time.
Thus, the mean time that a type 2 individual spends at each state is given by:

\begin{equation}\label{eq:time_in_state_blocking_time}
    c(u,v) = 
    \begin{cases}
        \frac{1}{\min(v,C) \mu}, & \text{if } v = N\\
        \frac{1}{\lambda_1 + \min(v,C) \mu}, & \text{otherwise}
    \end{cases}
\end{equation}
 
In equation (\ref{eq:time_in_state_blocking_time}), both service completions and 
type 1 arrivals are considered. 
Thus, from a blocked individual's perspective whenever the system moves from one 
state \((u,v)\)
to another state it can either:

\begin{itemize}
    \item be because of a service being completed: we will denote the probability 
    of this happening by \(p_s(u,v)\). 
    \item be because of an arrival of an individual of type 1: denoting such 
    probability by \(p_o(u,v)\).
\end{itemize}
The probabilities are given by:

\begin{equation*}
    p_s(u,v) = \frac{\min(v,C)\mu}{\lambda_1 + \min(v,C)\mu}, \qquad
    p_o(u,v) = \frac{\lambda_1}{\lambda_1 + \min(v,C)\mu}
\end{equation*}


Having defined \(c(u,v)\) and \(S_b\) a formula for the blocking time that is
expected to occur at each state can be given by:

\begin{equation}\label{eq:general_blocking_time_at_each_state}
    b(u,v) = 
    \begin{cases} 
        0, & \textbf{if } (u,v) \notin S_b \\
        c(u,v) + b(u - 1, v), & \textbf{if } v = N = T\\
        c(u,v) + b(u, v-1), & \textbf{if } v = N \neq T \\
        c(u,v) + p_s(u,v) b(u-1, v) + p_o(u,v) b(u, v+1), & \textbf{if } u > 0 
        \textbf{ and } \vspace{-0.2cm} \\ 
        & \quad v = T \\
        c(u,v) + p_s(u,v) b(u, v-1) + p_o(u,v) b(u, v+1), & \textbf{otherwise} \\
    \end{cases}
\end{equation}

A direct approach will be used to solve this equation here. 
By enumerating all equations of ((\ref{eq:general_blocking_time_at_each_state})) 
for all states \((u,v)\) that belong in \(S_b\) 
a system of linear equations arises where the unknown variables are all the 
\(b(u,v)\) terms. 
Note here that these equations correspond to all blocking states as defined in
(\ref{eq:set_of_blocking_states}). 
Equations that correspond to non-blocking states have a value of \(0\) as 
defined in (\ref{eq:general_blocking_time_at_each_state})
The general form of the equation in terms of \(C,T,N \text{ and } M\) is given by: 

\begin{align}
    b(1,T) \quad &= \quad c(1, T) + p_o b(1, T + 1) \label{eq:first_eq_of_blocking_general}\\
    b(1,T + 1) \quad &= \quad c(1, T + 1) + p_s b(1, T) + p_o b(1, T + 1) \\
    b(1,T + 2) \quad &= \quad c(1, T + 2) + p_s b(1, T + 1) + p_o b(1, T + 3) \\
    & \ \, \vdots \nonumber \\
    b(1, N) \quad &= \quad c(1, N) + b(1, N - 1) \\
    b(2, T) \quad &= \quad c(2, T) + p_s b(1, T) + p_o b(2, T + 1) \\
    b(2, T + 1) \quad &= \quad c(2, T + 1) + p_s b(2, T) + p_o b(2, T + 2) \\
    & \ \, \vdots \nonumber \\
    b(M - 1, N) \quad &= \quad c(M, N - 1) + b(M, N-1) \\ 
    b(M, T) \quad &= \quad c(T, N) + p_s b(T-1, N) + p_o b(T, N+1) \\
    & \ \, \vdots \nonumber \\
    b(M, N) \quad &= \quad c(M, N) + b(M, N-1) \label{eq:last_eq_of_blocking_general}
\end{align}

The equivalent matrix notation of the linear system of equations 
((\ref{eq:first_eq_of_blocking_general})) - ((\ref{eq:last_eq_of_blocking_general}))
is given by \(Zx=y\), where:
\begin{equation}\label{eq:general_algebaric_approach_blocking_time}
    \scalebox{0.73}{\(
        Z = 
        \begin{pmatrix}
            -1 & p_o & 0 & \dots & 0 & 0 & 0 & 0 & 0 & \dots & 0 & 0 \\ %(1,T)
            p_s & -1 & p_o & \dots & 0 & 0 & 0 & 0 & 0 & \dots & 0 & 0 \\ %(1,T+1)
            0 & p_s & -1 & \dots & 0 & 0 & 0 & 0 & 0 & \dots & 0 & 0 \\ %(1,T+2)
            \vdots & \vdots & \vdots & \ddots & \vdots & \vdots & \vdots & 
            \vdots & \vdots & \ddots & \vdots & \vdots \\ 
            0 & 0 & 0 & \dots & 1 & -1 & 0 & 0 & 0 & \dots & 0 & 0 \\ %(1,N)
            p_s & 0 & 0 & \dots & 0 & 0 & -1 & p_o & 0 & \dots & 0 & 0 \\ %(2,T)
            0 & 0 & 0 & \dots & 0 & 0 & p_s & -1 & p_o & \dots & 0 & 0 \\ %(2,T+1)
            \vdots & \vdots & \vdots & \ddots & \vdots & \vdots & \vdots & 
            \vdots & \vdots & \ddots & \vdots & \vdots \\ 
            0 & 0 & 0 & \dots & 0 & 0 & 0 & 0 & 0 & \dots & 1 & -1 \\ %(M,N)
        \end{pmatrix},
        x = 
        \begin{pmatrix}
            b(1,T) \\
            b(1,T+1) \\
            b(1,T+2) \\
            \vdots \\
            b(1,N) \\
            b(2,T) \\
            b(2,T+1) \\
            \vdots \\
            b(M,N) \\
        \end{pmatrix}, 
        y= 
        \begin{pmatrix}
            -c(1,T) \\
            -c(1,T+1) \\
            -c(1,T+2) \\
            \vdots \\
            -c(1,N) \\
            -c(2,T) \\
            -c(2,T+1) \\
            \vdots \\
            -c(M,N) \\
        \end{pmatrix}
    \)}
\end{equation}

Thus, having calculated the mean blocking time for all blocking states \(b(u,v)\), 
it only remains to put them together in a formula.
The resultant formula for the mean blocking time is given by:

\begin{equation}\label{eq:algebraic_blocking_time}
    B = \frac{\sum_{(u,v) \in S_A} \pi_{(u,v)} \; b(u,v)}{\sum_{(u,v) \in S_A} 
    \pi_{(u,v)}}
\end{equation}



To illustrate how the described formula works consider a Markov model where 
\(C=2, T=2, N=4, M=2\) (figure \ref{fig:example_algeb_blocking}). 
The equations that correspond to such a model are shown in 
((\ref{eq:first_eq_of_blocking_example}))-((\ref{eq:last_eq_of_blocking_example})) 
and their equivalent matrix notation form is shown in 
(\ref{eq:example_algebaric_approach_blocking_time}).

\begin{minipage}{.5\textwidth}
    \begin{figure}[H]
        \scalebox{0.6}{\documentclass{article}

\usepackage{amsmath}
\usepackage{amsfonts} 
\usepackage{geometry}
\usepackage{multicol}
\usepackage{float}
% \usepackage{mathtools}
% \usepackage{graphicx}
% \usepackage{soul}
% \usepackage{indentfirst}
\usepackage{tikz}
\usetikzlibrary{calc, automata, chains, arrows.meta, math}
\setcounter{MaxMatrixCols}{20}


\title{A game theoretic model of the behavioural gaming that takes place at the EMS - ED interface}

\author{
    Michalis Panayides, 
    Paul Harper, 
    Vince Knight
}

\begin{document}

\maketitle

\input{Abstract/main.tex}


\newpage
\tableofcontents

\newpage
\input{Introduction/main.tex}

\newpage
\input{Game_theory_component/main.tex}

\newpage
\input{MarkovChain/markov_chain_model/main.tex}
\input{MarkovChain/expressions_from_pi/main.tex}
\input{MarkovChain/markov_example/main.tex}

\newpage
\input{BehaviouralMethodology/main.tex}

\newpage
\input{Application_EMS_ED/main.tex}

\newpage
\input{Conclusion/main.tex}


\end{document}}
        \caption{
            \centering{Example of Markov chain with \(C=2, T=2, N=4, M=2\)}
        }
        \label{fig:example_algeb_blocking}
    \end{figure}
\end{minipage}
\begin{minipage}{.43\textwidth}
    \begin{align}
        b(1,2) &= c(1,2) + p_o b(1,3) \label{eq:first_eq_of_blocking_example} \\
        b(1,3) &= c(1,3) + p_s b(1,2) \nonumber \\ &+ p_o b(1,4) \\
        b(1,4) &= c(1,4) + b(1,3) \\
        b(2,2) &= c(2,2) + p_s b(1,2) \nonumber \\ &+ p_o b(2,3) \\
        b(2,3) &= c(2,3) + p_s b(2,2) \nonumber \\ &+ p_o b(1,4) \\
        b(2,4) &= c(2,4) + b(2,3) \label{eq:last_eq_of_blocking_example}
    \end{align}
\end{minipage}

\begin{equation}\label{eq:example_algebaric_approach_blocking_time}
    Z=
    \begin{pmatrix}
        -1 & p_o & 0 & 0 & 0 & 0 \\ %(1,2)
        p_s & -1 & p_o & 0 & 0 & 0 \\ %(1,3)
        0 & 1 & -1 & 0 & 0 & 0 \\ %(1,4)
        p_s & 0 & 0 & -1 & p_o & 0\\ %(2,2)
        0 & 0 & 0 & p_s & -1 & p_o \\ %(2,3)
        0 & 0 & 0 & 0 & 1 & -1 \\ %(2,4)
    \end{pmatrix},
    x=
    \begin{pmatrix}
        b(1,2) \\
        b(1,3) \\
        b(1,4) \\
        b(2,2) \\
        b(2,3) \\
        b(2,4) \\
    \end{pmatrix}, 
    y=
    \begin{pmatrix}
        -c(1,2) \\
        -c(1,3) \\
        -c(1,4) \\
        -c(2,2) \\
        -c(2,3) \\
        -c(2,4) \\
    \end{pmatrix}
\end{equation}


\documentclass{article}

\usepackage{amsmath}
\usepackage{amsfonts} 
\usepackage{geometry}
\usepackage{multicol}
\usepackage{float}
% \usepackage{mathtools}
% \usepackage{graphicx}
% \usepackage{soul}
% \usepackage{indentfirst}
\usepackage{tikz}
\usetikzlibrary{calc, automata, chains, arrows.meta, math}
\setcounter{MaxMatrixCols}{20}


\title{A game theoretic model of the behavioural gaming that takes place at the EMS - ED interface}

\author{
    Michalis Panayides, 
    Paul Harper, 
    Vince Knight
}

\begin{document}

\maketitle

\documentclass{article}

\usepackage{amsmath}
\usepackage{amsfonts} 
\usepackage{geometry}
\usepackage{multicol}
\usepackage{float}
% \usepackage{mathtools}
% \usepackage{graphicx}
% \usepackage{soul}
% \usepackage{indentfirst}
\usepackage{tikz}
\usetikzlibrary{calc, automata, chains, arrows.meta, math}
\setcounter{MaxMatrixCols}{20}


\title{A game theoretic model of the behavioural gaming that takes place at the EMS - ED interface}

\author{
    Michalis Panayides, 
    Paul Harper, 
    Vince Knight
}

\begin{document}

\maketitle

\documentclass{article}

\usepackage{amsmath}
\usepackage{amsfonts} 
\usepackage{geometry}
\usepackage{multicol}
\usepackage{float}
% \usepackage{mathtools}
% \usepackage{graphicx}
% \usepackage{soul}
% \usepackage{indentfirst}
\usepackage{tikz}
\usetikzlibrary{calc, automata, chains, arrows.meta, math}
\setcounter{MaxMatrixCols}{20}


\title{A game theoretic model of the behavioural gaming that takes place at the EMS - ED interface}

\author{
    Michalis Panayides, 
    Paul Harper, 
    Vince Knight
}

\begin{document}

\maketitle

\input{Abstract/main.tex}


\newpage
\tableofcontents

\newpage
\input{Introduction/main.tex}

\newpage
\input{Game_theory_component/main.tex}

\newpage
\input{MarkovChain/markov_chain_model/main.tex}
\input{MarkovChain/expressions_from_pi/main.tex}
\input{MarkovChain/markov_example/main.tex}

\newpage
\input{BehaviouralMethodology/main.tex}

\newpage
\input{Application_EMS_ED/main.tex}

\newpage
\input{Conclusion/main.tex}


\end{document}


\newpage
\tableofcontents

\newpage
\documentclass{article}

\usepackage{amsmath}
\usepackage{amsfonts} 
\usepackage{geometry}
\usepackage{multicol}
\usepackage{float}
% \usepackage{mathtools}
% \usepackage{graphicx}
% \usepackage{soul}
% \usepackage{indentfirst}
\usepackage{tikz}
\usetikzlibrary{calc, automata, chains, arrows.meta, math}
\setcounter{MaxMatrixCols}{20}


\title{A game theoretic model of the behavioural gaming that takes place at the EMS - ED interface}

\author{
    Michalis Panayides, 
    Paul Harper, 
    Vince Knight
}

\begin{document}

\maketitle

\input{Abstract/main.tex}


\newpage
\tableofcontents

\newpage
\input{Introduction/main.tex}

\newpage
\input{Game_theory_component/main.tex}

\newpage
\input{MarkovChain/markov_chain_model/main.tex}
\input{MarkovChain/expressions_from_pi/main.tex}
\input{MarkovChain/markov_example/main.tex}

\newpage
\input{BehaviouralMethodology/main.tex}

\newpage
\input{Application_EMS_ED/main.tex}

\newpage
\input{Conclusion/main.tex}


\end{document}

\newpage
\documentclass{article}

\usepackage{amsmath}
\usepackage{amsfonts} 
\usepackage{geometry}
\usepackage{multicol}
\usepackage{float}
% \usepackage{mathtools}
% \usepackage{graphicx}
% \usepackage{soul}
% \usepackage{indentfirst}
\usepackage{tikz}
\usetikzlibrary{calc, automata, chains, arrows.meta, math}
\setcounter{MaxMatrixCols}{20}


\title{A game theoretic model of the behavioural gaming that takes place at the EMS - ED interface}

\author{
    Michalis Panayides, 
    Paul Harper, 
    Vince Knight
}

\begin{document}

\maketitle

\input{Abstract/main.tex}


\newpage
\tableofcontents

\newpage
\input{Introduction/main.tex}

\newpage
\input{Game_theory_component/main.tex}

\newpage
\input{MarkovChain/markov_chain_model/main.tex}
\input{MarkovChain/expressions_from_pi/main.tex}
\input{MarkovChain/markov_example/main.tex}

\newpage
\input{BehaviouralMethodology/main.tex}

\newpage
\input{Application_EMS_ED/main.tex}

\newpage
\input{Conclusion/main.tex}


\end{document}

\newpage
\documentclass{article}

\usepackage{amsmath}
\usepackage{amsfonts} 
\usepackage{geometry}
\usepackage{multicol}
\usepackage{float}
% \usepackage{mathtools}
% \usepackage{graphicx}
% \usepackage{soul}
% \usepackage{indentfirst}
\usepackage{tikz}
\usetikzlibrary{calc, automata, chains, arrows.meta, math}
\setcounter{MaxMatrixCols}{20}


\title{A game theoretic model of the behavioural gaming that takes place at the EMS - ED interface}

\author{
    Michalis Panayides, 
    Paul Harper, 
    Vince Knight
}

\begin{document}

\maketitle

\input{Abstract/main.tex}


\newpage
\tableofcontents

\newpage
\input{Introduction/main.tex}

\newpage
\input{Game_theory_component/main.tex}

\newpage
\input{MarkovChain/markov_chain_model/main.tex}
\input{MarkovChain/expressions_from_pi/main.tex}
\input{MarkovChain/markov_example/main.tex}

\newpage
\input{BehaviouralMethodology/main.tex}

\newpage
\input{Application_EMS_ED/main.tex}

\newpage
\input{Conclusion/main.tex}


\end{document}
\subsection{Performance Measures}
One may easily derive the average number of individuals that are at any given state 
using \( pi \). 
The average number of individuals in state \( i \) can be calculated by multiplying 
the number of individuals that are present in state \( i \) with the probability 
of being at that particular state (i.e \(\pi_i (u_i + v_i)\)). 
Using this logic it is possible to calculate any performance measures that are related 
to the mean number of individuals in the system.


Average number of people in the system: 
\begin{equation}
    L = \sum_{i=1}^{|\pi|} \pi_i (u_i + v_i)
\end{equation} 

Average number of people in the service centre: 
\begin{equation}
    L_H = \sum_{i=1}^{|\pi|} \pi_i v_i
\end{equation}

Average number of people in the buffer centre:
\begin{equation}
    L_A = \sum_{i=1}^{|\pi|} \pi_i u_i
\end{equation}

Consequently getting the performance measures that are related to the duration of 
time is not as straightforward. 
Such performance measures are the mean waiting time in the system and the mean time 
blocked in the system. 
Under the scope of this study three approaches have been considered to calculate these 
performance measures; a direct approach, a recursive algorithm and consequently a
closed-form formula.

The research question that needs to be answered here is: ``When a class 1/2 
individuals enters the system, what is the expected time that they will have to 
wait?''. 
In order to formulate the answer to that question one needs to consider all possible 
scenarios of what state the system can be in when an individual arrives. 
Furthermore, different formulas arises for class 1 individuals 
and a different one for class 2 individuals.

\subsubsection{Mean waiting time} 
Upon closer inspection of the recursive formula a more compact formula can arise. 
The equivalent closed-form formula eliminates the need for recursion and thus makes 
the computation of waiting times much more efficient. 
Just like in the recursive part there are two formulas; one for \textit{class 1} 
and one for class 2 individuals. 
The formulas are given by:

\begin{equation} \label{eq:closed_form_waiting_others}
    W^{(1)} = \frac{\sum_{\substack{(u,v) \, \in S_A^{(1)} \\ v \geq C}} 
    \frac{1}{C \mu} \times (v-C+1) \times \pi(u,v)}{\sum_{(u,v) \, 
    \in S_A^{(1)}} \pi(u,v)}
\end{equation}
    
\begin{equation}\label{eq:closed_form_waiting_ambulance}
    W^{(2)} = \frac{\sum_{\substack{(u,v) \, \in S_A^{(2)} \\ min(v,T) \geq C}} 
    \frac{1}{C \mu} \times (\min(v+1,T)-C) \times \pi(u,v)}{\sum_{(u,v) \, 
    \in S_A^{(2)}} \pi(u,v)}
\end{equation}

Note here that the summation, in both equations \ref{eq:closed_form_waiting_others} 
and \ref{eq:closed_form_waiting_ambulance}, goes through all states in the set of 
accepting 
states of either class 1 or class 2 individuals respectively, where a wait 
incurs. 
In equation \ref{eq:closed_form_waiting_others} that includes all states \((u,v)\) 
in the set of accepting states of class 1 individuals such that \( v \geq C\); i.e. 
whenever an arrival occurs and the system is at a state where the number of individuals 
in the system is more than or equal to $C$. 
Consequently, for the states that are included in the summation the expression 
\( v-C+1 \) indicates the amount of people in service one would have to wait for 
upon arrival at the hospital.

Additionally, the minimisation function in equation 
\ref{eq:closed_form_waiting_ambulance} 
ensures that when a class 2 individual arrives at any state 
that is greater than the predetermined threshold, the wait that the individual will 
have to endure remains the same. 
In essence, the expression \(\min(v+1,T) - C\) returns the number of people in line 
in front of a particular individual upon arrival.


\subsubsection{Overall Waiting Time}

Consequently, the overall waiting time should can be estimated by a linear combination 
of the waiting times of class 1 and class 2 individuals. 
The overall waiting time can be then given by the following equation where \(c_1\) 
and \(c_2\) are the coefficients of each individual's type waiting time:

\begin{equation}\label{overall_waiting_time_coeff}
    W = c_1 W^{(1)} + c_2 W^{(2)}
\end{equation}

The two coefficients represent the proportion of individuals of each type that 
traversed through the model. 
Theoretically, getting these percentages should be as simple as looking at the arrival 
rates of each type but in practise if the service centre or the buffer centre 
is full, some individuals may be lost to the system. 
Thus, one should account for the probability that an individual is lost to the system. 
This probability can be easily calculated by using the two sets of accepting states 
\(S_A^{(2)}\) and \(S_A^{(1)}\) defined earlier in equations.
Let us define here the probability, for either class type, that an individual 
is not lost in the system by:

\begin{equation*}
    P(L'_1) = \sum_{(u,v) \, \in S_A^{(1)}} \pi(u,v) \hspace{2cm}
    P(L'_2) = \sum_{(u,v) \, \in S_A^{(2)}} \pi(u,v)
\end{equation*}

Having defined these probabilities one may combine them with the arrival rates of 
each class type in such a way to get the expected proportions of class 1 and 
class 2 individuals in the model. 
Thus, by using these values as the coefficient of equation 
\ref{overall_waiting_time_coeff} 
the resultant equation can be used to get the overall waiting time. 
Note here that the equation below gets the overall waiting time for both the recursive 
and the closed-form formula.

\begin{equation}\label{overall_waiting_time}
    W = \frac{\lambda_1 P(L'_1)}{\lambda_2 P(L'_2) + \lambda_1 P(L'_1)} W^{(1)} + 
    \frac{\lambda_2 P(L'_2)}{\lambda_2 P(L'_2) + \lambda_1 P(L'_1)} W^{(2)}
\end{equation}



\subsubsection{Mean blocking time}
Unlike the waiting time, the blocking time is only calculated for class 2 individuals.  
That is because class 1 individuals cannot be blocked. 
Thus, one only needs to consider the pathway of class 2 individuals to get the 
mean blocking time of the system. 
Blocking occurs at states \((u,v)\) where \(u > 0 \). 
Thus, the set of blocking states can be defined as:

\begin{equation*}
    S_b = \{(u,v) \in S \; | \; u > 0\}
\end{equation*}
 
In order to not consider individuals that will be lost to the system, the set of 
accepting states needs to be taken into account. The set of accepting states is given by:

\begin{equation*}
    S_A^{(2)}=
    \begin{cases}
        \{(u, v) \in S \; | \; u < M \} & \textbf{if } T \leq N\\
        \{(u, v) \in S \; | \; v < N \} & \textbf{otherwise}
    \end{cases}
\end{equation*}

For the waiting time formula,
the mean sojourn time for each state was considered,
ignoring any arrivals. Here, the same approach is used but ignoring only class 2
arrivals. That is because for the waiting time formula, once an individual enters 
the service centre (i.e. starts waiting) any individual arriving after them will 
not affect their
pathway. That is not the case for blocking time. When a class 2 individual is 
blocked, 
any class 1 individual that arrives will cause the blocked individual to remain 
blocked for more time. Therefore, class 1 arrivals are considered here:

\begin{equation}\label{eq:time_in_state_blocking_time}
    c(u,v) = 
    \begin{cases}
        \frac{1}{\min(v,C) \mu}, & \text{if } v = C\\
        \frac{1}{\min(v,C) \mu + \lambda_1}, & \text{otherwise}
    \end{cases}
\end{equation}
 
In equation \ref{eq:time_in_state_blocking_time}, both service completions and 
class 1 arrivals are considered. 
Thus, from a blocked individual's perspective whenever the system moves from one 
state \((u,v)\)
to another state it can either:

\begin{itemize}
    \item be because of a service being completed: we will denote the probability 
    of this happening by \(p_s(u,v)\). 
    \item be because of an arrival of an individual of class 1: denoting such 
    probability by \(p_o(u,v)\).
\end{itemize}
The probabilities are given by:

\begin{equation*}
    p_s(u,v) = \frac{\min(v,C)\mu}{\lambda_1 + \min(v,C)\mu}, \qquad
    p_o(u,v) = \frac{\lambda_1}{\lambda_1 + \min(v,C)\mu}
\end{equation*}


Having defined \(c(u,v)\) and \(S_b\) a formula for the blocking time that is
expected to occur at each state can be given by:

\begin{equation}\label{eq:blocking-time-at-each-state}
    b(u,v) = 
    \begin{cases} 
        0, & \textbf{if } (u,v) \notin S_b \\
        c(u,v) + b(u - 1, v), & \textbf{if } v = N = T\\
        c(u,v) + b(u, v-1), & \textbf{if } v = N \neq T \\
        c(u,v) + p_s(u,v) b(u-1, v) + p_o(u,v) b(u, v+1), & \textbf{if } u > 0 
        \textbf{ and } v = T \\
        c(u,v) + p_s(u,v) b(u, v-1) + p_o(u,v) b(u, v+1), & \textbf{otherwise} \\
    \end{cases}
\end{equation}

Equation 
(\ref{eq:blocking-time-at-each-state}) will not be solved recursively. 
A direct approach will be used to solve this equation here. 
By enumerating all equations of (\ref{eq:blocking-time-at-each-state}) for all 
states \((u,v)\) that belong in \(S_b\) 
a system of linear equations arises where the unknown variables are all the \(b(u,v)\)
terms.
For instance, let us consider a Markov model where \(C=2, T=3, N=6, M=2\). 
The Markov model is shown in Figure \ref{fig:example-algeb-blocking}
and the equivalent equations are 
(\ref{eq:first_eq_of_blocking_example})-(\ref{eq:last_eq_of_blocking_example}).
The equations considered here are only the ones that correspond to the blocking 
states.

\begin{multicols*}{2}
    \begin{figure}[H]
        \scalebox{0.50}{\input{MarkovChain/expressions_from_pi/example_model_2362/main.tex}}
        \caption{Example of Markov chain}
        \label{fig:example-algeb-blocking}
    \end{figure}
    \columnbreak
    \begin{align}
        b(1,2) &= c(1,2) + p_o b(1,3) \label{eq:first_eq_of_blocking_example} \\
        b(1,3) &= c(1,3) + p_s b(1,2) + p_o b(1,4) \\
        b(1,4) &= c(1,4) + b(1,3) \\
        b(2,2) &= c(2,2) + p_s b(1,2) + p_o b(2,3) \\
        b(2,3) &= c(2,3) + p_s b(2,2) + p_o b(1,4) \\
        b(2,4) &= c(2,4) + b(2,3)\label{eq:last_eq_of_blocking_example}
    \end{align}
\end{multicols*}

Additionally, the above equations can be transformed into a linear system of the 
form \(Zx=y\) where:

\begin{equation}\label{eq:example-algebaric-approach-blocking-time}
    Z=
    \begin{pmatrix}
        -1 & p_o & 0 & 0 & 0 & 0 \\ %(1,2)
        p_s & -1 & p_o & 0 & 0 & 0 \\ %(1,3)
        0 & 1 & -1 & 0 & 0 & 0 \\ %(1,4)
        p_s & 0 & 0 & -1 & p_o & 0\\ %(2,2)
        0 & 0 & 0 & p_s & -1 & p_o \\ %(2,3)
        0 & 0 & 0 & 0 & 1 & -1 \\ %(2,4)
    \end{pmatrix},
    x=
    \begin{pmatrix}
        b(1,2) \\
        b(1,3) \\
        b(1,4) \\
        b(2,2) \\
        b(2,3) \\
        b(2,4) \\
    \end{pmatrix}, 
    y=
    \begin{pmatrix}
        -c(1,2) \\
        -c(1,3) \\
        -c(1,4) \\
        -c(2,2) \\
        -c(2,3) \\
        -c(2,4) \\
    \end{pmatrix}
\end{equation}

A more generalised form of the equations in 
(\ref{eq:example-algebaric-approach-blocking-time})
can thus be given for any value of \(C,T,N,M\) by:

\begin{align}
    b(1,T) =& c(1, T) + p_o b(1, T + 1) \label{eq:first_eq_of_blocking_general}\\
    b(1,T + 1) =& c(1, T + 1) + p_s(1, T) + p_o b(1, T + 1) \\
    b(1,T + 2) =& c(1, T + 2) + p_s(1, T + 1) + p_o b(1, T + 3) \\
    & \vdots \nonumber \\
    b(1, N) =& c(1, N) + b(1, N - 1) \\
    b(2, T) =& c(2, T) + p_s b(1, T) + p_o b(2, T + 1) \\
    b(2, T + 1) =& c(2, T + 1) + p_s b(2, T) + p_o b(2, T + 2) \\
    & \vdots \nonumber \\
    b(M, T) =& c(M, T) + b(M, T-1) \label{eq:last_eq_of_blocking_general}
\end{align}

The equivalent matrix form of the linear system of equations 
(\ref{eq:first_eq_of_blocking_general}) - (\ref{eq:last_eq_of_blocking_general})
is given by \(Zx=y\), where:
\begin{equation}\label{eq:general-algebaric-approach-blocking-time}
    \scalebox{0.9}{
        \(
        Z = 
        \begin{pmatrix}
            -1 & p_o & 0 & \dots & 0 & 0 & 0 & 0 & 0 & \dots & 0 & 0 \\ %(1,T)
            p_s & -1 & p_o & \dots & 0 & 0 & 0 & 0 & 0 & \dots & 0 & 0 \\ %(1,T+1)
            0 & p_s & -1 & \dots & 0 & 0 & 0 & 0 & 0 & \dots & 0 & 0 \\ %(1,T+2)
            \vdots & \vdots & \vdots & \ddots & \vdots & \vdots & \vdots & \vdots & 
            \vdots & \ddots & \vdots & \vdots \\ 
            0 & 0 & 0 & \dots & 1 & -1 & 0 & 0 & 0 & \dots & 0 & 0 \\ %(1,N)
            p_s & 0 & 0 & \dots & 0 & 0 & -1 & p_o & 0 & \dots & 0 & 0 \\ %(2,T)
            0 & 0 & 0 & \dots & 0 & 0 & p_s & -1 & p_o & \dots & 0 & 0 \\ %(2,T+1)
            \vdots & \vdots & \vdots & \ddots & \vdots & \vdots & \vdots & \vdots & 
            \vdots & \ddots & \vdots & \vdots \\ 
            0 & 0 & 0 & \dots & 0 & 0 & 0 & 0 & 0 & \dots & 1 & -1 \\ %(M,T)
        \end{pmatrix},
        x = 
        \begin{pmatrix}
            b(1,T) \\
            b(1,T+1) \\
            b(1,T+2) \\
            \vdots \\
            b(1,N) \\
            b(2,T) \\
            b(2,T+1) \\
            \vdots \\
            b(M,T) \\
        \end{pmatrix}, 
        y= 
        \begin{pmatrix}
            -c(1,T) \\
            -c(1,T+1) \\
            -c(1,T+2) \\
            \vdots \\
            -c(1,N) \\
            -c(2,T) \\
            -c(2,T+1) \\
            \vdots \\
            -c(M,T) \\
        \end{pmatrix}
        \)
    }
\end{equation}

Thus, having calculated the mean blocking time for all blocking states \(b(u,v)\), 
it only remains to put them together in a formula.
The resultant blocking time formula is given by:

\begin{equation}\label{eq:algebraic-blocking-time}
    B = \frac{\sum_{(u,v) \in S_A} \pi_{(u,v)} \; b(u,v)}{\sum_{(u,v) \in S_A} 
    \pi_{(u,v)}}
\end{equation}

\documentclass{article}

\usepackage{amsmath}
\usepackage{amsfonts} 
\usepackage{geometry}
\usepackage{multicol}
\usepackage{float}
% \usepackage{mathtools}
% \usepackage{graphicx}
% \usepackage{soul}
% \usepackage{indentfirst}
\usepackage{tikz}
\usetikzlibrary{calc, automata, chains, arrows.meta, math}
\setcounter{MaxMatrixCols}{20}


\title{A game theoretic model of the behavioural gaming that takes place at the EMS - ED interface}

\author{
    Michalis Panayides, 
    Paul Harper, 
    Vince Knight
}

\begin{document}

\maketitle

\input{Abstract/main.tex}


\newpage
\tableofcontents

\newpage
\input{Introduction/main.tex}

\newpage
\input{Game_theory_component/main.tex}

\newpage
\input{MarkovChain/markov_chain_model/main.tex}
\input{MarkovChain/expressions_from_pi/main.tex}
\input{MarkovChain/markov_example/main.tex}

\newpage
\input{BehaviouralMethodology/main.tex}

\newpage
\input{Application_EMS_ED/main.tex}

\newpage
\input{Conclusion/main.tex}


\end{document}

\newpage
\documentclass{article}

\usepackage{amsmath}
\usepackage{amsfonts} 
\usepackage{geometry}
\usepackage{multicol}
\usepackage{float}
% \usepackage{mathtools}
% \usepackage{graphicx}
% \usepackage{soul}
% \usepackage{indentfirst}
\usepackage{tikz}
\usetikzlibrary{calc, automata, chains, arrows.meta, math}
\setcounter{MaxMatrixCols}{20}


\title{A game theoretic model of the behavioural gaming that takes place at the EMS - ED interface}

\author{
    Michalis Panayides, 
    Paul Harper, 
    Vince Knight
}

\begin{document}

\maketitle

\input{Abstract/main.tex}


\newpage
\tableofcontents

\newpage
\input{Introduction/main.tex}

\newpage
\input{Game_theory_component/main.tex}

\newpage
\input{MarkovChain/markov_chain_model/main.tex}
\input{MarkovChain/expressions_from_pi/main.tex}
\input{MarkovChain/markov_example/main.tex}

\newpage
\input{BehaviouralMethodology/main.tex}

\newpage
\input{Application_EMS_ED/main.tex}

\newpage
\input{Conclusion/main.tex}


\end{document}

\newpage
\section{EMS-ED application}

\subsection{Application}

\subsection{Data analysis of generated problem}

\newpage
\documentclass{article}

\usepackage{amsmath}
\usepackage{amsfonts} 
\usepackage{geometry}
\usepackage{multicol}
\usepackage{float}
% \usepackage{mathtools}
% \usepackage{graphicx}
% \usepackage{soul}
% \usepackage{indentfirst}
\usepackage{tikz}
\usetikzlibrary{calc, automata, chains, arrows.meta, math}
\setcounter{MaxMatrixCols}{20}


\title{A game theoretic model of the behavioural gaming that takes place at the EMS - ED interface}

\author{
    Michalis Panayides, 
    Paul Harper, 
    Vince Knight
}

\begin{document}

\maketitle

\input{Abstract/main.tex}


\newpage
\tableofcontents

\newpage
\input{Introduction/main.tex}

\newpage
\input{Game_theory_component/main.tex}

\newpage
\input{MarkovChain/markov_chain_model/main.tex}
\input{MarkovChain/expressions_from_pi/main.tex}
\input{MarkovChain/markov_example/main.tex}

\newpage
\input{BehaviouralMethodology/main.tex}

\newpage
\input{Application_EMS_ED/main.tex}

\newpage
\input{Conclusion/main.tex}


\end{document}


\end{document}


\newpage
\tableofcontents

\newpage
\documentclass{article}

\usepackage{amsmath}
\usepackage{amsfonts} 
\usepackage{geometry}
\usepackage{multicol}
\usepackage{float}
% \usepackage{mathtools}
% \usepackage{graphicx}
% \usepackage{soul}
% \usepackage{indentfirst}
\usepackage{tikz}
\usetikzlibrary{calc, automata, chains, arrows.meta, math}
\setcounter{MaxMatrixCols}{20}


\title{A game theoretic model of the behavioural gaming that takes place at the EMS - ED interface}

\author{
    Michalis Panayides, 
    Paul Harper, 
    Vince Knight
}

\begin{document}

\maketitle

\documentclass{article}

\usepackage{amsmath}
\usepackage{amsfonts} 
\usepackage{geometry}
\usepackage{multicol}
\usepackage{float}
% \usepackage{mathtools}
% \usepackage{graphicx}
% \usepackage{soul}
% \usepackage{indentfirst}
\usepackage{tikz}
\usetikzlibrary{calc, automata, chains, arrows.meta, math}
\setcounter{MaxMatrixCols}{20}


\title{A game theoretic model of the behavioural gaming that takes place at the EMS - ED interface}

\author{
    Michalis Panayides, 
    Paul Harper, 
    Vince Knight
}

\begin{document}

\maketitle

\input{Abstract/main.tex}


\newpage
\tableofcontents

\newpage
\input{Introduction/main.tex}

\newpage
\input{Game_theory_component/main.tex}

\newpage
\input{MarkovChain/markov_chain_model/main.tex}
\input{MarkovChain/expressions_from_pi/main.tex}
\input{MarkovChain/markov_example/main.tex}

\newpage
\input{BehaviouralMethodology/main.tex}

\newpage
\input{Application_EMS_ED/main.tex}

\newpage
\input{Conclusion/main.tex}


\end{document}


\newpage
\tableofcontents

\newpage
\documentclass{article}

\usepackage{amsmath}
\usepackage{amsfonts} 
\usepackage{geometry}
\usepackage{multicol}
\usepackage{float}
% \usepackage{mathtools}
% \usepackage{graphicx}
% \usepackage{soul}
% \usepackage{indentfirst}
\usepackage{tikz}
\usetikzlibrary{calc, automata, chains, arrows.meta, math}
\setcounter{MaxMatrixCols}{20}


\title{A game theoretic model of the behavioural gaming that takes place at the EMS - ED interface}

\author{
    Michalis Panayides, 
    Paul Harper, 
    Vince Knight
}

\begin{document}

\maketitle

\input{Abstract/main.tex}


\newpage
\tableofcontents

\newpage
\input{Introduction/main.tex}

\newpage
\input{Game_theory_component/main.tex}

\newpage
\input{MarkovChain/markov_chain_model/main.tex}
\input{MarkovChain/expressions_from_pi/main.tex}
\input{MarkovChain/markov_example/main.tex}

\newpage
\input{BehaviouralMethodology/main.tex}

\newpage
\input{Application_EMS_ED/main.tex}

\newpage
\input{Conclusion/main.tex}


\end{document}

\newpage
\documentclass{article}

\usepackage{amsmath}
\usepackage{amsfonts} 
\usepackage{geometry}
\usepackage{multicol}
\usepackage{float}
% \usepackage{mathtools}
% \usepackage{graphicx}
% \usepackage{soul}
% \usepackage{indentfirst}
\usepackage{tikz}
\usetikzlibrary{calc, automata, chains, arrows.meta, math}
\setcounter{MaxMatrixCols}{20}


\title{A game theoretic model of the behavioural gaming that takes place at the EMS - ED interface}

\author{
    Michalis Panayides, 
    Paul Harper, 
    Vince Knight
}

\begin{document}

\maketitle

\input{Abstract/main.tex}


\newpage
\tableofcontents

\newpage
\input{Introduction/main.tex}

\newpage
\input{Game_theory_component/main.tex}

\newpage
\input{MarkovChain/markov_chain_model/main.tex}
\input{MarkovChain/expressions_from_pi/main.tex}
\input{MarkovChain/markov_example/main.tex}

\newpage
\input{BehaviouralMethodology/main.tex}

\newpage
\input{Application_EMS_ED/main.tex}

\newpage
\input{Conclusion/main.tex}


\end{document}

\newpage
\documentclass{article}

\usepackage{amsmath}
\usepackage{amsfonts} 
\usepackage{geometry}
\usepackage{multicol}
\usepackage{float}
% \usepackage{mathtools}
% \usepackage{graphicx}
% \usepackage{soul}
% \usepackage{indentfirst}
\usepackage{tikz}
\usetikzlibrary{calc, automata, chains, arrows.meta, math}
\setcounter{MaxMatrixCols}{20}


\title{A game theoretic model of the behavioural gaming that takes place at the EMS - ED interface}

\author{
    Michalis Panayides, 
    Paul Harper, 
    Vince Knight
}

\begin{document}

\maketitle

\input{Abstract/main.tex}


\newpage
\tableofcontents

\newpage
\input{Introduction/main.tex}

\newpage
\input{Game_theory_component/main.tex}

\newpage
\input{MarkovChain/markov_chain_model/main.tex}
\input{MarkovChain/expressions_from_pi/main.tex}
\input{MarkovChain/markov_example/main.tex}

\newpage
\input{BehaviouralMethodology/main.tex}

\newpage
\input{Application_EMS_ED/main.tex}

\newpage
\input{Conclusion/main.tex}


\end{document}
\subsection{Performance Measures}
One may easily derive the average number of individuals that are at any given state 
using \( pi \). 
The average number of individuals in state \( i \) can be calculated by multiplying 
the number of individuals that are present in state \( i \) with the probability 
of being at that particular state (i.e \(\pi_i (u_i + v_i)\)). 
Using this logic it is possible to calculate any performance measures that are related 
to the mean number of individuals in the system.


Average number of people in the system: 
\begin{equation}
    L = \sum_{i=1}^{|\pi|} \pi_i (u_i + v_i)
\end{equation} 

Average number of people in the service centre: 
\begin{equation}
    L_H = \sum_{i=1}^{|\pi|} \pi_i v_i
\end{equation}

Average number of people in the buffer centre:
\begin{equation}
    L_A = \sum_{i=1}^{|\pi|} \pi_i u_i
\end{equation}

Consequently getting the performance measures that are related to the duration of 
time is not as straightforward. 
Such performance measures are the mean waiting time in the system and the mean time 
blocked in the system. 
Under the scope of this study three approaches have been considered to calculate these 
performance measures; a direct approach, a recursive algorithm and consequently a
closed-form formula.

The research question that needs to be answered here is: ``When a class 1/2 
individuals enters the system, what is the expected time that they will have to 
wait?''. 
In order to formulate the answer to that question one needs to consider all possible 
scenarios of what state the system can be in when an individual arrives. 
Furthermore, different formulas arises for class 1 individuals 
and a different one for class 2 individuals.

\subsubsection{Mean waiting time} 
Upon closer inspection of the recursive formula a more compact formula can arise. 
The equivalent closed-form formula eliminates the need for recursion and thus makes 
the computation of waiting times much more efficient. 
Just like in the recursive part there are two formulas; one for \textit{class 1} 
and one for class 2 individuals. 
The formulas are given by:

\begin{equation} \label{eq:closed_form_waiting_others}
    W^{(1)} = \frac{\sum_{\substack{(u,v) \, \in S_A^{(1)} \\ v \geq C}} 
    \frac{1}{C \mu} \times (v-C+1) \times \pi(u,v)}{\sum_{(u,v) \, 
    \in S_A^{(1)}} \pi(u,v)}
\end{equation}
    
\begin{equation}\label{eq:closed_form_waiting_ambulance}
    W^{(2)} = \frac{\sum_{\substack{(u,v) \, \in S_A^{(2)} \\ min(v,T) \geq C}} 
    \frac{1}{C \mu} \times (\min(v+1,T)-C) \times \pi(u,v)}{\sum_{(u,v) \, 
    \in S_A^{(2)}} \pi(u,v)}
\end{equation}

Note here that the summation, in both equations \ref{eq:closed_form_waiting_others} 
and \ref{eq:closed_form_waiting_ambulance}, goes through all states in the set of 
accepting 
states of either class 1 or class 2 individuals respectively, where a wait 
incurs. 
In equation \ref{eq:closed_form_waiting_others} that includes all states \((u,v)\) 
in the set of accepting states of class 1 individuals such that \( v \geq C\); i.e. 
whenever an arrival occurs and the system is at a state where the number of individuals 
in the system is more than or equal to $C$. 
Consequently, for the states that are included in the summation the expression 
\( v-C+1 \) indicates the amount of people in service one would have to wait for 
upon arrival at the hospital.

Additionally, the minimisation function in equation 
\ref{eq:closed_form_waiting_ambulance} 
ensures that when a class 2 individual arrives at any state 
that is greater than the predetermined threshold, the wait that the individual will 
have to endure remains the same. 
In essence, the expression \(\min(v+1,T) - C\) returns the number of people in line 
in front of a particular individual upon arrival.


\subsubsection{Overall Waiting Time}

Consequently, the overall waiting time should can be estimated by a linear combination 
of the waiting times of class 1 and class 2 individuals. 
The overall waiting time can be then given by the following equation where \(c_1\) 
and \(c_2\) are the coefficients of each individual's type waiting time:

\begin{equation}\label{overall_waiting_time_coeff}
    W = c_1 W^{(1)} + c_2 W^{(2)}
\end{equation}

The two coefficients represent the proportion of individuals of each type that 
traversed through the model. 
Theoretically, getting these percentages should be as simple as looking at the arrival 
rates of each type but in practise if the service centre or the buffer centre 
is full, some individuals may be lost to the system. 
Thus, one should account for the probability that an individual is lost to the system. 
This probability can be easily calculated by using the two sets of accepting states 
\(S_A^{(2)}\) and \(S_A^{(1)}\) defined earlier in equations.
Let us define here the probability, for either class type, that an individual 
is not lost in the system by:

\begin{equation*}
    P(L'_1) = \sum_{(u,v) \, \in S_A^{(1)}} \pi(u,v) \hspace{2cm}
    P(L'_2) = \sum_{(u,v) \, \in S_A^{(2)}} \pi(u,v)
\end{equation*}

Having defined these probabilities one may combine them with the arrival rates of 
each class type in such a way to get the expected proportions of class 1 and 
class 2 individuals in the model. 
Thus, by using these values as the coefficient of equation 
\ref{overall_waiting_time_coeff} 
the resultant equation can be used to get the overall waiting time. 
Note here that the equation below gets the overall waiting time for both the recursive 
and the closed-form formula.

\begin{equation}\label{overall_waiting_time}
    W = \frac{\lambda_1 P(L'_1)}{\lambda_2 P(L'_2) + \lambda_1 P(L'_1)} W^{(1)} + 
    \frac{\lambda_2 P(L'_2)}{\lambda_2 P(L'_2) + \lambda_1 P(L'_1)} W^{(2)}
\end{equation}



\subsubsection{Mean blocking time}
Unlike the waiting time, the blocking time is only calculated for class 2 individuals.  
That is because class 1 individuals cannot be blocked. 
Thus, one only needs to consider the pathway of class 2 individuals to get the 
mean blocking time of the system. 
Blocking occurs at states \((u,v)\) where \(u > 0 \). 
Thus, the set of blocking states can be defined as:

\begin{equation*}
    S_b = \{(u,v) \in S \; | \; u > 0\}
\end{equation*}
 
In order to not consider individuals that will be lost to the system, the set of 
accepting states needs to be taken into account. The set of accepting states is given by:

\begin{equation*}
    S_A^{(2)}=
    \begin{cases}
        \{(u, v) \in S \; | \; u < M \} & \textbf{if } T \leq N\\
        \{(u, v) \in S \; | \; v < N \} & \textbf{otherwise}
    \end{cases}
\end{equation*}

For the waiting time formula,
the mean sojourn time for each state was considered,
ignoring any arrivals. Here, the same approach is used but ignoring only class 2
arrivals. That is because for the waiting time formula, once an individual enters 
the service centre (i.e. starts waiting) any individual arriving after them will 
not affect their
pathway. That is not the case for blocking time. When a class 2 individual is 
blocked, 
any class 1 individual that arrives will cause the blocked individual to remain 
blocked for more time. Therefore, class 1 arrivals are considered here:

\begin{equation}\label{eq:time_in_state_blocking_time}
    c(u,v) = 
    \begin{cases}
        \frac{1}{\min(v,C) \mu}, & \text{if } v = C\\
        \frac{1}{\min(v,C) \mu + \lambda_1}, & \text{otherwise}
    \end{cases}
\end{equation}
 
In equation \ref{eq:time_in_state_blocking_time}, both service completions and 
class 1 arrivals are considered. 
Thus, from a blocked individual's perspective whenever the system moves from one 
state \((u,v)\)
to another state it can either:

\begin{itemize}
    \item be because of a service being completed: we will denote the probability 
    of this happening by \(p_s(u,v)\). 
    \item be because of an arrival of an individual of class 1: denoting such 
    probability by \(p_o(u,v)\).
\end{itemize}
The probabilities are given by:

\begin{equation*}
    p_s(u,v) = \frac{\min(v,C)\mu}{\lambda_1 + \min(v,C)\mu}, \qquad
    p_o(u,v) = \frac{\lambda_1}{\lambda_1 + \min(v,C)\mu}
\end{equation*}


Having defined \(c(u,v)\) and \(S_b\) a formula for the blocking time that is
expected to occur at each state can be given by:

\begin{equation}\label{eq:blocking-time-at-each-state}
    b(u,v) = 
    \begin{cases} 
        0, & \textbf{if } (u,v) \notin S_b \\
        c(u,v) + b(u - 1, v), & \textbf{if } v = N = T\\
        c(u,v) + b(u, v-1), & \textbf{if } v = N \neq T \\
        c(u,v) + p_s(u,v) b(u-1, v) + p_o(u,v) b(u, v+1), & \textbf{if } u > 0 
        \textbf{ and } v = T \\
        c(u,v) + p_s(u,v) b(u, v-1) + p_o(u,v) b(u, v+1), & \textbf{otherwise} \\
    \end{cases}
\end{equation}

Equation 
(\ref{eq:blocking-time-at-each-state}) will not be solved recursively. 
A direct approach will be used to solve this equation here. 
By enumerating all equations of (\ref{eq:blocking-time-at-each-state}) for all 
states \((u,v)\) that belong in \(S_b\) 
a system of linear equations arises where the unknown variables are all the \(b(u,v)\)
terms.
For instance, let us consider a Markov model where \(C=2, T=3, N=6, M=2\). 
The Markov model is shown in Figure \ref{fig:example-algeb-blocking}
and the equivalent equations are 
(\ref{eq:first_eq_of_blocking_example})-(\ref{eq:last_eq_of_blocking_example}).
The equations considered here are only the ones that correspond to the blocking 
states.

\begin{multicols*}{2}
    \begin{figure}[H]
        \scalebox{0.50}{\input{MarkovChain/expressions_from_pi/example_model_2362/main.tex}}
        \caption{Example of Markov chain}
        \label{fig:example-algeb-blocking}
    \end{figure}
    \columnbreak
    \begin{align}
        b(1,2) &= c(1,2) + p_o b(1,3) \label{eq:first_eq_of_blocking_example} \\
        b(1,3) &= c(1,3) + p_s b(1,2) + p_o b(1,4) \\
        b(1,4) &= c(1,4) + b(1,3) \\
        b(2,2) &= c(2,2) + p_s b(1,2) + p_o b(2,3) \\
        b(2,3) &= c(2,3) + p_s b(2,2) + p_o b(1,4) \\
        b(2,4) &= c(2,4) + b(2,3)\label{eq:last_eq_of_blocking_example}
    \end{align}
\end{multicols*}

Additionally, the above equations can be transformed into a linear system of the 
form \(Zx=y\) where:

\begin{equation}\label{eq:example-algebaric-approach-blocking-time}
    Z=
    \begin{pmatrix}
        -1 & p_o & 0 & 0 & 0 & 0 \\ %(1,2)
        p_s & -1 & p_o & 0 & 0 & 0 \\ %(1,3)
        0 & 1 & -1 & 0 & 0 & 0 \\ %(1,4)
        p_s & 0 & 0 & -1 & p_o & 0\\ %(2,2)
        0 & 0 & 0 & p_s & -1 & p_o \\ %(2,3)
        0 & 0 & 0 & 0 & 1 & -1 \\ %(2,4)
    \end{pmatrix},
    x=
    \begin{pmatrix}
        b(1,2) \\
        b(1,3) \\
        b(1,4) \\
        b(2,2) \\
        b(2,3) \\
        b(2,4) \\
    \end{pmatrix}, 
    y=
    \begin{pmatrix}
        -c(1,2) \\
        -c(1,3) \\
        -c(1,4) \\
        -c(2,2) \\
        -c(2,3) \\
        -c(2,4) \\
    \end{pmatrix}
\end{equation}

A more generalised form of the equations in 
(\ref{eq:example-algebaric-approach-blocking-time})
can thus be given for any value of \(C,T,N,M\) by:

\begin{align}
    b(1,T) =& c(1, T) + p_o b(1, T + 1) \label{eq:first_eq_of_blocking_general}\\
    b(1,T + 1) =& c(1, T + 1) + p_s(1, T) + p_o b(1, T + 1) \\
    b(1,T + 2) =& c(1, T + 2) + p_s(1, T + 1) + p_o b(1, T + 3) \\
    & \vdots \nonumber \\
    b(1, N) =& c(1, N) + b(1, N - 1) \\
    b(2, T) =& c(2, T) + p_s b(1, T) + p_o b(2, T + 1) \\
    b(2, T + 1) =& c(2, T + 1) + p_s b(2, T) + p_o b(2, T + 2) \\
    & \vdots \nonumber \\
    b(M, T) =& c(M, T) + b(M, T-1) \label{eq:last_eq_of_blocking_general}
\end{align}

The equivalent matrix form of the linear system of equations 
(\ref{eq:first_eq_of_blocking_general}) - (\ref{eq:last_eq_of_blocking_general})
is given by \(Zx=y\), where:
\begin{equation}\label{eq:general-algebaric-approach-blocking-time}
    \scalebox{0.9}{
        \(
        Z = 
        \begin{pmatrix}
            -1 & p_o & 0 & \dots & 0 & 0 & 0 & 0 & 0 & \dots & 0 & 0 \\ %(1,T)
            p_s & -1 & p_o & \dots & 0 & 0 & 0 & 0 & 0 & \dots & 0 & 0 \\ %(1,T+1)
            0 & p_s & -1 & \dots & 0 & 0 & 0 & 0 & 0 & \dots & 0 & 0 \\ %(1,T+2)
            \vdots & \vdots & \vdots & \ddots & \vdots & \vdots & \vdots & \vdots & 
            \vdots & \ddots & \vdots & \vdots \\ 
            0 & 0 & 0 & \dots & 1 & -1 & 0 & 0 & 0 & \dots & 0 & 0 \\ %(1,N)
            p_s & 0 & 0 & \dots & 0 & 0 & -1 & p_o & 0 & \dots & 0 & 0 \\ %(2,T)
            0 & 0 & 0 & \dots & 0 & 0 & p_s & -1 & p_o & \dots & 0 & 0 \\ %(2,T+1)
            \vdots & \vdots & \vdots & \ddots & \vdots & \vdots & \vdots & \vdots & 
            \vdots & \ddots & \vdots & \vdots \\ 
            0 & 0 & 0 & \dots & 0 & 0 & 0 & 0 & 0 & \dots & 1 & -1 \\ %(M,T)
        \end{pmatrix},
        x = 
        \begin{pmatrix}
            b(1,T) \\
            b(1,T+1) \\
            b(1,T+2) \\
            \vdots \\
            b(1,N) \\
            b(2,T) \\
            b(2,T+1) \\
            \vdots \\
            b(M,T) \\
        \end{pmatrix}, 
        y= 
        \begin{pmatrix}
            -c(1,T) \\
            -c(1,T+1) \\
            -c(1,T+2) \\
            \vdots \\
            -c(1,N) \\
            -c(2,T) \\
            -c(2,T+1) \\
            \vdots \\
            -c(M,T) \\
        \end{pmatrix}
        \)
    }
\end{equation}

Thus, having calculated the mean blocking time for all blocking states \(b(u,v)\), 
it only remains to put them together in a formula.
The resultant blocking time formula is given by:

\begin{equation}\label{eq:algebraic-blocking-time}
    B = \frac{\sum_{(u,v) \in S_A} \pi_{(u,v)} \; b(u,v)}{\sum_{(u,v) \in S_A} 
    \pi_{(u,v)}}
\end{equation}

\documentclass{article}

\usepackage{amsmath}
\usepackage{amsfonts} 
\usepackage{geometry}
\usepackage{multicol}
\usepackage{float}
% \usepackage{mathtools}
% \usepackage{graphicx}
% \usepackage{soul}
% \usepackage{indentfirst}
\usepackage{tikz}
\usetikzlibrary{calc, automata, chains, arrows.meta, math}
\setcounter{MaxMatrixCols}{20}


\title{A game theoretic model of the behavioural gaming that takes place at the EMS - ED interface}

\author{
    Michalis Panayides, 
    Paul Harper, 
    Vince Knight
}

\begin{document}

\maketitle

\input{Abstract/main.tex}


\newpage
\tableofcontents

\newpage
\input{Introduction/main.tex}

\newpage
\input{Game_theory_component/main.tex}

\newpage
\input{MarkovChain/markov_chain_model/main.tex}
\input{MarkovChain/expressions_from_pi/main.tex}
\input{MarkovChain/markov_example/main.tex}

\newpage
\input{BehaviouralMethodology/main.tex}

\newpage
\input{Application_EMS_ED/main.tex}

\newpage
\input{Conclusion/main.tex}


\end{document}

\newpage
\documentclass{article}

\usepackage{amsmath}
\usepackage{amsfonts} 
\usepackage{geometry}
\usepackage{multicol}
\usepackage{float}
% \usepackage{mathtools}
% \usepackage{graphicx}
% \usepackage{soul}
% \usepackage{indentfirst}
\usepackage{tikz}
\usetikzlibrary{calc, automata, chains, arrows.meta, math}
\setcounter{MaxMatrixCols}{20}


\title{A game theoretic model of the behavioural gaming that takes place at the EMS - ED interface}

\author{
    Michalis Panayides, 
    Paul Harper, 
    Vince Knight
}

\begin{document}

\maketitle

\input{Abstract/main.tex}


\newpage
\tableofcontents

\newpage
\input{Introduction/main.tex}

\newpage
\input{Game_theory_component/main.tex}

\newpage
\input{MarkovChain/markov_chain_model/main.tex}
\input{MarkovChain/expressions_from_pi/main.tex}
\input{MarkovChain/markov_example/main.tex}

\newpage
\input{BehaviouralMethodology/main.tex}

\newpage
\input{Application_EMS_ED/main.tex}

\newpage
\input{Conclusion/main.tex}


\end{document}

\newpage
\section{EMS-ED application}

\subsection{Application}

\subsection{Data analysis of generated problem}

\newpage
\documentclass{article}

\usepackage{amsmath}
\usepackage{amsfonts} 
\usepackage{geometry}
\usepackage{multicol}
\usepackage{float}
% \usepackage{mathtools}
% \usepackage{graphicx}
% \usepackage{soul}
% \usepackage{indentfirst}
\usepackage{tikz}
\usetikzlibrary{calc, automata, chains, arrows.meta, math}
\setcounter{MaxMatrixCols}{20}


\title{A game theoretic model of the behavioural gaming that takes place at the EMS - ED interface}

\author{
    Michalis Panayides, 
    Paul Harper, 
    Vince Knight
}

\begin{document}

\maketitle

\input{Abstract/main.tex}


\newpage
\tableofcontents

\newpage
\input{Introduction/main.tex}

\newpage
\input{Game_theory_component/main.tex}

\newpage
\input{MarkovChain/markov_chain_model/main.tex}
\input{MarkovChain/expressions_from_pi/main.tex}
\input{MarkovChain/markov_example/main.tex}

\newpage
\input{BehaviouralMethodology/main.tex}

\newpage
\input{Application_EMS_ED/main.tex}

\newpage
\input{Conclusion/main.tex}


\end{document}


\end{document}

\newpage
\documentclass{article}

\usepackage{amsmath}
\usepackage{amsfonts} 
\usepackage{geometry}
\usepackage{multicol}
\usepackage{float}
% \usepackage{mathtools}
% \usepackage{graphicx}
% \usepackage{soul}
% \usepackage{indentfirst}
\usepackage{tikz}
\usetikzlibrary{calc, automata, chains, arrows.meta, math}
\setcounter{MaxMatrixCols}{20}


\title{A game theoretic model of the behavioural gaming that takes place at the EMS - ED interface}

\author{
    Michalis Panayides, 
    Paul Harper, 
    Vince Knight
}

\begin{document}

\maketitle

\documentclass{article}

\usepackage{amsmath}
\usepackage{amsfonts} 
\usepackage{geometry}
\usepackage{multicol}
\usepackage{float}
% \usepackage{mathtools}
% \usepackage{graphicx}
% \usepackage{soul}
% \usepackage{indentfirst}
\usepackage{tikz}
\usetikzlibrary{calc, automata, chains, arrows.meta, math}
\setcounter{MaxMatrixCols}{20}


\title{A game theoretic model of the behavioural gaming that takes place at the EMS - ED interface}

\author{
    Michalis Panayides, 
    Paul Harper, 
    Vince Knight
}

\begin{document}

\maketitle

\input{Abstract/main.tex}


\newpage
\tableofcontents

\newpage
\input{Introduction/main.tex}

\newpage
\input{Game_theory_component/main.tex}

\newpage
\input{MarkovChain/markov_chain_model/main.tex}
\input{MarkovChain/expressions_from_pi/main.tex}
\input{MarkovChain/markov_example/main.tex}

\newpage
\input{BehaviouralMethodology/main.tex}

\newpage
\input{Application_EMS_ED/main.tex}

\newpage
\input{Conclusion/main.tex}


\end{document}


\newpage
\tableofcontents

\newpage
\documentclass{article}

\usepackage{amsmath}
\usepackage{amsfonts} 
\usepackage{geometry}
\usepackage{multicol}
\usepackage{float}
% \usepackage{mathtools}
% \usepackage{graphicx}
% \usepackage{soul}
% \usepackage{indentfirst}
\usepackage{tikz}
\usetikzlibrary{calc, automata, chains, arrows.meta, math}
\setcounter{MaxMatrixCols}{20}


\title{A game theoretic model of the behavioural gaming that takes place at the EMS - ED interface}

\author{
    Michalis Panayides, 
    Paul Harper, 
    Vince Knight
}

\begin{document}

\maketitle

\input{Abstract/main.tex}


\newpage
\tableofcontents

\newpage
\input{Introduction/main.tex}

\newpage
\input{Game_theory_component/main.tex}

\newpage
\input{MarkovChain/markov_chain_model/main.tex}
\input{MarkovChain/expressions_from_pi/main.tex}
\input{MarkovChain/markov_example/main.tex}

\newpage
\input{BehaviouralMethodology/main.tex}

\newpage
\input{Application_EMS_ED/main.tex}

\newpage
\input{Conclusion/main.tex}


\end{document}

\newpage
\documentclass{article}

\usepackage{amsmath}
\usepackage{amsfonts} 
\usepackage{geometry}
\usepackage{multicol}
\usepackage{float}
% \usepackage{mathtools}
% \usepackage{graphicx}
% \usepackage{soul}
% \usepackage{indentfirst}
\usepackage{tikz}
\usetikzlibrary{calc, automata, chains, arrows.meta, math}
\setcounter{MaxMatrixCols}{20}


\title{A game theoretic model of the behavioural gaming that takes place at the EMS - ED interface}

\author{
    Michalis Panayides, 
    Paul Harper, 
    Vince Knight
}

\begin{document}

\maketitle

\input{Abstract/main.tex}


\newpage
\tableofcontents

\newpage
\input{Introduction/main.tex}

\newpage
\input{Game_theory_component/main.tex}

\newpage
\input{MarkovChain/markov_chain_model/main.tex}
\input{MarkovChain/expressions_from_pi/main.tex}
\input{MarkovChain/markov_example/main.tex}

\newpage
\input{BehaviouralMethodology/main.tex}

\newpage
\input{Application_EMS_ED/main.tex}

\newpage
\input{Conclusion/main.tex}


\end{document}

\newpage
\documentclass{article}

\usepackage{amsmath}
\usepackage{amsfonts} 
\usepackage{geometry}
\usepackage{multicol}
\usepackage{float}
% \usepackage{mathtools}
% \usepackage{graphicx}
% \usepackage{soul}
% \usepackage{indentfirst}
\usepackage{tikz}
\usetikzlibrary{calc, automata, chains, arrows.meta, math}
\setcounter{MaxMatrixCols}{20}


\title{A game theoretic model of the behavioural gaming that takes place at the EMS - ED interface}

\author{
    Michalis Panayides, 
    Paul Harper, 
    Vince Knight
}

\begin{document}

\maketitle

\input{Abstract/main.tex}


\newpage
\tableofcontents

\newpage
\input{Introduction/main.tex}

\newpage
\input{Game_theory_component/main.tex}

\newpage
\input{MarkovChain/markov_chain_model/main.tex}
\input{MarkovChain/expressions_from_pi/main.tex}
\input{MarkovChain/markov_example/main.tex}

\newpage
\input{BehaviouralMethodology/main.tex}

\newpage
\input{Application_EMS_ED/main.tex}

\newpage
\input{Conclusion/main.tex}


\end{document}
\subsection{Performance Measures}
One may easily derive the average number of individuals that are at any given state 
using \( pi \). 
The average number of individuals in state \( i \) can be calculated by multiplying 
the number of individuals that are present in state \( i \) with the probability 
of being at that particular state (i.e \(\pi_i (u_i + v_i)\)). 
Using this logic it is possible to calculate any performance measures that are related 
to the mean number of individuals in the system.


Average number of people in the system: 
\begin{equation}
    L = \sum_{i=1}^{|\pi|} \pi_i (u_i + v_i)
\end{equation} 

Average number of people in the service centre: 
\begin{equation}
    L_H = \sum_{i=1}^{|\pi|} \pi_i v_i
\end{equation}

Average number of people in the buffer centre:
\begin{equation}
    L_A = \sum_{i=1}^{|\pi|} \pi_i u_i
\end{equation}

Consequently getting the performance measures that are related to the duration of 
time is not as straightforward. 
Such performance measures are the mean waiting time in the system and the mean time 
blocked in the system. 
Under the scope of this study three approaches have been considered to calculate these 
performance measures; a direct approach, a recursive algorithm and consequently a
closed-form formula.

The research question that needs to be answered here is: ``When a class 1/2 
individuals enters the system, what is the expected time that they will have to 
wait?''. 
In order to formulate the answer to that question one needs to consider all possible 
scenarios of what state the system can be in when an individual arrives. 
Furthermore, different formulas arises for class 1 individuals 
and a different one for class 2 individuals.

\subsubsection{Mean waiting time} 
Upon closer inspection of the recursive formula a more compact formula can arise. 
The equivalent closed-form formula eliminates the need for recursion and thus makes 
the computation of waiting times much more efficient. 
Just like in the recursive part there are two formulas; one for \textit{class 1} 
and one for class 2 individuals. 
The formulas are given by:

\begin{equation} \label{eq:closed_form_waiting_others}
    W^{(1)} = \frac{\sum_{\substack{(u,v) \, \in S_A^{(1)} \\ v \geq C}} 
    \frac{1}{C \mu} \times (v-C+1) \times \pi(u,v)}{\sum_{(u,v) \, 
    \in S_A^{(1)}} \pi(u,v)}
\end{equation}
    
\begin{equation}\label{eq:closed_form_waiting_ambulance}
    W^{(2)} = \frac{\sum_{\substack{(u,v) \, \in S_A^{(2)} \\ min(v,T) \geq C}} 
    \frac{1}{C \mu} \times (\min(v+1,T)-C) \times \pi(u,v)}{\sum_{(u,v) \, 
    \in S_A^{(2)}} \pi(u,v)}
\end{equation}

Note here that the summation, in both equations \ref{eq:closed_form_waiting_others} 
and \ref{eq:closed_form_waiting_ambulance}, goes through all states in the set of 
accepting 
states of either class 1 or class 2 individuals respectively, where a wait 
incurs. 
In equation \ref{eq:closed_form_waiting_others} that includes all states \((u,v)\) 
in the set of accepting states of class 1 individuals such that \( v \geq C\); i.e. 
whenever an arrival occurs and the system is at a state where the number of individuals 
in the system is more than or equal to $C$. 
Consequently, for the states that are included in the summation the expression 
\( v-C+1 \) indicates the amount of people in service one would have to wait for 
upon arrival at the hospital.

Additionally, the minimisation function in equation 
\ref{eq:closed_form_waiting_ambulance} 
ensures that when a class 2 individual arrives at any state 
that is greater than the predetermined threshold, the wait that the individual will 
have to endure remains the same. 
In essence, the expression \(\min(v+1,T) - C\) returns the number of people in line 
in front of a particular individual upon arrival.


\subsubsection{Overall Waiting Time}

Consequently, the overall waiting time should can be estimated by a linear combination 
of the waiting times of class 1 and class 2 individuals. 
The overall waiting time can be then given by the following equation where \(c_1\) 
and \(c_2\) are the coefficients of each individual's type waiting time:

\begin{equation}\label{overall_waiting_time_coeff}
    W = c_1 W^{(1)} + c_2 W^{(2)}
\end{equation}

The two coefficients represent the proportion of individuals of each type that 
traversed through the model. 
Theoretically, getting these percentages should be as simple as looking at the arrival 
rates of each type but in practise if the service centre or the buffer centre 
is full, some individuals may be lost to the system. 
Thus, one should account for the probability that an individual is lost to the system. 
This probability can be easily calculated by using the two sets of accepting states 
\(S_A^{(2)}\) and \(S_A^{(1)}\) defined earlier in equations.
Let us define here the probability, for either class type, that an individual 
is not lost in the system by:

\begin{equation*}
    P(L'_1) = \sum_{(u,v) \, \in S_A^{(1)}} \pi(u,v) \hspace{2cm}
    P(L'_2) = \sum_{(u,v) \, \in S_A^{(2)}} \pi(u,v)
\end{equation*}

Having defined these probabilities one may combine them with the arrival rates of 
each class type in such a way to get the expected proportions of class 1 and 
class 2 individuals in the model. 
Thus, by using these values as the coefficient of equation 
\ref{overall_waiting_time_coeff} 
the resultant equation can be used to get the overall waiting time. 
Note here that the equation below gets the overall waiting time for both the recursive 
and the closed-form formula.

\begin{equation}\label{overall_waiting_time}
    W = \frac{\lambda_1 P(L'_1)}{\lambda_2 P(L'_2) + \lambda_1 P(L'_1)} W^{(1)} + 
    \frac{\lambda_2 P(L'_2)}{\lambda_2 P(L'_2) + \lambda_1 P(L'_1)} W^{(2)}
\end{equation}



\subsubsection{Mean blocking time}
Unlike the waiting time, the blocking time is only calculated for class 2 individuals.  
That is because class 1 individuals cannot be blocked. 
Thus, one only needs to consider the pathway of class 2 individuals to get the 
mean blocking time of the system. 
Blocking occurs at states \((u,v)\) where \(u > 0 \). 
Thus, the set of blocking states can be defined as:

\begin{equation*}
    S_b = \{(u,v) \in S \; | \; u > 0\}
\end{equation*}
 
In order to not consider individuals that will be lost to the system, the set of 
accepting states needs to be taken into account. The set of accepting states is given by:

\begin{equation*}
    S_A^{(2)}=
    \begin{cases}
        \{(u, v) \in S \; | \; u < M \} & \textbf{if } T \leq N\\
        \{(u, v) \in S \; | \; v < N \} & \textbf{otherwise}
    \end{cases}
\end{equation*}

For the waiting time formula,
the mean sojourn time for each state was considered,
ignoring any arrivals. Here, the same approach is used but ignoring only class 2
arrivals. That is because for the waiting time formula, once an individual enters 
the service centre (i.e. starts waiting) any individual arriving after them will 
not affect their
pathway. That is not the case for blocking time. When a class 2 individual is 
blocked, 
any class 1 individual that arrives will cause the blocked individual to remain 
blocked for more time. Therefore, class 1 arrivals are considered here:

\begin{equation}\label{eq:time_in_state_blocking_time}
    c(u,v) = 
    \begin{cases}
        \frac{1}{\min(v,C) \mu}, & \text{if } v = C\\
        \frac{1}{\min(v,C) \mu + \lambda_1}, & \text{otherwise}
    \end{cases}
\end{equation}
 
In equation \ref{eq:time_in_state_blocking_time}, both service completions and 
class 1 arrivals are considered. 
Thus, from a blocked individual's perspective whenever the system moves from one 
state \((u,v)\)
to another state it can either:

\begin{itemize}
    \item be because of a service being completed: we will denote the probability 
    of this happening by \(p_s(u,v)\). 
    \item be because of an arrival of an individual of class 1: denoting such 
    probability by \(p_o(u,v)\).
\end{itemize}
The probabilities are given by:

\begin{equation*}
    p_s(u,v) = \frac{\min(v,C)\mu}{\lambda_1 + \min(v,C)\mu}, \qquad
    p_o(u,v) = \frac{\lambda_1}{\lambda_1 + \min(v,C)\mu}
\end{equation*}


Having defined \(c(u,v)\) and \(S_b\) a formula for the blocking time that is
expected to occur at each state can be given by:

\begin{equation}\label{eq:blocking-time-at-each-state}
    b(u,v) = 
    \begin{cases} 
        0, & \textbf{if } (u,v) \notin S_b \\
        c(u,v) + b(u - 1, v), & \textbf{if } v = N = T\\
        c(u,v) + b(u, v-1), & \textbf{if } v = N \neq T \\
        c(u,v) + p_s(u,v) b(u-1, v) + p_o(u,v) b(u, v+1), & \textbf{if } u > 0 
        \textbf{ and } v = T \\
        c(u,v) + p_s(u,v) b(u, v-1) + p_o(u,v) b(u, v+1), & \textbf{otherwise} \\
    \end{cases}
\end{equation}

Equation 
(\ref{eq:blocking-time-at-each-state}) will not be solved recursively. 
A direct approach will be used to solve this equation here. 
By enumerating all equations of (\ref{eq:blocking-time-at-each-state}) for all 
states \((u,v)\) that belong in \(S_b\) 
a system of linear equations arises where the unknown variables are all the \(b(u,v)\)
terms.
For instance, let us consider a Markov model where \(C=2, T=3, N=6, M=2\). 
The Markov model is shown in Figure \ref{fig:example-algeb-blocking}
and the equivalent equations are 
(\ref{eq:first_eq_of_blocking_example})-(\ref{eq:last_eq_of_blocking_example}).
The equations considered here are only the ones that correspond to the blocking 
states.

\begin{multicols*}{2}
    \begin{figure}[H]
        \scalebox{0.50}{\input{MarkovChain/expressions_from_pi/example_model_2362/main.tex}}
        \caption{Example of Markov chain}
        \label{fig:example-algeb-blocking}
    \end{figure}
    \columnbreak
    \begin{align}
        b(1,2) &= c(1,2) + p_o b(1,3) \label{eq:first_eq_of_blocking_example} \\
        b(1,3) &= c(1,3) + p_s b(1,2) + p_o b(1,4) \\
        b(1,4) &= c(1,4) + b(1,3) \\
        b(2,2) &= c(2,2) + p_s b(1,2) + p_o b(2,3) \\
        b(2,3) &= c(2,3) + p_s b(2,2) + p_o b(1,4) \\
        b(2,4) &= c(2,4) + b(2,3)\label{eq:last_eq_of_blocking_example}
    \end{align}
\end{multicols*}

Additionally, the above equations can be transformed into a linear system of the 
form \(Zx=y\) where:

\begin{equation}\label{eq:example-algebaric-approach-blocking-time}
    Z=
    \begin{pmatrix}
        -1 & p_o & 0 & 0 & 0 & 0 \\ %(1,2)
        p_s & -1 & p_o & 0 & 0 & 0 \\ %(1,3)
        0 & 1 & -1 & 0 & 0 & 0 \\ %(1,4)
        p_s & 0 & 0 & -1 & p_o & 0\\ %(2,2)
        0 & 0 & 0 & p_s & -1 & p_o \\ %(2,3)
        0 & 0 & 0 & 0 & 1 & -1 \\ %(2,4)
    \end{pmatrix},
    x=
    \begin{pmatrix}
        b(1,2) \\
        b(1,3) \\
        b(1,4) \\
        b(2,2) \\
        b(2,3) \\
        b(2,4) \\
    \end{pmatrix}, 
    y=
    \begin{pmatrix}
        -c(1,2) \\
        -c(1,3) \\
        -c(1,4) \\
        -c(2,2) \\
        -c(2,3) \\
        -c(2,4) \\
    \end{pmatrix}
\end{equation}

A more generalised form of the equations in 
(\ref{eq:example-algebaric-approach-blocking-time})
can thus be given for any value of \(C,T,N,M\) by:

\begin{align}
    b(1,T) =& c(1, T) + p_o b(1, T + 1) \label{eq:first_eq_of_blocking_general}\\
    b(1,T + 1) =& c(1, T + 1) + p_s(1, T) + p_o b(1, T + 1) \\
    b(1,T + 2) =& c(1, T + 2) + p_s(1, T + 1) + p_o b(1, T + 3) \\
    & \vdots \nonumber \\
    b(1, N) =& c(1, N) + b(1, N - 1) \\
    b(2, T) =& c(2, T) + p_s b(1, T) + p_o b(2, T + 1) \\
    b(2, T + 1) =& c(2, T + 1) + p_s b(2, T) + p_o b(2, T + 2) \\
    & \vdots \nonumber \\
    b(M, T) =& c(M, T) + b(M, T-1) \label{eq:last_eq_of_blocking_general}
\end{align}

The equivalent matrix form of the linear system of equations 
(\ref{eq:first_eq_of_blocking_general}) - (\ref{eq:last_eq_of_blocking_general})
is given by \(Zx=y\), where:
\begin{equation}\label{eq:general-algebaric-approach-blocking-time}
    \scalebox{0.9}{
        \(
        Z = 
        \begin{pmatrix}
            -1 & p_o & 0 & \dots & 0 & 0 & 0 & 0 & 0 & \dots & 0 & 0 \\ %(1,T)
            p_s & -1 & p_o & \dots & 0 & 0 & 0 & 0 & 0 & \dots & 0 & 0 \\ %(1,T+1)
            0 & p_s & -1 & \dots & 0 & 0 & 0 & 0 & 0 & \dots & 0 & 0 \\ %(1,T+2)
            \vdots & \vdots & \vdots & \ddots & \vdots & \vdots & \vdots & \vdots & 
            \vdots & \ddots & \vdots & \vdots \\ 
            0 & 0 & 0 & \dots & 1 & -1 & 0 & 0 & 0 & \dots & 0 & 0 \\ %(1,N)
            p_s & 0 & 0 & \dots & 0 & 0 & -1 & p_o & 0 & \dots & 0 & 0 \\ %(2,T)
            0 & 0 & 0 & \dots & 0 & 0 & p_s & -1 & p_o & \dots & 0 & 0 \\ %(2,T+1)
            \vdots & \vdots & \vdots & \ddots & \vdots & \vdots & \vdots & \vdots & 
            \vdots & \ddots & \vdots & \vdots \\ 
            0 & 0 & 0 & \dots & 0 & 0 & 0 & 0 & 0 & \dots & 1 & -1 \\ %(M,T)
        \end{pmatrix},
        x = 
        \begin{pmatrix}
            b(1,T) \\
            b(1,T+1) \\
            b(1,T+2) \\
            \vdots \\
            b(1,N) \\
            b(2,T) \\
            b(2,T+1) \\
            \vdots \\
            b(M,T) \\
        \end{pmatrix}, 
        y= 
        \begin{pmatrix}
            -c(1,T) \\
            -c(1,T+1) \\
            -c(1,T+2) \\
            \vdots \\
            -c(1,N) \\
            -c(2,T) \\
            -c(2,T+1) \\
            \vdots \\
            -c(M,T) \\
        \end{pmatrix}
        \)
    }
\end{equation}

Thus, having calculated the mean blocking time for all blocking states \(b(u,v)\), 
it only remains to put them together in a formula.
The resultant blocking time formula is given by:

\begin{equation}\label{eq:algebraic-blocking-time}
    B = \frac{\sum_{(u,v) \in S_A} \pi_{(u,v)} \; b(u,v)}{\sum_{(u,v) \in S_A} 
    \pi_{(u,v)}}
\end{equation}

\documentclass{article}

\usepackage{amsmath}
\usepackage{amsfonts} 
\usepackage{geometry}
\usepackage{multicol}
\usepackage{float}
% \usepackage{mathtools}
% \usepackage{graphicx}
% \usepackage{soul}
% \usepackage{indentfirst}
\usepackage{tikz}
\usetikzlibrary{calc, automata, chains, arrows.meta, math}
\setcounter{MaxMatrixCols}{20}


\title{A game theoretic model of the behavioural gaming that takes place at the EMS - ED interface}

\author{
    Michalis Panayides, 
    Paul Harper, 
    Vince Knight
}

\begin{document}

\maketitle

\input{Abstract/main.tex}


\newpage
\tableofcontents

\newpage
\input{Introduction/main.tex}

\newpage
\input{Game_theory_component/main.tex}

\newpage
\input{MarkovChain/markov_chain_model/main.tex}
\input{MarkovChain/expressions_from_pi/main.tex}
\input{MarkovChain/markov_example/main.tex}

\newpage
\input{BehaviouralMethodology/main.tex}

\newpage
\input{Application_EMS_ED/main.tex}

\newpage
\input{Conclusion/main.tex}


\end{document}

\newpage
\documentclass{article}

\usepackage{amsmath}
\usepackage{amsfonts} 
\usepackage{geometry}
\usepackage{multicol}
\usepackage{float}
% \usepackage{mathtools}
% \usepackage{graphicx}
% \usepackage{soul}
% \usepackage{indentfirst}
\usepackage{tikz}
\usetikzlibrary{calc, automata, chains, arrows.meta, math}
\setcounter{MaxMatrixCols}{20}


\title{A game theoretic model of the behavioural gaming that takes place at the EMS - ED interface}

\author{
    Michalis Panayides, 
    Paul Harper, 
    Vince Knight
}

\begin{document}

\maketitle

\input{Abstract/main.tex}


\newpage
\tableofcontents

\newpage
\input{Introduction/main.tex}

\newpage
\input{Game_theory_component/main.tex}

\newpage
\input{MarkovChain/markov_chain_model/main.tex}
\input{MarkovChain/expressions_from_pi/main.tex}
\input{MarkovChain/markov_example/main.tex}

\newpage
\input{BehaviouralMethodology/main.tex}

\newpage
\input{Application_EMS_ED/main.tex}

\newpage
\input{Conclusion/main.tex}


\end{document}

\newpage
\section{EMS-ED application}

\subsection{Application}

\subsection{Data analysis of generated problem}

\newpage
\documentclass{article}

\usepackage{amsmath}
\usepackage{amsfonts} 
\usepackage{geometry}
\usepackage{multicol}
\usepackage{float}
% \usepackage{mathtools}
% \usepackage{graphicx}
% \usepackage{soul}
% \usepackage{indentfirst}
\usepackage{tikz}
\usetikzlibrary{calc, automata, chains, arrows.meta, math}
\setcounter{MaxMatrixCols}{20}


\title{A game theoretic model of the behavioural gaming that takes place at the EMS - ED interface}

\author{
    Michalis Panayides, 
    Paul Harper, 
    Vince Knight
}

\begin{document}

\maketitle

\input{Abstract/main.tex}


\newpage
\tableofcontents

\newpage
\input{Introduction/main.tex}

\newpage
\input{Game_theory_component/main.tex}

\newpage
\input{MarkovChain/markov_chain_model/main.tex}
\input{MarkovChain/expressions_from_pi/main.tex}
\input{MarkovChain/markov_example/main.tex}

\newpage
\input{BehaviouralMethodology/main.tex}

\newpage
\input{Application_EMS_ED/main.tex}

\newpage
\input{Conclusion/main.tex}


\end{document}


\end{document}

\newpage
\documentclass{article}

\usepackage{amsmath}
\usepackage{amsfonts} 
\usepackage{geometry}
\usepackage{multicol}
\usepackage{float}
% \usepackage{mathtools}
% \usepackage{graphicx}
% \usepackage{soul}
% \usepackage{indentfirst}
\usepackage{tikz}
\usetikzlibrary{calc, automata, chains, arrows.meta, math}
\setcounter{MaxMatrixCols}{20}


\title{A game theoretic model of the behavioural gaming that takes place at the EMS - ED interface}

\author{
    Michalis Panayides, 
    Paul Harper, 
    Vince Knight
}

\begin{document}

\maketitle

\documentclass{article}

\usepackage{amsmath}
\usepackage{amsfonts} 
\usepackage{geometry}
\usepackage{multicol}
\usepackage{float}
% \usepackage{mathtools}
% \usepackage{graphicx}
% \usepackage{soul}
% \usepackage{indentfirst}
\usepackage{tikz}
\usetikzlibrary{calc, automata, chains, arrows.meta, math}
\setcounter{MaxMatrixCols}{20}


\title{A game theoretic model of the behavioural gaming that takes place at the EMS - ED interface}

\author{
    Michalis Panayides, 
    Paul Harper, 
    Vince Knight
}

\begin{document}

\maketitle

\input{Abstract/main.tex}


\newpage
\tableofcontents

\newpage
\input{Introduction/main.tex}

\newpage
\input{Game_theory_component/main.tex}

\newpage
\input{MarkovChain/markov_chain_model/main.tex}
\input{MarkovChain/expressions_from_pi/main.tex}
\input{MarkovChain/markov_example/main.tex}

\newpage
\input{BehaviouralMethodology/main.tex}

\newpage
\input{Application_EMS_ED/main.tex}

\newpage
\input{Conclusion/main.tex}


\end{document}


\newpage
\tableofcontents

\newpage
\documentclass{article}

\usepackage{amsmath}
\usepackage{amsfonts} 
\usepackage{geometry}
\usepackage{multicol}
\usepackage{float}
% \usepackage{mathtools}
% \usepackage{graphicx}
% \usepackage{soul}
% \usepackage{indentfirst}
\usepackage{tikz}
\usetikzlibrary{calc, automata, chains, arrows.meta, math}
\setcounter{MaxMatrixCols}{20}


\title{A game theoretic model of the behavioural gaming that takes place at the EMS - ED interface}

\author{
    Michalis Panayides, 
    Paul Harper, 
    Vince Knight
}

\begin{document}

\maketitle

\input{Abstract/main.tex}


\newpage
\tableofcontents

\newpage
\input{Introduction/main.tex}

\newpage
\input{Game_theory_component/main.tex}

\newpage
\input{MarkovChain/markov_chain_model/main.tex}
\input{MarkovChain/expressions_from_pi/main.tex}
\input{MarkovChain/markov_example/main.tex}

\newpage
\input{BehaviouralMethodology/main.tex}

\newpage
\input{Application_EMS_ED/main.tex}

\newpage
\input{Conclusion/main.tex}


\end{document}

\newpage
\documentclass{article}

\usepackage{amsmath}
\usepackage{amsfonts} 
\usepackage{geometry}
\usepackage{multicol}
\usepackage{float}
% \usepackage{mathtools}
% \usepackage{graphicx}
% \usepackage{soul}
% \usepackage{indentfirst}
\usepackage{tikz}
\usetikzlibrary{calc, automata, chains, arrows.meta, math}
\setcounter{MaxMatrixCols}{20}


\title{A game theoretic model of the behavioural gaming that takes place at the EMS - ED interface}

\author{
    Michalis Panayides, 
    Paul Harper, 
    Vince Knight
}

\begin{document}

\maketitle

\input{Abstract/main.tex}


\newpage
\tableofcontents

\newpage
\input{Introduction/main.tex}

\newpage
\input{Game_theory_component/main.tex}

\newpage
\input{MarkovChain/markov_chain_model/main.tex}
\input{MarkovChain/expressions_from_pi/main.tex}
\input{MarkovChain/markov_example/main.tex}

\newpage
\input{BehaviouralMethodology/main.tex}

\newpage
\input{Application_EMS_ED/main.tex}

\newpage
\input{Conclusion/main.tex}


\end{document}

\newpage
\documentclass{article}

\usepackage{amsmath}
\usepackage{amsfonts} 
\usepackage{geometry}
\usepackage{multicol}
\usepackage{float}
% \usepackage{mathtools}
% \usepackage{graphicx}
% \usepackage{soul}
% \usepackage{indentfirst}
\usepackage{tikz}
\usetikzlibrary{calc, automata, chains, arrows.meta, math}
\setcounter{MaxMatrixCols}{20}


\title{A game theoretic model of the behavioural gaming that takes place at the EMS - ED interface}

\author{
    Michalis Panayides, 
    Paul Harper, 
    Vince Knight
}

\begin{document}

\maketitle

\input{Abstract/main.tex}


\newpage
\tableofcontents

\newpage
\input{Introduction/main.tex}

\newpage
\input{Game_theory_component/main.tex}

\newpage
\input{MarkovChain/markov_chain_model/main.tex}
\input{MarkovChain/expressions_from_pi/main.tex}
\input{MarkovChain/markov_example/main.tex}

\newpage
\input{BehaviouralMethodology/main.tex}

\newpage
\input{Application_EMS_ED/main.tex}

\newpage
\input{Conclusion/main.tex}


\end{document}
\subsection{Performance Measures}
One may easily derive the average number of individuals that are at any given state 
using \( pi \). 
The average number of individuals in state \( i \) can be calculated by multiplying 
the number of individuals that are present in state \( i \) with the probability 
of being at that particular state (i.e \(\pi_i (u_i + v_i)\)). 
Using this logic it is possible to calculate any performance measures that are related 
to the mean number of individuals in the system.


Average number of people in the system: 
\begin{equation}
    L = \sum_{i=1}^{|\pi|} \pi_i (u_i + v_i)
\end{equation} 

Average number of people in the service centre: 
\begin{equation}
    L_H = \sum_{i=1}^{|\pi|} \pi_i v_i
\end{equation}

Average number of people in the buffer centre:
\begin{equation}
    L_A = \sum_{i=1}^{|\pi|} \pi_i u_i
\end{equation}

Consequently getting the performance measures that are related to the duration of 
time is not as straightforward. 
Such performance measures are the mean waiting time in the system and the mean time 
blocked in the system. 
Under the scope of this study three approaches have been considered to calculate these 
performance measures; a direct approach, a recursive algorithm and consequently a
closed-form formula.

The research question that needs to be answered here is: ``When a class 1/2 
individuals enters the system, what is the expected time that they will have to 
wait?''. 
In order to formulate the answer to that question one needs to consider all possible 
scenarios of what state the system can be in when an individual arrives. 
Furthermore, different formulas arises for class 1 individuals 
and a different one for class 2 individuals.

\subsubsection{Mean waiting time} 
Upon closer inspection of the recursive formula a more compact formula can arise. 
The equivalent closed-form formula eliminates the need for recursion and thus makes 
the computation of waiting times much more efficient. 
Just like in the recursive part there are two formulas; one for \textit{class 1} 
and one for class 2 individuals. 
The formulas are given by:

\begin{equation} \label{eq:closed_form_waiting_others}
    W^{(1)} = \frac{\sum_{\substack{(u,v) \, \in S_A^{(1)} \\ v \geq C}} 
    \frac{1}{C \mu} \times (v-C+1) \times \pi(u,v)}{\sum_{(u,v) \, 
    \in S_A^{(1)}} \pi(u,v)}
\end{equation}
    
\begin{equation}\label{eq:closed_form_waiting_ambulance}
    W^{(2)} = \frac{\sum_{\substack{(u,v) \, \in S_A^{(2)} \\ min(v,T) \geq C}} 
    \frac{1}{C \mu} \times (\min(v+1,T)-C) \times \pi(u,v)}{\sum_{(u,v) \, 
    \in S_A^{(2)}} \pi(u,v)}
\end{equation}

Note here that the summation, in both equations \ref{eq:closed_form_waiting_others} 
and \ref{eq:closed_form_waiting_ambulance}, goes through all states in the set of 
accepting 
states of either class 1 or class 2 individuals respectively, where a wait 
incurs. 
In equation \ref{eq:closed_form_waiting_others} that includes all states \((u,v)\) 
in the set of accepting states of class 1 individuals such that \( v \geq C\); i.e. 
whenever an arrival occurs and the system is at a state where the number of individuals 
in the system is more than or equal to $C$. 
Consequently, for the states that are included in the summation the expression 
\( v-C+1 \) indicates the amount of people in service one would have to wait for 
upon arrival at the hospital.

Additionally, the minimisation function in equation 
\ref{eq:closed_form_waiting_ambulance} 
ensures that when a class 2 individual arrives at any state 
that is greater than the predetermined threshold, the wait that the individual will 
have to endure remains the same. 
In essence, the expression \(\min(v+1,T) - C\) returns the number of people in line 
in front of a particular individual upon arrival.


\subsubsection{Overall Waiting Time}

Consequently, the overall waiting time should can be estimated by a linear combination 
of the waiting times of class 1 and class 2 individuals. 
The overall waiting time can be then given by the following equation where \(c_1\) 
and \(c_2\) are the coefficients of each individual's type waiting time:

\begin{equation}\label{overall_waiting_time_coeff}
    W = c_1 W^{(1)} + c_2 W^{(2)}
\end{equation}

The two coefficients represent the proportion of individuals of each type that 
traversed through the model. 
Theoretically, getting these percentages should be as simple as looking at the arrival 
rates of each type but in practise if the service centre or the buffer centre 
is full, some individuals may be lost to the system. 
Thus, one should account for the probability that an individual is lost to the system. 
This probability can be easily calculated by using the two sets of accepting states 
\(S_A^{(2)}\) and \(S_A^{(1)}\) defined earlier in equations.
Let us define here the probability, for either class type, that an individual 
is not lost in the system by:

\begin{equation*}
    P(L'_1) = \sum_{(u,v) \, \in S_A^{(1)}} \pi(u,v) \hspace{2cm}
    P(L'_2) = \sum_{(u,v) \, \in S_A^{(2)}} \pi(u,v)
\end{equation*}

Having defined these probabilities one may combine them with the arrival rates of 
each class type in such a way to get the expected proportions of class 1 and 
class 2 individuals in the model. 
Thus, by using these values as the coefficient of equation 
\ref{overall_waiting_time_coeff} 
the resultant equation can be used to get the overall waiting time. 
Note here that the equation below gets the overall waiting time for both the recursive 
and the closed-form formula.

\begin{equation}\label{overall_waiting_time}
    W = \frac{\lambda_1 P(L'_1)}{\lambda_2 P(L'_2) + \lambda_1 P(L'_1)} W^{(1)} + 
    \frac{\lambda_2 P(L'_2)}{\lambda_2 P(L'_2) + \lambda_1 P(L'_1)} W^{(2)}
\end{equation}



\subsubsection{Mean blocking time}
Unlike the waiting time, the blocking time is only calculated for class 2 individuals.  
That is because class 1 individuals cannot be blocked. 
Thus, one only needs to consider the pathway of class 2 individuals to get the 
mean blocking time of the system. 
Blocking occurs at states \((u,v)\) where \(u > 0 \). 
Thus, the set of blocking states can be defined as:

\begin{equation*}
    S_b = \{(u,v) \in S \; | \; u > 0\}
\end{equation*}
 
In order to not consider individuals that will be lost to the system, the set of 
accepting states needs to be taken into account. The set of accepting states is given by:

\begin{equation*}
    S_A^{(2)}=
    \begin{cases}
        \{(u, v) \in S \; | \; u < M \} & \textbf{if } T \leq N\\
        \{(u, v) \in S \; | \; v < N \} & \textbf{otherwise}
    \end{cases}
\end{equation*}

For the waiting time formula,
the mean sojourn time for each state was considered,
ignoring any arrivals. Here, the same approach is used but ignoring only class 2
arrivals. That is because for the waiting time formula, once an individual enters 
the service centre (i.e. starts waiting) any individual arriving after them will 
not affect their
pathway. That is not the case for blocking time. When a class 2 individual is 
blocked, 
any class 1 individual that arrives will cause the blocked individual to remain 
blocked for more time. Therefore, class 1 arrivals are considered here:

\begin{equation}\label{eq:time_in_state_blocking_time}
    c(u,v) = 
    \begin{cases}
        \frac{1}{\min(v,C) \mu}, & \text{if } v = C\\
        \frac{1}{\min(v,C) \mu + \lambda_1}, & \text{otherwise}
    \end{cases}
\end{equation}
 
In equation \ref{eq:time_in_state_blocking_time}, both service completions and 
class 1 arrivals are considered. 
Thus, from a blocked individual's perspective whenever the system moves from one 
state \((u,v)\)
to another state it can either:

\begin{itemize}
    \item be because of a service being completed: we will denote the probability 
    of this happening by \(p_s(u,v)\). 
    \item be because of an arrival of an individual of class 1: denoting such 
    probability by \(p_o(u,v)\).
\end{itemize}
The probabilities are given by:

\begin{equation*}
    p_s(u,v) = \frac{\min(v,C)\mu}{\lambda_1 + \min(v,C)\mu}, \qquad
    p_o(u,v) = \frac{\lambda_1}{\lambda_1 + \min(v,C)\mu}
\end{equation*}


Having defined \(c(u,v)\) and \(S_b\) a formula for the blocking time that is
expected to occur at each state can be given by:

\begin{equation}\label{eq:blocking-time-at-each-state}
    b(u,v) = 
    \begin{cases} 
        0, & \textbf{if } (u,v) \notin S_b \\
        c(u,v) + b(u - 1, v), & \textbf{if } v = N = T\\
        c(u,v) + b(u, v-1), & \textbf{if } v = N \neq T \\
        c(u,v) + p_s(u,v) b(u-1, v) + p_o(u,v) b(u, v+1), & \textbf{if } u > 0 
        \textbf{ and } v = T \\
        c(u,v) + p_s(u,v) b(u, v-1) + p_o(u,v) b(u, v+1), & \textbf{otherwise} \\
    \end{cases}
\end{equation}

Equation 
(\ref{eq:blocking-time-at-each-state}) will not be solved recursively. 
A direct approach will be used to solve this equation here. 
By enumerating all equations of (\ref{eq:blocking-time-at-each-state}) for all 
states \((u,v)\) that belong in \(S_b\) 
a system of linear equations arises where the unknown variables are all the \(b(u,v)\)
terms.
For instance, let us consider a Markov model where \(C=2, T=3, N=6, M=2\). 
The Markov model is shown in Figure \ref{fig:example-algeb-blocking}
and the equivalent equations are 
(\ref{eq:first_eq_of_blocking_example})-(\ref{eq:last_eq_of_blocking_example}).
The equations considered here are only the ones that correspond to the blocking 
states.

\begin{multicols*}{2}
    \begin{figure}[H]
        \scalebox{0.50}{\input{MarkovChain/expressions_from_pi/example_model_2362/main.tex}}
        \caption{Example of Markov chain}
        \label{fig:example-algeb-blocking}
    \end{figure}
    \columnbreak
    \begin{align}
        b(1,2) &= c(1,2) + p_o b(1,3) \label{eq:first_eq_of_blocking_example} \\
        b(1,3) &= c(1,3) + p_s b(1,2) + p_o b(1,4) \\
        b(1,4) &= c(1,4) + b(1,3) \\
        b(2,2) &= c(2,2) + p_s b(1,2) + p_o b(2,3) \\
        b(2,3) &= c(2,3) + p_s b(2,2) + p_o b(1,4) \\
        b(2,4) &= c(2,4) + b(2,3)\label{eq:last_eq_of_blocking_example}
    \end{align}
\end{multicols*}

Additionally, the above equations can be transformed into a linear system of the 
form \(Zx=y\) where:

\begin{equation}\label{eq:example-algebaric-approach-blocking-time}
    Z=
    \begin{pmatrix}
        -1 & p_o & 0 & 0 & 0 & 0 \\ %(1,2)
        p_s & -1 & p_o & 0 & 0 & 0 \\ %(1,3)
        0 & 1 & -1 & 0 & 0 & 0 \\ %(1,4)
        p_s & 0 & 0 & -1 & p_o & 0\\ %(2,2)
        0 & 0 & 0 & p_s & -1 & p_o \\ %(2,3)
        0 & 0 & 0 & 0 & 1 & -1 \\ %(2,4)
    \end{pmatrix},
    x=
    \begin{pmatrix}
        b(1,2) \\
        b(1,3) \\
        b(1,4) \\
        b(2,2) \\
        b(2,3) \\
        b(2,4) \\
    \end{pmatrix}, 
    y=
    \begin{pmatrix}
        -c(1,2) \\
        -c(1,3) \\
        -c(1,4) \\
        -c(2,2) \\
        -c(2,3) \\
        -c(2,4) \\
    \end{pmatrix}
\end{equation}

A more generalised form of the equations in 
(\ref{eq:example-algebaric-approach-blocking-time})
can thus be given for any value of \(C,T,N,M\) by:

\begin{align}
    b(1,T) =& c(1, T) + p_o b(1, T + 1) \label{eq:first_eq_of_blocking_general}\\
    b(1,T + 1) =& c(1, T + 1) + p_s(1, T) + p_o b(1, T + 1) \\
    b(1,T + 2) =& c(1, T + 2) + p_s(1, T + 1) + p_o b(1, T + 3) \\
    & \vdots \nonumber \\
    b(1, N) =& c(1, N) + b(1, N - 1) \\
    b(2, T) =& c(2, T) + p_s b(1, T) + p_o b(2, T + 1) \\
    b(2, T + 1) =& c(2, T + 1) + p_s b(2, T) + p_o b(2, T + 2) \\
    & \vdots \nonumber \\
    b(M, T) =& c(M, T) + b(M, T-1) \label{eq:last_eq_of_blocking_general}
\end{align}

The equivalent matrix form of the linear system of equations 
(\ref{eq:first_eq_of_blocking_general}) - (\ref{eq:last_eq_of_blocking_general})
is given by \(Zx=y\), where:
\begin{equation}\label{eq:general-algebaric-approach-blocking-time}
    \scalebox{0.9}{
        \(
        Z = 
        \begin{pmatrix}
            -1 & p_o & 0 & \dots & 0 & 0 & 0 & 0 & 0 & \dots & 0 & 0 \\ %(1,T)
            p_s & -1 & p_o & \dots & 0 & 0 & 0 & 0 & 0 & \dots & 0 & 0 \\ %(1,T+1)
            0 & p_s & -1 & \dots & 0 & 0 & 0 & 0 & 0 & \dots & 0 & 0 \\ %(1,T+2)
            \vdots & \vdots & \vdots & \ddots & \vdots & \vdots & \vdots & \vdots & 
            \vdots & \ddots & \vdots & \vdots \\ 
            0 & 0 & 0 & \dots & 1 & -1 & 0 & 0 & 0 & \dots & 0 & 0 \\ %(1,N)
            p_s & 0 & 0 & \dots & 0 & 0 & -1 & p_o & 0 & \dots & 0 & 0 \\ %(2,T)
            0 & 0 & 0 & \dots & 0 & 0 & p_s & -1 & p_o & \dots & 0 & 0 \\ %(2,T+1)
            \vdots & \vdots & \vdots & \ddots & \vdots & \vdots & \vdots & \vdots & 
            \vdots & \ddots & \vdots & \vdots \\ 
            0 & 0 & 0 & \dots & 0 & 0 & 0 & 0 & 0 & \dots & 1 & -1 \\ %(M,T)
        \end{pmatrix},
        x = 
        \begin{pmatrix}
            b(1,T) \\
            b(1,T+1) \\
            b(1,T+2) \\
            \vdots \\
            b(1,N) \\
            b(2,T) \\
            b(2,T+1) \\
            \vdots \\
            b(M,T) \\
        \end{pmatrix}, 
        y= 
        \begin{pmatrix}
            -c(1,T) \\
            -c(1,T+1) \\
            -c(1,T+2) \\
            \vdots \\
            -c(1,N) \\
            -c(2,T) \\
            -c(2,T+1) \\
            \vdots \\
            -c(M,T) \\
        \end{pmatrix}
        \)
    }
\end{equation}

Thus, having calculated the mean blocking time for all blocking states \(b(u,v)\), 
it only remains to put them together in a formula.
The resultant blocking time formula is given by:

\begin{equation}\label{eq:algebraic-blocking-time}
    B = \frac{\sum_{(u,v) \in S_A} \pi_{(u,v)} \; b(u,v)}{\sum_{(u,v) \in S_A} 
    \pi_{(u,v)}}
\end{equation}

\documentclass{article}

\usepackage{amsmath}
\usepackage{amsfonts} 
\usepackage{geometry}
\usepackage{multicol}
\usepackage{float}
% \usepackage{mathtools}
% \usepackage{graphicx}
% \usepackage{soul}
% \usepackage{indentfirst}
\usepackage{tikz}
\usetikzlibrary{calc, automata, chains, arrows.meta, math}
\setcounter{MaxMatrixCols}{20}


\title{A game theoretic model of the behavioural gaming that takes place at the EMS - ED interface}

\author{
    Michalis Panayides, 
    Paul Harper, 
    Vince Knight
}

\begin{document}

\maketitle

\input{Abstract/main.tex}


\newpage
\tableofcontents

\newpage
\input{Introduction/main.tex}

\newpage
\input{Game_theory_component/main.tex}

\newpage
\input{MarkovChain/markov_chain_model/main.tex}
\input{MarkovChain/expressions_from_pi/main.tex}
\input{MarkovChain/markov_example/main.tex}

\newpage
\input{BehaviouralMethodology/main.tex}

\newpage
\input{Application_EMS_ED/main.tex}

\newpage
\input{Conclusion/main.tex}


\end{document}

\newpage
\documentclass{article}

\usepackage{amsmath}
\usepackage{amsfonts} 
\usepackage{geometry}
\usepackage{multicol}
\usepackage{float}
% \usepackage{mathtools}
% \usepackage{graphicx}
% \usepackage{soul}
% \usepackage{indentfirst}
\usepackage{tikz}
\usetikzlibrary{calc, automata, chains, arrows.meta, math}
\setcounter{MaxMatrixCols}{20}


\title{A game theoretic model of the behavioural gaming that takes place at the EMS - ED interface}

\author{
    Michalis Panayides, 
    Paul Harper, 
    Vince Knight
}

\begin{document}

\maketitle

\input{Abstract/main.tex}


\newpage
\tableofcontents

\newpage
\input{Introduction/main.tex}

\newpage
\input{Game_theory_component/main.tex}

\newpage
\input{MarkovChain/markov_chain_model/main.tex}
\input{MarkovChain/expressions_from_pi/main.tex}
\input{MarkovChain/markov_example/main.tex}

\newpage
\input{BehaviouralMethodology/main.tex}

\newpage
\input{Application_EMS_ED/main.tex}

\newpage
\input{Conclusion/main.tex}


\end{document}

\newpage
\section{EMS-ED application}

\subsection{Application}

\subsection{Data analysis of generated problem}

\newpage
\documentclass{article}

\usepackage{amsmath}
\usepackage{amsfonts} 
\usepackage{geometry}
\usepackage{multicol}
\usepackage{float}
% \usepackage{mathtools}
% \usepackage{graphicx}
% \usepackage{soul}
% \usepackage{indentfirst}
\usepackage{tikz}
\usetikzlibrary{calc, automata, chains, arrows.meta, math}
\setcounter{MaxMatrixCols}{20}


\title{A game theoretic model of the behavioural gaming that takes place at the EMS - ED interface}

\author{
    Michalis Panayides, 
    Paul Harper, 
    Vince Knight
}

\begin{document}

\maketitle

\input{Abstract/main.tex}


\newpage
\tableofcontents

\newpage
\input{Introduction/main.tex}

\newpage
\input{Game_theory_component/main.tex}

\newpage
\input{MarkovChain/markov_chain_model/main.tex}
\input{MarkovChain/expressions_from_pi/main.tex}
\input{MarkovChain/markov_example/main.tex}

\newpage
\input{BehaviouralMethodology/main.tex}

\newpage
\input{Application_EMS_ED/main.tex}

\newpage
\input{Conclusion/main.tex}


\end{document}


\end{document}
\subsection{Performance Measures}
One may easily derive the average number of individuals that are at any given state 
using \( pi \). 
The average number of individuals in state \( i \) can be calculated by multiplying 
the number of individuals that are present in state \( i \) with the probability 
of being at that particular state (i.e \(\pi_i (u_i + v_i)\)). 
Using this logic it is possible to calculate any performance measures that are related 
to the mean number of individuals in the system.


Average number of people in the system: 
\begin{equation}
    L = \sum_{i=1}^{|\pi|} \pi_i (u_i + v_i)
\end{equation} 

Average number of people in the service centre: 
\begin{equation}
    L_H = \sum_{i=1}^{|\pi|} \pi_i v_i
\end{equation}

Average number of people in the buffer centre:
\begin{equation}
    L_A = \sum_{i=1}^{|\pi|} \pi_i u_i
\end{equation}

Consequently getting the performance measures that are related to the duration of 
time is not as straightforward. 
Such performance measures are the mean waiting time in the system and the mean time 
blocked in the system. 
Under the scope of this study three approaches have been considered to calculate these 
performance measures; a direct approach, a recursive algorithm and consequently a
closed-form formula.

The research question that needs to be answered here is: ``When a class 1/2 
individuals enters the system, what is the expected time that they will have to 
wait?''. 
In order to formulate the answer to that question one needs to consider all possible 
scenarios of what state the system can be in when an individual arrives. 
Furthermore, different formulas arises for class 1 individuals 
and a different one for class 2 individuals.

\subsubsection{Mean waiting time} 
Upon closer inspection of the recursive formula a more compact formula can arise. 
The equivalent closed-form formula eliminates the need for recursion and thus makes 
the computation of waiting times much more efficient. 
Just like in the recursive part there are two formulas; one for \textit{class 1} 
and one for class 2 individuals. 
The formulas are given by:

\begin{equation} \label{eq:closed_form_waiting_others}
    W^{(1)} = \frac{\sum_{\substack{(u,v) \, \in S_A^{(1)} \\ v \geq C}} 
    \frac{1}{C \mu} \times (v-C+1) \times \pi(u,v)}{\sum_{(u,v) \, 
    \in S_A^{(1)}} \pi(u,v)}
\end{equation}
    
\begin{equation}\label{eq:closed_form_waiting_ambulance}
    W^{(2)} = \frac{\sum_{\substack{(u,v) \, \in S_A^{(2)} \\ min(v,T) \geq C}} 
    \frac{1}{C \mu} \times (\min(v+1,T)-C) \times \pi(u,v)}{\sum_{(u,v) \, 
    \in S_A^{(2)}} \pi(u,v)}
\end{equation}

Note here that the summation, in both equations \ref{eq:closed_form_waiting_others} 
and \ref{eq:closed_form_waiting_ambulance}, goes through all states in the set of 
accepting 
states of either class 1 or class 2 individuals respectively, where a wait 
incurs. 
In equation \ref{eq:closed_form_waiting_others} that includes all states \((u,v)\) 
in the set of accepting states of class 1 individuals such that \( v \geq C\); i.e. 
whenever an arrival occurs and the system is at a state where the number of individuals 
in the system is more than or equal to $C$. 
Consequently, for the states that are included in the summation the expression 
\( v-C+1 \) indicates the amount of people in service one would have to wait for 
upon arrival at the hospital.

Additionally, the minimisation function in equation 
\ref{eq:closed_form_waiting_ambulance} 
ensures that when a class 2 individual arrives at any state 
that is greater than the predetermined threshold, the wait that the individual will 
have to endure remains the same. 
In essence, the expression \(\min(v+1,T) - C\) returns the number of people in line 
in front of a particular individual upon arrival.


\subsubsection{Overall Waiting Time}

Consequently, the overall waiting time should can be estimated by a linear combination 
of the waiting times of class 1 and class 2 individuals. 
The overall waiting time can be then given by the following equation where \(c_1\) 
and \(c_2\) are the coefficients of each individual's type waiting time:

\begin{equation}\label{overall_waiting_time_coeff}
    W = c_1 W^{(1)} + c_2 W^{(2)}
\end{equation}

The two coefficients represent the proportion of individuals of each type that 
traversed through the model. 
Theoretically, getting these percentages should be as simple as looking at the arrival 
rates of each type but in practise if the service centre or the buffer centre 
is full, some individuals may be lost to the system. 
Thus, one should account for the probability that an individual is lost to the system. 
This probability can be easily calculated by using the two sets of accepting states 
\(S_A^{(2)}\) and \(S_A^{(1)}\) defined earlier in equations.
Let us define here the probability, for either class type, that an individual 
is not lost in the system by:

\begin{equation*}
    P(L'_1) = \sum_{(u,v) \, \in S_A^{(1)}} \pi(u,v) \hspace{2cm}
    P(L'_2) = \sum_{(u,v) \, \in S_A^{(2)}} \pi(u,v)
\end{equation*}

Having defined these probabilities one may combine them with the arrival rates of 
each class type in such a way to get the expected proportions of class 1 and 
class 2 individuals in the model. 
Thus, by using these values as the coefficient of equation 
\ref{overall_waiting_time_coeff} 
the resultant equation can be used to get the overall waiting time. 
Note here that the equation below gets the overall waiting time for both the recursive 
and the closed-form formula.

\begin{equation}\label{overall_waiting_time}
    W = \frac{\lambda_1 P(L'_1)}{\lambda_2 P(L'_2) + \lambda_1 P(L'_1)} W^{(1)} + 
    \frac{\lambda_2 P(L'_2)}{\lambda_2 P(L'_2) + \lambda_1 P(L'_1)} W^{(2)}
\end{equation}



\subsubsection{Mean blocking time}
Unlike the waiting time, the blocking time is only calculated for class 2 individuals.  
That is because class 1 individuals cannot be blocked. 
Thus, one only needs to consider the pathway of class 2 individuals to get the 
mean blocking time of the system. 
Blocking occurs at states \((u,v)\) where \(u > 0 \). 
Thus, the set of blocking states can be defined as:

\begin{equation*}
    S_b = \{(u,v) \in S \; | \; u > 0\}
\end{equation*}
 
In order to not consider individuals that will be lost to the system, the set of 
accepting states needs to be taken into account. The set of accepting states is given by:

\begin{equation*}
    S_A^{(2)}=
    \begin{cases}
        \{(u, v) \in S \; | \; u < M \} & \textbf{if } T \leq N\\
        \{(u, v) \in S \; | \; v < N \} & \textbf{otherwise}
    \end{cases}
\end{equation*}

For the waiting time formula,
the mean sojourn time for each state was considered,
ignoring any arrivals. Here, the same approach is used but ignoring only class 2
arrivals. That is because for the waiting time formula, once an individual enters 
the service centre (i.e. starts waiting) any individual arriving after them will 
not affect their
pathway. That is not the case for blocking time. When a class 2 individual is 
blocked, 
any class 1 individual that arrives will cause the blocked individual to remain 
blocked for more time. Therefore, class 1 arrivals are considered here:

\begin{equation}\label{eq:time_in_state_blocking_time}
    c(u,v) = 
    \begin{cases}
        \frac{1}{\min(v,C) \mu}, & \text{if } v = C\\
        \frac{1}{\min(v,C) \mu + \lambda_1}, & \text{otherwise}
    \end{cases}
\end{equation}
 
In equation \ref{eq:time_in_state_blocking_time}, both service completions and 
class 1 arrivals are considered. 
Thus, from a blocked individual's perspective whenever the system moves from one 
state \((u,v)\)
to another state it can either:

\begin{itemize}
    \item be because of a service being completed: we will denote the probability 
    of this happening by \(p_s(u,v)\). 
    \item be because of an arrival of an individual of class 1: denoting such 
    probability by \(p_o(u,v)\).
\end{itemize}
The probabilities are given by:

\begin{equation*}
    p_s(u,v) = \frac{\min(v,C)\mu}{\lambda_1 + \min(v,C)\mu}, \qquad
    p_o(u,v) = \frac{\lambda_1}{\lambda_1 + \min(v,C)\mu}
\end{equation*}


Having defined \(c(u,v)\) and \(S_b\) a formula for the blocking time that is
expected to occur at each state can be given by:

\begin{equation}\label{eq:blocking-time-at-each-state}
    b(u,v) = 
    \begin{cases} 
        0, & \textbf{if } (u,v) \notin S_b \\
        c(u,v) + b(u - 1, v), & \textbf{if } v = N = T\\
        c(u,v) + b(u, v-1), & \textbf{if } v = N \neq T \\
        c(u,v) + p_s(u,v) b(u-1, v) + p_o(u,v) b(u, v+1), & \textbf{if } u > 0 
        \textbf{ and } v = T \\
        c(u,v) + p_s(u,v) b(u, v-1) + p_o(u,v) b(u, v+1), & \textbf{otherwise} \\
    \end{cases}
\end{equation}

Equation 
(\ref{eq:blocking-time-at-each-state}) will not be solved recursively. 
A direct approach will be used to solve this equation here. 
By enumerating all equations of (\ref{eq:blocking-time-at-each-state}) for all 
states \((u,v)\) that belong in \(S_b\) 
a system of linear equations arises where the unknown variables are all the \(b(u,v)\)
terms.
For instance, let us consider a Markov model where \(C=2, T=3, N=6, M=2\). 
The Markov model is shown in Figure \ref{fig:example-algeb-blocking}
and the equivalent equations are 
(\ref{eq:first_eq_of_blocking_example})-(\ref{eq:last_eq_of_blocking_example}).
The equations considered here are only the ones that correspond to the blocking 
states.

\begin{multicols*}{2}
    \begin{figure}[H]
        \scalebox{0.50}{\documentclass{article}

\usepackage{amsmath}
\usepackage{amsfonts} 
\usepackage{geometry}
\usepackage{multicol}
\usepackage{float}
% \usepackage{mathtools}
% \usepackage{graphicx}
% \usepackage{soul}
% \usepackage{indentfirst}
\usepackage{tikz}
\usetikzlibrary{calc, automata, chains, arrows.meta, math}
\setcounter{MaxMatrixCols}{20}


\title{A game theoretic model of the behavioural gaming that takes place at the EMS - ED interface}

\author{
    Michalis Panayides, 
    Paul Harper, 
    Vince Knight
}

\begin{document}

\maketitle

\input{Abstract/main.tex}


\newpage
\tableofcontents

\newpage
\input{Introduction/main.tex}

\newpage
\input{Game_theory_component/main.tex}

\newpage
\input{MarkovChain/markov_chain_model/main.tex}
\input{MarkovChain/expressions_from_pi/main.tex}
\input{MarkovChain/markov_example/main.tex}

\newpage
\input{BehaviouralMethodology/main.tex}

\newpage
\input{Application_EMS_ED/main.tex}

\newpage
\input{Conclusion/main.tex}


\end{document}}
        \caption{Example of Markov chain}
        \label{fig:example-algeb-blocking}
    \end{figure}
    \columnbreak
    \begin{align}
        b(1,2) &= c(1,2) + p_o b(1,3) \label{eq:first_eq_of_blocking_example} \\
        b(1,3) &= c(1,3) + p_s b(1,2) + p_o b(1,4) \\
        b(1,4) &= c(1,4) + b(1,3) \\
        b(2,2) &= c(2,2) + p_s b(1,2) + p_o b(2,3) \\
        b(2,3) &= c(2,3) + p_s b(2,2) + p_o b(1,4) \\
        b(2,4) &= c(2,4) + b(2,3)\label{eq:last_eq_of_blocking_example}
    \end{align}
\end{multicols*}

Additionally, the above equations can be transformed into a linear system of the 
form \(Zx=y\) where:

\begin{equation}\label{eq:example-algebaric-approach-blocking-time}
    Z=
    \begin{pmatrix}
        -1 & p_o & 0 & 0 & 0 & 0 \\ %(1,2)
        p_s & -1 & p_o & 0 & 0 & 0 \\ %(1,3)
        0 & 1 & -1 & 0 & 0 & 0 \\ %(1,4)
        p_s & 0 & 0 & -1 & p_o & 0\\ %(2,2)
        0 & 0 & 0 & p_s & -1 & p_o \\ %(2,3)
        0 & 0 & 0 & 0 & 1 & -1 \\ %(2,4)
    \end{pmatrix},
    x=
    \begin{pmatrix}
        b(1,2) \\
        b(1,3) \\
        b(1,4) \\
        b(2,2) \\
        b(2,3) \\
        b(2,4) \\
    \end{pmatrix}, 
    y=
    \begin{pmatrix}
        -c(1,2) \\
        -c(1,3) \\
        -c(1,4) \\
        -c(2,2) \\
        -c(2,3) \\
        -c(2,4) \\
    \end{pmatrix}
\end{equation}

A more generalised form of the equations in 
(\ref{eq:example-algebaric-approach-blocking-time})
can thus be given for any value of \(C,T,N,M\) by:

\begin{align}
    b(1,T) =& c(1, T) + p_o b(1, T + 1) \label{eq:first_eq_of_blocking_general}\\
    b(1,T + 1) =& c(1, T + 1) + p_s(1, T) + p_o b(1, T + 1) \\
    b(1,T + 2) =& c(1, T + 2) + p_s(1, T + 1) + p_o b(1, T + 3) \\
    & \vdots \nonumber \\
    b(1, N) =& c(1, N) + b(1, N - 1) \\
    b(2, T) =& c(2, T) + p_s b(1, T) + p_o b(2, T + 1) \\
    b(2, T + 1) =& c(2, T + 1) + p_s b(2, T) + p_o b(2, T + 2) \\
    & \vdots \nonumber \\
    b(M, T) =& c(M, T) + b(M, T-1) \label{eq:last_eq_of_blocking_general}
\end{align}

The equivalent matrix form of the linear system of equations 
(\ref{eq:first_eq_of_blocking_general}) - (\ref{eq:last_eq_of_blocking_general})
is given by \(Zx=y\), where:
\begin{equation}\label{eq:general-algebaric-approach-blocking-time}
    \scalebox{0.9}{
        \(
        Z = 
        \begin{pmatrix}
            -1 & p_o & 0 & \dots & 0 & 0 & 0 & 0 & 0 & \dots & 0 & 0 \\ %(1,T)
            p_s & -1 & p_o & \dots & 0 & 0 & 0 & 0 & 0 & \dots & 0 & 0 \\ %(1,T+1)
            0 & p_s & -1 & \dots & 0 & 0 & 0 & 0 & 0 & \dots & 0 & 0 \\ %(1,T+2)
            \vdots & \vdots & \vdots & \ddots & \vdots & \vdots & \vdots & \vdots & 
            \vdots & \ddots & \vdots & \vdots \\ 
            0 & 0 & 0 & \dots & 1 & -1 & 0 & 0 & 0 & \dots & 0 & 0 \\ %(1,N)
            p_s & 0 & 0 & \dots & 0 & 0 & -1 & p_o & 0 & \dots & 0 & 0 \\ %(2,T)
            0 & 0 & 0 & \dots & 0 & 0 & p_s & -1 & p_o & \dots & 0 & 0 \\ %(2,T+1)
            \vdots & \vdots & \vdots & \ddots & \vdots & \vdots & \vdots & \vdots & 
            \vdots & \ddots & \vdots & \vdots \\ 
            0 & 0 & 0 & \dots & 0 & 0 & 0 & 0 & 0 & \dots & 1 & -1 \\ %(M,T)
        \end{pmatrix},
        x = 
        \begin{pmatrix}
            b(1,T) \\
            b(1,T+1) \\
            b(1,T+2) \\
            \vdots \\
            b(1,N) \\
            b(2,T) \\
            b(2,T+1) \\
            \vdots \\
            b(M,T) \\
        \end{pmatrix}, 
        y= 
        \begin{pmatrix}
            -c(1,T) \\
            -c(1,T+1) \\
            -c(1,T+2) \\
            \vdots \\
            -c(1,N) \\
            -c(2,T) \\
            -c(2,T+1) \\
            \vdots \\
            -c(M,T) \\
        \end{pmatrix}
        \)
    }
\end{equation}

Thus, having calculated the mean blocking time for all blocking states \(b(u,v)\), 
it only remains to put them together in a formula.
The resultant blocking time formula is given by:

\begin{equation}\label{eq:algebraic-blocking-time}
    B = \frac{\sum_{(u,v) \in S_A} \pi_{(u,v)} \; b(u,v)}{\sum_{(u,v) \in S_A} 
    \pi_{(u,v)}}
\end{equation}

\documentclass{article}

\usepackage{amsmath}
\usepackage{amsfonts} 
\usepackage{geometry}
\usepackage{multicol}
\usepackage{float}
% \usepackage{mathtools}
% \usepackage{graphicx}
% \usepackage{soul}
% \usepackage{indentfirst}
\usepackage{tikz}
\usetikzlibrary{calc, automata, chains, arrows.meta, math}
\setcounter{MaxMatrixCols}{20}


\title{A game theoretic model of the behavioural gaming that takes place at the EMS - ED interface}

\author{
    Michalis Panayides, 
    Paul Harper, 
    Vince Knight
}

\begin{document}

\maketitle

\documentclass{article}

\usepackage{amsmath}
\usepackage{amsfonts} 
\usepackage{geometry}
\usepackage{multicol}
\usepackage{float}
% \usepackage{mathtools}
% \usepackage{graphicx}
% \usepackage{soul}
% \usepackage{indentfirst}
\usepackage{tikz}
\usetikzlibrary{calc, automata, chains, arrows.meta, math}
\setcounter{MaxMatrixCols}{20}


\title{A game theoretic model of the behavioural gaming that takes place at the EMS - ED interface}

\author{
    Michalis Panayides, 
    Paul Harper, 
    Vince Knight
}

\begin{document}

\maketitle

\input{Abstract/main.tex}


\newpage
\tableofcontents

\newpage
\input{Introduction/main.tex}

\newpage
\input{Game_theory_component/main.tex}

\newpage
\input{MarkovChain/markov_chain_model/main.tex}
\input{MarkovChain/expressions_from_pi/main.tex}
\input{MarkovChain/markov_example/main.tex}

\newpage
\input{BehaviouralMethodology/main.tex}

\newpage
\input{Application_EMS_ED/main.tex}

\newpage
\input{Conclusion/main.tex}


\end{document}


\newpage
\tableofcontents

\newpage
\documentclass{article}

\usepackage{amsmath}
\usepackage{amsfonts} 
\usepackage{geometry}
\usepackage{multicol}
\usepackage{float}
% \usepackage{mathtools}
% \usepackage{graphicx}
% \usepackage{soul}
% \usepackage{indentfirst}
\usepackage{tikz}
\usetikzlibrary{calc, automata, chains, arrows.meta, math}
\setcounter{MaxMatrixCols}{20}


\title{A game theoretic model of the behavioural gaming that takes place at the EMS - ED interface}

\author{
    Michalis Panayides, 
    Paul Harper, 
    Vince Knight
}

\begin{document}

\maketitle

\input{Abstract/main.tex}


\newpage
\tableofcontents

\newpage
\input{Introduction/main.tex}

\newpage
\input{Game_theory_component/main.tex}

\newpage
\input{MarkovChain/markov_chain_model/main.tex}
\input{MarkovChain/expressions_from_pi/main.tex}
\input{MarkovChain/markov_example/main.tex}

\newpage
\input{BehaviouralMethodology/main.tex}

\newpage
\input{Application_EMS_ED/main.tex}

\newpage
\input{Conclusion/main.tex}


\end{document}

\newpage
\documentclass{article}

\usepackage{amsmath}
\usepackage{amsfonts} 
\usepackage{geometry}
\usepackage{multicol}
\usepackage{float}
% \usepackage{mathtools}
% \usepackage{graphicx}
% \usepackage{soul}
% \usepackage{indentfirst}
\usepackage{tikz}
\usetikzlibrary{calc, automata, chains, arrows.meta, math}
\setcounter{MaxMatrixCols}{20}


\title{A game theoretic model of the behavioural gaming that takes place at the EMS - ED interface}

\author{
    Michalis Panayides, 
    Paul Harper, 
    Vince Knight
}

\begin{document}

\maketitle

\input{Abstract/main.tex}


\newpage
\tableofcontents

\newpage
\input{Introduction/main.tex}

\newpage
\input{Game_theory_component/main.tex}

\newpage
\input{MarkovChain/markov_chain_model/main.tex}
\input{MarkovChain/expressions_from_pi/main.tex}
\input{MarkovChain/markov_example/main.tex}

\newpage
\input{BehaviouralMethodology/main.tex}

\newpage
\input{Application_EMS_ED/main.tex}

\newpage
\input{Conclusion/main.tex}


\end{document}

\newpage
\documentclass{article}

\usepackage{amsmath}
\usepackage{amsfonts} 
\usepackage{geometry}
\usepackage{multicol}
\usepackage{float}
% \usepackage{mathtools}
% \usepackage{graphicx}
% \usepackage{soul}
% \usepackage{indentfirst}
\usepackage{tikz}
\usetikzlibrary{calc, automata, chains, arrows.meta, math}
\setcounter{MaxMatrixCols}{20}


\title{A game theoretic model of the behavioural gaming that takes place at the EMS - ED interface}

\author{
    Michalis Panayides, 
    Paul Harper, 
    Vince Knight
}

\begin{document}

\maketitle

\input{Abstract/main.tex}


\newpage
\tableofcontents

\newpage
\input{Introduction/main.tex}

\newpage
\input{Game_theory_component/main.tex}

\newpage
\input{MarkovChain/markov_chain_model/main.tex}
\input{MarkovChain/expressions_from_pi/main.tex}
\input{MarkovChain/markov_example/main.tex}

\newpage
\input{BehaviouralMethodology/main.tex}

\newpage
\input{Application_EMS_ED/main.tex}

\newpage
\input{Conclusion/main.tex}


\end{document}
\subsection{Performance Measures}
One may easily derive the average number of individuals that are at any given state 
using \( pi \). 
The average number of individuals in state \( i \) can be calculated by multiplying 
the number of individuals that are present in state \( i \) with the probability 
of being at that particular state (i.e \(\pi_i (u_i + v_i)\)). 
Using this logic it is possible to calculate any performance measures that are related 
to the mean number of individuals in the system.


Average number of people in the system: 
\begin{equation}
    L = \sum_{i=1}^{|\pi|} \pi_i (u_i + v_i)
\end{equation} 

Average number of people in the service centre: 
\begin{equation}
    L_H = \sum_{i=1}^{|\pi|} \pi_i v_i
\end{equation}

Average number of people in the buffer centre:
\begin{equation}
    L_A = \sum_{i=1}^{|\pi|} \pi_i u_i
\end{equation}

Consequently getting the performance measures that are related to the duration of 
time is not as straightforward. 
Such performance measures are the mean waiting time in the system and the mean time 
blocked in the system. 
Under the scope of this study three approaches have been considered to calculate these 
performance measures; a direct approach, a recursive algorithm and consequently a
closed-form formula.

The research question that needs to be answered here is: ``When a class 1/2 
individuals enters the system, what is the expected time that they will have to 
wait?''. 
In order to formulate the answer to that question one needs to consider all possible 
scenarios of what state the system can be in when an individual arrives. 
Furthermore, different formulas arises for class 1 individuals 
and a different one for class 2 individuals.

\subsubsection{Mean waiting time} 
Upon closer inspection of the recursive formula a more compact formula can arise. 
The equivalent closed-form formula eliminates the need for recursion and thus makes 
the computation of waiting times much more efficient. 
Just like in the recursive part there are two formulas; one for \textit{class 1} 
and one for class 2 individuals. 
The formulas are given by:

\begin{equation} \label{eq:closed_form_waiting_others}
    W^{(1)} = \frac{\sum_{\substack{(u,v) \, \in S_A^{(1)} \\ v \geq C}} 
    \frac{1}{C \mu} \times (v-C+1) \times \pi(u,v)}{\sum_{(u,v) \, 
    \in S_A^{(1)}} \pi(u,v)}
\end{equation}
    
\begin{equation}\label{eq:closed_form_waiting_ambulance}
    W^{(2)} = \frac{\sum_{\substack{(u,v) \, \in S_A^{(2)} \\ min(v,T) \geq C}} 
    \frac{1}{C \mu} \times (\min(v+1,T)-C) \times \pi(u,v)}{\sum_{(u,v) \, 
    \in S_A^{(2)}} \pi(u,v)}
\end{equation}

Note here that the summation, in both equations \ref{eq:closed_form_waiting_others} 
and \ref{eq:closed_form_waiting_ambulance}, goes through all states in the set of 
accepting 
states of either class 1 or class 2 individuals respectively, where a wait 
incurs. 
In equation \ref{eq:closed_form_waiting_others} that includes all states \((u,v)\) 
in the set of accepting states of class 1 individuals such that \( v \geq C\); i.e. 
whenever an arrival occurs and the system is at a state where the number of individuals 
in the system is more than or equal to $C$. 
Consequently, for the states that are included in the summation the expression 
\( v-C+1 \) indicates the amount of people in service one would have to wait for 
upon arrival at the hospital.

Additionally, the minimisation function in equation 
\ref{eq:closed_form_waiting_ambulance} 
ensures that when a class 2 individual arrives at any state 
that is greater than the predetermined threshold, the wait that the individual will 
have to endure remains the same. 
In essence, the expression \(\min(v+1,T) - C\) returns the number of people in line 
in front of a particular individual upon arrival.


\subsubsection{Overall Waiting Time}

Consequently, the overall waiting time should can be estimated by a linear combination 
of the waiting times of class 1 and class 2 individuals. 
The overall waiting time can be then given by the following equation where \(c_1\) 
and \(c_2\) are the coefficients of each individual's type waiting time:

\begin{equation}\label{overall_waiting_time_coeff}
    W = c_1 W^{(1)} + c_2 W^{(2)}
\end{equation}

The two coefficients represent the proportion of individuals of each type that 
traversed through the model. 
Theoretically, getting these percentages should be as simple as looking at the arrival 
rates of each type but in practise if the service centre or the buffer centre 
is full, some individuals may be lost to the system. 
Thus, one should account for the probability that an individual is lost to the system. 
This probability can be easily calculated by using the two sets of accepting states 
\(S_A^{(2)}\) and \(S_A^{(1)}\) defined earlier in equations.
Let us define here the probability, for either class type, that an individual 
is not lost in the system by:

\begin{equation*}
    P(L'_1) = \sum_{(u,v) \, \in S_A^{(1)}} \pi(u,v) \hspace{2cm}
    P(L'_2) = \sum_{(u,v) \, \in S_A^{(2)}} \pi(u,v)
\end{equation*}

Having defined these probabilities one may combine them with the arrival rates of 
each class type in such a way to get the expected proportions of class 1 and 
class 2 individuals in the model. 
Thus, by using these values as the coefficient of equation 
\ref{overall_waiting_time_coeff} 
the resultant equation can be used to get the overall waiting time. 
Note here that the equation below gets the overall waiting time for both the recursive 
and the closed-form formula.

\begin{equation}\label{overall_waiting_time}
    W = \frac{\lambda_1 P(L'_1)}{\lambda_2 P(L'_2) + \lambda_1 P(L'_1)} W^{(1)} + 
    \frac{\lambda_2 P(L'_2)}{\lambda_2 P(L'_2) + \lambda_1 P(L'_1)} W^{(2)}
\end{equation}



\subsubsection{Mean blocking time}
Unlike the waiting time, the blocking time is only calculated for class 2 individuals.  
That is because class 1 individuals cannot be blocked. 
Thus, one only needs to consider the pathway of class 2 individuals to get the 
mean blocking time of the system. 
Blocking occurs at states \((u,v)\) where \(u > 0 \). 
Thus, the set of blocking states can be defined as:

\begin{equation*}
    S_b = \{(u,v) \in S \; | \; u > 0\}
\end{equation*}
 
In order to not consider individuals that will be lost to the system, the set of 
accepting states needs to be taken into account. The set of accepting states is given by:

\begin{equation*}
    S_A^{(2)}=
    \begin{cases}
        \{(u, v) \in S \; | \; u < M \} & \textbf{if } T \leq N\\
        \{(u, v) \in S \; | \; v < N \} & \textbf{otherwise}
    \end{cases}
\end{equation*}

For the waiting time formula,
the mean sojourn time for each state was considered,
ignoring any arrivals. Here, the same approach is used but ignoring only class 2
arrivals. That is because for the waiting time formula, once an individual enters 
the service centre (i.e. starts waiting) any individual arriving after them will 
not affect their
pathway. That is not the case for blocking time. When a class 2 individual is 
blocked, 
any class 1 individual that arrives will cause the blocked individual to remain 
blocked for more time. Therefore, class 1 arrivals are considered here:

\begin{equation}\label{eq:time_in_state_blocking_time}
    c(u,v) = 
    \begin{cases}
        \frac{1}{\min(v,C) \mu}, & \text{if } v = C\\
        \frac{1}{\min(v,C) \mu + \lambda_1}, & \text{otherwise}
    \end{cases}
\end{equation}
 
In equation \ref{eq:time_in_state_blocking_time}, both service completions and 
class 1 arrivals are considered. 
Thus, from a blocked individual's perspective whenever the system moves from one 
state \((u,v)\)
to another state it can either:

\begin{itemize}
    \item be because of a service being completed: we will denote the probability 
    of this happening by \(p_s(u,v)\). 
    \item be because of an arrival of an individual of class 1: denoting such 
    probability by \(p_o(u,v)\).
\end{itemize}
The probabilities are given by:

\begin{equation*}
    p_s(u,v) = \frac{\min(v,C)\mu}{\lambda_1 + \min(v,C)\mu}, \qquad
    p_o(u,v) = \frac{\lambda_1}{\lambda_1 + \min(v,C)\mu}
\end{equation*}


Having defined \(c(u,v)\) and \(S_b\) a formula for the blocking time that is
expected to occur at each state can be given by:

\begin{equation}\label{eq:blocking-time-at-each-state}
    b(u,v) = 
    \begin{cases} 
        0, & \textbf{if } (u,v) \notin S_b \\
        c(u,v) + b(u - 1, v), & \textbf{if } v = N = T\\
        c(u,v) + b(u, v-1), & \textbf{if } v = N \neq T \\
        c(u,v) + p_s(u,v) b(u-1, v) + p_o(u,v) b(u, v+1), & \textbf{if } u > 0 
        \textbf{ and } v = T \\
        c(u,v) + p_s(u,v) b(u, v-1) + p_o(u,v) b(u, v+1), & \textbf{otherwise} \\
    \end{cases}
\end{equation}

Equation 
(\ref{eq:blocking-time-at-each-state}) will not be solved recursively. 
A direct approach will be used to solve this equation here. 
By enumerating all equations of (\ref{eq:blocking-time-at-each-state}) for all 
states \((u,v)\) that belong in \(S_b\) 
a system of linear equations arises where the unknown variables are all the \(b(u,v)\)
terms.
For instance, let us consider a Markov model where \(C=2, T=3, N=6, M=2\). 
The Markov model is shown in Figure \ref{fig:example-algeb-blocking}
and the equivalent equations are 
(\ref{eq:first_eq_of_blocking_example})-(\ref{eq:last_eq_of_blocking_example}).
The equations considered here are only the ones that correspond to the blocking 
states.

\begin{multicols*}{2}
    \begin{figure}[H]
        \scalebox{0.50}{\input{MarkovChain/expressions_from_pi/example_model_2362/main.tex}}
        \caption{Example of Markov chain}
        \label{fig:example-algeb-blocking}
    \end{figure}
    \columnbreak
    \begin{align}
        b(1,2) &= c(1,2) + p_o b(1,3) \label{eq:first_eq_of_blocking_example} \\
        b(1,3) &= c(1,3) + p_s b(1,2) + p_o b(1,4) \\
        b(1,4) &= c(1,4) + b(1,3) \\
        b(2,2) &= c(2,2) + p_s b(1,2) + p_o b(2,3) \\
        b(2,3) &= c(2,3) + p_s b(2,2) + p_o b(1,4) \\
        b(2,4) &= c(2,4) + b(2,3)\label{eq:last_eq_of_blocking_example}
    \end{align}
\end{multicols*}

Additionally, the above equations can be transformed into a linear system of the 
form \(Zx=y\) where:

\begin{equation}\label{eq:example-algebaric-approach-blocking-time}
    Z=
    \begin{pmatrix}
        -1 & p_o & 0 & 0 & 0 & 0 \\ %(1,2)
        p_s & -1 & p_o & 0 & 0 & 0 \\ %(1,3)
        0 & 1 & -1 & 0 & 0 & 0 \\ %(1,4)
        p_s & 0 & 0 & -1 & p_o & 0\\ %(2,2)
        0 & 0 & 0 & p_s & -1 & p_o \\ %(2,3)
        0 & 0 & 0 & 0 & 1 & -1 \\ %(2,4)
    \end{pmatrix},
    x=
    \begin{pmatrix}
        b(1,2) \\
        b(1,3) \\
        b(1,4) \\
        b(2,2) \\
        b(2,3) \\
        b(2,4) \\
    \end{pmatrix}, 
    y=
    \begin{pmatrix}
        -c(1,2) \\
        -c(1,3) \\
        -c(1,4) \\
        -c(2,2) \\
        -c(2,3) \\
        -c(2,4) \\
    \end{pmatrix}
\end{equation}

A more generalised form of the equations in 
(\ref{eq:example-algebaric-approach-blocking-time})
can thus be given for any value of \(C,T,N,M\) by:

\begin{align}
    b(1,T) =& c(1, T) + p_o b(1, T + 1) \label{eq:first_eq_of_blocking_general}\\
    b(1,T + 1) =& c(1, T + 1) + p_s(1, T) + p_o b(1, T + 1) \\
    b(1,T + 2) =& c(1, T + 2) + p_s(1, T + 1) + p_o b(1, T + 3) \\
    & \vdots \nonumber \\
    b(1, N) =& c(1, N) + b(1, N - 1) \\
    b(2, T) =& c(2, T) + p_s b(1, T) + p_o b(2, T + 1) \\
    b(2, T + 1) =& c(2, T + 1) + p_s b(2, T) + p_o b(2, T + 2) \\
    & \vdots \nonumber \\
    b(M, T) =& c(M, T) + b(M, T-1) \label{eq:last_eq_of_blocking_general}
\end{align}

The equivalent matrix form of the linear system of equations 
(\ref{eq:first_eq_of_blocking_general}) - (\ref{eq:last_eq_of_blocking_general})
is given by \(Zx=y\), where:
\begin{equation}\label{eq:general-algebaric-approach-blocking-time}
    \scalebox{0.9}{
        \(
        Z = 
        \begin{pmatrix}
            -1 & p_o & 0 & \dots & 0 & 0 & 0 & 0 & 0 & \dots & 0 & 0 \\ %(1,T)
            p_s & -1 & p_o & \dots & 0 & 0 & 0 & 0 & 0 & \dots & 0 & 0 \\ %(1,T+1)
            0 & p_s & -1 & \dots & 0 & 0 & 0 & 0 & 0 & \dots & 0 & 0 \\ %(1,T+2)
            \vdots & \vdots & \vdots & \ddots & \vdots & \vdots & \vdots & \vdots & 
            \vdots & \ddots & \vdots & \vdots \\ 
            0 & 0 & 0 & \dots & 1 & -1 & 0 & 0 & 0 & \dots & 0 & 0 \\ %(1,N)
            p_s & 0 & 0 & \dots & 0 & 0 & -1 & p_o & 0 & \dots & 0 & 0 \\ %(2,T)
            0 & 0 & 0 & \dots & 0 & 0 & p_s & -1 & p_o & \dots & 0 & 0 \\ %(2,T+1)
            \vdots & \vdots & \vdots & \ddots & \vdots & \vdots & \vdots & \vdots & 
            \vdots & \ddots & \vdots & \vdots \\ 
            0 & 0 & 0 & \dots & 0 & 0 & 0 & 0 & 0 & \dots & 1 & -1 \\ %(M,T)
        \end{pmatrix},
        x = 
        \begin{pmatrix}
            b(1,T) \\
            b(1,T+1) \\
            b(1,T+2) \\
            \vdots \\
            b(1,N) \\
            b(2,T) \\
            b(2,T+1) \\
            \vdots \\
            b(M,T) \\
        \end{pmatrix}, 
        y= 
        \begin{pmatrix}
            -c(1,T) \\
            -c(1,T+1) \\
            -c(1,T+2) \\
            \vdots \\
            -c(1,N) \\
            -c(2,T) \\
            -c(2,T+1) \\
            \vdots \\
            -c(M,T) \\
        \end{pmatrix}
        \)
    }
\end{equation}

Thus, having calculated the mean blocking time for all blocking states \(b(u,v)\), 
it only remains to put them together in a formula.
The resultant blocking time formula is given by:

\begin{equation}\label{eq:algebraic-blocking-time}
    B = \frac{\sum_{(u,v) \in S_A} \pi_{(u,v)} \; b(u,v)}{\sum_{(u,v) \in S_A} 
    \pi_{(u,v)}}
\end{equation}

\documentclass{article}

\usepackage{amsmath}
\usepackage{amsfonts} 
\usepackage{geometry}
\usepackage{multicol}
\usepackage{float}
% \usepackage{mathtools}
% \usepackage{graphicx}
% \usepackage{soul}
% \usepackage{indentfirst}
\usepackage{tikz}
\usetikzlibrary{calc, automata, chains, arrows.meta, math}
\setcounter{MaxMatrixCols}{20}


\title{A game theoretic model of the behavioural gaming that takes place at the EMS - ED interface}

\author{
    Michalis Panayides, 
    Paul Harper, 
    Vince Knight
}

\begin{document}

\maketitle

\input{Abstract/main.tex}


\newpage
\tableofcontents

\newpage
\input{Introduction/main.tex}

\newpage
\input{Game_theory_component/main.tex}

\newpage
\input{MarkovChain/markov_chain_model/main.tex}
\input{MarkovChain/expressions_from_pi/main.tex}
\input{MarkovChain/markov_example/main.tex}

\newpage
\input{BehaviouralMethodology/main.tex}

\newpage
\input{Application_EMS_ED/main.tex}

\newpage
\input{Conclusion/main.tex}


\end{document}

\newpage
\documentclass{article}

\usepackage{amsmath}
\usepackage{amsfonts} 
\usepackage{geometry}
\usepackage{multicol}
\usepackage{float}
% \usepackage{mathtools}
% \usepackage{graphicx}
% \usepackage{soul}
% \usepackage{indentfirst}
\usepackage{tikz}
\usetikzlibrary{calc, automata, chains, arrows.meta, math}
\setcounter{MaxMatrixCols}{20}


\title{A game theoretic model of the behavioural gaming that takes place at the EMS - ED interface}

\author{
    Michalis Panayides, 
    Paul Harper, 
    Vince Knight
}

\begin{document}

\maketitle

\input{Abstract/main.tex}


\newpage
\tableofcontents

\newpage
\input{Introduction/main.tex}

\newpage
\input{Game_theory_component/main.tex}

\newpage
\input{MarkovChain/markov_chain_model/main.tex}
\input{MarkovChain/expressions_from_pi/main.tex}
\input{MarkovChain/markov_example/main.tex}

\newpage
\input{BehaviouralMethodology/main.tex}

\newpage
\input{Application_EMS_ED/main.tex}

\newpage
\input{Conclusion/main.tex}


\end{document}

\newpage
\section{EMS-ED application}

\subsection{Application}

\subsection{Data analysis of generated problem}

\newpage
\documentclass{article}

\usepackage{amsmath}
\usepackage{amsfonts} 
\usepackage{geometry}
\usepackage{multicol}
\usepackage{float}
% \usepackage{mathtools}
% \usepackage{graphicx}
% \usepackage{soul}
% \usepackage{indentfirst}
\usepackage{tikz}
\usetikzlibrary{calc, automata, chains, arrows.meta, math}
\setcounter{MaxMatrixCols}{20}


\title{A game theoretic model of the behavioural gaming that takes place at the EMS - ED interface}

\author{
    Michalis Panayides, 
    Paul Harper, 
    Vince Knight
}

\begin{document}

\maketitle

\input{Abstract/main.tex}


\newpage
\tableofcontents

\newpage
\input{Introduction/main.tex}

\newpage
\input{Game_theory_component/main.tex}

\newpage
\input{MarkovChain/markov_chain_model/main.tex}
\input{MarkovChain/expressions_from_pi/main.tex}
\input{MarkovChain/markov_example/main.tex}

\newpage
\input{BehaviouralMethodology/main.tex}

\newpage
\input{Application_EMS_ED/main.tex}

\newpage
\input{Conclusion/main.tex}


\end{document}


\end{document}

\newpage
\documentclass{article}

\usepackage{amsmath}
\usepackage{amsfonts} 
\usepackage{geometry}
\usepackage{multicol}
\usepackage{float}
% \usepackage{mathtools}
% \usepackage{graphicx}
% \usepackage{soul}
% \usepackage{indentfirst}
\usepackage{tikz}
\usetikzlibrary{calc, automata, chains, arrows.meta, math}
\setcounter{MaxMatrixCols}{20}


\title{A game theoretic model of the behavioural gaming that takes place at the EMS - ED interface}

\author{
    Michalis Panayides, 
    Paul Harper, 
    Vince Knight
}

\begin{document}

\maketitle

\documentclass{article}

\usepackage{amsmath}
\usepackage{amsfonts} 
\usepackage{geometry}
\usepackage{multicol}
\usepackage{float}
% \usepackage{mathtools}
% \usepackage{graphicx}
% \usepackage{soul}
% \usepackage{indentfirst}
\usepackage{tikz}
\usetikzlibrary{calc, automata, chains, arrows.meta, math}
\setcounter{MaxMatrixCols}{20}


\title{A game theoretic model of the behavioural gaming that takes place at the EMS - ED interface}

\author{
    Michalis Panayides, 
    Paul Harper, 
    Vince Knight
}

\begin{document}

\maketitle

\input{Abstract/main.tex}


\newpage
\tableofcontents

\newpage
\input{Introduction/main.tex}

\newpage
\input{Game_theory_component/main.tex}

\newpage
\input{MarkovChain/markov_chain_model/main.tex}
\input{MarkovChain/expressions_from_pi/main.tex}
\input{MarkovChain/markov_example/main.tex}

\newpage
\input{BehaviouralMethodology/main.tex}

\newpage
\input{Application_EMS_ED/main.tex}

\newpage
\input{Conclusion/main.tex}


\end{document}


\newpage
\tableofcontents

\newpage
\documentclass{article}

\usepackage{amsmath}
\usepackage{amsfonts} 
\usepackage{geometry}
\usepackage{multicol}
\usepackage{float}
% \usepackage{mathtools}
% \usepackage{graphicx}
% \usepackage{soul}
% \usepackage{indentfirst}
\usepackage{tikz}
\usetikzlibrary{calc, automata, chains, arrows.meta, math}
\setcounter{MaxMatrixCols}{20}


\title{A game theoretic model of the behavioural gaming that takes place at the EMS - ED interface}

\author{
    Michalis Panayides, 
    Paul Harper, 
    Vince Knight
}

\begin{document}

\maketitle

\input{Abstract/main.tex}


\newpage
\tableofcontents

\newpage
\input{Introduction/main.tex}

\newpage
\input{Game_theory_component/main.tex}

\newpage
\input{MarkovChain/markov_chain_model/main.tex}
\input{MarkovChain/expressions_from_pi/main.tex}
\input{MarkovChain/markov_example/main.tex}

\newpage
\input{BehaviouralMethodology/main.tex}

\newpage
\input{Application_EMS_ED/main.tex}

\newpage
\input{Conclusion/main.tex}


\end{document}

\newpage
\documentclass{article}

\usepackage{amsmath}
\usepackage{amsfonts} 
\usepackage{geometry}
\usepackage{multicol}
\usepackage{float}
% \usepackage{mathtools}
% \usepackage{graphicx}
% \usepackage{soul}
% \usepackage{indentfirst}
\usepackage{tikz}
\usetikzlibrary{calc, automata, chains, arrows.meta, math}
\setcounter{MaxMatrixCols}{20}


\title{A game theoretic model of the behavioural gaming that takes place at the EMS - ED interface}

\author{
    Michalis Panayides, 
    Paul Harper, 
    Vince Knight
}

\begin{document}

\maketitle

\input{Abstract/main.tex}


\newpage
\tableofcontents

\newpage
\input{Introduction/main.tex}

\newpage
\input{Game_theory_component/main.tex}

\newpage
\input{MarkovChain/markov_chain_model/main.tex}
\input{MarkovChain/expressions_from_pi/main.tex}
\input{MarkovChain/markov_example/main.tex}

\newpage
\input{BehaviouralMethodology/main.tex}

\newpage
\input{Application_EMS_ED/main.tex}

\newpage
\input{Conclusion/main.tex}


\end{document}

\newpage
\documentclass{article}

\usepackage{amsmath}
\usepackage{amsfonts} 
\usepackage{geometry}
\usepackage{multicol}
\usepackage{float}
% \usepackage{mathtools}
% \usepackage{graphicx}
% \usepackage{soul}
% \usepackage{indentfirst}
\usepackage{tikz}
\usetikzlibrary{calc, automata, chains, arrows.meta, math}
\setcounter{MaxMatrixCols}{20}


\title{A game theoretic model of the behavioural gaming that takes place at the EMS - ED interface}

\author{
    Michalis Panayides, 
    Paul Harper, 
    Vince Knight
}

\begin{document}

\maketitle

\input{Abstract/main.tex}


\newpage
\tableofcontents

\newpage
\input{Introduction/main.tex}

\newpage
\input{Game_theory_component/main.tex}

\newpage
\input{MarkovChain/markov_chain_model/main.tex}
\input{MarkovChain/expressions_from_pi/main.tex}
\input{MarkovChain/markov_example/main.tex}

\newpage
\input{BehaviouralMethodology/main.tex}

\newpage
\input{Application_EMS_ED/main.tex}

\newpage
\input{Conclusion/main.tex}


\end{document}
\subsection{Performance Measures}
One may easily derive the average number of individuals that are at any given state 
using \( pi \). 
The average number of individuals in state \( i \) can be calculated by multiplying 
the number of individuals that are present in state \( i \) with the probability 
of being at that particular state (i.e \(\pi_i (u_i + v_i)\)). 
Using this logic it is possible to calculate any performance measures that are related 
to the mean number of individuals in the system.


Average number of people in the system: 
\begin{equation}
    L = \sum_{i=1}^{|\pi|} \pi_i (u_i + v_i)
\end{equation} 

Average number of people in the service centre: 
\begin{equation}
    L_H = \sum_{i=1}^{|\pi|} \pi_i v_i
\end{equation}

Average number of people in the buffer centre:
\begin{equation}
    L_A = \sum_{i=1}^{|\pi|} \pi_i u_i
\end{equation}

Consequently getting the performance measures that are related to the duration of 
time is not as straightforward. 
Such performance measures are the mean waiting time in the system and the mean time 
blocked in the system. 
Under the scope of this study three approaches have been considered to calculate these 
performance measures; a direct approach, a recursive algorithm and consequently a
closed-form formula.

The research question that needs to be answered here is: ``When a class 1/2 
individuals enters the system, what is the expected time that they will have to 
wait?''. 
In order to formulate the answer to that question one needs to consider all possible 
scenarios of what state the system can be in when an individual arrives. 
Furthermore, different formulas arises for class 1 individuals 
and a different one for class 2 individuals.

\subsubsection{Mean waiting time} 
Upon closer inspection of the recursive formula a more compact formula can arise. 
The equivalent closed-form formula eliminates the need for recursion and thus makes 
the computation of waiting times much more efficient. 
Just like in the recursive part there are two formulas; one for \textit{class 1} 
and one for class 2 individuals. 
The formulas are given by:

\begin{equation} \label{eq:closed_form_waiting_others}
    W^{(1)} = \frac{\sum_{\substack{(u,v) \, \in S_A^{(1)} \\ v \geq C}} 
    \frac{1}{C \mu} \times (v-C+1) \times \pi(u,v)}{\sum_{(u,v) \, 
    \in S_A^{(1)}} \pi(u,v)}
\end{equation}
    
\begin{equation}\label{eq:closed_form_waiting_ambulance}
    W^{(2)} = \frac{\sum_{\substack{(u,v) \, \in S_A^{(2)} \\ min(v,T) \geq C}} 
    \frac{1}{C \mu} \times (\min(v+1,T)-C) \times \pi(u,v)}{\sum_{(u,v) \, 
    \in S_A^{(2)}} \pi(u,v)}
\end{equation}

Note here that the summation, in both equations \ref{eq:closed_form_waiting_others} 
and \ref{eq:closed_form_waiting_ambulance}, goes through all states in the set of 
accepting 
states of either class 1 or class 2 individuals respectively, where a wait 
incurs. 
In equation \ref{eq:closed_form_waiting_others} that includes all states \((u,v)\) 
in the set of accepting states of class 1 individuals such that \( v \geq C\); i.e. 
whenever an arrival occurs and the system is at a state where the number of individuals 
in the system is more than or equal to $C$. 
Consequently, for the states that are included in the summation the expression 
\( v-C+1 \) indicates the amount of people in service one would have to wait for 
upon arrival at the hospital.

Additionally, the minimisation function in equation 
\ref{eq:closed_form_waiting_ambulance} 
ensures that when a class 2 individual arrives at any state 
that is greater than the predetermined threshold, the wait that the individual will 
have to endure remains the same. 
In essence, the expression \(\min(v+1,T) - C\) returns the number of people in line 
in front of a particular individual upon arrival.


\subsubsection{Overall Waiting Time}

Consequently, the overall waiting time should can be estimated by a linear combination 
of the waiting times of class 1 and class 2 individuals. 
The overall waiting time can be then given by the following equation where \(c_1\) 
and \(c_2\) are the coefficients of each individual's type waiting time:

\begin{equation}\label{overall_waiting_time_coeff}
    W = c_1 W^{(1)} + c_2 W^{(2)}
\end{equation}

The two coefficients represent the proportion of individuals of each type that 
traversed through the model. 
Theoretically, getting these percentages should be as simple as looking at the arrival 
rates of each type but in practise if the service centre or the buffer centre 
is full, some individuals may be lost to the system. 
Thus, one should account for the probability that an individual is lost to the system. 
This probability can be easily calculated by using the two sets of accepting states 
\(S_A^{(2)}\) and \(S_A^{(1)}\) defined earlier in equations.
Let us define here the probability, for either class type, that an individual 
is not lost in the system by:

\begin{equation*}
    P(L'_1) = \sum_{(u,v) \, \in S_A^{(1)}} \pi(u,v) \hspace{2cm}
    P(L'_2) = \sum_{(u,v) \, \in S_A^{(2)}} \pi(u,v)
\end{equation*}

Having defined these probabilities one may combine them with the arrival rates of 
each class type in such a way to get the expected proportions of class 1 and 
class 2 individuals in the model. 
Thus, by using these values as the coefficient of equation 
\ref{overall_waiting_time_coeff} 
the resultant equation can be used to get the overall waiting time. 
Note here that the equation below gets the overall waiting time for both the recursive 
and the closed-form formula.

\begin{equation}\label{overall_waiting_time}
    W = \frac{\lambda_1 P(L'_1)}{\lambda_2 P(L'_2) + \lambda_1 P(L'_1)} W^{(1)} + 
    \frac{\lambda_2 P(L'_2)}{\lambda_2 P(L'_2) + \lambda_1 P(L'_1)} W^{(2)}
\end{equation}



\subsubsection{Mean blocking time}
Unlike the waiting time, the blocking time is only calculated for class 2 individuals.  
That is because class 1 individuals cannot be blocked. 
Thus, one only needs to consider the pathway of class 2 individuals to get the 
mean blocking time of the system. 
Blocking occurs at states \((u,v)\) where \(u > 0 \). 
Thus, the set of blocking states can be defined as:

\begin{equation*}
    S_b = \{(u,v) \in S \; | \; u > 0\}
\end{equation*}
 
In order to not consider individuals that will be lost to the system, the set of 
accepting states needs to be taken into account. The set of accepting states is given by:

\begin{equation*}
    S_A^{(2)}=
    \begin{cases}
        \{(u, v) \in S \; | \; u < M \} & \textbf{if } T \leq N\\
        \{(u, v) \in S \; | \; v < N \} & \textbf{otherwise}
    \end{cases}
\end{equation*}

For the waiting time formula,
the mean sojourn time for each state was considered,
ignoring any arrivals. Here, the same approach is used but ignoring only class 2
arrivals. That is because for the waiting time formula, once an individual enters 
the service centre (i.e. starts waiting) any individual arriving after them will 
not affect their
pathway. That is not the case for blocking time. When a class 2 individual is 
blocked, 
any class 1 individual that arrives will cause the blocked individual to remain 
blocked for more time. Therefore, class 1 arrivals are considered here:

\begin{equation}\label{eq:time_in_state_blocking_time}
    c(u,v) = 
    \begin{cases}
        \frac{1}{\min(v,C) \mu}, & \text{if } v = C\\
        \frac{1}{\min(v,C) \mu + \lambda_1}, & \text{otherwise}
    \end{cases}
\end{equation}
 
In equation \ref{eq:time_in_state_blocking_time}, both service completions and 
class 1 arrivals are considered. 
Thus, from a blocked individual's perspective whenever the system moves from one 
state \((u,v)\)
to another state it can either:

\begin{itemize}
    \item be because of a service being completed: we will denote the probability 
    of this happening by \(p_s(u,v)\). 
    \item be because of an arrival of an individual of class 1: denoting such 
    probability by \(p_o(u,v)\).
\end{itemize}
The probabilities are given by:

\begin{equation*}
    p_s(u,v) = \frac{\min(v,C)\mu}{\lambda_1 + \min(v,C)\mu}, \qquad
    p_o(u,v) = \frac{\lambda_1}{\lambda_1 + \min(v,C)\mu}
\end{equation*}


Having defined \(c(u,v)\) and \(S_b\) a formula for the blocking time that is
expected to occur at each state can be given by:

\begin{equation}\label{eq:blocking-time-at-each-state}
    b(u,v) = 
    \begin{cases} 
        0, & \textbf{if } (u,v) \notin S_b \\
        c(u,v) + b(u - 1, v), & \textbf{if } v = N = T\\
        c(u,v) + b(u, v-1), & \textbf{if } v = N \neq T \\
        c(u,v) + p_s(u,v) b(u-1, v) + p_o(u,v) b(u, v+1), & \textbf{if } u > 0 
        \textbf{ and } v = T \\
        c(u,v) + p_s(u,v) b(u, v-1) + p_o(u,v) b(u, v+1), & \textbf{otherwise} \\
    \end{cases}
\end{equation}

Equation 
(\ref{eq:blocking-time-at-each-state}) will not be solved recursively. 
A direct approach will be used to solve this equation here. 
By enumerating all equations of (\ref{eq:blocking-time-at-each-state}) for all 
states \((u,v)\) that belong in \(S_b\) 
a system of linear equations arises where the unknown variables are all the \(b(u,v)\)
terms.
For instance, let us consider a Markov model where \(C=2, T=3, N=6, M=2\). 
The Markov model is shown in Figure \ref{fig:example-algeb-blocking}
and the equivalent equations are 
(\ref{eq:first_eq_of_blocking_example})-(\ref{eq:last_eq_of_blocking_example}).
The equations considered here are only the ones that correspond to the blocking 
states.

\begin{multicols*}{2}
    \begin{figure}[H]
        \scalebox{0.50}{\input{MarkovChain/expressions_from_pi/example_model_2362/main.tex}}
        \caption{Example of Markov chain}
        \label{fig:example-algeb-blocking}
    \end{figure}
    \columnbreak
    \begin{align}
        b(1,2) &= c(1,2) + p_o b(1,3) \label{eq:first_eq_of_blocking_example} \\
        b(1,3) &= c(1,3) + p_s b(1,2) + p_o b(1,4) \\
        b(1,4) &= c(1,4) + b(1,3) \\
        b(2,2) &= c(2,2) + p_s b(1,2) + p_o b(2,3) \\
        b(2,3) &= c(2,3) + p_s b(2,2) + p_o b(1,4) \\
        b(2,4) &= c(2,4) + b(2,3)\label{eq:last_eq_of_blocking_example}
    \end{align}
\end{multicols*}

Additionally, the above equations can be transformed into a linear system of the 
form \(Zx=y\) where:

\begin{equation}\label{eq:example-algebaric-approach-blocking-time}
    Z=
    \begin{pmatrix}
        -1 & p_o & 0 & 0 & 0 & 0 \\ %(1,2)
        p_s & -1 & p_o & 0 & 0 & 0 \\ %(1,3)
        0 & 1 & -1 & 0 & 0 & 0 \\ %(1,4)
        p_s & 0 & 0 & -1 & p_o & 0\\ %(2,2)
        0 & 0 & 0 & p_s & -1 & p_o \\ %(2,3)
        0 & 0 & 0 & 0 & 1 & -1 \\ %(2,4)
    \end{pmatrix},
    x=
    \begin{pmatrix}
        b(1,2) \\
        b(1,3) \\
        b(1,4) \\
        b(2,2) \\
        b(2,3) \\
        b(2,4) \\
    \end{pmatrix}, 
    y=
    \begin{pmatrix}
        -c(1,2) \\
        -c(1,3) \\
        -c(1,4) \\
        -c(2,2) \\
        -c(2,3) \\
        -c(2,4) \\
    \end{pmatrix}
\end{equation}

A more generalised form of the equations in 
(\ref{eq:example-algebaric-approach-blocking-time})
can thus be given for any value of \(C,T,N,M\) by:

\begin{align}
    b(1,T) =& c(1, T) + p_o b(1, T + 1) \label{eq:first_eq_of_blocking_general}\\
    b(1,T + 1) =& c(1, T + 1) + p_s(1, T) + p_o b(1, T + 1) \\
    b(1,T + 2) =& c(1, T + 2) + p_s(1, T + 1) + p_o b(1, T + 3) \\
    & \vdots \nonumber \\
    b(1, N) =& c(1, N) + b(1, N - 1) \\
    b(2, T) =& c(2, T) + p_s b(1, T) + p_o b(2, T + 1) \\
    b(2, T + 1) =& c(2, T + 1) + p_s b(2, T) + p_o b(2, T + 2) \\
    & \vdots \nonumber \\
    b(M, T) =& c(M, T) + b(M, T-1) \label{eq:last_eq_of_blocking_general}
\end{align}

The equivalent matrix form of the linear system of equations 
(\ref{eq:first_eq_of_blocking_general}) - (\ref{eq:last_eq_of_blocking_general})
is given by \(Zx=y\), where:
\begin{equation}\label{eq:general-algebaric-approach-blocking-time}
    \scalebox{0.9}{
        \(
        Z = 
        \begin{pmatrix}
            -1 & p_o & 0 & \dots & 0 & 0 & 0 & 0 & 0 & \dots & 0 & 0 \\ %(1,T)
            p_s & -1 & p_o & \dots & 0 & 0 & 0 & 0 & 0 & \dots & 0 & 0 \\ %(1,T+1)
            0 & p_s & -1 & \dots & 0 & 0 & 0 & 0 & 0 & \dots & 0 & 0 \\ %(1,T+2)
            \vdots & \vdots & \vdots & \ddots & \vdots & \vdots & \vdots & \vdots & 
            \vdots & \ddots & \vdots & \vdots \\ 
            0 & 0 & 0 & \dots & 1 & -1 & 0 & 0 & 0 & \dots & 0 & 0 \\ %(1,N)
            p_s & 0 & 0 & \dots & 0 & 0 & -1 & p_o & 0 & \dots & 0 & 0 \\ %(2,T)
            0 & 0 & 0 & \dots & 0 & 0 & p_s & -1 & p_o & \dots & 0 & 0 \\ %(2,T+1)
            \vdots & \vdots & \vdots & \ddots & \vdots & \vdots & \vdots & \vdots & 
            \vdots & \ddots & \vdots & \vdots \\ 
            0 & 0 & 0 & \dots & 0 & 0 & 0 & 0 & 0 & \dots & 1 & -1 \\ %(M,T)
        \end{pmatrix},
        x = 
        \begin{pmatrix}
            b(1,T) \\
            b(1,T+1) \\
            b(1,T+2) \\
            \vdots \\
            b(1,N) \\
            b(2,T) \\
            b(2,T+1) \\
            \vdots \\
            b(M,T) \\
        \end{pmatrix}, 
        y= 
        \begin{pmatrix}
            -c(1,T) \\
            -c(1,T+1) \\
            -c(1,T+2) \\
            \vdots \\
            -c(1,N) \\
            -c(2,T) \\
            -c(2,T+1) \\
            \vdots \\
            -c(M,T) \\
        \end{pmatrix}
        \)
    }
\end{equation}

Thus, having calculated the mean blocking time for all blocking states \(b(u,v)\), 
it only remains to put them together in a formula.
The resultant blocking time formula is given by:

\begin{equation}\label{eq:algebraic-blocking-time}
    B = \frac{\sum_{(u,v) \in S_A} \pi_{(u,v)} \; b(u,v)}{\sum_{(u,v) \in S_A} 
    \pi_{(u,v)}}
\end{equation}

\documentclass{article}

\usepackage{amsmath}
\usepackage{amsfonts} 
\usepackage{geometry}
\usepackage{multicol}
\usepackage{float}
% \usepackage{mathtools}
% \usepackage{graphicx}
% \usepackage{soul}
% \usepackage{indentfirst}
\usepackage{tikz}
\usetikzlibrary{calc, automata, chains, arrows.meta, math}
\setcounter{MaxMatrixCols}{20}


\title{A game theoretic model of the behavioural gaming that takes place at the EMS - ED interface}

\author{
    Michalis Panayides, 
    Paul Harper, 
    Vince Knight
}

\begin{document}

\maketitle

\input{Abstract/main.tex}


\newpage
\tableofcontents

\newpage
\input{Introduction/main.tex}

\newpage
\input{Game_theory_component/main.tex}

\newpage
\input{MarkovChain/markov_chain_model/main.tex}
\input{MarkovChain/expressions_from_pi/main.tex}
\input{MarkovChain/markov_example/main.tex}

\newpage
\input{BehaviouralMethodology/main.tex}

\newpage
\input{Application_EMS_ED/main.tex}

\newpage
\input{Conclusion/main.tex}


\end{document}

\newpage
\documentclass{article}

\usepackage{amsmath}
\usepackage{amsfonts} 
\usepackage{geometry}
\usepackage{multicol}
\usepackage{float}
% \usepackage{mathtools}
% \usepackage{graphicx}
% \usepackage{soul}
% \usepackage{indentfirst}
\usepackage{tikz}
\usetikzlibrary{calc, automata, chains, arrows.meta, math}
\setcounter{MaxMatrixCols}{20}


\title{A game theoretic model of the behavioural gaming that takes place at the EMS - ED interface}

\author{
    Michalis Panayides, 
    Paul Harper, 
    Vince Knight
}

\begin{document}

\maketitle

\input{Abstract/main.tex}


\newpage
\tableofcontents

\newpage
\input{Introduction/main.tex}

\newpage
\input{Game_theory_component/main.tex}

\newpage
\input{MarkovChain/markov_chain_model/main.tex}
\input{MarkovChain/expressions_from_pi/main.tex}
\input{MarkovChain/markov_example/main.tex}

\newpage
\input{BehaviouralMethodology/main.tex}

\newpage
\input{Application_EMS_ED/main.tex}

\newpage
\input{Conclusion/main.tex}


\end{document}

\newpage
\section{EMS-ED application}

\subsection{Application}

\subsection{Data analysis of generated problem}

\newpage
\documentclass{article}

\usepackage{amsmath}
\usepackage{amsfonts} 
\usepackage{geometry}
\usepackage{multicol}
\usepackage{float}
% \usepackage{mathtools}
% \usepackage{graphicx}
% \usepackage{soul}
% \usepackage{indentfirst}
\usepackage{tikz}
\usetikzlibrary{calc, automata, chains, arrows.meta, math}
\setcounter{MaxMatrixCols}{20}


\title{A game theoretic model of the behavioural gaming that takes place at the EMS - ED interface}

\author{
    Michalis Panayides, 
    Paul Harper, 
    Vince Knight
}

\begin{document}

\maketitle

\input{Abstract/main.tex}


\newpage
\tableofcontents

\newpage
\input{Introduction/main.tex}

\newpage
\input{Game_theory_component/main.tex}

\newpage
\input{MarkovChain/markov_chain_model/main.tex}
\input{MarkovChain/expressions_from_pi/main.tex}
\input{MarkovChain/markov_example/main.tex}

\newpage
\input{BehaviouralMethodology/main.tex}

\newpage
\input{Application_EMS_ED/main.tex}

\newpage
\input{Conclusion/main.tex}


\end{document}


\end{document}

\newpage
\section{EMS-ED application}

\subsection{Application}

\subsection{Data analysis of generated problem}

\newpage
\documentclass{article}

\usepackage{amsmath}
\usepackage{amsfonts} 
\usepackage{geometry}
\usepackage{multicol}
\usepackage{float}
% \usepackage{mathtools}
% \usepackage{graphicx}
% \usepackage{soul}
% \usepackage{indentfirst}
\usepackage{tikz}
\usetikzlibrary{calc, automata, chains, arrows.meta, math}
\setcounter{MaxMatrixCols}{20}


\title{A game theoretic model of the behavioural gaming that takes place at the EMS - ED interface}

\author{
    Michalis Panayides, 
    Paul Harper, 
    Vince Knight
}

\begin{document}

\maketitle

\documentclass{article}

\usepackage{amsmath}
\usepackage{amsfonts} 
\usepackage{geometry}
\usepackage{multicol}
\usepackage{float}
% \usepackage{mathtools}
% \usepackage{graphicx}
% \usepackage{soul}
% \usepackage{indentfirst}
\usepackage{tikz}
\usetikzlibrary{calc, automata, chains, arrows.meta, math}
\setcounter{MaxMatrixCols}{20}


\title{A game theoretic model of the behavioural gaming that takes place at the EMS - ED interface}

\author{
    Michalis Panayides, 
    Paul Harper, 
    Vince Knight
}

\begin{document}

\maketitle

\input{Abstract/main.tex}


\newpage
\tableofcontents

\newpage
\input{Introduction/main.tex}

\newpage
\input{Game_theory_component/main.tex}

\newpage
\input{MarkovChain/markov_chain_model/main.tex}
\input{MarkovChain/expressions_from_pi/main.tex}
\input{MarkovChain/markov_example/main.tex}

\newpage
\input{BehaviouralMethodology/main.tex}

\newpage
\input{Application_EMS_ED/main.tex}

\newpage
\input{Conclusion/main.tex}


\end{document}


\newpage
\tableofcontents

\newpage
\documentclass{article}

\usepackage{amsmath}
\usepackage{amsfonts} 
\usepackage{geometry}
\usepackage{multicol}
\usepackage{float}
% \usepackage{mathtools}
% \usepackage{graphicx}
% \usepackage{soul}
% \usepackage{indentfirst}
\usepackage{tikz}
\usetikzlibrary{calc, automata, chains, arrows.meta, math}
\setcounter{MaxMatrixCols}{20}


\title{A game theoretic model of the behavioural gaming that takes place at the EMS - ED interface}

\author{
    Michalis Panayides, 
    Paul Harper, 
    Vince Knight
}

\begin{document}

\maketitle

\input{Abstract/main.tex}


\newpage
\tableofcontents

\newpage
\input{Introduction/main.tex}

\newpage
\input{Game_theory_component/main.tex}

\newpage
\input{MarkovChain/markov_chain_model/main.tex}
\input{MarkovChain/expressions_from_pi/main.tex}
\input{MarkovChain/markov_example/main.tex}

\newpage
\input{BehaviouralMethodology/main.tex}

\newpage
\input{Application_EMS_ED/main.tex}

\newpage
\input{Conclusion/main.tex}


\end{document}

\newpage
\documentclass{article}

\usepackage{amsmath}
\usepackage{amsfonts} 
\usepackage{geometry}
\usepackage{multicol}
\usepackage{float}
% \usepackage{mathtools}
% \usepackage{graphicx}
% \usepackage{soul}
% \usepackage{indentfirst}
\usepackage{tikz}
\usetikzlibrary{calc, automata, chains, arrows.meta, math}
\setcounter{MaxMatrixCols}{20}


\title{A game theoretic model of the behavioural gaming that takes place at the EMS - ED interface}

\author{
    Michalis Panayides, 
    Paul Harper, 
    Vince Knight
}

\begin{document}

\maketitle

\input{Abstract/main.tex}


\newpage
\tableofcontents

\newpage
\input{Introduction/main.tex}

\newpage
\input{Game_theory_component/main.tex}

\newpage
\input{MarkovChain/markov_chain_model/main.tex}
\input{MarkovChain/expressions_from_pi/main.tex}
\input{MarkovChain/markov_example/main.tex}

\newpage
\input{BehaviouralMethodology/main.tex}

\newpage
\input{Application_EMS_ED/main.tex}

\newpage
\input{Conclusion/main.tex}


\end{document}

\newpage
\documentclass{article}

\usepackage{amsmath}
\usepackage{amsfonts} 
\usepackage{geometry}
\usepackage{multicol}
\usepackage{float}
% \usepackage{mathtools}
% \usepackage{graphicx}
% \usepackage{soul}
% \usepackage{indentfirst}
\usepackage{tikz}
\usetikzlibrary{calc, automata, chains, arrows.meta, math}
\setcounter{MaxMatrixCols}{20}


\title{A game theoretic model of the behavioural gaming that takes place at the EMS - ED interface}

\author{
    Michalis Panayides, 
    Paul Harper, 
    Vince Knight
}

\begin{document}

\maketitle

\input{Abstract/main.tex}


\newpage
\tableofcontents

\newpage
\input{Introduction/main.tex}

\newpage
\input{Game_theory_component/main.tex}

\newpage
\input{MarkovChain/markov_chain_model/main.tex}
\input{MarkovChain/expressions_from_pi/main.tex}
\input{MarkovChain/markov_example/main.tex}

\newpage
\input{BehaviouralMethodology/main.tex}

\newpage
\input{Application_EMS_ED/main.tex}

\newpage
\input{Conclusion/main.tex}


\end{document}
\subsection{Performance Measures}
One may easily derive the average number of individuals that are at any given state 
using \( pi \). 
The average number of individuals in state \( i \) can be calculated by multiplying 
the number of individuals that are present in state \( i \) with the probability 
of being at that particular state (i.e \(\pi_i (u_i + v_i)\)). 
Using this logic it is possible to calculate any performance measures that are related 
to the mean number of individuals in the system.


Average number of people in the system: 
\begin{equation}
    L = \sum_{i=1}^{|\pi|} \pi_i (u_i + v_i)
\end{equation} 

Average number of people in the service centre: 
\begin{equation}
    L_H = \sum_{i=1}^{|\pi|} \pi_i v_i
\end{equation}

Average number of people in the buffer centre:
\begin{equation}
    L_A = \sum_{i=1}^{|\pi|} \pi_i u_i
\end{equation}

Consequently getting the performance measures that are related to the duration of 
time is not as straightforward. 
Such performance measures are the mean waiting time in the system and the mean time 
blocked in the system. 
Under the scope of this study three approaches have been considered to calculate these 
performance measures; a direct approach, a recursive algorithm and consequently a
closed-form formula.

The research question that needs to be answered here is: ``When a class 1/2 
individuals enters the system, what is the expected time that they will have to 
wait?''. 
In order to formulate the answer to that question one needs to consider all possible 
scenarios of what state the system can be in when an individual arrives. 
Furthermore, different formulas arises for class 1 individuals 
and a different one for class 2 individuals.

\subsubsection{Mean waiting time} 
Upon closer inspection of the recursive formula a more compact formula can arise. 
The equivalent closed-form formula eliminates the need for recursion and thus makes 
the computation of waiting times much more efficient. 
Just like in the recursive part there are two formulas; one for \textit{class 1} 
and one for class 2 individuals. 
The formulas are given by:

\begin{equation} \label{eq:closed_form_waiting_others}
    W^{(1)} = \frac{\sum_{\substack{(u,v) \, \in S_A^{(1)} \\ v \geq C}} 
    \frac{1}{C \mu} \times (v-C+1) \times \pi(u,v)}{\sum_{(u,v) \, 
    \in S_A^{(1)}} \pi(u,v)}
\end{equation}
    
\begin{equation}\label{eq:closed_form_waiting_ambulance}
    W^{(2)} = \frac{\sum_{\substack{(u,v) \, \in S_A^{(2)} \\ min(v,T) \geq C}} 
    \frac{1}{C \mu} \times (\min(v+1,T)-C) \times \pi(u,v)}{\sum_{(u,v) \, 
    \in S_A^{(2)}} \pi(u,v)}
\end{equation}

Note here that the summation, in both equations \ref{eq:closed_form_waiting_others} 
and \ref{eq:closed_form_waiting_ambulance}, goes through all states in the set of 
accepting 
states of either class 1 or class 2 individuals respectively, where a wait 
incurs. 
In equation \ref{eq:closed_form_waiting_others} that includes all states \((u,v)\) 
in the set of accepting states of class 1 individuals such that \( v \geq C\); i.e. 
whenever an arrival occurs and the system is at a state where the number of individuals 
in the system is more than or equal to $C$. 
Consequently, for the states that are included in the summation the expression 
\( v-C+1 \) indicates the amount of people in service one would have to wait for 
upon arrival at the hospital.

Additionally, the minimisation function in equation 
\ref{eq:closed_form_waiting_ambulance} 
ensures that when a class 2 individual arrives at any state 
that is greater than the predetermined threshold, the wait that the individual will 
have to endure remains the same. 
In essence, the expression \(\min(v+1,T) - C\) returns the number of people in line 
in front of a particular individual upon arrival.


\subsubsection{Overall Waiting Time}

Consequently, the overall waiting time should can be estimated by a linear combination 
of the waiting times of class 1 and class 2 individuals. 
The overall waiting time can be then given by the following equation where \(c_1\) 
and \(c_2\) are the coefficients of each individual's type waiting time:

\begin{equation}\label{overall_waiting_time_coeff}
    W = c_1 W^{(1)} + c_2 W^{(2)}
\end{equation}

The two coefficients represent the proportion of individuals of each type that 
traversed through the model. 
Theoretically, getting these percentages should be as simple as looking at the arrival 
rates of each type but in practise if the service centre or the buffer centre 
is full, some individuals may be lost to the system. 
Thus, one should account for the probability that an individual is lost to the system. 
This probability can be easily calculated by using the two sets of accepting states 
\(S_A^{(2)}\) and \(S_A^{(1)}\) defined earlier in equations.
Let us define here the probability, for either class type, that an individual 
is not lost in the system by:

\begin{equation*}
    P(L'_1) = \sum_{(u,v) \, \in S_A^{(1)}} \pi(u,v) \hspace{2cm}
    P(L'_2) = \sum_{(u,v) \, \in S_A^{(2)}} \pi(u,v)
\end{equation*}

Having defined these probabilities one may combine them with the arrival rates of 
each class type in such a way to get the expected proportions of class 1 and 
class 2 individuals in the model. 
Thus, by using these values as the coefficient of equation 
\ref{overall_waiting_time_coeff} 
the resultant equation can be used to get the overall waiting time. 
Note here that the equation below gets the overall waiting time for both the recursive 
and the closed-form formula.

\begin{equation}\label{overall_waiting_time}
    W = \frac{\lambda_1 P(L'_1)}{\lambda_2 P(L'_2) + \lambda_1 P(L'_1)} W^{(1)} + 
    \frac{\lambda_2 P(L'_2)}{\lambda_2 P(L'_2) + \lambda_1 P(L'_1)} W^{(2)}
\end{equation}



\subsubsection{Mean blocking time}
Unlike the waiting time, the blocking time is only calculated for class 2 individuals.  
That is because class 1 individuals cannot be blocked. 
Thus, one only needs to consider the pathway of class 2 individuals to get the 
mean blocking time of the system. 
Blocking occurs at states \((u,v)\) where \(u > 0 \). 
Thus, the set of blocking states can be defined as:

\begin{equation*}
    S_b = \{(u,v) \in S \; | \; u > 0\}
\end{equation*}
 
In order to not consider individuals that will be lost to the system, the set of 
accepting states needs to be taken into account. The set of accepting states is given by:

\begin{equation*}
    S_A^{(2)}=
    \begin{cases}
        \{(u, v) \in S \; | \; u < M \} & \textbf{if } T \leq N\\
        \{(u, v) \in S \; | \; v < N \} & \textbf{otherwise}
    \end{cases}
\end{equation*}

For the waiting time formula,
the mean sojourn time for each state was considered,
ignoring any arrivals. Here, the same approach is used but ignoring only class 2
arrivals. That is because for the waiting time formula, once an individual enters 
the service centre (i.e. starts waiting) any individual arriving after them will 
not affect their
pathway. That is not the case for blocking time. When a class 2 individual is 
blocked, 
any class 1 individual that arrives will cause the blocked individual to remain 
blocked for more time. Therefore, class 1 arrivals are considered here:

\begin{equation}\label{eq:time_in_state_blocking_time}
    c(u,v) = 
    \begin{cases}
        \frac{1}{\min(v,C) \mu}, & \text{if } v = C\\
        \frac{1}{\min(v,C) \mu + \lambda_1}, & \text{otherwise}
    \end{cases}
\end{equation}
 
In equation \ref{eq:time_in_state_blocking_time}, both service completions and 
class 1 arrivals are considered. 
Thus, from a blocked individual's perspective whenever the system moves from one 
state \((u,v)\)
to another state it can either:

\begin{itemize}
    \item be because of a service being completed: we will denote the probability 
    of this happening by \(p_s(u,v)\). 
    \item be because of an arrival of an individual of class 1: denoting such 
    probability by \(p_o(u,v)\).
\end{itemize}
The probabilities are given by:

\begin{equation*}
    p_s(u,v) = \frac{\min(v,C)\mu}{\lambda_1 + \min(v,C)\mu}, \qquad
    p_o(u,v) = \frac{\lambda_1}{\lambda_1 + \min(v,C)\mu}
\end{equation*}


Having defined \(c(u,v)\) and \(S_b\) a formula for the blocking time that is
expected to occur at each state can be given by:

\begin{equation}\label{eq:blocking-time-at-each-state}
    b(u,v) = 
    \begin{cases} 
        0, & \textbf{if } (u,v) \notin S_b \\
        c(u,v) + b(u - 1, v), & \textbf{if } v = N = T\\
        c(u,v) + b(u, v-1), & \textbf{if } v = N \neq T \\
        c(u,v) + p_s(u,v) b(u-1, v) + p_o(u,v) b(u, v+1), & \textbf{if } u > 0 
        \textbf{ and } v = T \\
        c(u,v) + p_s(u,v) b(u, v-1) + p_o(u,v) b(u, v+1), & \textbf{otherwise} \\
    \end{cases}
\end{equation}

Equation 
(\ref{eq:blocking-time-at-each-state}) will not be solved recursively. 
A direct approach will be used to solve this equation here. 
By enumerating all equations of (\ref{eq:blocking-time-at-each-state}) for all 
states \((u,v)\) that belong in \(S_b\) 
a system of linear equations arises where the unknown variables are all the \(b(u,v)\)
terms.
For instance, let us consider a Markov model where \(C=2, T=3, N=6, M=2\). 
The Markov model is shown in Figure \ref{fig:example-algeb-blocking}
and the equivalent equations are 
(\ref{eq:first_eq_of_blocking_example})-(\ref{eq:last_eq_of_blocking_example}).
The equations considered here are only the ones that correspond to the blocking 
states.

\begin{multicols*}{2}
    \begin{figure}[H]
        \scalebox{0.50}{\input{MarkovChain/expressions_from_pi/example_model_2362/main.tex}}
        \caption{Example of Markov chain}
        \label{fig:example-algeb-blocking}
    \end{figure}
    \columnbreak
    \begin{align}
        b(1,2) &= c(1,2) + p_o b(1,3) \label{eq:first_eq_of_blocking_example} \\
        b(1,3) &= c(1,3) + p_s b(1,2) + p_o b(1,4) \\
        b(1,4) &= c(1,4) + b(1,3) \\
        b(2,2) &= c(2,2) + p_s b(1,2) + p_o b(2,3) \\
        b(2,3) &= c(2,3) + p_s b(2,2) + p_o b(1,4) \\
        b(2,4) &= c(2,4) + b(2,3)\label{eq:last_eq_of_blocking_example}
    \end{align}
\end{multicols*}

Additionally, the above equations can be transformed into a linear system of the 
form \(Zx=y\) where:

\begin{equation}\label{eq:example-algebaric-approach-blocking-time}
    Z=
    \begin{pmatrix}
        -1 & p_o & 0 & 0 & 0 & 0 \\ %(1,2)
        p_s & -1 & p_o & 0 & 0 & 0 \\ %(1,3)
        0 & 1 & -1 & 0 & 0 & 0 \\ %(1,4)
        p_s & 0 & 0 & -1 & p_o & 0\\ %(2,2)
        0 & 0 & 0 & p_s & -1 & p_o \\ %(2,3)
        0 & 0 & 0 & 0 & 1 & -1 \\ %(2,4)
    \end{pmatrix},
    x=
    \begin{pmatrix}
        b(1,2) \\
        b(1,3) \\
        b(1,4) \\
        b(2,2) \\
        b(2,3) \\
        b(2,4) \\
    \end{pmatrix}, 
    y=
    \begin{pmatrix}
        -c(1,2) \\
        -c(1,3) \\
        -c(1,4) \\
        -c(2,2) \\
        -c(2,3) \\
        -c(2,4) \\
    \end{pmatrix}
\end{equation}

A more generalised form of the equations in 
(\ref{eq:example-algebaric-approach-blocking-time})
can thus be given for any value of \(C,T,N,M\) by:

\begin{align}
    b(1,T) =& c(1, T) + p_o b(1, T + 1) \label{eq:first_eq_of_blocking_general}\\
    b(1,T + 1) =& c(1, T + 1) + p_s(1, T) + p_o b(1, T + 1) \\
    b(1,T + 2) =& c(1, T + 2) + p_s(1, T + 1) + p_o b(1, T + 3) \\
    & \vdots \nonumber \\
    b(1, N) =& c(1, N) + b(1, N - 1) \\
    b(2, T) =& c(2, T) + p_s b(1, T) + p_o b(2, T + 1) \\
    b(2, T + 1) =& c(2, T + 1) + p_s b(2, T) + p_o b(2, T + 2) \\
    & \vdots \nonumber \\
    b(M, T) =& c(M, T) + b(M, T-1) \label{eq:last_eq_of_blocking_general}
\end{align}

The equivalent matrix form of the linear system of equations 
(\ref{eq:first_eq_of_blocking_general}) - (\ref{eq:last_eq_of_blocking_general})
is given by \(Zx=y\), where:
\begin{equation}\label{eq:general-algebaric-approach-blocking-time}
    \scalebox{0.9}{
        \(
        Z = 
        \begin{pmatrix}
            -1 & p_o & 0 & \dots & 0 & 0 & 0 & 0 & 0 & \dots & 0 & 0 \\ %(1,T)
            p_s & -1 & p_o & \dots & 0 & 0 & 0 & 0 & 0 & \dots & 0 & 0 \\ %(1,T+1)
            0 & p_s & -1 & \dots & 0 & 0 & 0 & 0 & 0 & \dots & 0 & 0 \\ %(1,T+2)
            \vdots & \vdots & \vdots & \ddots & \vdots & \vdots & \vdots & \vdots & 
            \vdots & \ddots & \vdots & \vdots \\ 
            0 & 0 & 0 & \dots & 1 & -1 & 0 & 0 & 0 & \dots & 0 & 0 \\ %(1,N)
            p_s & 0 & 0 & \dots & 0 & 0 & -1 & p_o & 0 & \dots & 0 & 0 \\ %(2,T)
            0 & 0 & 0 & \dots & 0 & 0 & p_s & -1 & p_o & \dots & 0 & 0 \\ %(2,T+1)
            \vdots & \vdots & \vdots & \ddots & \vdots & \vdots & \vdots & \vdots & 
            \vdots & \ddots & \vdots & \vdots \\ 
            0 & 0 & 0 & \dots & 0 & 0 & 0 & 0 & 0 & \dots & 1 & -1 \\ %(M,T)
        \end{pmatrix},
        x = 
        \begin{pmatrix}
            b(1,T) \\
            b(1,T+1) \\
            b(1,T+2) \\
            \vdots \\
            b(1,N) \\
            b(2,T) \\
            b(2,T+1) \\
            \vdots \\
            b(M,T) \\
        \end{pmatrix}, 
        y= 
        \begin{pmatrix}
            -c(1,T) \\
            -c(1,T+1) \\
            -c(1,T+2) \\
            \vdots \\
            -c(1,N) \\
            -c(2,T) \\
            -c(2,T+1) \\
            \vdots \\
            -c(M,T) \\
        \end{pmatrix}
        \)
    }
\end{equation}

Thus, having calculated the mean blocking time for all blocking states \(b(u,v)\), 
it only remains to put them together in a formula.
The resultant blocking time formula is given by:

\begin{equation}\label{eq:algebraic-blocking-time}
    B = \frac{\sum_{(u,v) \in S_A} \pi_{(u,v)} \; b(u,v)}{\sum_{(u,v) \in S_A} 
    \pi_{(u,v)}}
\end{equation}

\documentclass{article}

\usepackage{amsmath}
\usepackage{amsfonts} 
\usepackage{geometry}
\usepackage{multicol}
\usepackage{float}
% \usepackage{mathtools}
% \usepackage{graphicx}
% \usepackage{soul}
% \usepackage{indentfirst}
\usepackage{tikz}
\usetikzlibrary{calc, automata, chains, arrows.meta, math}
\setcounter{MaxMatrixCols}{20}


\title{A game theoretic model of the behavioural gaming that takes place at the EMS - ED interface}

\author{
    Michalis Panayides, 
    Paul Harper, 
    Vince Knight
}

\begin{document}

\maketitle

\input{Abstract/main.tex}


\newpage
\tableofcontents

\newpage
\input{Introduction/main.tex}

\newpage
\input{Game_theory_component/main.tex}

\newpage
\input{MarkovChain/markov_chain_model/main.tex}
\input{MarkovChain/expressions_from_pi/main.tex}
\input{MarkovChain/markov_example/main.tex}

\newpage
\input{BehaviouralMethodology/main.tex}

\newpage
\input{Application_EMS_ED/main.tex}

\newpage
\input{Conclusion/main.tex}


\end{document}

\newpage
\documentclass{article}

\usepackage{amsmath}
\usepackage{amsfonts} 
\usepackage{geometry}
\usepackage{multicol}
\usepackage{float}
% \usepackage{mathtools}
% \usepackage{graphicx}
% \usepackage{soul}
% \usepackage{indentfirst}
\usepackage{tikz}
\usetikzlibrary{calc, automata, chains, arrows.meta, math}
\setcounter{MaxMatrixCols}{20}


\title{A game theoretic model of the behavioural gaming that takes place at the EMS - ED interface}

\author{
    Michalis Panayides, 
    Paul Harper, 
    Vince Knight
}

\begin{document}

\maketitle

\input{Abstract/main.tex}


\newpage
\tableofcontents

\newpage
\input{Introduction/main.tex}

\newpage
\input{Game_theory_component/main.tex}

\newpage
\input{MarkovChain/markov_chain_model/main.tex}
\input{MarkovChain/expressions_from_pi/main.tex}
\input{MarkovChain/markov_example/main.tex}

\newpage
\input{BehaviouralMethodology/main.tex}

\newpage
\input{Application_EMS_ED/main.tex}

\newpage
\input{Conclusion/main.tex}


\end{document}

\newpage
\section{EMS-ED application}

\subsection{Application}

\subsection{Data analysis of generated problem}

\newpage
\documentclass{article}

\usepackage{amsmath}
\usepackage{amsfonts} 
\usepackage{geometry}
\usepackage{multicol}
\usepackage{float}
% \usepackage{mathtools}
% \usepackage{graphicx}
% \usepackage{soul}
% \usepackage{indentfirst}
\usepackage{tikz}
\usetikzlibrary{calc, automata, chains, arrows.meta, math}
\setcounter{MaxMatrixCols}{20}


\title{A game theoretic model of the behavioural gaming that takes place at the EMS - ED interface}

\author{
    Michalis Panayides, 
    Paul Harper, 
    Vince Knight
}

\begin{document}

\maketitle

\input{Abstract/main.tex}


\newpage
\tableofcontents

\newpage
\input{Introduction/main.tex}

\newpage
\input{Game_theory_component/main.tex}

\newpage
\input{MarkovChain/markov_chain_model/main.tex}
\input{MarkovChain/expressions_from_pi/main.tex}
\input{MarkovChain/markov_example/main.tex}

\newpage
\input{BehaviouralMethodology/main.tex}

\newpage
\input{Application_EMS_ED/main.tex}

\newpage
\input{Conclusion/main.tex}


\end{document}


\end{document}


\end{document}
