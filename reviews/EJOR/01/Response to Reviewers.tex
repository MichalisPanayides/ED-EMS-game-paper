\documentclass{article}

\title{Response to reviewers}
\author{A game theoretic model of the behavioural gaming that takes \\
place at the ED-EMS interface}

\begin{document}
    \maketitle

    The authors would like to thank all the reviewers for their
    constructive feedback and for
    the time they took to review the manuscript.
    Their suggestions were extremely helpful and we believe this has resulted
    in a much improved manuscript.
    Below is a point by point response and summary of changes made, which have
    been grouped together if reviewers provided similar
    suggestions/improvements.


    \line(1, 0){0.92\textwidth}
    \begin{itemize}
        \item Reviewer 1 - Comment 2: \textit{Combine Sections 2 and 5
        together to demonstrate the underlying queueing models and the
        EMS - ED interface counterpart.
        The current arrangement is not easy to follow for finding the right
        explanations for each part of the assumptions adopted in the
        article.}

        \item Reviewer 3 - Comment 1: \textit{Sections 1 and 2 are
        difficult to read.
        It would be better to focus on the main contribution and its
        motivation in Section 1, and to put the literature review in a
        separate Section.}
    \end{itemize}
    The paper has been completely restructured.
    Sections 2 and 5 have been combined and the game theoretic model is now
    presented along with the results in one section.
    In addition, section 1 contains only the introduction while the
    literature review has moved to section 2.
    Finally, the queueing model section now comes before the game theoretic
    section.
    The current structure is: Introduction; Literature Review; A queueing
    model for the ED-EMS interface; Strategic manipulation of the ED-EMS
    interface; Results.
    We thank the reviewers for their suggestions and hope that the major
    restructuring now ensures that the manuscript flows in a much more coherent
    way.

    \line(1, 0){0.92\textwidth}
    \begin{itemize}
        \item Reviewer 3 - Comment 2: \textit{Since the application in mind
        is the EMS-ED interface, it would be better to write the paper
        immediately from this perspective, instead of first abstracting it
        and then applying it to that problem.}
        \item Reviewer 3 - Comment 8: \textit{Figs 7-11: Row player and
        Column Player are unclear: hospital 1 and hospital 2 would be a
        better choice I presume.}
    \end{itemize}
    The paper is presented with the application of the model as the focus.
    All sections are now presented from the perspective of the EMS-ED
    interface.
    In the game theoretic section instead of row/column player we now use
    hospital A and hospital B.

    \line(1, 0){0.92\textwidth}
    \begin{itemize}
        \item Reviewer 1 - Comment 4: \textit{To highlight the contributions
        of this work, it is necessary to compare the novelty of this work
        with previous results in the literature.}
    \end{itemize}
    In the new literature review section we now directly compare the results
    from previous work with the results from this work to highlight the
    contributions of this work.

    \line(1, 0){0.92\textwidth}
    \begin{itemize}
        \item Reviewer 3 - Comment 6: \textit{Figure 3: about solving the
        Markov chain: the Markov chain is truncated to solve it. Therefore,
        probably N and M are not too small. What about the numerical
        complexity? Because if you just solve it, you have to invert the
        matrix? Is there some structure in the Markov chain that can be
        exploited if N and M would be large?}
        \item Reviewer 3 - Comment 7: \textit{Figures 4-6: why do you
        compare your calculated values with simulation?
        Is it to evaluate the truncations?
        Or are you not sure about the formulas?}
    \end{itemize}
    This observation was really helpful and lead to us conducting more numerical
    experiments to clarify the point of truncating the state space.
    As the values of \(N\) and \(M\) increase, the truncated Markov chain
    becomes more complex to solve.
    We have touched upon building a closed-form formula for the steady state
    probabilities but that is still a work in progress and not something we are
    able to include in this manuscript.
    Figures 4-6 are there to demonstrate the accuracy of our constructed
    model against its discrete event simulation equivalent and these have been
    expanded on.
    An additional paragraph is now included that discusses the truncation of
    the model
    and the previous figures have been replaced with ones that compare the
    results from the truncated Markov model with the results from the
    untruncated simulation model.
    We feel that these extra numeric results
    help clarify the paper and are thankful for the reviewer comments.

    \line(1, 0){0.92\textwidth}
    \begin{itemize}
        \item Reviewer 3 - Comment 3: \textit{In section 2, a diagram
        of the full system (including the two queues and the EMS) would
        increase readability.}
    \end{itemize}
    A full system diagram is now added in the game theoretic model section.

    \line(1, 0){0.92\textwidth}
    \begin{itemize}
        \item Reviewer 3 - Comment 9: \textit{Page 22 scenario at bottom of
        the page: this scenario does not seem like a realistic scenario:
        the mean number of patients offered per time unit is larger than
        that can be treated, meaning that many patients would be rejected
        by the hospital. Please comment on the reality of the selected
        scenarios}
    \end{itemize}
    The queueing systems in the particular example are meant to be extremely
    busy systems.
    In the new manuscript this is more clearly stated.

    \line(1, 0){0.92\textwidth}
    \begin{itemize}
        \item Reviewer 3 - Comment 5: \textit{A key modelling assumption is
        that a fraction \(p_A\) of patients is distributed to hospital A.
        Some comments on how realistic this is should be added. Do
        distributers have a choice? Don't they just have to choose the
        nearest hospital? Do they really do it randomly, don' t they take
        into account queues at the hospitals (e.g. join the shortest
        ambulance queue)? what about urgency of patients? I would expect
        patients that come in are prioritized since they are probably more
        urgent.}
    \end{itemize}
    An additional paragraph has been added that touches upon the possibility of
    more complex queueing models where the hospital distance or patient's
    priority type are considered.

    \line(1, 0){0.92\textwidth}
    \begin{itemize}
        \item Reviewer 3 - Comment 4: \textit{The authors claim their model
        has applications in a range of sectors and settings, but have given
        no examples.}
    \end{itemize}
    Examples of other applications have been added in the conclusion section
    of the manuscript.

    \line(1, 0){0.92\textwidth}
    \begin{itemize}
        \item Reviewer 1 - Comment 3: \textit{In the game-theoretic
        formulation and analysis part, it is better to present the
        essential components of a normal-form representation, including the
        players in the game, the strategies available to each player and the
        payoff received by each player for each combination of strategies
        that could be chosen by the players, etc.}
    \end{itemize}
    In the new manuscript the normal form game is explicitly defined in the
    game theoretic section. The players, strategies and payoffs are clearly
    defined.

    \line(1, 0){0.92\textwidth}
    \begin{itemize}
        \item Reviewer 1 - Comment 1: \textit{Rewrite the Highlights (for
        review) of the paper, and really capture the sense of work.}
    \end{itemize}
    Highlights have been rewritten to be more explicit in terms of the work
    that was done.

    \line(1, 0){0.92\textwidth}
    \begin{itemize}
        \item Reviewer 1 - Comment 5: \textit{There are some mistakes/typos
        with English grammar and writing in the article.}
    \end{itemize}
    We have thoroughly checked the grammar and spelling of the manuscript.


\end{document}

