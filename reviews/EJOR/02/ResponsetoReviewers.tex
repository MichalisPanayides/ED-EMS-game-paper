\documentclass{article}

\title{Response to reviewers}
\author{A game theoretic model of the behavioural gaming that takes \\
place at the ED-EMS interface}

\begin{document}
    \maketitle

    The authors would like to thank once again all the reviewers for their
    constructive feedback and for
    the time they took to review the manuscript.
    We are also pleased to see that the reviewers feel that the revision has addressed their comments and that the manuscript presents scientifically correct and useful results.
    We have made the remaining requested minor changes and also provide a response to one of the 
    reviewers whose remaining concern is about the fit of this paper for EJOR.


    \line(1, 0){0.92\textwidth}
    \begin{itemize}
        \item Reviewer 1: \textit{The following two papers are related to the
        current work and I suggest the authors add them in the references with
        citations in the literature review: \\
        1. Chen, W., Z. G. Zhang, and X. Chen. 2020. "On Two-tier Healthcare
        System Under Capacity Constraint." International Journal of Production
        Research 58 (12): 3744-3764. \\
        2. Wang, J., Z. Wang, Z. G. Zhang and F. Wang. 2021.
        ``Efficiency-quality trade-off in allocating resource to public
        healthcare systems.'' International Journal of Production Research, \\
        DOI:10.1080/00207543.2021.1992529}
    \end{itemize}

    These two papers are now cited in the manuscript at the literature review
    section.

    \line(1, 0){0.92\textwidth}
    \begin{itemize}
        \item Reviewer 2: I gave a look into the new version and the comments
        made by the authors following the reviews they received.
        My criticism towards the paper as stated in the first round stays.
        It is basically the question of what are the standards of EJOR.
        If a paper which states a interesting model and gives details on
        numerical procedures for solving it but without any theorems and
        insights is good enough, then it should be accepted as the paper is
        doing a good job of what it promises.
        Otherwise, they should look for a more applied journal where what they
        do is sufficient.
    \end{itemize}

    We are grateful for the opportunity to respond to this as it seems that the 
    remaining concern is the fit of this work in EJOR and not the work itself.
    Firstly, this paper does provide theoretical results
    and insights.
    Section 3 provides detailed theoretical expressions of the queueing model's
    performance measures.
    Expressions of the waiting time, blocking time and proportion of
    individuals within target have been derived and are presented in the paper.
    Also, section 4.2 contains a list of applied scenarios where insights 
    are presented.
    A busy model is examined and insights on ways to increase the relative
    efficiency of the model are presented.

    We do understand the main point being raised by the reviewer in that this paper is not a purely theoretic one.
    For the benefits of the handling editor who we believe is left with the decision of whether or not the paper is in the scope of EJOR, here is a list of a number of papers published in EJOR that have a similar applied emphasis:
    \begin{itemize}
        \item A queueing model for general group screening policies and dynamic
        item arrivals \\
        https://doi.org/10.1016/j.ejor.2010.05.042
        \item Selfish routing in public services: \\
        https://doi.org/10.1016/j.ejor.2013.04.003
        \item The effect of ambulance relocations on the performance of
        ambulance service providers \\
        https://doi.org/10.1016/j.ejor.2015.12.022
        \item Robust and stochastic formulations for ambulance deployment and
        dispatch \\
        https://doi.org/10.1016/j.ejor.2019.05.011
        \item A Markovian queueing model for ambulance offload delays \\
        https://doi.org/10.1016/j.ejor.2012.11.030
        \item Models and algorithms for an integrated vessel scheduling and tug
        assignment problem within a canal harbor \\
        https://doi.org/10.1016/j.ejor.2021.10.037
        \item The application of operational research to European air traffic
        flow management - understanding the context \\
        https://doi.org/10.1016/S0377-2217(99)00084-3
    \end{itemize}
    
    Finally, note that the aims and scope of EJOR do not emphasise that a paper
    should present a new theorem in order to be published.
    Here's the aim and scope of EJOR as listed in their website: \\[0.5cm]
    \textit{
        The European Journal of Operational Research (EJOR) publishes high
        quality, original papers that contribute to the methodology of operational
        research (OR) and to the practice of decision making. EJOR contains the
        following types of papers:
        \begin{itemize}
            \item Invited Reviews, explaining to the general OR audience the developments in an OR topic over the recent years
            \item Innovative Applications of OR, describing novel ways to solve real problems
            \item Theory and Methodology Papers, presenting original research results contributing to the methodology of OR and to its theoretical foundations,
            \item Short Communications, if they correct important errors found in papers previously published in EJOR
        \end{itemize}
    }

\end{document}